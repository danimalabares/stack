\input{preamble}

\begin{document}

\title{Algebraic Geometry}
\maketitle

\phantomsection
\label{section-phantom}
\hfill
\href{http://github.com/danimalabares/stack}{github.com/danimalabares/stack}

\tableofcontents

\section{Sheaves}
\label{section-sheaves}

For a definition of presheaf,
see Categories, Definition \ref{categories-definition-presheaf}.

\begin{definition}
\label{definition-sheaf}
Let $X$ be a topological space.
\begin{enumerate}
\item A {\it sheaf $\mathcal{F}$ of sets on $X$} is a presheaf
of sets which satisfies the following additional property: Given
any open covering $U = \bigcup_{i \in I} U_i$ and any collection
of sections $s_i \in \mathcal{F}(U_i)$, $i \in I$ such that
$\forall i, j\in I$
$$
s_i|_{U_i \cap U_j} = s_j|_{U_i \cap U_j}
$$
there exists a unique section $s \in \mathcal{F}(U)$ such that
$s_i = s|_{U_i}$ for all $i \in I$.
\item A {\it morphism of sheaves of sets} is simply a
morphism of presheaves of sets.
\item The category of sheaves of sets on $X$ is denoted
$\Sh(X)$.
\end{enumerate}
\end{definition}


\medskip\noindent
Let $X$ be a topological space. Let $x \in X$ be a point.
Let $\mathcal{F}$ be a presheaf of sets on $X$.
The {\it stalk of $\mathcal{F}$ at $x$} is the set
$$
\mathcal{F}_x
=
\colim_{x\in U} \mathcal{F}(U)
$$
where the colimit is over the set of open neighbourhoods
$U$ of $x$ in $X$. The set of open neighbourhoods is
partially ordered by (reverse) inclusion:
We say $U \geq U' \Leftrightarrow U \subset U'$.
The transition maps in the system are
given by the restriction maps of $\mathcal{F}$.
See Categories, Section \ref{categories-section-posets-limits}
for notation and terminology regarding (co)limits over systems.
Note that the colimit is a directed colimit.
Thus it is easy to describe $\mathcal{F}_x$. Namely,
$$
\mathcal{F}_x
=
\{
(U, s)
\mid
x\in U, s\in \mathcal{F}(U)
\}/\sim
$$
with equivalence relation given by $(U, s) \sim (U', s')$ if and only if
there exists an open $U'' \subset U \cap U'$ with $x \in U''$ and
$s|_{U''} = s'|_{U''}$. Given a pair $(U, s)$ we sometimes denote
$s_x$ the element of $\mathcal{F}_x$ corresponding to the equivalence
class of $(U, x)$. We sometimes use the phrase
``image of $s$ in $\mathcal{F}_x$'' to denote $s_x$.
For example, given two pairs $(U, s)$ and $(U', s')$ we sometimes
say ``$s$ is equal to $s'$ in $\mathcal{F}_x$'' to indicate
that $s_x = s'_x$. Other authors use the terminology
``germ of $s$ at $x$''.

\section{Abelian sheaves}
\label{section-abelian-sheaves}

The following may be used to define the ideal
sheaf of a variety:

\begin{lemma}
\label{lemma-sheaves-valued-on-groups-have-kernels}
Let $X$ be a topological space and
$\mathcal{F}$ and $\mathcal{G}$ be sheaves over $X$
with values on $\mathit{Grp}$.
For every morphism of sheaves
$f:\mathcal{F}\to \mathcal{G}$,
\begin{align*}
\Ker f: \mathit{Open}_X^{\text{op}} &\longrightarrow \mathit{Sets} \\
U &\longmapsto \Ker f(U)\\
(i:V \to U)&\longmapsto 
\substack{
\Ker f(i):\Ker f(U) \to \Ker f(V)\\
x \mapsto x}
\end{align*}
is a sheaf over $X$.
\end{lemma}

\begin{proof}
First observe that the correspondence
on morphisms is well-defined. Indeed, 
$\Ker f(U)\subset \mathcal{F}(U) \subset \mathcal{F}(V)$ 
when $V \subset U$, and we just apply $f(U)$ 
to notice that $\Ker f(V) \subset \Ker f(U)$.

To see this is a presheaf notice it is obvious
that the identity is mapped to the identity
by definition of the correspondence of morphisms.
It is also obvious that composition is preserved.

To see it is a sheaf consider an open
cover $U_i$ of $U$, and elements $x_i \in \Ker f(U_i)$.
Then use the property of $\mathcal{F}$ being
a sheaf to reconstruct an element $x \in \mathcal{F}(U)$,
whose image under $f$ will be mapped to the
identity element of $\mathcal{G}(U)$ because
it does so in every point of the cover of $U$.
Thus $x$ is in $\Ker f(U)$ as desired.
\end{proof}

\medskip\noindent
More formally,

\begin{definition}
\label{definition-abelian-presheaves}
Let $X$ be a topological space.
\begin{enumerate}
\item A {\it presheaf of abelian groups on $X$} or an
{\it abelian presheaf over $X$}
is a presheaf of sets $\mathcal{F}$ such that for each open
$U \subset X$ the set $\mathcal{F}(U)$ is endowed with
the structure of an abelian group, and such that all restriction
maps $\rho^U_V$ are homomorphisms of abelian groups, see
Lemma \ref{lemma-abelian-presheaves} above.
\item A {\it morphism of abelian presheaves over $X$}
$\varphi : \mathcal{F} \to \mathcal{G}$ is a morphism of presheaves
of sets which induces
a homomorphism of abelian groups $\mathcal{F}(U) \to \mathcal{G}(U)$
for every open $U \subset X$.
\item The category of presheaves of abelian groups on $X$ is denoted
$\textit{PAb}(X)$.
\end{enumerate}
\end{definition}

\medskip\noindent
\begin{definition}
\label{definition-abelian-sheaf}
Let $X$ be a topological space.
\begin{enumerate}
\item An {\it abelian sheaf on $X$} or
{\it sheaf of abelian groups on $X$}
is an abelian presheaf on $X$ such that the underlying presheaf of
sets is a sheaf.
\item The category of sheaves of abelian groups
is denoted $\textit{Ab}(X)$.
\end{enumerate}
\end{definition}

\noindent
Let $X$ be a topological space.
In the case of an abelian presheaf $\mathcal{F}$ the sheaf
condition with regards to an open covering $U = \bigcup U_i$
is often expressed by saying that the complex of abelian groups
$$
0 \to \mathcal{F}(U)
\to \prod\nolimits_i \mathcal{F}(U_i)
\to \prod\nolimits_{(i_0, i_1)} \mathcal{F}(U_{i_0} \cap U_{i_1})
$$
is exact. The first map is the usual one, whereas the second
maps the element $(s_i)_{i \in I}$ to the element
$$
(
s_{i_0}|_{U_{i_0} \cap U_{i_1}} -
s_{i_1}|_{U_{i_0} \cap U_{i_1}}
)_{(i_0, i_1)}
\in \prod\nolimits_{(i_0, i_1)} \mathcal{F}(U_{i_0} \cap U_{i_1})
$$

In fact, the notion of kernel of a sheaf
is not really defined as I did in the beginning of this section,
but in the next one, along with several other
important things.


\section{The abelian category of sheaves of modules}
\label{section-kernels}

\noindent
I guess that the reason to introduce
coherent sheaves is not the search for an
abelian category, after all.
Looks like the pathologies avoided
by the definition of coherence are not
so obvious---something like ``wildly infinitely generated''.

\section{Tensor product of sheaves}
\label{section-tensor product of sheaves}

Here's my unexpected encounter with the definition of tensor product of sheaves.
It's not the ``fiber is tensor product of fibers'' construction, but actually
just some notion of ``change of ring'' sheaf that ends up being adjoint to
some ``restriction'' sheaf. The setting is a mapping of presheaves {\it of
rings} over a space $X$… (I think the usual definition is this one taking
$\mathcal{O}_1$ as the other presheaf we want to tensor).

Immediately after introducing this notion there's the definition of sheaf, then
stalks, abelian sheaves, some other notions like an ``algebraic structure'' and
then tensor product will be defined after sheafification---because the following
definition is in general not a sheaf.

Furthermore, I add that Vakil leaves it as an exercise to define the tensor
product of two $\mathcal{O}_X$ modules (with a hint of defining the presheaf
tensor product and sheafifying), which makes me think that after all it {\it is}
just the intuitive definition. Before diving in, also by Vakil (Exercise 26.K):
the stalk of the tensor product is the tensor product of the stalks.

\medskip\noindent

Suppose that $\mathcal{O}_1 \to \mathcal{O}_2$ is a
morphism of presheaves of rings on $X$. In this case,
if $\mathcal{F}$ is a presheaf of $\mathcal{O}_2$-modules
then we can think of $\mathcal{F}$ as a presheaf of
$\mathcal{O}_1$-modules by using the composition
$$
\mathcal{O}_1 \times \mathcal{F}
\to
\mathcal{O}_2 \times \mathcal{F}
\to
\mathcal{F}.
$$
We sometimes denote this by $\mathcal{F}_{\mathcal{O}_1}$
to indicate the restriction of rings. We call this
the {\it restriction of $\mathcal{F}$}. We obtain the
restriction functor
$$
\textit{PMod}(\mathcal{O}_2)
\longrightarrow
\textit{PMod}(\mathcal{O}_1)
$$

\medskip\noindent
On the other hand, given a presheaf of $\mathcal{O}_1$-modules
$\mathcal{G}$
we can construct a presheaf of $\mathcal{O}_2$-modules
$\mathcal{O}_2 \otimes_{p, \mathcal{O}_1} \mathcal{G}$
by the rule
$$
\left(\mathcal{O}_2 \otimes_{p, \mathcal{O}_1} \mathcal{G}\right)(U)
=
\mathcal{O}_2(U) \otimes_{\mathcal{O}_1(U)} \mathcal{G}(U)
$$
The index $p$ stands for ``presheaf'' and not ``point''.
This presheaf is called the tensor product presheaf. We obtain
the {\it change of rings} functor
$$
\textit{PMod}(\mathcal{O}_1)
\longrightarrow
\textit{PMod}(\mathcal{O}_2)
$$

\begin{lemma}
\label{lemma-adjointness-tensor-restrict-presheaves}
With $X$, $\mathcal{O}_1$, $\mathcal{O}_2$, $\mathcal{F}$ and
$\mathcal{G}$ as above there exists a canonical bijection
$$
\Hom_{\mathcal{O}_1}(\mathcal{G}, \mathcal{F}_{\mathcal{O}_1})
=
\Hom_{\mathcal{O}_2}(
\mathcal{O}_2 \otimes_{p, \mathcal{O}_1} \mathcal{G},
\mathcal{F}
)
$$
In other words, the restriction and change of rings functors
are adjoint to each other.
\end{lemma}

\begin{proof}
This follows from the fact that for a ring map
$A \to B$ the restriction functor and the change
of ring functor are adjoint to each other.
\end{proof}

\medskip\noindent
Tipologia dos feixes.

\begin{definition}
\label{definition-tipologia-dos-feixes}
A sheaf of $\mathcal{A}$-modules $\mathcal{F}$ over a sheaf of rings 
$\mathcal{A}$ (on a topological space $X$) is called
\begin{itemize}
\item 
\end{itemize}
\end{definition}

\section{Closed immersions}
\label{section-closed-immersion}

\noindent
Closed immersions of ringes spaces are in 
\href{https://stacks.math.columbia.edu/tag/01C1}{01C1}.

\begin{definition}
\label{definition-closed-immersion}
A map of ringed spaces
$i:(Z,\mathcal{O}_Z) \to (X,\mathcal{O}_X)$ 
is a {\it closed immersion}
if $i$ is injective,
its image is closed (I think these two are 
``closed immersion of topological spaces'')
and the induced map $\mathcal{O}_X \to i_*\mathcal{O}_Z$
is surjective; denote its kernel by $\mathcal{I}$.
To make sure that if $(X,\mathcal{O}_X)$ is a scheme
then so is $(Z,\mathcal{O}_Z)$ with add the extra condition
that the $\mathcal{O}_X$-module $\mathcal{I}$ is
locally generated by sections.
\end{definition}

\section{Restricting sheaves to subschemes}
\label{section-restriction}

\noindent
Upshot about restricting sheaves to subschemes:
the sheaves on $Z$ correspond essentially
(in the categorical sense)
to sheaves on $X$ with support on $Z$.

More explicitly, from \href{https://stacks.math.columbia.edu/tag/01QY}{01QY}:
the pushforward of the inclusion (a closed immersion of schemes)
$i:Z \to X$, with kernel on induced map $\mathcal{O}_Z \to \mathcal{O}_X$ 
denoted by $I$, is an exact, fully faithful functor
$$
i_*:\QCoh(\mathcal{O}_Z) \to \QCoh(\mathcal{O}_X)
$$
with essential image the quasi-coherent $\mathcal{O}_X$ modules
$\mathcal{G}$ such that $\mathcal{I}\mathcal{G}=0$.

It makes sense to think of this operation as
$\mathcal{F}|_Z=i^*\mathcal{F}=\mathcal{F}\otimes\mathcal{O}_Z$
for $\mathcal{F}\in\QCoh(X)$.
Indeed: tensoring by the ideal sheaf would kill the information on $Z$,
and tensoring with the structure sheaf does the opposite
--- it's the sheaves on $X$ that are no fun outside $Z$,
so basically sheaves on $Z$.

\medskip\noindent
Here's a more detailed discussion:

In understanding the cohomology of sheaves when
we restrict them to submanifolds (see Complex Geometry
Exercise \ref{complex-geometry-exercise-deformations-curve-in-k3},
I found Stacks Project 
\href{https://stacks.math.columbia.edu/tag/01AW}{01AW}
section on closed immersions and abelian sheaves.
First we have a few propositions stating that
sheaves on the subscheme $Z \subset X$ can be pushed
to sheaves on $X$.

And then there's the converse: 
what I would like to call the restriction functor $\mathcal{H}$
but rather is called the {\it sub-module of sections with support in $Z$}.
In Remark \href{https://stacks.math.columbia.edu/tag/01AY}{01AY}
(and then in Remark \href{https://stacks.math.columbia.edu/tag/0G6N}{0G6N}
for ringed spaces; these seem to be analogue constructions…)
we see that for an abelian sheaf $\mathcal{F}$ on $X$
we can define an abelian sheaf $\mathcal{H}_Z(\mathcal{F})$
which although is a sheaf on $X$ 
we can just think it's a sheaf on $Z$ - that's beacause
$$
\mathcal{H}_Z(\mathcal{F})(U)
=\{s \in \mathcal{F}(U))| \text{the support of $s$ is contained in }Z\cap U\}.
$$
We can just think that it's the functions
that vanish outside the subvariety.
Which doesn't really make sense:
a function vanishing outside the submanifold will most likely
vanish also in the submanifold! Just by continuity,
the submanifold is in the closure of the open set!
I guess what I mean to say is that I don't mind if
the functions do not vanish outside $Z$…
I just want to consider them as functions on $Z$…

\medskip\noindent
Here's another attempt in understanding how 
to restrict sheaves to submanifolds.
First and foremost, lacking a reference this,
I understand the symbol $\mathcal{F}|_Z$ for a sheaf
on $X$ and a subscheme  $Z \subset X$ to mean the
pullback of $\mathcal{F}$ by the inclusion.
Then look at \href{https://stacks.math.columbia.edu/tag/01QY}{01QY}
(more basically, as above, 
\href{https://stacks.math.columbia.edu/tag/01AW}{01AW}…
in fact there appear to be several version of this statement)
to recall that sheaves of the subscheme $Z$ should correspond
to sheaves on $X$ with support on $Z$.
{\bf So essentially $\mathcal{F}|_Z$ should be a version of $\mathcal{F}$ 
but with support on $Z$.}
The key point here is that this is non other than
$\mathcal{F} \otimes \mathcal{O}_Z$.
Why? Because when multiply with the ideal sheaf of $Z$ we obtain
$(\mathcal{F} \otimes \mathcal{O}_Z)\mathcal{I}=0$
since sections of $\mathcal{I}$ 
vanish in $\mathcal{O}_Z=\mathcal{O}_X/\mathcal{I}$.
That is, the support of $\mathcal{F} \otimes \mathcal{O}_Z$ indeed
contained in $Z$: if the sections become zero when
multiplied by sections of $\mathcal{I}$ 
(sections of $\mathcal{I}$ are only nonzero outside $Z$)
it means that all the action was inside $Z$.
that outside $Z$ 

I think the upshot here is that structure sheaves
act as some sort of ``contour'' for the geometric object:
they vanish {\it outside} the geometric object,
they tell us what is the shape of the object
by describing its surroundings, rather than its interior.
This is exactly how algebraic/analytic sets are defined,
and this why, when we want to restrict a sheaf
to a subscheme, we tensor with the structure sheaf:
tensor by the ideal sheaf would kill the information on $Z$,
and tensoring with the structure sheaf does the opposite.




\section{The category of affine schemes}
\label{section-category-affine-schemes}

\noindent
This might be a cornerstone of algebraic geometry.
The point is that once we take affine charts,
we re in the affine category. And this category turns
out to be equivalent to the category of rings. /o/
See below for some important properties of the correspondence;
namely fibred products and tensor product counterparts.

\section{Proj of a graded ring}
\label{section-proj}

\noindent
(Stacks Project 
\href{https://stacks.math.columbia.edu/tag/01M3}{01M3},
Proj of a graded ring.)
This section may be taken as the preamble to
the question: what is projective space anyway?
Yes, projective space is Proj of a graded ring $S$.
And yes, projective space is a scheme.
So in particular it's locally affine.

(Stacks Project 
\href{https://stacks.math.columbia.edu/tag/01MX}{01MX},
Functoriality of Proj.) 
This is reminiscent of the olden days.
The point is that Proj is covariant,
i.e. for rings  $S' \subset S$ we have
$\text{Proj}S \subset \text{Proj}S$
(under the right conditions,
which is the point of this section),
meaning that, indeed,
{\bf Projs are projective schemes}.


\section{Invertible sheaves on Proj (twists!)}
\label{section-invertible-on-proj}

\noindent
As for the construction of the twisted sheaves:
for a projective space $\text{Proj} S$,
we just compute the twisted $S$ module
$S(m)$ by the formula $S(m)_d=S_{m+d}$ 
and turn that into a sheaf $\widetilde{S(m)}$.

Here's the construction of the twisted
sheaves $\mathcal{O}_X(n)$.
The point is that there is a good way
to pass from an $S$-module $M$ to an
$\mathcal{O}_X$-module, called $\widetilde{M}$.
So basically you just define the twists
by $M(d)_n=M_{n+d}$ and apply that construction.

Pick an element of degree $d$.
Then think of the module generated by this element.
The least degree piece of such a module
is the degree $d$ piece of the original module,
that is we have $M(-d)$.


\section{Logarithmic derivations}
\label{section-logarithmic-derivations}

\noindent
Just to record a notion of logarithmic derivation
from Vladimiro Benedetti, Danielle Faenzi and Simone.

Let $F$ be a homogneous polynomial of degree $d$.
The {\it logarithmic derivations} of $F$ are
$$
\text{Der}_U(-\text{log}F)
=\{\partial \in \text{Der}_U
:\partial F \in \left<F\right>_U\}
$$
where $U=\mathbb{C}[x_1,…,x_e]$.

\section{Gonality}
\label{section-gonality}

\noindent
From a talk by Juliana Coelho at Bandoleiros 2025:

A curve $C$ is {\it $k$-gonal} if there is a map $C \to \mathbb{P}^1$ of degree $k$.

A curve is {\it stable} if it is nodal and $\text{Aut}(C)$ is finite.
Equivalently, the rational components of $C$ meet the
rest of the curve in at least three points.

We may study the moduli of $k$-stable curves,
or the space of $k$-gonal structures.

\section{Reduced schemes}
\label{section-reduced}

\noindent
So far a reduced scheme,
for me,
is just a scheme where there are
no ``repetitions'':
$(x^2)$ is not reduced because
it gives the same information as $(x)$.
Algebraically this means that the
coordinate ring is reduced,
which is defined by asking
that there are no nilpotent elements,
i.e. elements such that some power vanishes.
(There's no definition of reduced ring in Stacks Project
since it's taken as part of the basic algebraic knowledge).

The definition is on stalks but the first lemma
shows it's equivalent to define it on any open set.
This is in virtue of some ``injectivity'' lemma
from the ring at $U$ to the product of all stalks parametrized in $U$;
essentially that if a section vanishes when projected to all
the stalks then it's zero.

\begin{definition}
\label{definition-reduced}
Let $X$ be a scheme. We say $X$ is {\it reduced} if every local ring
$\mathcal{O}_{X, x}$ is reduced.
\end{definition}


\begin{proof}
Assume that $X$ is reduced.
Let $f \in \mathcal{O}_X(U)$ be a section such that $f^n = 0$.
Then the image of $f$ in $\mathcal{O}_{U, u}$ is zero for
all $u \in U$. Hence $f$ is zero, see
Sheaves, Lemma \ref{sheaves-lemma-sheaf-subset-stalks}.
Conversely, assume that $\mathcal{O}_X(U)$ is reduced
for all opens $U$. Pick any nonzero element $f \in \mathcal{O}_{X, x}$.
Any representative $(U, f \in \mathcal{O}(U))$  of $f$ is nonzero and
hence not nilpotent. Hence $f$ is not nilpotent in $\mathcal{O}_{X, x}$.
\end{proof}

\section{Dominant morphisms}
\label{section-dominant}

\noindent
The definition of a morphism of schemes being dominant is a little
different from what you might expect if you are used to the notion
of a dominant morphism of varieties.

\begin{definition}
\label{definition-dominant}
A morphism $f : X \to S$ of schemes is called {\it dominant} if the
image of $f$ is a dense subset of $S$.
\end{definition}

\section{Morphisms of finite type}
\label{section-finite-type}

\noindent
Recall that a ring map $R \to A$ is said to be of finite type if
$A$ is isomorphic to a quotient of $R[x_1, \ldots, x_n]$ as an $R$-algebra, see
Algebra, Definition \ref{algebra-definition-finite-type}.

\begin{definition}
\label{definition-finite-type}
Let $f : X \to S$ be a morphism of schemes.
\begin{enumerate}
\item We say that $f$ is of {\it finite type at $x \in X$} if
there exists an affine open neighbourhood $\Spec(A) = U \subset X$
of $x$ and an affine open $\Spec(R) = V \subset S$
with $f(U) \subset V$ such that the induced ring map
$R \to A$ is of finite type.
\item We say that $f$ is {\it locally of finite type} if it is
of finite type at every point of $X$.
\item We say that $f$ is of {\it finite type} if it is locally of
finite type and quasi-compact.
\end{enumerate}
\end{definition}

\section{Flat morphisms}
\label{section-flat-morphisms}

\noindent
The essential technical property
for for defining flatness
is the preservation of exact sequences.
Right-exactness is true in general;
it follows from currying in category
of commutative rings.
But the functor $-\otimes_R N$ does not
preserve injectivity of maps---
that's the point of flatness.

\begin{lemma}[Internal Hom for R-modules]
\label{lemma-hom-from-tensor-product}
For any three $R$-modules $M, N, P$,
$$
\Hom_R(M \otimes_R N, P) \cong \Hom_R(M, \Hom_R(N, P))
$$
\end{lemma}

\begin{proof}
An $R$-linear map $\hat{f}\in \Hom_R(M \otimes_R N, P)$ corresponds to an
$R$-bilinear map $f : M \times N \to P$. For
each $x\in M$ the mapping $y\mapsto f(x, y)$ is $R$-linear by the universal
property. Thus $f$ corresponds to a
map $\phi_f : M \to \Hom_R(N, P)$. This map is $R$-linear since
$$
\phi_f(ax + y)(z) =
f(ax + y, z) = af(x, z)+f(y, z) =
(a\phi_f(x)+\phi_f(y))(z),
$$
for all $a \in R$, $x \in M$, $y \in M$ and
$z \in N$. Conversely, any
$f \in \Hom_R(M, \Hom_R(N, P))$ defines an $R$-bilinear
map $M \times N \to P$, namely $(x, y)\mapsto f(x)(y)$.
So this is a natural one-to-one correspondence between the
two modules
$\Hom_R(M \otimes_R N, P)$ and $\Hom_R(M, \Hom_R(N, P))$.
\end{proof}

\begin{lemma}[Tensor product is right exact]
\label{lemma-tensor-product-exact}
Let
\begin{align*}
M_1\xrightarrow{f} M_2\xrightarrow{g} M_3 \to 0
\end{align*}
be an exact sequence of $R$-modules and homomorphisms, and let $N$ be any
$R$-module. Then the sequence
\begin{equation}
\label{equation-2ndex}
M_1\otimes N\xrightarrow{f \otimes 1} M_2\otimes N \xrightarrow{g \otimes 1}
M_3\otimes N \to 0
\end{equation}
is exact. In other words, the functor $- \otimes_R N$ is
{\it right exact}, in the sense that tensoring
each term in the original right exact sequence preserves the exactness.
\end{lemma}

\begin{proof}
For every $R$-module $P$
we apply the functor $\Hom(-, \Hom(N, P))$ to the first exact
sequence. We obtain
$$
0 \to
\Hom(M_3, \Hom(N, P)) \to
\Hom(M_2, \Hom(N, P)) \to
\Hom(M_1, \Hom(N, P))
$$
which is exact by Lemma \ref{lemma-hom-exact} (1).
By Lemma \ref{lemma-hom-from-tensor-product} this becomes the sequence
$$
0 \to \Hom(M_3 \otimes N, P) \to
\Hom(M_2 \otimes N, P) \to \Hom(M_1 \otimes N, P)
$$
which is therefore also exact. Then using
Lemma \ref{lemma-hom-exact} (1) again, we arrive at the desired exact sequence.
\end{proof}

\begin{remark}
\label{remark-tensor-product-not-exact}
However, tensor product does NOT preserve exact sequences in general.
In other words, if $M_1 \to M_2 \to M_3$ is
exact, then it is not necessarily true that
$M_1 \otimes N \to M_2 \otimes N \to M_3 \otimes N$
is exact for arbitrary $R$-module $N$.
\end{remark}

\begin{example}
\label{example-tensor-product-not-exact}
Consider the injective map $2 : \mathbf{Z}\to \mathbf{Z}$
viewed as a map of $\mathbf{Z}$-modules.
Let $N = \mathbf{Z}/2$. Then the induced map
$\mathbf{Z} \otimes \mathbf{Z}/2 \to \mathbf{Z} \otimes \mathbf{Z}/2$
is NOT injective. This is because for
$x \otimes y\in \mathbf{Z} \otimes \mathbf{Z}/2$,
$$
(2 \otimes 1)(x \otimes y) = 2x \otimes y = x \otimes 2y = x \otimes 0 = 0
$$
Therefore the induced map is the zero map while $\mathbf{Z} \otimes N\neq 0$.
\end{example}

\begin{definition}
\label{definition-flat-module}
For $R$-modules $N$, if the
functor $-\otimes_R N$ is exact, i.e. tensoring
with $N$ preserves all exact
sequences, then $N$ is said to be {\it flat} $R$-module.
We will discuss this later in Section \ref{section-flat}.
\end{definition}

\medskip\noindent
Epiphany:
in the category of commutative rings
pushout is tensor product.
So think of $\Spec $ as a functor
from $\mathit{CRing}^{\text{op}}$ to $\Sch$,
then pushout goes to pullback,
and what's an example of a pullback?
Fibre!
So, the coordinate ring of a fiber
is essentially given by the residue field
at the point that parametrizes it!
(tensored with the coordinate ring
of the deformation space).

There is a lot of information on Stacks Project about flatness.
It looks like the heart of the concept is 
captured in the commutative-algebraic notion of preserving
exact sequences:

\begin{definition}
\label{definition-flat}
Let $R$ be a ring.
\begin{enumerate}
\item An $R$-module $M$ is called {\it flat} if whenever
$N_1 \to N_2 \to N_3$ is an exact sequence of $R$-modules
the sequence $M \otimes_R N_1 \to M \otimes_R N_2 \to M \otimes_R N_3$
is exact as well.
\item An $R$-module $M$ is called {\it faithfully flat} if the
complex of $R$-modules
$N_1 \to N_2 \to N_3$ is exact if and only if
the sequence $M \otimes_R N_1 \to M \otimes_R N_2 \to M \otimes_R N_3$
is exact.
\item A ring map $R \to S$ is called {\it flat} if
$S$ is flat as an $R$-module.
\item A ring map $R \to S$ is called {\it faithfully flat} if
$S$ is faithfully flat as an $R$-module.
\end{enumerate}
\end{definition}

\medskip\noindent
Recall that a module $M$ over a ring $R$ is {\it flat} if the functor
$-\otimes_R M : \text{Mod}_R \to \text{Mod}_R$ is exact. A ring map
$R \to A$ is said to be {\it flat} if $A$ is flat as an $R$-module.

\section{Singularities}
\label{section-singularities}

As in \cite{sea},
a Noetherian local ring $A$ is {\it regular} 
if $\dim A = \dim_k\mathfrak{m}/\mathfrak{m}^2$.
The stalk $\mathcal{O}_{X,p}$ is
always local
so we say that $p$ is {\it regular} 
if its stalk is regular.

\section{Birational morphisms}
\label{section-birational}

\noindent
Recall that for
\href{https://www.mathnet.ru/php/
presentation.phtml?option_lang=eng&presentid=337}{Kuznetsov}
a birational map is that is an isomorphism on two open subsets
of the varieties…

So far I get: they are isomorphisms on a dense open set,
but more formally, here in Stacks, they give
{\bf isomorphisms on the stalks of the generic points}.

\medskip\noindent
You may be used to the notion of a birational map of varieties
having the property that it is an isomorphism over an open subset
of the target. However, in general a birational morphism may
not be an isomorphism over any nonempty open, see
Example \ref{example-birational-not-iso-over-open}.
Here is the formal definition.

\begin{definition}
\label{definition-birational}
\begin{reference}
\cite[(2.2.9)]{EGA1}
\end{reference}
Let $X$, $Y$ be schemes. Assume $X$ and $Y$ have finitely many
irreducible components. We say a morphism $f : X \to Y$ is
{\it birational} if
\begin{enumerate}
\item $f$ induces a bijection between the set of generic points
of irreducible components of $X$ and the set of generic points
of the irreducible components of $Y$, and
\item for every generic point $\eta \in X$ of an irreducible component
of $X$ the local ring map
$\mathcal{O}_{Y, f(\eta)} \to \mathcal{O}_{X, \eta}$
is an isomorphism.
\end{enumerate}
\end{definition}

\section{Ampleness}
\label{section-ampleness}

First is this lemma that comes from modules.tex. I think these sets $X_s$ are
the base points of the bundle. Because look: image of $s$ just means consider
the section $s$ of the line bundle as a germ near $x$. Now a line bundle is a
locally free rank-1 $\mathcal{O}_X$-module, so its sections, like $s$, may be
multiplied by germs of functions in the maximal ring $\mathfrak{m}_x$, i.e. the
functions that vanish at $x$. So $X_s$ is the vanishing locus of the section
$s$. If $s(x)\neq 0$, obviously $s
\not\in\mathfrak{m}_x\mathcal{L}_x$, so $x\in X_s$. Conversely, I would like to
show that if $s(x)=0$ then  $s\in\mathfrak{m}_x\mathcal{L}_x$ but I'm not sure
how. It's like: a vector field with a zero can be multiplied by a function that
vanishes at the point, sure, but what's this function?

\begin{lemma}
\label{lemma-s-open}
From modules.tex.
\begin{slogan}
A (local) trivialisation of a linebundle
is the same as a (local) nonvanishing section.
\end{slogan}
Let $X$ be a ringed space. Assume that each stalk $\mathcal{O}_{X, x}$
is a local ring with maximal ideal $\mathfrak m_x$.
Let $\mathcal{L}$ be an invertible $\mathcal{O}_X$-module.
For any section $s \in \Gamma(X, \mathcal{L})$ the set
$$
X_s = \{x \in X \mid \text{image }s \not\in \mathfrak m_x\mathcal{L}_x\}
$$
is open in $X$. The map $s : \mathcal{O}_{X_s} \to \mathcal{L}|_{X_s}$
is an isomorphism, and there exists a section $s'$
of $\mathcal{L}^{\otimes -1}$ over $X_s$ such that $s' (s|_{X_s}) = 1$.
\end{lemma}

\begin{proof}
Suppose $x \in X_s$.
We have an isomorphism
$$
\mathcal{L}_x \otimes_{\mathcal{O}_{X, x}} (\mathcal{L}^{\otimes -1})_x
\longrightarrow
\mathcal{O}_{X, x}
$$
by Lemma \ref{lemma-constructions-invertible}.
Both $\mathcal{L}_x$ and $(\mathcal{L}^{\otimes -1})_x$
are free $\mathcal{O}_{X, x}$-modules of rank $1$. We conclude
from Algebra, Nakayama's Lemma \ref{algebra-lemma-NAK} that
$s_x$ is a basis for $\mathcal{L}_x$. Hence there exists
a basis element $t_x \in (\mathcal{L}^{\otimes -1})_x$
such that $s_x \otimes t_x$ maps to $1$.
Choose an open neighbourhood $U$ of
$x$ such that $t_x$ comes from a section $t$
of $\mathcal{L}^{\otimes -1}$ over $U$ and such that
$s \otimes t$ maps to $1 \in \mathcal{O}_X(U)$.
Clearly, for every $x' \in U$ we see that $s$ generates
the module $\mathcal{L}_{x'}$. Hence $U \subset X_s$.
This proves that $X_s$ is open. Moreover, the section
$t$ constructed over $U$ above is unique, and hence
these glue to give the section $s'$ of the lemma.
\end{proof}

Recall from Modules, Lemma \ref{modules-lemma-s-open}
that given an invertible sheaf $\mathcal{L}$ on a locally ringed
space $X$, and given a global section $s$ of $\mathcal{L}$
the set $X_s = \{x \in X \mid s \not \in \mathfrak m_x\mathcal{L}_x\}$
is open. A general remark is that
$X_s \cap X_{s'} = X_{ss'}$, where $ss'$ denote
the section $s \otimes s' \in \Gamma(X, \mathcal{L} \otimes \mathcal{L}')$.

\begin{definition}
\label{definition-ample}
\begin{reference}
\cite[II Definition 4.5.3]{EGA}
\end{reference}
Let $X$ be a scheme.
Let $\mathcal{L}$ be an invertible $\mathcal{O}_X$-module.
We say $\mathcal{L}$ is {\it ample} if
\begin{enumerate}
\item $X$ is quasi-compact, and
\item for every $x \in X$ there exists an $n \geq 1$
and $s \in \Gamma(X, \mathcal{L}^{\otimes n})$ such
that $x \in X_s$ and $X_s$ is affine.
\end{enumerate}
\end{definition}

\begin{exercise}
\label{exercise-ample-bundle-on-K3}
Let $L$ be an ample bundle on a K3 surface $M$. Prove that
$\mathcal{L}^{\otimes 2}$ is globally generated (that is, for each $x\in M$
there exsits a section $h \in H^{0}(L^{\otimes 2})$ which does not vanish in
$x$).
\end{exercise}

\begin{proof}
This just asks that the $n$ in Definition \ref{definition-ample} is $2$ for all
$x\in X$. Because, again, $x\in X_s$ means that $s(x)\neq 0$ because if it was,
then we could somehow write $s$ as a product of a vanishing function on
$\mathfrak{m}_x$ and a local frame of $\Gamma(X,\mathcal{L})$. But I guess for
the exercise do this: a line bundle is {\it ample} if there is $n$ such that the
canonical embedding (cf Lemma \ref{lemma-map-into-proj}) is an embedding, i.e.
that $\mathcal{L}^{\otimes n}$ is {\it very ample}. (Interestingly, the notion
very ampleness is defined in morphisms.tex.)
\end{proof}

Now we pass to the part where ampleness gives you an {\bf open immersion} to
some projective space. Because, it's only very ampleness that gives an
embedding, right? (Actually I think here in stacks project there are no
embeddings but closed immersions.)

\begin{definition}
\label{definition-gamma-star}
From modules.tex. Let $(X, \mathcal{O}_X)$ be a ringed space.
Given an invertible sheaf $\mathcal{L}$ on $X$ we define
the {\it associated graded ring} to be
$$
\Gamma_*(X, \mathcal{L})
=
\bigoplus\nolimits_{n \geq 0} \Gamma(X, \mathcal{L}^{\otimes n})
$$
Given a sheaf of $\mathcal{O}_X$-modules $\mathcal{F}$ we set
$$
\Gamma_*(X, \mathcal{L}, \mathcal{F})
=
\bigoplus\nolimits_{n \in \mathbf{Z}} \Gamma(X,
\mathcal{F} \otimes_{\mathcal{O}_X} \mathcal{L}^{\otimes n})
$$
which we think of as a graded $\Gamma_*(X, \mathcal{L})$-module.
\end{definition}

\begin{lemma}
\label{lemma-map-into-proj}
Let $X$ be a scheme.
Let $\mathcal{L}$ be an invertible $\mathcal{O}_X$-module.
Set $S = \Gamma_*(X, \mathcal{L})$ as a graded ring.
If every point of $X$ is contained in one of the
open subschemes $X_s$, for some $s \in S_{+}$ homogeneous, then
there is a canonical morphism of schemes
$$
f : X \longrightarrow Y = \text{Proj}(S),
$$
to the homogeneous spectrum of $S$ (see
Constructions, Section \ref{constructions-section-proj}).
This morphism has the following properties
\begin{enumerate}
\item $f^{-1}(D_{+}(s)) = X_s$ for any $s \in S_{+}$ homogeneous,
\item there are $\mathcal{O}_X$-module maps
$f^*\mathcal{O}_Y(n) \to \mathcal{L}^{\otimes n}$
compatible with multiplication maps, see
Constructions, Equation (\ref{constructions-equation-multiply}),
\item the composition
$S_n \to \Gamma(Y, \mathcal{O}_Y(n)) \to \Gamma(X, \mathcal{L}^{\otimes n})$
is the identity map, and
\item for every $x \in X$ there is an integer $d \geq 1$
and an open neighbourhood $U \subset X$ of $x$
such that $f^*\mathcal{O}_Y(dn)|_U \to \mathcal{L}^{\otimes dn}|_U$
is an isomorphism for all $n \in \mathbf{Z}$.
\end{enumerate}
\end{lemma}

\section{Adjunction formulas}
\label{section-adjunction formulas}

There are several statements called adjunction formula
in different texts. All of them concern ``subvarieties'',
that is, closed embedded subschemes.

\begin{exercise}[Genus formula for a curve on a surface]
\label{exercise-genus-formula-for-curve-on-surface}
Let $C \to X$ be a closed embedded subscheme 
of dimension $1$ (as a topological space, i.e. pure dimension)
inside a smooth surface $X$.
Then $2p_a-2=(\mathcal{O}_X(C),\mathcal{O}_X(C))$.
\end{exercise}

\begin{proof}
Consider the ideal sheaf exact sequence
$$
\xymatrix{
0\ar[r]&\mathcal{O}_X(-C)\ar[r]&\mathcal{O}_X\ar[r]&\mathcal{O}_C\ar[r]&0
}
$$
This sequence splits since there is an obvious inverse morphism
to the inclusion $\mathcal{O}_X(-C)\to \mathcal{O}_X$, namely
mapping a function $f$ to 
Then $\mathcal{O}_X\cong \mathcal{O}_X(-C) \oplus \mathcal{O}_C$.
 $\chi(\mathcal{O}_X)=\chi$
\end{proof}

\section{Normalization}
\label{section-normalization}

\begin{definition}
\label{definition-normalization-X-in-Y}
Let $f : Y \to X$ be a quasi-compact and quasi-separated morphism of schemes.
Let $\mathcal{O}'$ be the integral closure of $\mathcal{O}_X$ in
$f_*\mathcal{O}_Y$. The {\it normalization of $X$ in $Y$} is the
scheme\footnote{The scheme $X'$ need not be normal, for example if
$Y = X$ and $f = \text{id}_X$, then $X' = X$.}
$$
\nu : X' = \underline{\Spec}_X(\mathcal{O}') \to X
$$
over $X$. It comes equipped with a natural factorization
$$
Y \xrightarrow{f'} X' \xrightarrow{\nu} X
$$
of the initial morphism $f$.
\end{definition}

\noindent
The factorization is the composition of the canonical morphism
$Y \to \underline{\Spec}_X(f_*\mathcal{O}_Y)$ (see
Constructions, Lemma
\ref{constructions-lemma-canonical-morphism})
and the morphism of relative spectra coming from the inclusion map
$\mathcal{O}' \to f_*\mathcal{O}_Y$. We can characterize the
normalization as follows.

\begin{lemma}
\label{lemma-characterize-normalization}
Let $f : Y \to X$ be a quasi-compact and quasi-separated morphism of schemes.
The factorization $f = \nu \circ f'$, where $\nu : X' \to X$ is the
normalization of $X$ in $Y$ is characterized by the following
two properties:
\begin{enumerate}
\item the morphism $\nu$ is integral, and
\item for any factorization $f = \pi \circ g$, with $\pi : Z \to X$
integral, there exists a commutative diagram
$$
\xymatrix{
Y \ar[d]_{f'} \ar[r]_g & Z \ar[d]^\pi \\
X' \ar[ru]^h \ar[r]^\nu & X
}
$$
for some unique morphism $h : X' \to Z$.
\end{enumerate}
Moreover, the morphism $f' : Y \to X'$ is dominant and in (2) the
morphism $h : X' \to Z$ is the normalization of $Z$ in $Y$.
\end{lemma}



\section{Reflexive sheaves}
\label{section-reflexive-sheaves}

\begin{slogan}
These are vector bundles except for a small locus.
\end{slogan}

\begin{definition}
\label{definition-reflexive}
Let $X$ be an integral locally Noetherian scheme. Let $\mathcal{F}$
be a coherent $\mathcal{O}_X$-module. The {\it reflexive hull}
of $\mathcal{F}$ is the $\mathcal{O}_X$-module
$$
\mathcal{F}^{**} = \SheafHom_{\mathcal{O}_X}(
\SheafHom_{\mathcal{O}_X}(\mathcal{F}, \mathcal{O}_X), \mathcal{O}_X)
$$
We say $\mathcal{F}$ is {\it reflexive} if the natural map
$j : \mathcal{F} \longrightarrow \mathcal{F}^{**}$
is an isomorphism.
\end{definition}

\begin{lemma}
\label{lemma-reflexive-torsion-free}
Let $X$ be an integral locally Noetherian scheme. Let $\mathcal{F}$
be a coherent $\mathcal{O}_X$-module.
\begin{enumerate}
\item If $\mathcal{F}$ is reflexive, then $\mathcal{F}$ is torsion free.
\item The map $j : \mathcal{F} \longrightarrow \mathcal{F}^{**}$
is injective if and only if $\mathcal{F}$ is torsion free.
\end{enumerate}
\end{lemma}

\begin{remark}[Talk at IMPA, 11 June]\leavevmode
\label{remark-reflexive-talk}
Torsion could also be defined so that the sheaf can inject onto its dual. In
this talk we discussed the moduli space of reflexive/torsion-free sheaves, which
turned out to be parametrized by $c_1, c_2$ and $c_3$. This was denoted by 
$R(c_1,c_2,c_3)$. Actually I think it may have been Manolache that proved the
existence of this moduli space.


Alan Muniz

Nesta palestra discutiremos a classificação de feixes reflexivos de posto dois e
seus espaços de módulos. Apresentaremos algumas ferramentas básicas usadas na
construção e determinação de tais feixes. Aplicaremos estas técnicas para o caso
de feixes com segunda classe de Chern igual a quatro, obtido recentemente em
colaboração com Marcos Jardim.
\end{remark}

\subsection{Distributions on manifolds}
\label{subsection-distributions-on-manifolds}

Here's the abstract from a talk by Marcos Jardim at Geometric Structures:

``I will revise the work done over the past 10 years with various collaborators
on distributions and foliations on 3-folds, especially on the projective space,
with a focus on properties of the tangent sheaf and singular scheme."

Here are two key ideas: if the distribution is codimension 1 we can write:
$$
\xymatrix{
0\ar[r]&F\ar[r]&TX\ar[r]^{\omega}&I_Z \otimes L\ar[r]&0
}
$$
where $L$ is a line bundle and $\omega \in H^{0}(\Omega_X \otimes L)$, and
$Z=\{p:\omega(p)=0\}$.

When codimension is 2 then $\mathcal{D}$ is given by a holomorphic vector field 
$\nu$: $T_p=\left<\nu(p)\right>$.

It can be encoded as an exact sequence
$$
\xymatrix{
	0\ar[r]&L\ar[r]^{\nu}&TX\ar[r]&N\ar[r]&0
}
$$
where $L$ is a line bundle and $\nu \in H^{0}(TX \otimes L^\vee)$;
  $Z=\{p | \nu(p) = 0\}$.

\begin{remark}
\label{remark-stauration}
Saturation means that $Z \subset X$ is a union of curves and points.
\end{remark}

And again, distributions are parametrized by Chern classes.

Two interesting open questions:
\begin{enumerate}
\item {\bf Conjeture.} if $\mathcal{D}$ is a codimension 1 foliation of degree
$d$ on $\mathbb{P}^3$, then $c_2(F)\leq d^2-d+1$ and bound is attained
by rational foliations of type $(1,d+1)$. (True for $d \leq 2$.)
\item {\bf Conjecture (with Pepe Seade).} $\mathcal{D}$ is a codimension 1 
foliation on a smooth projective 3-fold, then $\text{Sing}\mathcal{D}$ is connected.
\end{enumerate}

\begin{theorem}[Jardim-Muniz]
\label{theorem-jardim-muniz}
Conditions on Chern classes used to understand moduli space $R(c_1,c_2,c_3)$.
$c_2=4$ gives (?). For $c_3\leq 6$, possible ``spectrum" exists…
\end{theorem}

\section{Stability}
\label{section-stability}

{\bf Question.} What is stability?

\begin{enumerate}
\item Stable objects in an abelian category are the ``building blocks":
we can reconstruct the whole category from them.
\item  An abelian subcategory (hart) $\subset$ a triangulated
\item stability defined via stability function on $\mathcal{A}$.
\item Q. Can we reconstruct $\mathcal{T}$ from the semistable elements of
$\mathcal{A}$
\item {\bf Example.} $\mathcal{A}=\text{Coh}X$ is heart of $D^b(X)$
w/ funny function.
\item Stability condition is hart + stability function.
\item Bridgeland Stabl:= the stability conditions are a complex manifold
of complex codimension $\text{rk}\Lambda$:
$$
 \mathcal{Z}:\text{Stab}(\mathcal{T})
\longrightarrow \text{Hom}(\Lambda,\mathbb{C})
$$
\item There's a chamber structure; moduli space changes across chambers.
\item I think we typically think of vectors in $\text{Hom}(\Lambda,\mathbb{C})$ 
as Chern classes, to characterize the moduli spaces.
\item Existence: given a projective variety $X$, are there stability
conditions on $D^\text{b}(X)$? Yes for fano 3fold pic rk 1.
\item Moduli spaces: is $M_\sigma(v)$ a projective scheme? Cannot use
usual git techniques to study. A stack!
\item Picture: blue + black are walls. Q. What are $\beta$ and $\alpha$?
\item thm: bridgeland stable = gieseker stable ?
\item Q. slope stability = bridgeland stability? A. Not always.
\item DT/PT correspondance: only one wall between PT and G chambers
\medskip
\item Polynomial stability function. This is an asymptotic version of BS.
\item There are some $\rho$'s. Arrangements of $\rho_i$ are polynomial 
stability conditions on a threefold.
\item Pata-Thomas introduced stability for rank 1 objects.
Bayer compares the---wall. Q. Same for Bridge S---only one wall?
\item Recall Gieseker stability.
\medskip
\item Def. A {\it stable triple}: when
$\text{gcd}(\text{ch}_0,\text{ch}_2,\text{ch}_3)=1$, every PT stable object
comes from three conditions (missing).
\item What happens when you cross the blue wall? 
Both $\mathcal{G}$ and $\mathcal{T}$ are projective. What happens at the blue
wall?
\medskip
\item For $X$ smooth threefold with $\text{rk}\text{Pic}=1$, $\mathcal{G}(v)$,
$v=(r,0,0,-n)$, $\mathcal{G}(v)$ is a known sheaf object and
$\mathcal{T}=\emptyset$.
\item $X$ sm 3 rkpic1, Fake wall; $\mathcal{G}=\mathcal{T}$.
\item (Extra.) red circle is a wall for a weaker form of stability.
\end{enumerate}
\begin{definition}[Talk at impa]
\label{definition-slope-stability}
A rank-2 sheaf $\mathcal{F}$ is {\it semistable (stable)}if
$H^{0}(\mathcal{F}(t))=0$ for $-t \geq(>) \frac{c_1F}{2}$
\end{definition}

Compare with

\begin{definition}[moduli-curves.tex]
\label{definition-semistable}
Let $f : X \to S$ be a family of curves.
We say $f$ is a {\it semistable family of curves} if
\begin{enumerate}
\item $X \to S$ is a prestable family of curves, and
\item $X_s$ has genus $\geq 1$ and
does not have a rational tail for all $s \in S$.
\end{enumerate}
\end{definition}

\begin{itemize}
\item The twistor diagram.
$$
\xymatrix{
&  \ell_\infty \ar@{.>}[dr]\ar[dl]\\
\mathbb{C}P^{2} \ar@{^{(}->}[rr]& &
\mathbb{C}P^{3}\ar[dd]_{\substack{\text{twistor} \\ \text{map}}}\\ \\
\mathbb{C}^2\ar@{^{(}->}[u]
\mathbb{R}^4 \ar[u]\ar[rr]& &S^4
}
$$
\item Instantons are solutions to some Yang-Mills solution.
\item Expository paper of Donaldson arXiv:2205.08639
\item ADHM construction, 1978. The first appearance of Algebraic Geometry in
Mathematical Physics. See Hitchin-Kobayashi correspondence.
\item Donaldson: ``unashamedly computational''.
\item Expository paper by Simon.
\item Take bundle $(E,\nabla)$ with an anti-self dual (ADS) 
connection on $S^4$ and pullback to $\mathbb{C}P^{3}$ via the
twistor map
$$
\tau[x:y:z:w]=[x+jy:z+jw] \qquad \text{note: $\tau^{-1}(p)=\mathbb{C}P^{1}$}
$$
Then:
\begin{itemize}
\item Restriction to fibres are trivial.
\item Invariant under anti holomorphic involution (check this formula!) 
$[x:y:z:w] \mapsto [-y:x:-w:z]$.
\item $\mathcal{E}$ is also an instanton bundle.
\item Penrose Transform: $H^1(\mathcal{E}(-2)) \cong \Ker \Delta=0$ where
$\Delta$ is a Laplacian.
\item Definition of instanton sheaf on $\mathbb{P}^3$ via $c_1(E)=0$ and some
vanishing of cohomologies.
\item Passage from [differential equations? algebraic geometry?] 
to linear algebra: via {\it monads}. The point is that instanton sheaf is
equivalent to  ``$E$ being the cohomology of a linear monad''; theorem by Horps
in the 60's, and is the main tool used by ADHM. Indeed, ADHM equations come from
the cohomology sequences of the so-called monads.
\item Mathamatical inst bund:= locally free instanton sheaves.
\end{itemize}
\end{itemize}

\noindent
{\bf Properties.}
\begin{itemize}
\item The only instanton of rank 1 on $\mathbb{P}^3$ is the structure sheaf.
\item non-trivial rank 2 locally free instanton sheaf ois  $\mu_0stable$
\item double dual is locally free and also instanton
\item non-trivial rank 2 instanton sheaaf is Fieseker stable
therefore it makes sance fo define moduls space of instanton sheaves as an open
subset of $\mathcal{M}(c)=\mathcal{G}(k,0,2,0)$.
\end{itemize}

Then studied the irreducibility (Tikhomirov) and smoothness (Jardim-Verbitsky,
2014. Uses ``3rd hyperkähler quotient'') of $\mathcal{I}(c)$, the moduli
space of rank 2 locally gree instanton sheaves of charge $c.$ But nobody likes
this results; want new proofs.

In contrast, $\mathcal{M}(c)$ of rank 2-instanton sheaves of charge $c$ is not
irreducible in general!
$\mathcal{M}(1)$ and $\mathcal{M}(2)$ are irreducible, $\mathcal{M}(3)$ has
exactly 2 irreducible components of dimension 21;
 $\mathcal{M}(4)$ has 4 irredicuble components: 
the locally free is irreducible, and the other 3 that intersect the closure of
the locally
free, $\overline{\mathcal{I}(c)}$. 3 components of dimension 29 and one of dimension 32

\begin{remark}
\label{remark-}
In general the $\mu$ moduli space is not projective, but the Gieseker is.
\end{remark}


\noindent
{\bf Is $\mathcal{M}(c)$ connected?} Use $\mathbb{C}^*$ action. True for $c \leq
4$; every component intersects $\overline{\mathcal{I}(c)}$ in this range!

\begin{definition}[Elementary transformation]
\label{definition-elementary-transformation}
$F$ of rank 2 locally free instanton, $Q$ of rank 0 instanton with 1 dimensional
sheaf $h^p Q(-2)$ for $p = 0,1$. So in this conditions if we have an epimorphism
 $$
F \overset{\varphi,\text{ surj}}{\to}\} Q
$$
we get that $\Ker \varphi$ is an instanton.
\end{definition}

So  we might be interested in
$$
E \hookrightarrow E^{* *}\xrightarrow{\text{surj.}} Q_E
$$
Let's have a look at $\mathcal{M}(3)=\overline{\mathcal{I}(3)}\cup
\overline{C(0,1,3}$. Consider a generic line bundle of degree 0 on a planar
cubic (cwhich is encoded in that we have intersection of something of cimension
1 and something of simension 3), $L inn \text{Pic}^0(C)$ so we have an
epimorphism
$$
\xymatrix{
0\ar[r]&E\ar[r]&\mathcal{O}_{\mathbb{P}^3}\oplus
\mathcal{O}_{\mathbb{P}^3}\ar[r]&(i_* L)\ar[r]&0
}
$$
yielding a bundle $E$ as previously outlined.

So we are studying the components via pushforwards of sheaves on complete
intersection curves inside $\mathbb{P}^3$ 

``when we do alementary transformation of rational (not rationl?) we get
something on the boundary of locally free.''

\medskip\noindent
{\bf Perverse instanton sheaves.} Like an instnaton sheaf in monad description
but with some restrictions on the cohomologies. This leads to definition of {\it
$0$-rimensional instanton}, a perverse instanton such that $\mathcal{H}^0=0$.

\medskip\noindent
{\bf Instantons and quivers.}

\medskip\noindent
{\bf Framed instantons.} Fix a line $j:L\to \mathbb{P}^3$ and $E$ a perverse
sheaf; a {\it framing} at $L$ is an isomorphis [missing].

Apply GIT to the ADHM data to construct a moduli space 
$$
\mathcal{P}(r,c)=\mathcal{V}(r,c)^{\text{st}}/\!/\text{GL}(V_c)
$$
and it will follow that $\mathbb{P}(?,?)$ is connected.

\medskip\noindent
Considerations on quaternionic spaces lead to generalization of what has been
discussed so far to higher dimensions. I.e. an {\it instanton sheaf} on
$\mathbb{P}^n$ is… [other cohomological conditions]
sheaf

Why are instantons interesting? They are the simplest; may provide examples for
Bridgeland stability.

In Kuznetsov (2012) and Faenzi (2014) introduced {\it rank 2 instanton bundles
on Fano 3-folds}. An {\it instanton bundle} on $X$ is a $\mu$-stable … and some
Chern class is called the {\it charge}. An {\it
instanton sheaf} (introduced by Marcos-Gaia) is ….

 ``Since we are imposing $\mu$-stability on the defintion we can consider the
moduli $\mathcal{I}_X(c) \subset \mathcal{G}_X(2,-r_X,c,0)$''.

There's also monad representations as an ingredient.


\medskip\noindent
Here are two questions that invite us to join the instanton fever:

\medskip\noindent
{\bf Task 1.} Construct rank 2-instanton sheaves that do not deform into locally
free ones, and obtain the new irreducible components of
$\mathcal{G}(2,-r_X,c,0)$.

\medskip\noindent
{\bf Task 2.} Nonlocally instanton sheaves that can be deformed into non-locally
free ones: the {\it instanton boundary}
$\overline{\mathcal{I}(c)}/\mathcal{I}(X)$[formula right?]


\medskip\noindent
{\bf Recipe to construct your own instanton.}
\begin{enumerate}
\item (Make a bunch of instantons.) Find an appropriate curve to 
use Serre correspondence to find some rank 2 instanton sheaf:
$$
\xymatrix{
0\ar[r]&\mathcal{O}_{\mathbb{P}^3}(-1)\ar[r]&\underbrace{E}_{\exists
}\ar[r]&\mathcal{I}_{\sqcup \text{lines}}\ar[r]&0
}
$$
where $E$ is a locally free instanton of charge = the number of lines $-1$.

\item (Is your family of instantons generic?) You look at the $\text{Ext}$s.
``Therefore, the family of instantons only defines a locally closed subset
within a generically smooth irreducible component of $\mathcal{G}$''.

\item ``Find suitable rank 0 intranton sheaves to perform an elementary
transformations on the examples obtained in Step 1.'' But they are non-locally
free.

\item Now I have my instantons, I know they are locally free: but how to prove
that the elementary transformations deform to locally free ones? Looks like
the challenge is to prove that the deformation is locally free.
\end{enumerate}

In the papers by the group there are several particular cases when the
deformations are locally free. But they don't have a general result that would
work for 3-folds.

\medskip\noindent
{\bf What you need to call a thing an instanton.}
\begin{itemize}
\item Minimal cohomology possible; try to kill as much cohomology as you can.
\item Fixing $c_1$ (which may determine other Chern classes).
\item Some stability condition like $\mu$-stability or quiver stability. Here is
an example that is not $\mu$-semistable: 
$T\mathbb{P}^3(-1)\oplus\Omega_{\mathbb{P}^3}(1)$
\item Whenever possible, look for a monadic representation. (The monadic
representation comes from ADHM --- the beginnings of this theory. And it's still
here!)
\end{itemize}

\section{Coherent sheaves}
\label{section-coherent-sheaves}

\begin{lemma}
\label{lemma-quasi-coherent-affine-cohomology-zero}
\begin{slogan}
Serre vanishing: Higher cohomology vanishes on affine schemes
for quasi-coherent modules.
\end{slogan}
Let $X$ be a scheme.
Let $\mathcal{F}$ be a quasi-coherent $\mathcal{O}_X$-module.
For any affine open $U \subset X$ we have
$H^p(U, \mathcal{F}) = 0$ for all $p > 0$.
\end{lemma}

\begin{proof}
We are going to apply
Cohomology, Lemma \ref{cohomology-lemma-cech-vanish-basis}.
As our basis $\mathcal{B}$ for the topology of $X$ we are going to use
the affine opens of $X$.
As our set $\text{Cov}$ of open coverings we are going to use the standard
open coverings of affine opens of $X$.
Next we check that conditions (1), (2) and (3) of
Cohomology, Lemma \ref{cohomology-lemma-cech-vanish-basis}
hold. Note that the intersection of standard opens in an affine is
another standard open. Hence property (1) holds.
The coverings form a cofinal system of open coverings of any element
of $\mathcal{B}$, see
Schemes, Lemma \ref{schemes-lemma-standard-open}.
Hence (2) holds.
Finally, condition (3) of the lemma follows from
Lemma \ref{lemma-cech-cohomology-quasi-coherent-trivial}.
\end{proof}

\section{Hilbert polynomial}
\label{section-Hilbert-polynomial}

\noindent
The following lemma will be made obsolete by the more general
Lemma \ref{lemma-numerical-polynomial-from-euler}.

\begin{lemma}
\label{lemma-hilbert-polynomial}
Let $k$ be a field. Let $n \geq 0$. Let $\mathcal{F}$ be a coherent sheaf
on $\mathbf{P}^n_k$. The function
$$
d \longmapsto \chi(\mathbf{P}^n_k, \mathcal{F}(d))
$$
is a polynomial.
\end{lemma}

\begin{proof}
We prove this by induction on $n$. If $n = 0$, then
$\mathbf{P}^n_k = \Spec(k)$ and $\mathcal{F}(d) = \mathcal{F}$.
Hence in this case the function is constant, i.e., a polynomial
of degree $0$. Assume $n > 0$. By
Lemma \ref{lemma-euler-characteristic-extend-base-field}
we may assume $k$ is infinite. Apply
Lemma \ref{lemma-exact-sequence-induction}.
Applying Lemma \ref{lemma-euler-characteristic-additive}
to the twisted sequences
$0 \to \mathcal{F}(d - 1) \to \mathcal{F}(d) \to i_*\mathcal{G}(d) \to 0$
we obtain
$$
\chi(\mathbf{P}^n_k, \mathcal{F}(d)) -
\chi(\mathbf{P}^n_k, \mathcal{F}(d - 1)) =
\chi(H, \mathcal{G}(d))
$$
See Remark \ref{remark-exact-sequence-induction-cohomology}.
Since $H \cong \mathbf{P}^{n - 1}_k$
by induction the right hand side is a polynomial.
The lemma is finished by noting that any function
$f : \mathbf{Z} \to \mathbf{Z}$ with the property that the map
$d \mapsto f(d) - f(d - 1)$ is a polynomial, is itself a polynomial.
We omit the proof of this fact (hint: compare with
Algebra, Lemma \ref{algebra-lemma-numerical-polynomial}).
\end{proof}

\begin{definition}
\label{definition-hilbert-polynomial}
Let $k$ be a field. Let $n \geq 0$. Let $\mathcal{F}$ be a coherent sheaf
on $\mathbf{P}^n_k$. The function
$d \mapsto \chi(\mathbf{P}^n_k, \mathcal{F}(d))$ is called the
{\it Hilbert polynomial} of $\mathcal{F}$.
\end{definition}

\noindent
The Hilbert polynomial has coefficients in $\mathbf{Q}$ and not
in general in $\mathbf{Z}$. For example the Hilbert polynomial
of $\mathcal{O}_{\mathbf{P}^n_k}$ is
$$
d \longmapsto {d + n \choose n} = \frac{d^n}{n!} + \ldots
$$
This follows from the following lemma and the fact that
$$
H^0(\mathbf{P}^n_k, \mathcal{O}_{\mathbf{P}^n_k}(d)) = k[T_0, \ldots, T_n]_d
$$
(degree $d$ part) whose dimension over $k$ is ${d + n \choose n}$.

\begin{lemma}
\label{lemma-hilbert-polynomial-H0}
Let $k$ be a field. Let $n \geq 0$. Let $\mathcal{F}$ be a coherent sheaf
on $\mathbf{P}^n_k$ with Hilbert polynomial $P \in \mathbf{Q}[t]$.
Then
$$
P(d) = \dim_k H^0(\mathbf{P}^n_k, \mathcal{F}(d))
$$
for all $d \gg 0$.
\end{lemma}

\begin{proof}
This follows from the vanishing of cohomology of high enough twists
of $\mathcal{F}$. See
Cohomology of Schemes,
Lemma \ref{coherent-lemma-coherent-projective}.
\end{proof}


\medskip\noindent
For completeness I include earlier notes
from \cite{har} on the matter.

The fact that $M$ is finitely generated is what makes the following two
definitions make sense.

\begin{definition}
\label{definition-Hilbert-function}
The {\it Hilbert function} of a finitely generated graded $S=k[x_0,\ldots,x_r]$ 
-module $M$ is
$$
H_M(d)=\dim_kM_d
$$
\end{definition}

\begin{definition}
\label{definition-sysygy}
Define $F_0$ to be the free $S$-module on the generators of $M$. Elements in the
 kernel $M_1$ of the inclusion are called {\it sysygies}. By Hilbert's basis
theorem, $M_1$ is also finitely generated, so we may choose a set of generators
and repeat this process.
\end{definition}

\begin{theorem}[Hilbert Syzygy Theorem]
\label{theorem-Hilbert-syzygy}
\begin{reference}
\cite[Theorem 1.1]{sys}
\end{reference}
Any finitely generated $S$-module $M$ has a finite graded free resolution
$$
\xymatrix{
0\ar[r]&F_m\ar[r]^{\varphi_m}&\ar[r]&F_{m-q}\ar[r]&\cdots\ar[r]&
F_1\ar[r]^{\varphi_1}&F_0
}
$$
Moreover, we may take $m\leq r+1$, the number of variables in $S$.
\end{theorem}

\begin{lemma}
\label{lemma-Hilbert-function}
Suppose that $S=k[x_0,\ldots,x_r]$ is a polynomial ring. If the graded
$S$-module $M$ has finite free resolution
$$
\xymatrix{
0\ar[r]&F_m\ar[r]^{\varphi_m}&F_{m-1}\ar\cdots[r]&F_1\ar[r]^{\varphi_1}&
\ar[r]&F_0
}
$$
with each $F_i$ a finitely generated free module,
$F_i=\bigoplus_{j}S(-a_{i,j})$, then
\begin{equation}
\label{equation-Hilbert-function}
H_M(d)=\sum_{i=0}(-1)^i\sum_{j}\binom{r+d-a_{i,j}}{r}
\end{equation}
\end{lemma}

\begin{lemma}
\label{lemma-Hilbert-function-becomes-polynomial}
There is a polynomial $P_M(d)$ called the {\it Hilbert polynomial} such that, if
$M$ has free resolution as above, then $P_M(d)=H_M(d)$ for 
$d\geq\text{max}_{i,j}\{a_{i,j}-r\}$.
\end{lemma}

\begin{proof}
When $d$ satisfies this bound then the binomial coefficients in Eq.
\ref{equation-Hilbert-function} are polynomials of degree $r$ in $d$.
\end{proof}

\begin{theorem}[Hilbert-Serre]
\label{theorem-Hilbert-Serre}
\begin{reference}
\cite[I, Theorem 7.5]{hart}
\end{reference}
Let $M$ be a finitely generated graded $S=k[x_0,\ldots,x_n]$. Then there exists
a unique polynomial $p_M$ such that $p_M(\ell)=\dim S_\ell$ for large enough
$\ell$.
\end{theorem}

\begin{definition}
\label{definition-Hilbert-polynomial}
\begin{reference}
\cite[I, p. 52]{har}
\end{reference}
The polynomial $P_M$ of Hilbert-Serre Theorem \cite{Hilbert-Serre} is the {\it
Hilbert polynomial} of the finitely generated $k[x_0,\ldots,x_n]$-module $M$.
\end{definition}

\begin{definition}
\label{definition-degree-of-projective-variety}
\begin{reference}
\cite[p. 52]{hart}
\end{reference}
If $Y\subset \mathbb{P}^n$ is an algebraic set of dimension $r$, we define the
{\it degree of $Y$} to be $r!$ times the leading coefficient of the Hilbert
polynomial of the homogeneous coordinate ring $S(Y)$.
\end{definition}

\begin{exercise}
\label{exercise-very-ample-bundle-self-intersection-is-degree-of-surface}
\begin{reference}
\cite[V, Exercise 1.2]{har}
\end{reference}
Let $H$ be a very ample divisor on the surface $X$, corresponding to a
projective embedding $X\subseteq\mathbb{P}^N$. If we write the Hilbert
polynomial of $X$ as $P(z)=\frac{1}{2}az^2+bz+c$, show that $a=H^2$,
$b=\frac{1}{2}H^2+1-\pi$, where $\pi$ is the genus of a nonsingular curve
representing $H$, and $c=1+p_a$.
\end{exercise}

\section{Nakai-Moishezon Criterion}
\label{section-Nakai-Moishezon-criterion}

\begin{theorem}[Nakai-Moishezon Criterion]
\label{theorem-Nakai-Moishezon-criterion}
\begin{reference}
\cite[V, Theorem 1.10]{hart}
\end{reference}
A divisor $D$ on the surface $X$ is ample if and only if $D^2>0$ and $D.C>0$ for
all irreducible curves $C$ in $X$.
\end{theorem}

\begin{proof}
The direct implication is easy: since $D$ is ample,  $mD$ is very ample for some
$m$, so that $m^2D^2$ is the self-intersection number of $mD$. By exercise
\ref{exercise-very-ample-bundle-self-intersection-is-degree-of-surface},
 $D^2$ is the leading coefficient of the Hilbert polynomial of $X$ as a 
subscheme of $\mathbb{P}^n$. This means that $D^2$ is twice the leading
coefficient of the Hilbert polynomial of a projective variety for large enough
$m$, so that it must be a positive number (it's the dimension of one of the
graded components of the coordinate ring of the surface).
\end{proof}

\section{Hilbert scheme}
\label{section-Hilbert-scheme}

{\bf Upshot \cite[p. 6]{HarrMorr}.} 
We wish to parametrize subschemes of a projective space (or
perhaps a more general scheme?). Since there are too many such subschemes we
restrict ourselves to schemes with a given Hilbert polynomial, since the latter
``encodes the most important numerical invariants of schemes''. The Hilbert
scheme is introduced via a theorem by Grothendieck 
as the object that represents the functor $\mathbf{Hilb}_{P,r}$ 
that maps a reduced scheme $B$ to the
set of proper flat families
$$
\xymatrix{
\mathcal{X}\ar@{^{(}->}[r]^i\ar[rd]&\mathbb{P}^r \times B \ar[d]^{\pi_B}\\
& B
}
$$
with $\mathcal{X}$ having Hilbert polynomial $P$.

\begin{theorem}[Grothendieck, '66]
\label{theorem-Grothendieck}
The functor $\mathbf{Hilb}_{P,r}$ is representable by a projective scheme
$\mathcal{H}_{P,r}$.
\end{theorem}

SEE Hilbert schemes of subschemes.

\section{Deformation theory}
\label{section-deformation-theory}

\noindent
\cite{Sernesi-Overview} has a great way of motivating deformation
theory via the moduli problem.
Here let's just put the following important meaning for
``deformation theory'' (as opposed to studying flat families!):

\begin{quote}
``{\it Deformation theory} is the study of infinitesimal deformaions
as a tool to understand the local structure of the moduli space.
The goal is to be able to describe the restriction
of the universal family to a small neighborhood of $m \in \mathcal{M}$,
or, more precisely, its restriction to the germ of $M$ at $m$.

What is interesting here is that we can study order and infinitesimal
deformations even though the functor $F$ is not representable or simply we don’t
yet know it is. This is the most frequent case. Such an investigation will
reveal the infinitesimal properties at $[X]$ of a yet unknown global structure
on $M$ which will be hopefully understood at a subsequent stage of the
investigation.  In order words it turns out to be possible and convenient to
separate the {\it global moduli problem} from the {\it local moduli problem},
and deformation theory studies the latter, with the purpose of constructing a
family of deformations of a given object parametrized by the spectrum of a local
ring, and having properties as close as possible to a universal property.''
\end{quote}




\noindent
I start by reading Stacks Project.

\medskip\noindent
The first notion is {\it thickening of ringed spaces},
which I ultimately think of as a closed subscheme
$(X,\mathcal{O}_X) \to (X',\mathcal{O}_{X'})$
with a nilpotent ideal sheaf,
which in an imprecise way means
that $\mathcal{I}_X^n=0$ for some $n$,
and in a precise way its given on sections.

\medskip\noindent
The following definition is from [lucas-defos], which in turn comes from
\cite{Sernesi-deformations}

\begin{definition}
\label{definition-deformation}
Let $X$ be an algebraic $\mathbb{C}$-scheme.
\begin{enumerate}
\item A {\it deformation} of $X$ is a Cartesian diagram $\xi$
$$
\xymatrix{
X\ar[r]\ar[d]&\mathcal{X}\ar[d]^{\pi}\\
\Spec \mathbb{C}\ar[r]^s&S
}
$$
where $\pi$ is a flat surjective morphism of algebraic $\mathbb{C}$-schemes and
$S$ is connected. (Recall that flatness accounts for ``continuity''.)
\item A {\it local deformation} of $X$ is a deformation $\xi$ where
$S=\Spec A$ for $A$ a noetherian local $\mathbb{C}$ algebra with residue
field $\mathbb{C}$.
\item An {\it infinitesimal deformation of $X$} is a local deformation with $A$
an artinian local $\mathbb{C}$-algebra with residue field $\mathbb{C}$. $X$ is
called {\it rigid} if all infinitesimal deformations are trivial.
\item An {\it inifinitesimal deformation of order $n$} is an infinitesimal
deformation when $S=\Spec (\mathbb{C}[t]/(t^{n+1})$.
\end{enumerate}
\end{definition}

{\bf Upshot.} An interpretation of the so-called {\it dual numbers}
$k[t]/(t^2)$ (see \cite{Hartshorne-deformation}) as the tangent space of
something is thinking of Taylor polynomials: after quotienting by $t^2$ we loose
the tails of the polynomials and are left with the first derivative information
only - this justifies the use of the ``first order deformations'' term.
As for why using the dual numbers as the underlying ring
of the parameter space, suppose we have a series
$f(x)=\sum_{n=0}^\infty a_nx^n$.
Then
 \begin{align*}
f(x+\varepsilon)&=\sum_{n=0}^\infty a_n(x+\varepsilon)
&=\sum_{n=0}^\infty\sum_{k=0}^na_n\binom{n}{k}x^{n-k}\varepsilon^k\\
&=\underbrace{\sum_{n=0}^\infty a_nx^n}_{k=0}
+\underbrace{\varepsilon\sum_{n=1}^\infty na_nx^{n-1}}_{k=1}\\
&=f(x)+\varepsilon f'(x),
\end{align*}

\noindent
which says that
$$
f(x+\varepsilon)-f(x)=\varepsilon f'(x).
$$
(Further interpretation needed…)


There is the following interpretation in [continued-fractions] p. 39: the space
of first order deformation classes of $X$ is $D(\mathbb{C}[t]/(t^2)$. This is
said to ````represent'' the tangent space $\mathbb{T}^1_X$ of the hypothetical
deformation space of $X$''. (I put double quotations because the word
``represent'' is quoted in the original text.) Further, if $X$ is nonsingular
and compact, then $\mathbb{T}^1_X=H^{1}(X,T_X)$.

Which basically I interpret as: the dimension of the deformation space of a
smooth compact variety is $H^{1}(X,T_X)$.

\medskip\noindent
After a few months of the previous text in this section
I think I have understood the construction of
Kodaira-Spencer map for embedded deformations,
which is \cite[Proposition 3.2.1(ii)]{Sernesi}.

But I will actually briefly outline a particular case
of this, which is easier: \cite[Examples 3.2.4(ii)]{Sernesi}.
This the case when $X \subset Y$ is a Cartier divisor.
Then $X$ is cut out locally by a set of functions $f_i$
defined on an affine cover $U_i$ of $Y$.
Then we have a line bundle defined by the
transition functions $f_{ij}=f_i/f_j$,
and is such that the $f_i$ glue to a section
which vanishes along $X$
(see \cite[Theorem 3.39]{lec}).
This is the line bundle $\mathcal{O}_Y(X)$, which
by the adjunction formula is also the normal bundle $N_{X/Y}$
of $X$ in $Y$.

Now consider a first order deformation
$$
\xymatrix{
X\ar[r]\ar[d]
&\mathcal{X}\subset Y\times B_\varepsilon\ar[d]^\pi\\
0\ar[r]
&B_\varepsilon
}
$$

The great thing about first order deformations
is that the structure sheaf of $\mathcal{X}$
on an affine open $U_i$ of $Y$ 
looks like the tensor product 
$\mathcal{O}_{U_i}\otimes \mathbb{C}[\varepsilon]/(\varepsilon^2)$
(\href{https://stacks.math.columbia.edu/tag/01I4}{01I4}).
By definition of tensor product this just says
that sections look like $f+\varepsilon g$
for $f,g \in \Gamma(\mathcal{O}_{U_i})$.

If $\mathcal{X}$ is a Cartier divisor on $Y \times B_\varepsilon$
(it is; but I'm missing a formal proof),
it is given as the vanishing locus
of some local sections $f_i+\varepsilon g_i$.
(Perhaps the $f_i$ coincide with the distinguished
section of $X$.)
The line bundle associated to $\mathcal{X}$ 
is given by the cocycle
$$
F_{ij}=f_{ij}+\varepsilon+g_{ij}
$$
for $g_{ij}=\frac{g_i-f_{ij}g_j}{f_j}$.
Indeed, note that 
$\frac{1}{f_j+\varepsilon g_j}=\frac{1}{f_j}-\varepsilon \frac{g_j}{f^2_j}$ 
and compute $F_{ij}=(f_i+\varepsilon g_j)/(f_j+\varepsilon g_j)$.

Then
$$
f_i+\varepsilon g_i=(f_{ij}+\varepsilon g_{ij})(f_j+\varepsilon g_j)
$$
gives
$$
g_i=f_{ij}g_j+g_{ij}f_j
$$
and since $f_j$ vanishes along $X$,
we are left with $g_i=f_{ij}g_j$ 
which is how a section of the sheaf $N_{X/Y}$
is constructed by definition of sheaf.


\begin{example}
\label{example-moduli-space-of-nonsingular-Riemann-surfaces-of-genus-g}
\begin{reference}
\cite[Example 3.1]{continued-fractions}
\end{reference}
Fix $g \geq 2$. The dimension of the deformation space of a nonsingular
projective curve $X$ is $3g-3$. This is ``the dimension of the moduli space of
curves of genus $g$''.
 
We can compute this number by Riemann-Roch formula on the bundle $-K_X$. 
Indeed, since $X$ is a curve, $\Omega_X^1=K_X$ and thus $-K_X=T_X$. We get
\begin{align*}
h^0(-K)-h^0(K-(-K))&=\text{deg}(-K)-g+1\\
&=-2g+2-g+1\qquad \substack{\text{degree additive and}\\ \text{deg}K=2g-2}\\
&=-3g+3
\end{align*}
Now by Serre duality $h^0(K-(-K))=h^1(-K)=h^1(T_X)$, and it turns out that a
Riemann surface of genus $g \geq 2$ has no holomorphic vector fields, so that
$h^0(-K)=h^0(T_X)=0$.
\end{example}

\begin{exercise}
\label{exercise-deformations-of-curves-in-K3}
Let $C$ be a smooth genus $g$ curve which can be embedded in a K3 surface  $M$,
and $X$ the family of all deformations of $C$ in $M$.
\begin{enumerate}
\item Prove that $\dim X \leq  g$.
\item Let $\mathcal{X}_g$ be the space of all curves of genus $g$ (smooth?)
which can be possibly embedded to a K3 surface. Prove that each irreducible
component $Z$ of $\mathcal{X}_g$ satisfies $\dim_\mathbb{C} \leq  g+19$. Deduce
that there exists a compact complex curve which cannot be embedded in a K3
surface.
\end{enumerate}
\end{exercise}

\begin{exercise}
\label{exercise-}
Let $C$ be a smooth curve embedded in a K3 surface $X$. Show that the dimension
of the deformation space of $C$ is $\leq g$.
\end{exercise}

\begin{proof}
%\begin{enumerate}
%\item 
The deformation space of a variety is the space of isomorphism classes of
deformations as explained above. It turns out that there is a way to associate
1-cocyles of the tangent sheaf to deformations, so that in fact the deformation
space $\text{Def}_1$ is isomorphic to $H^{1}(X,\mathcal{T}_X)$ for any variety
$X$.

For our curve $C$ we thus know that the dimension of the space of deformations
(deformations not necessarily contained in $X$) is $h^1(\mathcal{T}_C)$.  The
family of deformations of $C$ that are contained in $X$ is the Hilbert scheme of
curves with fixed Hilbert polynomial $P(t)$ after quotienting by $\mathbb{C}s$,
where $s$ is the section whose vanishing locus is  $C$. 
This says that the number we are looking for is $h^{0}(X,\mathcal{O}(C))-1$.

Before showing that in fact $h^0(X,\mathcal{O}(C))=1+g$,
let's explain why this Hilbert scheme construction actually works.
Intuitively, any curve lying in the same component
as $C$ would be deformation-equivalent to $C$
--- at least we know that it's Hilbert polynomial is the same.
But it's not clear what's the deformation involving these two
(this question should be addressed!).
Further, this doesn't explain Vladimiro's
suggesion of ``taking quotient by $\mathbb{C}s$''.
To me it looks like I would just compute the dimension
of the tangent space of $\text{Hilb}^P(M)$ at $C$.

Now I will show that $h^0(X,\mathcal{O}(C))=1+g$ (see 
\cite[Lemma 1.2.1, Remark 1.2.2]{huk}).

Consider the ideal sheaf exact sequence twisted by
$\mathcal{O}(C)$:
$$
\xymatrix{
0\ar[r]&\mathcal{O}_X\ar[r]&\mathcal{O}(C)\ar[r]
&\mathcal{O}_X(C) \otimes \mathcal{O}_C=\mathcal{O}(C)|_{C}\ar[r]&0
}
$$
By $X$ being a K3 we know that $H^{1}(\mathcal{O}_X)=0$, so that we have the
short exact sequence in cohomology
$$
\xymatrix{
0\ar[r]&H^{0}(\mathcal{O}_X)\ar[r]&H^{0}(\mathcal{O}(C))\ar[r]
&H^{0}(\mathcal{O}(C)|_{C})\ar[r]&0
}
$$
so that $h^0(\mathcal{O}(C))=1+h^1(\mathcal{O}(C)|_{C})$. 
A version of adjunction formula says
$\omega_C \cong (K_X \otimes\mathcal{O}(C)|_{C}$, and using that 
$K_X=\mathcal{O}_X$ we obtain $h^0(\mathcal{O}(C)|_{C})=h^0(\omega_C)=g$.

To 

\medskip\noindent
For the record I put other thoughts I went through in solving this exercise.

Recall from \cite[p. 146]{gri} that the normal bundle
$\mathcal{N}$ of a hypersurface of a smooth variety $X$ satisfies
$\mathcal{N}^\vee\cong\mathcal{O}_X(-C)|_{C}$. Taking duals we get that
$\mathcal{N}\cong\mathcal{O}_X(C)|_{C}$.

By adjunction formula $2g-2=(\mathcal{O}(C),\mathcal{O}(C))$. 
Applying Riemann-Roch
to $\mathcal{O}(C)$ (which is by definition the dual of the ideal sheaf of $C$,
which is a line bundle on $X$), we obtain that
$\chi(\mathcal{O}(C))=2+\frac{1}{2}(\mathcal{O}(C),\mathcal{O}(C))$. Then
$\chi(\mathcal{O}(C))=g+1$.

Now recall that $\chi(\mathcal{O}(C))=h^0(\mathcal{O}(C)
-h^1(\mathcal{O}(C))+h^2(\mathcal{O}(C))$. By Serre duality and $X$ being a K3
surface we see that $h^2(\mathcal{O}(C))\cong h^{0}(\mathcal{O}(-C))$, which is
the ideal sheaf of $C$. Any section of such a sheaf would vanish along $C$, and
since $X$ is compact we conclude there cannot be any such section.

Now we show that also $h^1(\mathcal{O}(C))=0$ to conclude that
$h^0(\mathcal{O}(C))=g+1$. 

We thus conclude that $h^0(\mathcal{O}(C))$




Since $C$ is smooth we can use the normal exact sequence
$$
\xymatrix{
0\ar[r]&\mathcal{T}_C\ar[r]&\mathcal{T}_X|_{C}\ar[r]&\mathcal{N}\ar[r]&0
}
$$
Taking Euler characteristic we see that
$\chi(\mathcal{T}_C)+\chi(\mathcal{N})=\chi(\mathcal{T}_X|_{C})$.


This means that we should be done once we compute $\chi(\mathcal{T}_X|_{C})$. 
For this we can use Riemman-Roch formula for coherent sheaves on a curve, which
tells us that
$$
\text{deg}(\mathcal{T}_X|_{C})=\chi(\mathcal{T}_X|_{C})
-\text{rk}(\mathcal{T}_X|_{C})\cdot\chi(\mathcal{O}_C)
$$
But of course we know that since $C$ is a curve, by Serre duality we get that
$\chi(\mathcal{O}_C)=1-g$. (Indeed:
$h^1(\mathcal{O}_C)=h^0(\Omega^1_C):=p_a(C)$.) And now the question is what is
the degree. Apparently this is just the first Chern class $c_1$ of the
restricted tangent bundle. So what is it? And what is $h^0(\mathcal{T}_C)$ if
the genus is 0 or 1, and that's it.

I
know that $h^0(\mathcal{T}_X)=0$ and $h^1(\mathcal{T}_X)=20$ by $X$ being a K3
surface, but I'm not sure what happens when we restrict to $C$.

If $g \geq 2$ we know that $h^0(\mathcal{T}_C)=0$, so that
$\boxed{\chi(\mathcal{T}_C)=h^1(\mathcal{T}_C)}$.
\bigskip
To compute the
latter Euler characteristic, which is given by definition by
$\chi(\mathcal{T}_X|_{C})=h^0(\mathcal{T}_X|_{C})-h^1(\mathcal{T}_X|_{C})$, we
first note that $h^0(\mathcal{T}_X|_{C})=0$, because this is the dimension of
the global holomorphic vector fields on $X$
restricted to $C$, which is constant along $C$ since $X$ is
smooth. And then this number is actually zero by Hodge numbers of a K3 surface.

and taking cohomology long exact sequence we obtain
$$
\xymatrix{
\cdots\ar[r]&H^{0}(T_X)\ar[r]&H^{0}(\mathcal{N})\ar[r]&H^{1}(T_C)\ar[r]&\\
H^{1}(T_X)\ar[r]&H^{1}(\mathcal{N})\ar[r]& \cdots
}
$$
\medskip\noindent
Or is it $h^1(\mathcal{N})$? See \cite{HarrMorr}. If this was the case, then we
can use adjunction formula as above to get that
$2g-2=(\mathcal{N},\mathcal{N})=\text{deg}_C \mathcal{N}$. Then we may find
$h^0(\mathcal{N})$ via Riemann-Roch:
$$
h^0(\mathcal{N})-h^0(K_C-\mathcal{N})=\text{deg}\mathcal{N}-g+1
$$
Note that $h^0(N_C-\mathcal{N})=h^1(-K_C+N+K_C)=h^1(\mathcal{N})$ via Serre
duality, and by Riemann-Roch on a surface as above we see that
$$
g+1=\chi(\mathcal{N})=h^0(\mathcal{N})-h^1(\mathcal{N})
\implies h^1(\mathcal{N})=-g-1+h^0(\mathcal{N})
$$
so that
\begin{align*}
h^0(\mathcal{N})-(-g-1+h^0(\mathcal{N}))&=\text{deg}\mathcal{N}-g+1\\
\implies \text{oops! I lost $h^0(\mathcal{N})$ in this operation…}
\end{align*}
Maybe if I just use normal exact sequence and realise that
$$
h^0(\mathcal{N})=h^0(\mathcal{T}_X|_{C})-h^0(\mathcal{T}_C)
$$
I know that $h^0(\mathcal{T}_C)=0$ for $g>1$, so the question is how to compute
the restricted holomorphic vector fields.
%\end{enumerate}
\end{proof}

\medskip\noindent
Look for the ``Perfect obstruction theory''.
This is used by Bruzzo for deformations of (super)stacks,
work with E. Paiva and ---.

\section{Continued fractions}
\label{section-continued-fractions}

Definition of HJ continued fraction. For $i>2$ they are in bijection with
$\mathbb{Q}_{>1}$.

The basic diagram of this course starts with a surface $S$ (eg. Hirzebruch
surface $S=\mathbb{F}_m$). Blowing up leads to $X$, and contracting Wahl chains
on $X$ leads to $W$, a normal projective surface that has only Wahl
singularities. Then we construct $\mathbb{Q}$-Gorenstein smoothings $W_t$.
(These  $\mathbb{Q}$-Gorenstein smoothings have Milnor number =0.)

Continued fractions have minimal models:
\begin{itemize}
\item $[1,1]$ means a 0 curve,  $\mathbb{P}^1$.
\item $[1]$ means a  $-1$ curve,  $\mathbb{P}^1$.
\item For $\frac{m}{q}\in\mathbb{Q}_{>1}$, the continued fraction
$[e_1,\ldots,e_r]$ means a chain, which is a sequence of lines that intersect
transversally with $-e_1,\ldots,-e_r$. This is mapped to $\frac{1}{m}(1,q)$.
\end{itemize}

\medskip\noindent

{\bf Third lecture.}

Here's some slogans/recap:
\begin{enumerate}
\item The most important cyclic quotient singularities (c.q.s.) are Wahl
$\frac{1}{n^2}(na-1)$. There is a model to deal with this kind o singularities
using continued fractions. This is very silly but what I picked up is that ``you
add a 2 in the end and add +1 to the first number'', so for example
$[4]\rightsquigarrow[5,2]\rightsquigarrow[6,2,2]$. But on the second step the
$[5,2]$ also goes to $[2,5,3]$ in a way I don't understand. This is called the
Wahl algorithm.
\item (See [KSB88]) There is a notion of $M$-resolution, which is a drawing of 
several curves $\Gamma_i$ intersecting at points $P_i$ that may be Wahl 
singularities or 
smooth points with the key property that $\Gamma_i\cdot K\geq 0$. We have
``toric boundary for $P_i$''. These $M$-resolutions are in 1-1 correspondence
with smoothings of $\frac{1}{m}(1,q)$, and in turn in 1-1 correspondence with
continued fractions $K\left(\frac{m}{m-q}\right)=\{k_1,\ldots,k_s]:
1\leq k_i\leq b_i\;\forall i\}$ where $\frac{m}{m-q}=[b_1,\ldots,b_s]$.
\end{enumerate}

\medskip\noindent

Today we consider the fibers to be $W_t=\mathbb{P}^2$ and
try to find $W$. Set $m_1,m_2,m_3\in\mathbb{Z}_{>0}$. Define
$$
\mathbb{P}(m_1,m_2,m_3):=
\mathbb{P}^2/(\mathbb{Z}/m_1\oplus\mathbb{Z}/m_2\oplus\mathbb{Z}/m_3)
=\mathbb{C}^3\setminus\{0\}/(\lambda\in\mathbb{C}^*\lambda(x,y,z)
=(\lambda^{m_1}x,\lambda^{m_2}y,\lambda^{m_3}z))
$$
For $\text{gdc}(d,m_i)=1$ we have
$\mathbb{P}(dm_1,dm_2,dm_3)=\mathbb{P}(m_1,m_2,m_3)$.

For a triangle $xyz=0$ given by three lines $\Gamma_i$ we have cqs singularities
of the kind $\frac{1}{m_1}(m_2,m_3)$. In this case
$K_W=-(m_1+m_2+m_3)\xi=-\Gamma_1-\Gamma_2-\Gamma_3$ for
$\xi^2=\frac{1}{m_1m_2m_3}$, and $\text{Cl}(W)=\mathbb{Z}\left<\xi\right>$.
Since these are Wahl singularities, we must have that the $m_i$ are squares,
i.e. $m_i=n_i^2$ for some $n_i$. We must have:
\begin{align*}
K_W^2=(m_1+m_2+m_3)^2\frac{1}{m_1m_2m_3}=9&=K^2_{\mathbb{P}^2}\\
\implies (n_1^2+n_2^2+n_3^2)-9n_1^2n_2^2n_3^2&=0\\
\implies (n_1^2+n_2^2+n_3^2-2n_1n_2n_3)\cdot(\text{positive factor})&=0\\
\implies n_1^2+n_2^2+n_3^2&=3n_1n_2n_3
\end{align*}
The last equation is known as {\it Markov equation}.

\begin{example}
\label{example-Hirzebruch-surface}
For $\mathbb{P}(1,1,4)=W$, a triangle with a Wahl singularity
$\frac{1}{4}(1,1)$ in one vertex. Blowing up gives the Hirzebruch surface
$\mathbb{F}_4$, so that a minimal resolution is the triangle. Compare with
[\href{https://arxiv.org/pdf/2504.19929}{Hacking-Prokhorov-2010}]. This example
satisfies the Markov equation for $n_1=1,n_2=1,n_3=2$.
\end{example}

\begin{theorem}[[HP2010]]
\label{theorem-HP2010}
If $\mathbb{P}^2\rightsquigarrow W$ with only log terminal singularities then
$W$ is a partial $\mathbb{Q}$-Gorenstein smoothing of $\mathbb{P}(a^2,b^2,c^2)$
where $a^2+b^2+c^2=3abc$.
\end{theorem}
By the Markov equation condition all the singularities must be Wahl. The triple
$(a,b,c)$ is called {\it Markov triple}. Any permutation of a Markov triple is
another Markov triple. Is $(a,b,c)$ is Markov then so is $(a,b,3ab-c)$. This
allows to construct a {\it Markov tree}. There is so-called Markov conjecture
(due to Frobenius) still unsolved.

\section{Fano varieties}
\label{section-Fano-varieties}

\begin{definition}
\label{definition-Fano-variety}
A {\it Fano variety} is a projective variety with $-K_X$ ample.
\end{definition}

\noindent
The idea of the Fano index is that the
larger the index, the simpler is the variety.
In a talk by Jõao, it is defined as the larger integer
dividing $-K_X$ in $\text{Pic}(X)$.
\begin{definition}
\label{definition-Fano-index}
$$
r(X):=\text{min}\{r:\frac{c_1(X)}{r}\in H^{2}(X,\mathbb{Z})\}
$$
\end{definition}

\begin{exercise}
\label{exercise-Fano-vanishing-higher-cohomology}
By Kodaira vanishing theorem \ref{theorem-Kodaira-vanishing}, 
you can show that the cohomology $H^{i}(X,L)$ for
a Fano variety $X$ vanishes. You just have to put $L=\mathcal{O}(k)$ with $k\geq
-r$, where $r$ is the Fano index.
\end{exercise}

\begin{exercise}
\label{exercise-Pic-H2-Fano}
Show that  $\text{Pic}(X)\cong H^{2}(X,\mathbb{Z})$ holds for Fano varieties.
\end{exercise}

\begin{remark}
\label{remark-derived-category-of-Fano-3-folds-with-vanishing-simplicial
-cohomology}
\begin{reference}
Marcos Jardim, CIMPA 2025 Florianópolis, Lecture 2.
\end{reference}
If $H^3(X,\mathbb{Z})=0$ of a Fano 3-fold, then its derived category is
generated by 4 elements.
\end{remark}

\medskip\noindent
Araujo-Druel em 2017 classify
Fano foliations relating the Fano index (of the foliation)
with its rank.
João found conditions 
(integrability of the leaves)
to describe del Pezzo foliations
(a kind of Fano foliation)
with Picard rank 1.
Araujo (2019) classifies the leaves of a algebraically integrable
del Pezzo foliation (with no restriction on the (log
canonica) singularities!).


\section{Quivers}
\label{section-quivers}

\begin{definition}
\label{definition-quiver}
A {\it quiver} is a set of vertices $Q_0$, a set of arrows $Q_1$ equipped with
the maps of source $s$ and target $t$ that to each arrow they assign the point
that is source or target of the arrow.
\end{definition}

\begin{definition}
\label{definition-representation-of-quiver}
A {\it representation} of a quiver is a set of finite dimensional vector spaces
equipped with maps between them realising a given quiver (incomplete…).
\end{definition}

There is a notion of projective representation, which I missed to write. But it
is analogous to the injective representation:

\begin{definition}
\label{definition-injective-representation-of-quivers}
Given a quiver $Q$, the {\it injective representation} of $Q_0$ is given by, for
$i \in Q_0$,
$$
I(i)_j=\begin{cases}
k\qquad &i=j \\
k^{d'}\qquad &j \neq i
\end{cases}
$$
where $d'$ is the number of paths from $j$ to $i$.
\end{definition}

\section{Stacks}
\label{section-stacks}

My first definition of stack can be extracted
from

\begin{definition}
\label{definition-superstack}
A {\it superstack} is a stack over
the étale site $\text{SSch}$ of superschemes,
i.e. it is a category fibered in groupoids
over the category of superschemes,
the latter equipped with the 
étale topology, 
satisfying the descent condition.
\end{definition}

Here are some other definitions:

\begin{definition}
\label{definition-algebraic-stacks}
Let  $\mathfrak{X}$ be a stack over $\text{Sch}_{\text{ét}}$.
An {\it algebraic space} is
such that there exists morphism
$\mathcal{U} \to \mathfrak{X}$
where $\mathcal{U}$ is a scheme, that is
schematic, étale and injective (check this one).

$\mathfrak{X} \to y$ is {\it representable} if
there exists a scheme $\mathcal{U}$ and a map
$\mathcal{U} \to y$ such that the 
fibered product
$$
\xymatrix{
\mathcal{U} \times_y \mathfrak{X}\ar[r]\ar[d]\ar@{}[dr]|-{\lrcorner}&\mathfrak{X}\ar[d]\\
\mathcal{U}\ar[r]&y
}
$$
is an algebraic space.

Finally, a  stack is {\it algebraic} (resp. {\it Deligne-Mumford})
is there exists a 
representable surjective morphism  $\mathcal{U} \to \mathfrak{X}$ 
that is smooth (resp. étale).

A {\it stable map} over a projective
variety $X$ is an element of the first
Chow group $\beta \in A_1$, where
 $(C,g)$ is an algebraic curve and
$f:C \to X$ with $[f(C)]=\beta$.
\end{definition}

\noindent
The curves that are points under this map
(contractible) are {\bf stable}.

\section{Moduli}
\label{section-moduli}

\noindent
Pseudo-admissible covers may be the way Mumford orifinally developed stability
notions.

\noindent
This is a two-session minicourse by Ugo Bruzzo
in LEGALzinho 2025.

Let $X$ be a smooth complex projective variety,
$X \overset{j}{\hookrightarrow}\mathbb{P}^n$.
$\mathcal{O}_{\mathbb{P}^n}(1)$ is the class of hyperplanes of $\mathbb{P}^n$.
$\mathcal{L}=j^*\mathcal{O}_{\mathbb{P}^n}(1)$.
$c_1(\mathcal{L})\in H^2(X,\mathbb{Z})$ is
a {\it polarization}.

Let $\mathcal{E}$ be a coherent $\mathcal{O}_X$-module.
For example, $\mathcal{E}$ may be a locally free $\mathcal{O}_X$-module.
Recall that $H^k(X,\mathcal{E})$ are finite dimensional vector spaces
and they vanish for $k>\dim X=n$.
Recall that the {\it Euler characterstic} of $\mathcal{E}$ 
is given by
$\chi(\mathcal{E})=\sum_{k=0}^n(-1)^k\dim_{\mathbb{C}}H^k(X,\mathcal{E})$.

The {\it reduced Hilbert polynomial} of $\mathcal{E}$ is
$$
p_{\mathcal{E}}(k)=\frac{1}{\text{rk}\mathcal{E}}
\chi(\mathcal{E}\otimes_{\mathcal{O}_X}\mathcal{L}^k).
$$

\begin{definition}
\label{definition-stable-module}
Let $\mathcal{E}$ be a coherent $\mathcal{O}_X$-module
with no torsion and positive rank.
$\mathcal{E}$ is {\it stable} if for all $f \subset \mathcal{E}$ 
($0 \text{rk} f<\text{rk}\mathcal{E}$),
$$
p_f(k)<p_{\mathcal{E}}(k)\qquad \text{for } k\gg 0
$$
and {\it semi-stable} if we put $\leq$ instead of $<$.

\begin{lemma}
\label{lemma-stable}
If $\mathcal{E}$ is stable then $\mathcal{E}$ is irreducible
($\mathcal{E} \neq  \mathcal{E}_1 \oplus \mathcal{E}_2$)
and $\text{Aut}(\mathcal{E})\simeq \mathbb{C}^*$.
\end{lemma}

\end{definition} 

The {\it functor of moduli} is
\begin{align*}
\mathcal{M}: \text{Sch}_\mathbb{C} &\longrightarrow \mathsf{Set} \\
S
&\longmapsto
\left\{
\substack{\text{stable family}  \\ \text{of sheaves with} \\
\text{Hilbert polynomial} f\\
\text{parametrized by $S$}} \right\}\Big/\sim.
\end{align*}

\medskip\noindent
Riemann-Roch.
Let $C$ be a smooth, projective, connected curve
of genus $g$.
Let $\mathcal{E}$ be a coherent sheaf over $H^2(C,\mathbb{Z})$ 
with generator of $H^2(X,\mathbb{Z})$ 
$w=[pt] \in H^2(C,\mathbb{Z})$.
$c_1(\mathcal{E})=\text{deg}\mathcal{E}\cdot w$.

\begin{theorem}
\label{theorem-euler-characteristic-moduli}
\begin{align*}
\chi(\mathcal{E})
&=\int_C(1+(1-g)w)(\text{rk}\mathcal{E}+\text{deg}\mathcal{E}w)\\
&=\int_C \text{deg}\mathcal{E}+w+(1-g)\text{rk}\mathcal{E}\cdot w\\
&=\text{deg}(\mathcal{E})+1-g \text{rk}\mathcal{E}.
\end{align*}
\end{theorem}

\begin{example}
\label{example-hilbert-polynomial-sheaf}
\end{example}

\begin{theorem}
\label{theorem-moduli-suave}
Se $(r,d)$, $r>0$ are coprime,
the functor of moduli of stable bundles
over $C$ of rank $r$ and degree $d$ 
is representable, and the moduli space
$\mathcal{M}_C(\mathcal{E},d)$ is smooth.
\end{theorem}

\noindent
In the non coprime case this does not necessarily hold.

\begin{definition}
\label{definition-}
$\mathcal{M}$ is a {\it grosseiro} moduli space
if the funtor is not representable.
\end{definition}

\noindent
This means that although there is not
a universal family,
there is still a correspondence
of the closed points of $\mathcal{M}$ and the
equivalence classes of semistable sheaves
of rank $r$ and degree $d$.

\medskip\noindent
Now let's fix $\dim X=2$.
Let $\mathcal{E}$ be a stable $\mathcal{O}_X$-module.
Let $r=\text{rk}\mathcal{E}$,
and the {\it discriminant}
$$
\Delta(\mathcal{E})=c_2(\mathcal{E})-\frac{\mathcal{E}-1}{2r}c_1(\mathcal{E})^2
\in H^4(X,\mathbb{Q})\cong\mathbb{Q}.
$$
Here enter $c_1(\mathcal{E}) \in H^2(X,\mathbb{Z})$
and $c_2(\mathcal{E}) \in H^4(X,\mathbb{Z})$.

\begin{theorem}[Bogomolov]
\label{theorem-bogomolov}
$\Delta(\mathcal{E}) \geq 0$.
\end{theorem}

\noindent
It could be interesting to study the proof of this theorem.

\medskip\noindent
Miyaoka (1987). Consider a curve $C$ and a bundle $E$.
We can for any fibre, which is a vector space $V$,
consider its projectivization
$V\setminus\{0\}/\mathbb{C}^*=\mathbb{P}(V)$.
Then we can bundlize this construction ad obtain
$\mathbb{P}E \to C$.
Consider $\mathcal{O}_{\mathbb{P}E}(1)$.
We define
$$
\lambda=rc_1(\mathcal{O}_{\mathbb{P}E}(1))
-\pi^*c_1(E) \in H^2(\mathbb{P}E,\mathbb{Z}).
$$

\begin{theorem}
\label{theorem-semistable-nef}
$E$ is semistable if $\lambda$ is 
semicorrente efectiva.
\end{theorem}

\noindent
\begin{definition}
\label{definition-nef}
Consider again a projective smooth connected
curve $\Gamma$ and let $f:\Gamma \to \mathbb{P}E$.
We say $\lambda$ is {\it nef} if 
$$
\int_{\Gamma} f^*\lambda\geq 0.
$$
\end{definition}


\noindent
The following theorem was proved by Nakayama
(who is a different author than that of Nakayama's lemma)
in 1999 but was unkown to Bruzo and Daniel HR.
\begin{theorem}[Nakayama '99, Bruzzo-HR '04]
\label{theorem-bruzzo-daniel-nakayama}
The following are equivalent:
\begin{enumerate}
\item $\lambda$ is nef.
\item For all $f:\Gamma \to X$,
$f^*E$ is semi-stable.
\item $E$ is semistable with respect to 
the polarization, and  $\Delta(E)=0$.
\end{enumerate}
\end{theorem}

The polarization says how the surface is embedded in projective space,
it is $H=c_1(\mathcal{L})$. The stability notion is
$$
\mu(E)=\frac{c_1(E)\cdot H^{n-1}}{\text{rk}E}.
$$
\medskip\noindent
They have proved that for Higgs bundles 1 $\iff$ 2, $3 \implies 1,2$.
And also for $1,2 \implies 3$ but only for $r=2$ using the Higgs Grassmannian,
but the problem is very complicated for general rank.

\medskip\noindent
Second lecture. I think this might be a more general
definition of the Hilbert polynomial:
$$
\sum_{k=0}^n(-1)^k \dim H^k(X,\mathcal{E})=\chi(\mathcal{E})
=\int_X\text{ch}(\mathcal{E})\text{td}X
$$
And we also have:
$$
\chi(\mathcal{E}\otimes \mathcal{L}^*)
=\int_X (1+(1-g)w)(r+kr
\underbrace{c_1(\mathcal{L})}_{\text{deg}\mathcal{L}\cdot w}
+\text{deg}\mathcal{E}\cdot w.
$$
And then
$$
p_{\mathcal{E}}=k \text{deg} \mathcal{L} +1-g+\mu(\mathcal{E}).
$$



\medskip\noindent
Now we discuss Higgs bundles.
Let $X$ be ac onnected complex projective variety.

\begin{definition}
\label{definition-higgs-bundle}
A {\it Higgs bundle} on $X$ 
is a pair $\mathfrak{C}=(E,\phi)$ with
\begin{itemize}
\item $E$ a vector bundle on $X$ 
\item $\phi:$ a morphism (connection-like…)
\end{itemize}
\end{definition}

For the following see Nitsuze, FGA explained.
There following scheme is also representable
(I think this is the ``universal quotient'')

We have also consider the Grassmanization of a vector bundle.
We obtained a short exact sequence
and tried to compute the tensor product with
the holomorphic differentials.
The idea was to pullback the grassman bundle
given a map $f:Y \to X$.
The result is not satisfactory, the pulled back
thing may be strange. This is called the
{\it Higgs Grassmannian}.


Here's some notion also studied by Campana, Demailly:
numerically flat bnef bundle $E$.
Then we have that 
\begin{lemma}
\label{lemma-proof-hugo-bruzzo}
If $c_1=0$ then numerically flat bundle iff semistable.
\end{lemma}

\medskip\noindent
This is the beginning of a talk by Alan Muniz
at Bandoleros 2025 titled
``Rank 2-bundles on $\mathbb{P}^3$ with odd determinant''.
(Work in progress with A. Fontes.)

Problem: Classify rank-2 bundles on $\mathbb{P}^3$.

We consider bundles $E$ with chern classes
$c_1=-1$, $c_2=2n$ and $c_3=0$.
Let $\mathcal{B}(-1,2n)$ be the moduli space
of $\mu$-stable rank 2-bundles.

\begin{itemize}
\item $\mathcal{B}(-1,2)$ is irreducible of dimension 11
[Hartshorne-Sols '82, Manolache '81].

\item $\mathcal{B}(-1,4)$ has two components
[Barnica-Manolache '85].

\item $c_2\geq 6$, Tikhomirov (2014)
Fontes-Jardim (2023, 2024, 2025).
$\mathcal{B}(-1,6)$ has $\geq 4$ components.
\end{itemize}

\medskip\noindent
\subsection*{Serre correspondence}
How can we describe these spaces?
Let $E$ be a vector bundle of rank 2
on $\mathbb{P}^3$ and $\sigma \in H^0(E(r))$.
Then $C=(\sigma_0)\subset \mathbb{P}^3$ is
a curve * equipped
with an ``extension class''
$\zeta \in H^0(\omega_C(4-2r-c_1))$.
In fact, this correspondence
is 1-1, and we call it Serre correspondence.
We have that $\text{deg}C=c_2(E(r))$ and
$p_a(C)=1=\frac{c_2(E(r))(c_1(E(r))-4)}{2}$
To pass from the curve to the bundle
and viceversa we use
$$
\xymatrix{
0\ar[r]
&\mathcal{O}(-r)\ar[r]
&E\ar[r]
&\mathcal{I}_C(c_1+r)\ar[r]
&0
}
$$


\medskip\noindent
Upshot about Deligne-Mumford stacks:
the sit below a stack via a map that is étale.

\subsection{Framed moduli}
\label{subsection-framed-moduli}

\noindent
Framed instantons on the blow-up of $\mathbb{P}^3$ at a point.

Let $X$ be a projective variety and fix a polarization $H$.
Let $\mathcal{E}$ be a sheaf on $X$.
A {\it framed pair} is a $(\mathcal{F},\alpha)$ where
$\mathcal{F}$ is a coherent sheaf on $X$ 
and $\alpha$ is a {\it framing datum}, i.e.
a map $\alpha:\mathcal{F} \to \mathcal{E}$.

Set
$$
\mathcal{E}(\alpha)
=\begin{cases}
	1\qquad &\alpha\neq 0 \\
	0\qquad &\text{otherwise} .
\end{cases}
$$
Define the Hilbert polynomial by 
$$
P(k)=\chi(\mathcal{F} \otimes \mathcal{O}(kH)).
$$
For fied $\delta \in \mathbb{Q}[t]$ rational positive,
define
$$
P_{(\mathcal{F},\alpha)}=P_{\mathcal{F}}(k)-\varepsilon(\alpha)\delta.
$$

We can also define induced framings from subsheaves $\mathcal{F}' \subset
\mathcal{F}$, by asking that the inclusion commutes
with the framing datum $\alpha$.
Then we put
$$
\varepsilon(\alpha')
=\begin{cases}
	1\qquad &\mathcal{F}' \subseteq Ker \alpha \\
	0\qquad &\text{otherwise} .
\end{cases}
$$
We say that $\mathcal{F}$ is {\it $\delta$-(semi)stable}
if for any subpair $(\mathcal{F}',\alpha')$ with 
$\text{rk}\mathcal{F}'=r'<r=\text{rk}\mathcal{F}$
one has
$$
\frac{1}{r'}P_{(\mathcal{F}',\alpha')}(k)
\leq  \frac{1}{r}P_{(\mathcal{F},\alpha)}(k).
$$
If $\text{deg}(\delta)\leq \dim X$,
there are moduli functors $\mathcal{M}^{st}(X,\mathcal{E},\delta)
\subseteq \mathcal{M}^{\text{ss}}(X,\mathcal{E},\delta)$.

\begin{remark}
\label{remark-degree-greater-than-dimension}
If $\text{deg} \delta>\dim X$
then $\alpha$ is injective or zero, and $\mathcal{F}$ is semistable.
This gives quot schemes.
\end{remark}

\noindent
Huybrechts-L. In $\dim X\leq 2$
there exists a quasi-projective fine moduli space
for $\delta \in \mathbb{Q}[t]$ and $p \in \mathbb{Q}[k]$.

\medskip\noindent
Then there is a notion of good framing by Bruzz-M,
depending on $D$, an effective divisor on $X$.
They proved that having good framing is enough
to obtain a fine quasi-projective moduli space.

\begin{definition}
\label{definition-instanton}
A coherent sheaf $\mathcal{E}$ on $X=\tilde{\mathbb{P}^3}$ 
is an {\it instanton} if it is $\mu_{\mathcal{L}}$-semistable,
$c_1(\mathcal{E})=0$ and $H^0(\mathcal{E})=0=H^0(Ec(-3,2))$,
$H^1(\mathcal{E}(-2,1))=0=H^2(\mathcal{E}(-2,1))$ and
$H^3(\mathcal{E}(0,-1))=H^2(\mathcal{E}(-1,1))$.
\end{definition}

\noindent
Some examples are locally free sheaves of rank 2 (via Harrshorne-Serre
correspondence),
torsion-free rank 2 constructed via elementary
transformations.

\medskip\noindent
General result: for rank 2 instantons on Fano threefolds
that have a plane $E$, 
$\mathcal{E}|_{E}$ is $\mu$-semistable
if and only if $h^1(\mathcal{E}(-2,1) \otimes \mathcal{O}(-E))=0$.




\section{Campana}

A \textit{special} variaty should be thought of 
as the farthest notion possible from being of general type.

\begin{definition}[Campana]
$X$ complex projective.
$X$ is \textit{special} if $\forall i \leq  p \leq \dim X$,
for all line bundle $L \subset \Omega^P_X$,
$\kappa(L)<P$ (Kodaira dimension).
\end{definition}

\noindent
A conjecture by Campana says that
$X$ is special if and only if there exists  $f:\mathbb{C}\to X$
Zariski dense.
Compare this with Kobayashi hyperbolicity!
There are no lines on …?

The main tool for proving this kind of
``Bloch-Ochai" statements (i.e.
something of the kind: if $h^0(\Omega_X)> \dim X$
then there are no nonconstant $f:\mathbb{C}\to X$
Zariski dense)
is by passing through the Albanese variety.

In a talk [fill details]
we study a theorem of this kind but with the bound
given by $q^+(X)$ defined as the supremum of the $h^0(\hat{X},\pi^*\Omega_{X}$
for $\hat{X}\to X$ some (étale) cover.
Can we prove a BO stement with the condition
$q^+(X)>\dim X$?

Notion of special comes in: indeed
 $$
q^+(X)>\dim X \implies X \text{ not special }\implies 
\not\exists f:\mathbb{C}\to X \text{Zariski dense}.  
$$

The following theorem is not at all easy to prove:
\begin{theorem}[Campana]
\label{theorem-campana}
$\hat{X}\xrightarrow{\text{étale} }X$, then
$\hat{X}$ is special iff $X$ is special.
\end{theorem}

\noindent
My question: why does the abelianess of
the Albanese variety is important in this construction?

\bibliography{my}
\bibliographystyle{amsalpha}




\end{document}

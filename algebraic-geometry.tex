\IfFileExists{stacks-project.cls}{%
\documentclass{stacks-project}
}{%
\documentclass{amsart}
}

% For dealing with references we use the comment environment
\usepackage{verbatim}
\newenvironment{reference}{\comment}{\endcomment}
%\newenvironment{reference}{}{}
\newenvironment{slogan}{\comment}{\endcomment}
\newenvironment{history}{\comment}{\endcomment}

% For commutative diagrams we use Xy-pic
\usepackage[all]{xy}

% We use 2cell for 2-commutative diagrams.
\xyoption{2cell}
\UseAllTwocells

% We use multicol for the list of chapters between chapters
\usepackage{multicol}

% This is generally recommended for better output
\usepackage{lmodern}
\usepackage[T1]{fontenc}

% For cross-file-references
\usepackage{xr-hyper}

% Package for hypertext links:
\usepackage{hyperref}

% For any local file, say "hello.tex" you want to link to please
% use \externaldocument[hello-]{hello}
\externaldocument[introduction-]{introduction}
\externaldocument[conventions-]{conventions}
\externaldocument[sets-]{sets}
\externaldocument[categories-]{categories}
\externaldocument[topology-]{topology}
\externaldocument[sheaves-]{sheaves}
\externaldocument[sites-]{sites}
\externaldocument[stacks-]{stacks}
\externaldocument[fields-]{fields}
\externaldocument[algebra-]{algebra}
\externaldocument[brauer-]{brauer}
\externaldocument[homology-]{homology}
\externaldocument[derived-]{derived}
\externaldocument[simplicial-]{simplicial}
\externaldocument[more-algebra-]{more-algebra}
\externaldocument[smoothing-]{smoothing}
\externaldocument[modules-]{modules}
\externaldocument[sites-modules-]{sites-modules}
\externaldocument[injectives-]{injectives}
\externaldocument[cohomology-]{cohomology}
\externaldocument[sites-cohomology-]{sites-cohomology}
\externaldocument[dga-]{dga}
\externaldocument[dpa-]{dpa}
\externaldocument[sdga-]{sdga}
\externaldocument[hypercovering-]{hypercovering}
\externaldocument[schemes-]{schemes}
\externaldocument[constructions-]{constructions}
\externaldocument[properties-]{properties}
\externaldocument[morphisms-]{morphisms}
\externaldocument[coherent-]{coherent}
\externaldocument[divisors-]{divisors}
\externaldocument[limits-]{limits}
\externaldocument[varieties-]{varieties}
\externaldocument[topologies-]{topologies}
\externaldocument[descent-]{descent}
\externaldocument[perfect-]{perfect}
\externaldocument[more-morphisms-]{more-morphisms}
\externaldocument[flat-]{flat}
\externaldocument[groupoids-]{groupoids}
\externaldocument[more-groupoids-]{more-groupoids}
\externaldocument[etale-]{etale}
\externaldocument[chow-]{chow}
\externaldocument[intersection-]{intersection}
\externaldocument[pic-]{pic}
\externaldocument[weil-]{weil}
\externaldocument[adequate-]{adequate}
\externaldocument[dualizing-]{dualizing}
\externaldocument[duality-]{duality}
\externaldocument[discriminant-]{discriminant}
\externaldocument[derham-]{derham}
\externaldocument[local-cohomology-]{local-cohomology}
\externaldocument[algebraization-]{algebraization}
\externaldocument[curves-]{curves}
\externaldocument[resolve-]{resolve}
\externaldocument[models-]{models}
\externaldocument[functors-]{functors}
\externaldocument[equiv-]{equiv}
\externaldocument[pione-]{pione}
\externaldocument[etale-cohomology-]{etale-cohomology}
\externaldocument[proetale-]{proetale}
\externaldocument[relative-cycles-]{relative-cycles}
\externaldocument[more-etale-]{more-etale}
\externaldocument[trace-]{trace}
\externaldocument[crystalline-]{crystalline}
\externaldocument[spaces-]{spaces}
\externaldocument[spaces-properties-]{spaces-properties}
\externaldocument[spaces-morphisms-]{spaces-morphisms}
\externaldocument[decent-spaces-]{decent-spaces}
\externaldocument[spaces-cohomology-]{spaces-cohomology}
\externaldocument[spaces-limits-]{spaces-limits}
\externaldocument[spaces-divisors-]{spaces-divisors}
\externaldocument[spaces-over-fields-]{spaces-over-fields}
\externaldocument[spaces-topologies-]{spaces-topologies}
\externaldocument[spaces-descent-]{spaces-descent}
\externaldocument[spaces-perfect-]{spaces-perfect}
\externaldocument[spaces-more-morphisms-]{spaces-more-morphisms}
\externaldocument[spaces-flat-]{spaces-flat}
\externaldocument[spaces-groupoids-]{spaces-groupoids}
\externaldocument[spaces-more-groupoids-]{spaces-more-groupoids}
\externaldocument[bootstrap-]{bootstrap}
\externaldocument[spaces-pushouts-]{spaces-pushouts}
\externaldocument[spaces-chow-]{spaces-chow}
\externaldocument[groupoids-quotients-]{groupoids-quotients}
\externaldocument[spaces-more-cohomology-]{spaces-more-cohomology}
\externaldocument[spaces-simplicial-]{spaces-simplicial}
\externaldocument[spaces-duality-]{spaces-duality}
\externaldocument[formal-spaces-]{formal-spaces}
\externaldocument[restricted-]{restricted}
\externaldocument[spaces-resolve-]{spaces-resolve}
\externaldocument[formal-defos-]{formal-defos}
\externaldocument[defos-]{defos}
\externaldocument[cotangent-]{cotangent}
\externaldocument[examples-defos-]{examples-defos}
\externaldocument[algebraic-]{algebraic}
\externaldocument[examples-stacks-]{examples-stacks}
\externaldocument[stacks-sheaves-]{stacks-sheaves}
\externaldocument[criteria-]{criteria}
\externaldocument[artin-]{artin}
\externaldocument[quot-]{quot}
\externaldocument[stacks-properties-]{stacks-properties}
\externaldocument[stacks-morphisms-]{stacks-morphisms}
\externaldocument[stacks-limits-]{stacks-limits}
\externaldocument[stacks-cohomology-]{stacks-cohomology}
\externaldocument[stacks-perfect-]{stacks-perfect}
\externaldocument[stacks-introduction-]{stacks-introduction}
\externaldocument[stacks-more-morphisms-]{stacks-more-morphisms}
\externaldocument[stacks-geometry-]{stacks-geometry}
\externaldocument[moduli-]{moduli}
\externaldocument[moduli-curves-]{moduli-curves}
\externaldocument[examples-]{examples}
\externaldocument[exercises-]{exercises}
\externaldocument[guide-]{guide}
\externaldocument[desirables-]{desirables}
\externaldocument[coding-]{coding}
\externaldocument[obsolete-]{obsolete}
\externaldocument[fdl-]{fdl}
\externaldocument[index-]{index}

% Theorem environments.
%
\theoremstyle{plain}
\newtheorem{theorem}[subsection]{Theorem}
\newtheorem{proposition}[subsection]{Proposition}
\newtheorem{lemma}[subsection]{Lemma}

\theoremstyle{definition}
\newtheorem{definition}[subsection]{Definition}
\newtheorem{example}[subsection]{Example}
\newtheorem{exercise}[subsection]{Exercise}
\newtheorem{situation}[subsection]{Situation}

\theoremstyle{remark}
\newtheorem{remark}[subsection]{Remark}
\newtheorem{remarks}[subsection]{Remarks}

\numberwithin{equation}{subsection}

% Macros
%
\def\lim{\mathop{\mathrm{lim}}\nolimits}
\def\colim{\mathop{\mathrm{colim}}\nolimits}
\def\Spec{\mathop{\mathrm{Spec}}}
\def\Hom{\mathop{\mathrm{Hom}}\nolimits}
\def\Ext{\mathop{\mathrm{Ext}}\nolimits}
\def\SheafHom{\mathop{\mathcal{H}\!\mathit{om}}\nolimits}
\def\SheafExt{\mathop{\mathcal{E}\!\mathit{xt}}\nolimits}
\def\Sch{\mathit{Sch}}
\def\Mor{\mathop{\mathrm{Mor}}\nolimits}
\def\Ob{\mathop{\mathrm{Ob}}\nolimits}
\def\Sh{\mathop{\mathit{Sh}}\nolimits}
\def\NL{\mathop{N\!L}\nolimits}
\def\CH{\mathop{\mathrm{CH}}\nolimits}
\def\proetale{{pro\text{-}\acute{e}tale}}
\def\etale{{\acute{e}tale}}
\def\QCoh{\mathit{QCoh}}
\def\Ker{\mathop{\mathrm{Ker}}}
\def\Im{\mathop{\mathrm{Im}}}
\def\Coker{\mathop{\mathrm{Coker}}}
\def\Coim{\mathop{\mathrm{Coim}}}

% Boxtimes
%
\DeclareMathSymbol{\boxtimes}{\mathbin}{AMSa}{"02}

%
% Macros for moduli stacks/spaces
%
\def\QCohstack{\mathcal{QC}\!\mathit{oh}}
\def\Cohstack{\mathcal{C}\!\mathit{oh}}
\def\Spacesstack{\mathcal{S}\!\mathit{paces}}
\def\Quotfunctor{\mathrm{Quot}}
\def\Hilbfunctor{\mathrm{Hilb}}
\def\Curvesstack{\mathcal{C}\!\mathit{urves}}
\def\Polarizedstack{\mathcal{P}\!\mathit{olarized}}
\def\Complexesstack{\mathcal{C}\!\mathit{omplexes}}
% \Pic is the operator that assigns to X its picard group, usage \Pic(X)
% \Picardstack_{X/B} denotes the Picard stack of X over B
% \Picardfunctor_{X/B} denotes the Picard functor of X over B
\def\Pic{\mathop{\mathrm{Pic}}\nolimits}
\def\Picardstack{\mathcal{P}\!\mathit{ic}}
\def\Picardfunctor{\mathrm{Pic}}
\def\Deformationcategory{\mathcal{D}\!\mathit{ef}}

%Dani's additions
\usepackage{graphicx} %figures


\begin{document}

\title{Algebraic Geometry}
\maketitle

\phantomsection
\label{section-phantom}
\hfill
\href{http://github.com/danimalabares/stack}{github.com/danimalabares/stack}

\tableofcontents

\section{Sheaves}
\label{section-sheaves}

For a definition of presheaf,
see Categories, Definition \ref{categories-definition-presheaf}.

\begin{definition}
\label{definition-sheaf}
Let $X$ be a topological space.
\begin{enumerate}
\item A {\it sheaf $\mathcal{F}$ of sets on $X$} is a presheaf
of sets which satisfies the following additional property: Given
any open covering $U = \bigcup_{i \in I} U_i$ and any collection
of sections $s_i \in \mathcal{F}(U_i)$, $i \in I$ such that
$\forall i, j\in I$
$$
s_i|_{U_i \cap U_j} = s_j|_{U_i \cap U_j}
$$
there exists a unique section $s \in \mathcal{F}(U)$ such that
$s_i = s|_{U_i}$ for all $i \in I$.
\item A {\it morphism of sheaves of sets} is simply a
morphism of presheaves of sets.
\item The category of sheaves of sets on $X$ is denoted
$\Sh(X)$.
\end{enumerate}
\end{definition}


\medskip\noindent
Let $X$ be a topological space. Let $x \in X$ be a point.
Let $\mathcal{F}$ be a presheaf of sets on $X$.
The {\it stalk of $\mathcal{F}$ at $x$} is the set
$$
\mathcal{F}_x
=
\colim_{x\in U} \mathcal{F}(U)
$$
where the colimit is over the set of open neighbourhoods
$U$ of $x$ in $X$. The set of open neighbourhoods is
partially ordered by (reverse) inclusion:
We say $U \geq U' \Leftrightarrow U \subset U'$.
The transition maps in the system are
given by the restriction maps of $\mathcal{F}$.
See Categories, Section \ref{categories-section-posets-limits}
for notation and terminology regarding (co)limits over systems.
Note that the colimit is a directed colimit.
Thus it is easy to describe $\mathcal{F}_x$. Namely,
$$
\mathcal{F}_x
=
\{
(U, s)
\mid
x\in U, s\in \mathcal{F}(U)
\}/\sim
$$
with equivalence relation given by $(U, s) \sim (U', s')$ if and only if
there exists an open $U'' \subset U \cap U'$ with $x \in U''$ and
$s|_{U''} = s'|_{U''}$. Given a pair $(U, s)$ we sometimes denote
$s_x$ the element of $\mathcal{F}_x$ corresponding to the equivalence
class of $(U, x)$. We sometimes use the phrase
``image of $s$ in $\mathcal{F}_x$'' to denote $s_x$.
For example, given two pairs $(U, s)$ and $(U', s')$ we sometimes
say ``$s$ is equal to $s'$ in $\mathcal{F}_x$'' to indicate
that $s_x = s'_x$. Other authors use the terminology
``germ of $s$ at $x$''.

\section{Abelian sheaves}
\label{section-abelian-sheaves}

The following may be used to define the ideal
sheaf of a variety:

\begin{lemma}
\label{lemma-sheaves-valued-on-groups-have-kernels}
Let $X$ be a topological space and
$\mathcal{F}$ and $\mathcal{G}$ be sheaves over $X$
with values on $\mathit{Grp}$.
For every morphism of sheaves
$f:\mathcal{F}\to \mathcal{G}$,
\begin{align*}
\Ker f: \mathit{Open}_X^{\text{op}} &\longrightarrow \mathit{Sets} \\
U &\longmapsto \Ker f(U)\\
(i:V \to U)&\longmapsto 
\substack{
\Ker f(i):\Ker f(U) \to \Ker f(V)\\
x \mapsto x}
\end{align*}
is a sheaf over $X$.
\end{lemma}

\begin{proof}
First observe that the correspondence
on morphisms is well-defined. Indeed, 
$\Ker f(U)\subset \mathcal{F}(U) \subset \mathcal{F}(V)$ 
when $V \subset U$, and we just apply $f(U)$ 
to notice that $\Ker f(V) \subset \Ker f(U)$.

To see this is a presheaf notice it is obvious
that the identity is mapped to the identity
by definition of the correspondence of morphisms.
It is also obvious that composition is preserved.

To see it is a sheaf consider an open
cover $U_i$ of $U$, and elements $x_i \in \Ker f(U_i)$.
Then use the property of $\mathcal{F}$ being
a sheaf to reconstruct an element $x \in \mathcal{F}(U)$,
whose image under $f$ will be mapped to the
identity element of $\mathcal{G}(U)$ because
it does so in every point of the cover of $U$.
Thus $x$ is in $\Ker f(U)$ as desired.
\end{proof}

\medskip\noindent
More formally,

\begin{definition}
\label{definition-abelian-presheaves}
Let $X$ be a topological space.
\begin{enumerate}
\item A {\it presheaf of abelian groups on $X$} or an
{\it abelian presheaf over $X$}
is a presheaf of sets $\mathcal{F}$ such that for each open
$U \subset X$ the set $\mathcal{F}(U)$ is endowed with
the structure of an abelian group, and such that all restriction
maps $\rho^U_V$ are homomorphisms of abelian groups, see
Lemma \ref{lemma-abelian-presheaves} above.
\item A {\it morphism of abelian presheaves over $X$}
$\varphi : \mathcal{F} \to \mathcal{G}$ is a morphism of presheaves
of sets which induces
a homomorphism of abelian groups $\mathcal{F}(U) \to \mathcal{G}(U)$
for every open $U \subset X$.
\item The category of presheaves of abelian groups on $X$ is denoted
$\textit{PAb}(X)$.
\end{enumerate}
\end{definition}

\medskip\noindent
\begin{definition}
\label{definition-abelian-sheaf}
Let $X$ be a topological space.
\begin{enumerate}
\item An {\it abelian sheaf on $X$} or
{\it sheaf of abelian groups on $X$}
is an abelian presheaf on $X$ such that the underlying presheaf of
sets is a sheaf.
\item The category of sheaves of abelian groups
is denoted $\textit{Ab}(X)$.
\end{enumerate}
\end{definition}

\noindent
Let $X$ be a topological space.
In the case of an abelian presheaf $\mathcal{F}$ the sheaf
condition with regards to an open covering $U = \bigcup U_i$
is often expressed by saying that the complex of abelian groups
$$
0 \to \mathcal{F}(U)
\to \prod\nolimits_i \mathcal{F}(U_i)
\to \prod\nolimits_{(i_0, i_1)} \mathcal{F}(U_{i_0} \cap U_{i_1})
$$
is exact. The first map is the usual one, whereas the second
maps the element $(s_i)_{i \in I}$ to the element
$$
(
s_{i_0}|_{U_{i_0} \cap U_{i_1}} -
s_{i_1}|_{U_{i_0} \cap U_{i_1}}
)_{(i_0, i_1)}
\in \prod\nolimits_{(i_0, i_1)} \mathcal{F}(U_{i_0} \cap U_{i_1})
$$

In fact, the notion of kernel of a sheaf
is not really defined as I did in the beginning of this section,
but in the next one, along with several other
important things.

\section{The abelian category of sheaves of modules}
\label{section-kernels}

\noindent
I guess that the reason to introduce
coherent sheaves is not the search for an
abelian category, after all.
Looks like the pathologies avoided
by the definition of coherence are not
so obvious---something like ``wildly infinitely generated''.

\medskip\noindent
Let $(X, \mathcal{O}_X)$ be a ringed space, see
Sheaves, Definition \ref{sheaves-definition-ringed-space}.
Let $\mathcal{F}$, $\mathcal{G}$ be sheaves of $\mathcal{O}_X$-modules, see
Sheaves, Definition \ref{sheaves-definition-sheaf-modules}.
Let $\varphi, \psi : \mathcal{F} \to \mathcal{G}$
be morphisms of sheaves of $\mathcal{O}_X$-modules.
We define $\varphi + \psi : \mathcal{F} \to \mathcal{G}$
to be the map which on each open $U \subset X$ is the
sum of the maps induced by $\varphi$, $\psi$. This is
clearly again a map of sheaves of $\mathcal{O}_X$-modules.
It is also clear that composition of maps of
$\mathcal{O}_X$-modules is bilinear with respect to this
addition. Thus $\textit{Mod}(\mathcal{O}_X)$ is a pre-additive
category, see Homology, Definition \ref{homology-definition-preadditive}.

\medskip\noindent
We will denote $0$ the sheaf of $\mathcal{O}_X$-modules
which has constant value $\{0\}$ for all open $U \subset X$.
Clearly this is both a final and an initial object of
$\textit{Mod}(\mathcal{O}_X)$. Given a morphism
of $\mathcal{O}_X$-modules $\varphi : \mathcal{F} \to \mathcal{G}$
the following are equivalent:
(a) $\varphi$ is zero, (b) $\varphi$ factors through $0$,
(c) $\varphi$ is zero on sections over each open $U$, and
(d) $\varphi_x = 0$ for all $x \in X$. See
Sheaves, Lemma \ref{sheaves-lemma-points-exactness}.

\medskip\noindent
Moreover, given a pair
$\mathcal{F}$, $\mathcal{G}$ of sheaves of $\mathcal{O}_X$-modules
we may define the direct sum as
$$
\mathcal{F} \oplus \mathcal{G} = \mathcal{F} \times \mathcal{G}
$$
with obvious maps $(i, j, p, q)$ as in Homology, Definition
\ref{homology-definition-direct-sum}. Thus $\textit{Mod}(\mathcal{O}_X)$
is an additive category, see
Homology, Definition \ref{homology-definition-additive-category}.

\medskip\noindent
Let $\varphi : \mathcal{F} \to \mathcal{G}$ be a morphism
of $\mathcal{O}_X$-modules. We may define $\Ker(\varphi)$
to be the subsheaf of $\mathcal{F}$ with sections
$$
\Ker(\varphi)(U) =
\{ s \in \mathcal{F}(U) \mid \varphi(s) = 0 \text{ in } \mathcal{G}(U)\}
$$
for all open $U \subset X$. It is easy to see that this is indeed
a kernel in the category of $\mathcal{O}_X$-modules. In other words,
a morphism $\alpha : \mathcal{H} \to \mathcal{F}$ factors
through $\Ker(\varphi)$ if and only if $\varphi \circ \alpha = 0$.
Moreover, on the level of stalks we have
$\Ker(\varphi)_x = \Ker(\varphi_x)$.

\medskip\noindent
On the other hand, we define
$\Coker(\varphi)$ as the sheaf of $\mathcal{O}_X$-modules
associated to the presheaf of $\mathcal{O}_X$-modules defined
by the rule
$$
U
\longmapsto
\Coker(\mathcal{F}(U)\to \mathcal{G}(U)) =
\mathcal{G}(U)/\varphi(\mathcal{F}(U)).
$$
Since taking stalks commutes with taking sheafification, see
Sheaves, Lemma \ref{sheaves-lemma-stalk-sheafification} we
see that $\Coker(\varphi)_x = \Coker(\varphi_x)$.
Thus the map $\mathcal{G} \to \Coker(\varphi)$ is surjective
(as a map of sheaves of sets),
see Sheaves, Section \ref{sheaves-section-exactness-points}.
To show that this is a cokernel, note that if
$\beta : \mathcal{G} \to \mathcal{H}$ is a morphism of $\mathcal{O}_X$-modules
such that $\beta \circ \varphi$ is zero, then you get for every
open $U \subset X$ a map induced by $\beta$ from
$\mathcal{G}(U)/\varphi(\mathcal{F}(U))$ into $\mathcal{H}(U)$.
By the universal property of sheafification (see
Sheaves, Lemma \ref{sheaves-lemma-sheafification-presheaf-modules})
we obtain a canonical map $\Coker(\varphi) \to \mathcal{H}$
such that the original $\beta$ is equal to
the composition
$\mathcal{G} \to \Coker(\varphi) \to \mathcal{H}$.
The morphism $\Coker(\varphi) \to \mathcal{H}$ is unique
because of the surjectivity mentioned above.

\begin{lemma}
\label{lemma-abelian}
Let $(X, \mathcal{O}_X)$ be a ringed space. The category
$\textit{Mod}(\mathcal{O}_X)$ is an abelian category. Moreover
a complex
$$
\mathcal{F} \to \mathcal{G} \to \mathcal{H}
$$
is exact at $\mathcal{G}$ if and only if for all $x \in X$ the
complex
$$
\mathcal{F}_x \to \mathcal{G}_x \to \mathcal{H}_x
$$
is exact at $\mathcal{G}_x$.
\end{lemma}

\begin{proof}
By Homology, Definition \ref{homology-definition-abelian-category}
we have to show that image and coimage agree. By Sheaves,
Lemma \ref{sheaves-lemma-points-exactness} it is enough to show
that image and coimage have the same stalk at every $x \in X$.
By the constructions of kernels and cokernels above these stalks
are the coimage and image in the categories of $\mathcal{O}_{X, x}$-modules.
Thus we get the result from the fact that the category of modules
over a ring is abelian.
\end{proof}

\noindent
Actually the category $\textit{Mod}(\mathcal{O}_X)$ has many more properties.
Here are two constructions we can do.
\begin{enumerate}
\item Given any set $I$ and for each $i \in I$ a $\mathcal{O}_X$-module
we can form the product
$$
\prod\nolimits_{i \in I} \mathcal{F}_i
$$
which is the sheaf that associates to each open $U$ the
product of the modules $\mathcal{F}_i(U)$. This is also the
categorical product, as in
Categories, Definition \ref{categories-definition-product}.
\item Given any set $I$ and for each $i \in I$ a $\mathcal{O}_X$-module
we can form the direct sum
$$
\bigoplus\nolimits_{i \in I} \mathcal{F}_i
$$
which is the {\it sheafification} of the presheaf
that associates to each open $U$ the
direct sum of the modules $\mathcal{F}_i(U)$.
This is also the categorical coproduct, as in
Categories, Definition \ref{categories-definition-coproduct}.
To see this you use the universal property of sheafification.
\end{enumerate}
Using these we conclude that all limits and colimits exist in
$\textit{Mod}(\mathcal{O}_X)$.

\begin{lemma}
\label{lemma-limits-colimits}
Let $(X, \mathcal{O}_X)$ be a ringed space.
\begin{enumerate}
\item All limits exist in $\textit{Mod}(\mathcal{O}_X)$.
Limits are the same as the corresponding limits of presheaves of
$\mathcal{O}_X$-modules (i.e., commute with taking
sections over opens).
\item All colimits exist in $\textit{Mod}(\mathcal{O}_X)$.
Colimits are the sheafification of the corresponding colimit in
the category of presheaves. Taking colimits commutes with taking
stalks.
\item Filtered colimits are exact.
\item Finite direct sums are the same as the corresponding
finite direct sums of presheaves of $\mathcal{O}_X$-modules.
\end{enumerate}
\end{lemma}

\begin{proof}
As $\textit{Mod}(\mathcal{O}_X)$ is abelian (Lemma \ref{lemma-abelian})
it has all finite limits and colimits
(Homology, Lemma \ref{homology-lemma-colimit-abelian-category}).
Thus the existence of limits and colimits and their description follows from
the existence of products and coproducts and their description
(see discussion above) and
Categories, Lemmas \ref{categories-lemma-limits-products-equalizers} and
\ref{categories-lemma-colimits-coproducts-coequalizers}.
Since sheafification commutes with taking stalks we see that
colimits commute with taking stalks. Part (3) signifies that given
a system $0 \to \mathcal{F}_i \to \mathcal{G}_i \to \mathcal{H}_i \to 0$
of exact sequences of $\mathcal{O}_X$-modules over a directed set $I$
the sequence $0 \to \colim \mathcal{F}_i \to \colim \mathcal{G}_i \to
\colim \mathcal{H}_i \to 0$ is exact as well. Since we can check
exactness on stalks (Lemma \ref{lemma-abelian}) this follows from the case
of modules which is
Algebra, Lemma \ref{algebra-lemma-directed-colimit-exact}.
We omit the proof of (4).
\end{proof}

\noindent
The existence of limits and colimits
allows us to consider exactness properties of
functors defined on the category of $\mathcal{O}$-modules
in terms of limits and colimits, as in
Categories, Section \ref{categories-section-exact-functor}.
See Homology, Lemma \ref{homology-lemma-exact-functor} for a
description of exactness
properties in terms of short exact sequences.

\begin{lemma}
\label{lemma-exactness-pushforward-pullback}
Let $f : (X, \mathcal{O}_X) \to (Y, \mathcal{O}_Y)$
be a morphism of ringed spaces.
\begin{enumerate}
\item The functor
$f_* : \textit{Mod}(\mathcal{O}_X) \to \textit{Mod}(\mathcal{O}_Y)$
is left exact. In fact it commutes with all limits.
\item The functor
$f^* : \textit{Mod}(\mathcal{O}_Y) \to \textit{Mod}(\mathcal{O}_X)$
is right exact. In fact it commutes with all colimits.
\item Pullback $f^{-1} : \textit{Ab}(Y) \to \textit{Ab}(X)$
on abelian sheaves is exact.
\end{enumerate}
\end{lemma}

\begin{proof}
Parts (1) and (2) hold because $(f^*, f_*)$ is an adjoint pair
of functors, see
Sheaves, Lemma \ref{sheaves-lemma-adjoint-pullback-pushforward-modules}
and
Categories, Section \ref{categories-section-adjoint}.
Part (3) holds because exactness can be checked on stalks
(Lemma \ref{lemma-abelian})
and the description of stalks of the pullback, see
Sheaves, Lemma \ref{sheaves-lemma-pullback-abelian-stalk}.
\end{proof}

\begin{lemma}
\label{lemma-j-shriek-exact}
Let $j : U \to X$ be an open immersion of topological spaces.
The functor $j_! : \textit{Ab}(U) \to \textit{Ab}(X)$
is exact.
\end{lemma}

\begin{proof}
Follows from the description of stalks
given in Sheaves, Lemma \ref{sheaves-lemma-j-shriek-abelian}.
\end{proof}

\begin{lemma}
\label{lemma-section-direct-sum-quasi-compact}
Let $(X, \mathcal{O}_X)$ be a ringed space.
Let $I$ be a set. For $i \in I$, let $\mathcal{F}_i$
be a sheaf of $\mathcal{O}_X$-modules.
For $U \subset X$ quasi-compact open the map
$$
\bigoplus\nolimits_{i \in I} \mathcal{F}_i(U)
\longrightarrow
\left(\bigoplus\nolimits_{i \in I} \mathcal{F}_i\right)(U)
$$
is bijective.
\end{lemma}

\begin{proof}
If $s$ is an element of the right hand side, then
there exists an open covering $U = \bigcup_{j \in J} U_j$
such that $s|_{U_j}$ is a finite sum
$\sum_{i \in I_j} s_{ji}$ with $s_{ji} \in \mathcal{F}_i(U_j)$.
Because $U$ is quasi-compact we may assume that the
covering is finite, i.e., that $J$ is finite.
Then $I' = \bigcup_{j \in J} I_j$ is a finite subset of
$I$. Clearly, $s$ is a section of the subsheaf
$\bigoplus_{i \in I'} \mathcal{F}_i$. The result follows
from the fact that for a finite direct sum sheafification
is not needed, see Lemma \ref{lemma-limits-colimits} above.
\end{proof}






\section{Tensor product of sheaves}
\label{section-tensor product of sheaves}

Here's my unexpected encounter with the definition of tensor product of sheaves.
It's not the ``fiber is tensor product of fibers'' construction, but actually
just some notion of ``change of ring'' sheaf that ends up being adjoint to
some ``restriction'' sheaf. The setting is a mapping of presheaves {\it of
rings} over a space $X$… (I think the usual definition is this one taking
$\mathcal{O}_1$ as the other presheaf we want to tensor).

Immediately after introducing this notion there's the definition of sheaf, then
stalks, abelian sheaves, some other notions like an ``algebraic structure'' and
then tensor product will be defined after sheafification---because the following
definition is in general not a sheaf.

Furthermore, I add that Vakil leaves it as an exercise to define the tensor
product of two $\mathcal{O}_X$ modules (with a hint of defining the presheaf
tensor product and sheafifying), which makes me think that after all it {\it is}
just the intuitive definition. Before diving in, also by Vakil (Exercise 26.K):
the stalk of the tensor product is the tensor product of the stalks.

\medskip\noindent

Suppose that $\mathcal{O}_1 \to \mathcal{O}_2$ is a
morphism of presheaves of rings on $X$. In this case,
if $\mathcal{F}$ is a presheaf of $\mathcal{O}_2$-modules
then we can think of $\mathcal{F}$ as a presheaf of
$\mathcal{O}_1$-modules by using the composition
$$
\mathcal{O}_1 \times \mathcal{F}
\to
\mathcal{O}_2 \times \mathcal{F}
\to
\mathcal{F}.
$$
We sometimes denote this by $\mathcal{F}_{\mathcal{O}_1}$
to indicate the restriction of rings. We call this
the {\it restriction of $\mathcal{F}$}. We obtain the
restriction functor
$$
\textit{PMod}(\mathcal{O}_2)
\longrightarrow
\textit{PMod}(\mathcal{O}_1)
$$

\medskip\noindent
On the other hand, given a presheaf of $\mathcal{O}_1$-modules
$\mathcal{G}$
we can construct a presheaf of $\mathcal{O}_2$-modules
$\mathcal{O}_2 \otimes_{p, \mathcal{O}_1} \mathcal{G}$
by the rule
$$
\left(\mathcal{O}_2 \otimes_{p, \mathcal{O}_1} \mathcal{G}\right)(U)
=
\mathcal{O}_2(U) \otimes_{\mathcal{O}_1(U)} \mathcal{G}(U)
$$
The index $p$ stands for ``presheaf'' and not ``point''.
This presheaf is called the tensor product presheaf. We obtain
the {\it change of rings} functor
$$
\textit{PMod}(\mathcal{O}_1)
\longrightarrow
\textit{PMod}(\mathcal{O}_2)
$$

\begin{lemma}
\label{lemma-adjointness-tensor-restrict-presheaves}
With $X$, $\mathcal{O}_1$, $\mathcal{O}_2$, $\mathcal{F}$ and
$\mathcal{G}$ as above there exists a canonical bijection
$$
\Hom_{\mathcal{O}_1}(\mathcal{G}, \mathcal{F}_{\mathcal{O}_1})
=
\Hom_{\mathcal{O}_2}(
\mathcal{O}_2 \otimes_{p, \mathcal{O}_1} \mathcal{G},
\mathcal{F}
)
$$
In other words, the restriction and change of rings functors
are adjoint to each other.
\end{lemma}

\begin{proof}
This follows from the fact that for a ring map
$A \to B$ the restriction functor and the change
of ring functor are adjoint to each other.
\end{proof}

\subsection{Tipologia dos feixes}
\label{subsection-tipologia-dos-feixes}

\begin{definition}
\label{definition-tipologia-dos-feixes}
A sheaf of $\mathcal{A}$-modules $\mathcal{F}$ over a sheaf of rings 
$\mathcal{A}$ (on a topological space $X$) is called
\begin{itemize}
\item 
\end{itemize}
\end{definition}

\section{Locally ringed spaces}
\label{section-locally-ringed-spaces}

\section{Morphisms}
\label{section-morphisms}

Induced map on rings, etc.

\section{Dominant morphisms}
\label{section-dominant}

\noindent
The definition of a morphism of schemes being dominant is a little
different from what you might expect if you are used to the notion
of a dominant morphism of varieties.

\begin{definition}
\label{definition-dominant}
A morphism $f : X \to S$ of schemes is called {\it dominant} if the
image of $f$ is a dense subset of $S$.
\end{definition}

\section{Morphisms of finite type}
\label{section-finite-type}

\noindent
Recall that a ring map $R \to A$ is said to be of finite type if
$A$ is isomorphic to a quotient of $R[x_1, \ldots, x_n]$ as an $R$-algebra, see
Algebra, Definition \ref{algebra-definition-finite-type}.

\begin{definition}
\label{definition-finite-type}
Let $f : X \to S$ be a morphism of schemes.
\begin{enumerate}
\item We say that $f$ is of {\it finite type at $x \in X$} if
there exists an affine open neighbourhood $\Spec(A) = U \subset X$
of $x$ and an affine open $\Spec(R) = V \subset S$
with $f(U) \subset V$ such that the induced ring map
$R \to A$ is of finite type.
\item We say that $f$ is {\it locally of finite type} if it is
of finite type at every point of $X$.
\item We say that $f$ is of {\it finite type} if it is locally of
finite type and quasi-compact.
\end{enumerate}
\end{definition}

\section{Flat morphisms}
\label{section-flat-morphisms}

Epiphany:
in the category of commutative rings
pushout is tensor product.
So think of $\text{Spec}$ as a functor
from $\mathit{CRing}^{\text{op}}$ to $\Sch$,
then pushout goes to pullback,
and what's an example of a pullback?
Fibre!
So, the coordinate ring of a fiber
is essentially given by the residue field
at the point that parametrizes it!
(tensored with the coordinate ring
of the deformation space).

There is a lot of information on Stacks Project about flatness.
It looks like the heart of the concept is 
captured in the commutative-algebraic notion of preserving
exact sequences:

\begin{definition}
\label{definition-flat}
Let $R$ be a ring.
\begin{enumerate}
\item An $R$-module $M$ is called {\it flat} if whenever
$N_1 \to N_2 \to N_3$ is an exact sequence of $R$-modules
the sequence $M \otimes_R N_1 \to M \otimes_R N_2 \to M \otimes_R N_3$
is exact as well.
\item An $R$-module $M$ is called {\it faithfully flat} if the
complex of $R$-modules
$N_1 \to N_2 \to N_3$ is exact if and only if
the sequence $M \otimes_R N_1 \to M \otimes_R N_2 \to M \otimes_R N_3$
is exact.
\item A ring map $R \to S$ is called {\it flat} if
$S$ is flat as an $R$-module.
\item A ring map $R \to S$ is called {\it faithfully flat} if
$S$ is faithfully flat as an $R$-module.
\end{enumerate}
\end{definition}

\medskip\noindent
Recall that a module $M$ over a ring $R$ is {\it flat} if the functor
$-\otimes_R M : \text{Mod}_R \to \text{Mod}_R$ is exact. A ring map
$R \to A$ is said to be {\it flat} if $A$ is flat as an $R$-module.

\section{Invertible modules (line bundles)}
\label{section-invertible}

\noindent
Similarly to the case of modules over rings
(More on Algebra, Section \ref{more-algebra-section-picard})
we have the following definition.

\begin{definition}
\label{definition-invertible}
Let $(X, \mathcal{O}_X)$ be a ringed space. An
{\it invertible $\mathcal{O}_X$-module} is a sheaf
of $\mathcal{O}_X$-modules $\mathcal{L}$ such that
the functor
$$
\textit{Mod}(\mathcal{O}_X) \longrightarrow \textit{Mod}(\mathcal{O}_X),\quad
\mathcal{F} \longmapsto \mathcal{L} \otimes_{\mathcal{O}_X} \mathcal{F}
$$
is an equivalence of categories. We say that $\mathcal{L}$ is
{\it trivial} if it is isomorphic as an $\mathcal{O}_X$-module
to $\mathcal{O}_X$.
\end{definition}

\noindent
Lemma \ref{lemma-invertible-is-locally-free-rank-1}
below explains the relationship with locally free modules
of rank $1$.

\begin{lemma}
\label{lemma-invertible}
Let $(X, \mathcal{O}_X)$ be a ringed space. Let $\mathcal{L}$
be an $\mathcal{O}_X$-module. Equivalent are
\begin{enumerate}
\item $\mathcal{L}$ is invertible, and
\item there exists an $\mathcal{O}_X$-module $\mathcal{N}$
such that
$\mathcal{L} \otimes_{\mathcal{O}_X} \mathcal{N} \cong \mathcal{O}_X$.
\end{enumerate}
In this case $\mathcal{L}$ is locally a direct summand of a finite free
$\mathcal{O}_X$-module and the module $\mathcal{N}$ in (2) is isomorphic to
$\SheafHom_{\mathcal{O}_X}(\mathcal{L}, \mathcal{O}_X)$.
\end{lemma}

\begin{proof}
Assume (1). Then the functor $- \otimes_{\mathcal{O}_X} \mathcal{L}$
is essentially surjective, hence there exists an $\mathcal{O}_X$-module
$\mathcal{N}$ as in (2). If (2) holds, then the functor
$- \otimes_{\mathcal{O}_X} \mathcal{N}$ is a quasi-inverse
to the functor $- \otimes_{\mathcal{O}_X} \mathcal{L}$ and
we see that (1) holds.

\medskip\noindent
Assume (1) and (2) hold. Denote
$\psi : \mathcal{L} \otimes_{\mathcal{O}_X} \mathcal{N} \to \mathcal{O}_X$
the given isomorphism. Let $x \in X$. Choose an open neighbourhood
$U$ an integer $n \geq 1$ and sections $s_i \in \mathcal{L}(U)$,
$t_i \in \mathcal{N}(U)$ such that $\psi(\sum s_i \otimes t_i) = 1$.
Consider the isomorphisms
$$
\mathcal{L}|_U \to
\mathcal{L}|_U \otimes_{\mathcal{O}_U}
\mathcal{L}|_U \otimes_{\mathcal{O}_U} \mathcal{N}|_U \to \mathcal{L}|_U
$$
where the first arrow sends $s$ to $\sum s_i \otimes s \otimes t_i$
and the second arrow sends $s \otimes s' \otimes t$ to $\psi(s' \otimes t)s$.
We conclude that $s \mapsto \sum \psi(s \otimes t_i)s_i$ is
an automorphism of $\mathcal{L}|_U$. This automorphism factors as
$$
\mathcal{L}|_U \to \mathcal{O}_U^{\oplus n} \to \mathcal{L}|_U
$$
where the first arrow is given by
$s \mapsto (\psi(s \otimes t_1), \ldots, \psi(s \otimes t_n))$
and the second arrow by $(a_1, \ldots, a_n) \mapsto \sum a_i s_i$.
In this way we conclude that $\mathcal{L}|_U$ is a direct summand
of a finite free $\mathcal{O}_U$-module.

\medskip\noindent
Assume (1) and (2) hold. Consider the evaluation map
$$
\mathcal{L} \otimes_{\mathcal{O}_X}
\SheafHom_{\mathcal{O}_X}(\mathcal{L}, \mathcal{O}_X)
\longrightarrow \mathcal{O}_X
$$
To finish the proof of the lemma
we will show this is an isomorphism by checking it induces
isomorphisms on stalks. Let $x \in X$.
Since we know (by the previous paragraph)
that $\mathcal{L}$ is a finitely presented
$\mathcal{O}_X$-module
we can use Lemma \ref{lemma-stalk-internal-hom}
to see that it suffices to show that
$$
\mathcal{L}_x \otimes_{\mathcal{O}_{X, x}}
\Hom_{\mathcal{O}_{X, x}}(\mathcal{L}_x, \mathcal{O}_{X, x})
\longrightarrow \mathcal{O}_{X, x}
$$
is an isomorphism. Since
$\mathcal{L}_x \otimes_{\mathcal{O}_{X, x}} \mathcal{N}_x =
(\mathcal{L} \otimes_{\mathcal{O}_X} \mathcal{N})_x =
\mathcal{O}_{X, x}$ (Lemma \ref{lemma-stalk-tensor-product})
the desired result follows from
More on Algebra, Lemma \ref{more-algebra-lemma-invertible}.
\end{proof}

\begin{lemma}
\label{lemma-pullback-invertible}
Let $f : (X, \mathcal{O}_X) \to (Y, \mathcal{O}_Y)$ be a
morphism of ringed spaces. The pullback $f^*\mathcal{L}$ of an
invertible $\mathcal{O}_Y$-module is invertible.
\end{lemma}

\begin{proof}
By Lemma \ref{lemma-invertible}
there exists an $\mathcal{O}_Y$-module $\mathcal{N}$ such that
$\mathcal{L} \otimes_{\mathcal{O}_Y} \mathcal{N} \cong \mathcal{O}_Y$.
Pulling back we get
$f^*\mathcal{L} \otimes_{\mathcal{O}_X} f^*\mathcal{N} \cong \mathcal{O}_X$
by Lemma \ref{lemma-tensor-product-pullback}.
Thus $f^*\mathcal{L}$ is invertible by Lemma \ref{lemma-invertible}.
\end{proof}

\begin{lemma}
\label{lemma-invertible-is-locally-free-rank-1}
Let $(X, \mathcal{O}_X)$ be a ringed space. Any locally free
$\mathcal{O}_X$-module of rank $1$ is invertible.
If all stalks $\mathcal{O}_{X, x}$ are local rings, then
the converse holds as well (but in general this is not the case).
\end{lemma}

\begin{proof}
The parenthetical statement follows by considering a one point
space $X$ with sheaf of rings $\mathcal{O}_X$ given by a ring $R$.
Then invertible $\mathcal{O}_X$-modules correspond to invertible
$R$-modules, hence as soon as $\Pic(R)$ is not the trivial group,
then we get an example.

\medskip\noindent
Assume $\mathcal{L}$ is locally free of rank $1$ and consider the
evaluation map
$$
\mathcal{L} \otimes_{\mathcal{O}_X}
\SheafHom_{\mathcal{O}_X}(\mathcal{L}, \mathcal{O}_X)
\longrightarrow \mathcal{O}_X
$$
Looking over an open covering trivialization $\mathcal{L}$, we see
that this map is an isomorphism. Hence $\mathcal{L}$ is invertible
by Lemma \ref{lemma-invertible}.

\medskip\noindent
Assume all stalks $\mathcal{O}_{X, x}$ are local rings and $\mathcal{L}$
invertible. In the proof of Lemma \ref{lemma-invertible}
we have seen that $\mathcal{L}_x$ is an invertible
$\mathcal{O}_{X, x}$-module for all $x \in X$. Since
$\mathcal{O}_{X, x}$ is local, we see that
$\mathcal{L}_x \cong \mathcal{O}_{X, x}$
(More on Algebra, Section \ref{more-algebra-section-picard}).
Since $\mathcal{L}$ is of finite presentation by
Lemma \ref{lemma-invertible} we conclude that $\mathcal{L}$
is locally free of rank $1$ by
Lemma \ref{lemma-finite-presentation-stalk-free}.
\end{proof}

Now I introduce some of the properties of line bundles, Cartier divisors and so
on.

\begin{lemma}
\label{lemma-Cartier-effective-and-invertible-sheaf}
The ideal sheaf of an effective Cartier divisor (a subscheme locally defined by
the vanishing of a single function) is an invertible sheaf.
\end{lemma}

\begin{proof}
We just need to check that the generator of the ideal sheaf at any affine set is
not a zerodivisor. This follows from the ideal sheaf exact sequence, which
implies that multplication by the generator is injective:
 $$
\xymatrix{
0\ar[r]&I\cong A\ar[r]&A\ar[r]&A/I\ar[r]&0
}
$$
\end{proof}


\section{Ampleness}
\label{section-ampleness}

First is this lemma that comes from modules.tex. I think these sets $X_s$ are
the base points of the bundle. Because look: image of $s$ just means consider
the section $s$ of the line bundle as a germ near $x$. Now a line bundle is a
locally free rank-1 $\mathcal{O}_X$-module, so its sections, like $s$, may be
multiplied by germs of functions in the maximal ring $\mathfrak{m}_x$, i.e. the
functions that vanish at $x$. So $X_s$ is the vanishing locus of the section
$s$. If $s(x)\neq 0$, obviously $s
\not\in\mathfrak{m}_x\mathcal{L}_x$, so $x\in X_s$. Conversely, I would like to
show that if $s(x)=0$ then  $s\in\mathfrak{m}_x\mathcal{L}_x$ but I'm not sure
how. It's like: a vector field with a zero can be multiplied by a function that
vanishes at the point, sure, but what's this function?

\begin{lemma}
\label{lemma-s-open}
From modules.tex.
\begin{slogan}
A (local) trivialisation of a linebundle
is the same as a (local) nonvanishing section.
\end{slogan}
Let $X$ be a ringed space. Assume that each stalk $\mathcal{O}_{X, x}$
is a local ring with maximal ideal $\mathfrak m_x$.
Let $\mathcal{L}$ be an invertible $\mathcal{O}_X$-module.
For any section $s \in \Gamma(X, \mathcal{L})$ the set
$$
X_s = \{x \in X \mid \text{image }s \not\in \mathfrak m_x\mathcal{L}_x\}
$$
is open in $X$. The map $s : \mathcal{O}_{X_s} \to \mathcal{L}|_{X_s}$
is an isomorphism, and there exists a section $s'$
of $\mathcal{L}^{\otimes -1}$ over $X_s$ such that $s' (s|_{X_s}) = 1$.
\end{lemma}

\begin{proof}
Suppose $x \in X_s$.
We have an isomorphism
$$
\mathcal{L}_x \otimes_{\mathcal{O}_{X, x}} (\mathcal{L}^{\otimes -1})_x
\longrightarrow
\mathcal{O}_{X, x}
$$
by Lemma \ref{lemma-constructions-invertible}.
Both $\mathcal{L}_x$ and $(\mathcal{L}^{\otimes -1})_x$
are free $\mathcal{O}_{X, x}$-modules of rank $1$. We conclude
from Algebra, Nakayama's Lemma \ref{algebra-lemma-NAK} that
$s_x$ is a basis for $\mathcal{L}_x$. Hence there exists
a basis element $t_x \in (\mathcal{L}^{\otimes -1})_x$
such that $s_x \otimes t_x$ maps to $1$.
Choose an open neighbourhood $U$ of
$x$ such that $t_x$ comes from a section $t$
of $\mathcal{L}^{\otimes -1}$ over $U$ and such that
$s \otimes t$ maps to $1 \in \mathcal{O}_X(U)$.
Clearly, for every $x' \in U$ we see that $s$ generates
the module $\mathcal{L}_{x'}$. Hence $U \subset X_s$.
This proves that $X_s$ is open. Moreover, the section
$t$ constructed over $U$ above is unique, and hence
these glue to give the section $s'$ of the lemma.
\end{proof}

Recall from Modules, Lemma \ref{modules-lemma-s-open}
that given an invertible sheaf $\mathcal{L}$ on a locally ringed
space $X$, and given a global section $s$ of $\mathcal{L}$
the set $X_s = \{x \in X \mid s \not \in \mathfrak m_x\mathcal{L}_x\}$
is open. A general remark is that
$X_s \cap X_{s'} = X_{ss'}$, where $ss'$ denote
the section $s \otimes s' \in \Gamma(X, \mathcal{L} \otimes \mathcal{L}')$.

\begin{definition}
\label{definition-ample}
\begin{reference}
\cite[II Definition 4.5.3]{EGA}
\end{reference}
Let $X$ be a scheme.
Let $\mathcal{L}$ be an invertible $\mathcal{O}_X$-module.
We say $\mathcal{L}$ is {\it ample} if
\begin{enumerate}
\item $X$ is quasi-compact, and
\item for every $x \in X$ there exists an $n \geq 1$
and $s \in \Gamma(X, \mathcal{L}^{\otimes n})$ such
that $x \in X_s$ and $X_s$ is affine.
\end{enumerate}
\end{definition}

\begin{exercise}
\label{exercise-ample-bundle-on-K3}
Let $L$ be an ample bundle on a K3 surface $M$. Prove that
$\mathcal{L}^{\otimes 2}$ is globally generated (that is, for each $x\in M$
there exsits a section $h \in H^{0}(L^{\otimes 2})$ which does not vanish in
$x$).
\end{exercise}

\begin{proof}
This just asks that the $n$ in Definition \ref{definition-ample} is $2$ for all
$x\in X$. Because, again, $x\in X_s$ means that $s(x)\neq 0$ because if it was,
then we could somehow write $s$ as a product of a vanishing function on
$\mathfrak{m}_x$ and a local frame of $\Gamma(X,\mathcal{L})$. But I guess for
the exercise do this: a line bundle is {\it ample} if there is $n$ such that the
canonical embedding (cf Lemma \ref{lemma-map-into-proj}) is an embedding, i.e.
that $\mathcal{L}^{\otimes n}$ is {\it very ample}. (Interestingly, the notion
very ampleness is defined in morphisms.tex.)
\end{proof}

Now we pass to the part where ampleness gives you an {\bf open immersion} to
some projective space. Because, it's only very ampleness that gives an
embedding, right? (Actually I think here in stacks project there are no
embeddings but closed immersions.)

\begin{definition}
\label{definition-gamma-star}
From modules.tex. Let $(X, \mathcal{O}_X)$ be a ringed space.
Given an invertible sheaf $\mathcal{L}$ on $X$ we define
the {\it associated graded ring} to be
$$
\Gamma_*(X, \mathcal{L})
=
\bigoplus\nolimits_{n \geq 0} \Gamma(X, \mathcal{L}^{\otimes n})
$$
Given a sheaf of $\mathcal{O}_X$-modules $\mathcal{F}$ we set
$$
\Gamma_*(X, \mathcal{L}, \mathcal{F})
=
\bigoplus\nolimits_{n \in \mathbf{Z}} \Gamma(X,
\mathcal{F} \otimes_{\mathcal{O}_X} \mathcal{L}^{\otimes n})
$$
which we think of as a graded $\Gamma_*(X, \mathcal{L})$-module.
\end{definition}

\begin{lemma}
\label{lemma-map-into-proj}
Let $X$ be a scheme.
Let $\mathcal{L}$ be an invertible $\mathcal{O}_X$-module.
Set $S = \Gamma_*(X, \mathcal{L})$ as a graded ring.
If every point of $X$ is contained in one of the
open subschemes $X_s$, for some $s \in S_{+}$ homogeneous, then
there is a canonical morphism of schemes
$$
f : X \longrightarrow Y = \text{Proj}(S),
$$
to the homogeneous spectrum of $S$ (see
Constructions, Section \ref{constructions-section-proj}).
This morphism has the following properties
\begin{enumerate}
\item $f^{-1}(D_{+}(s)) = X_s$ for any $s \in S_{+}$ homogeneous,
\item there are $\mathcal{O}_X$-module maps
$f^*\mathcal{O}_Y(n) \to \mathcal{L}^{\otimes n}$
compatible with multiplication maps, see
Constructions, Equation (\ref{constructions-equation-multiply}),
\item the composition
$S_n \to \Gamma(Y, \mathcal{O}_Y(n)) \to \Gamma(X, \mathcal{L}^{\otimes n})$
is the identity map, and
\item for every $x \in X$ there is an integer $d \geq 1$
and an open neighbourhood $U \subset X$ of $x$
such that $f^*\mathcal{O}_Y(dn)|_U \to \mathcal{L}^{\otimes dn}|_U$
is an isomorphism for all $n \in \mathbf{Z}$.
\end{enumerate}
\end{lemma}

\begin{proof}
Denote $\psi : S \to \Gamma_*(X, \mathcal{L})$ the identity map.
We are going to use the triple
$(U(\psi), r_{\mathcal{L}, \psi}, \theta)$ of
Constructions, Lemma \ref{constructions-lemma-invertible-map-into-proj}.
By assumption the open subscheme $U(\psi)$ of equals $X$. Hence
$r_{\mathcal{L}, \psi} : U(\psi) \to Y$ is defined on all of $X$.
We set $f = r_{\mathcal{L}, \psi}$.
The maps in part (2) are the components of $\theta$.
Part (3) follows from condition (2) in the lemma cited above.
Part (1) follows from (3) combined with condition (1) in the lemma
cited above. Part (4) follows from the last statement in
Constructions, Lemma \ref{constructions-lemma-invertible-map-into-proj}
since the map $\alpha$ mentioned there is an isomorphism.
\end{proof}

\begin{lemma}
\label{lemma-map-into-proj-quasi-compact}
Let $X$ be a scheme. Let $\mathcal{L}$ be an invertible $\mathcal{O}_X$-module.
Set $S = \Gamma_*(X, \mathcal{L})$.
Assume (a) every point of $X$ is contained in one of the
open subschemes $X_s$, for some $s \in S_{+}$ homogeneous,
and (b) $X$ is quasi-compact. Then the canonical morphism of schemes
$f : X \longrightarrow \text{Proj}(S)$ of Lemma \ref{lemma-map-into-proj}
above is quasi-compact with dense image.
\end{lemma}

\begin{proof}
To prove $f$ is quasi-compact it suffices to show that $f^{-1}(D_{+}(s))$
is quasi-compact for any $s \in S_{+}$ homogeneous. Write
$X = \bigcup_{i = 1, \ldots, n} X_i$ as a finite union of
affine opens. By Lemma \ref{lemma-affine-cap-s-open} each intersection
$X_s \cap X_i$ is affine. Hence $X_s = \bigcup_{i = 1, \ldots, n} X_s \cap X_i$
is quasi-compact. Assume that the image of $f$ is not dense to get
a contradiction. Then, since the opens $D_+(s)$ with $s \in S_+$ homogeneous
form a basis for the topology on $\text{Proj}(S)$, we can find such
an $s$ with $D_+(s) \not = \emptyset$ and $f(X) \cap D_+(s) = \emptyset$.
By Lemma \ref{lemma-map-into-proj}
this means $X_s = \emptyset$. By Lemma \ref{lemma-invert-s-sections}
this means that a power $s^n$ is the zero section of
$\mathcal{L}^{\otimes n\deg(s)}$.
This in turn means that $D_+(s) = \emptyset$ which is the
desired contradiction.
\end{proof}

\begin{lemma}
\label{lemma-ample-immersion-into-proj}
Let $X$ be a scheme. Let $\mathcal{L}$ be an invertible $\mathcal{O}_X$-module.
Set $S = \Gamma_*(X, \mathcal{L})$.
Assume $\mathcal{L}$ is ample. Then the canonical morphism of schemes
$f : X \longrightarrow \text{Proj}(S)$ of Lemma \ref{lemma-map-into-proj}
is an open immersion with dense image.
\end{lemma}

\begin{proof}
By Lemma \ref{lemma-affine-s-opens-cover-quasi-separated} we see
that $X$ is quasi-separated. Choose finitely many
$s_1, \ldots, s_n \in S_{+}$ homogeneous
such that $X_{s_i}$ are affine, and $X = \bigcup X_{s_i}$.
Say $s_i$ has degree $d_i$. The inverse image of
$D_{+}(s_i)$ under $f$ is $X_{s_i}$, see Lemma \ref{lemma-map-into-proj}.
By Lemma \ref{lemma-invert-s-sections} the ring map
$$
(S^{(d_i)})_{(s_i)} = \Gamma(D_{+}(s_i), \mathcal{O}_{\text{Proj}(S)})
\longrightarrow
\Gamma(X_{s_i}, \mathcal{O}_X)
$$
is an isomorphism. Hence $f$ induces an isomorphism
$X_{s_i} \to D_{+}(s_i)$. Thus $f$ is an isomorphism of $X$ onto the open
subscheme $\bigcup_{i = 1, \ldots, n} D_{+}(s_i)$ of $\text{Proj}(S)$.
The image is dense by Lemma \ref{lemma-map-into-proj-quasi-compact}.
\end{proof}

\begin{lemma}
\label{lemma-open-in-proj-ample}
Let $X$ be a scheme.
Let $S$ be a graded ring. Assume $X$ is quasi-compact,
and assume there exists an open immersion
$$
j : X \longrightarrow Y = \text{Proj}(S).
$$
Then $j^*\mathcal{O}_Y(d)$ is an invertible ample sheaf
for some $d > 0$.
\end{lemma}

\begin{proof}
This is Constructions, Lemma \ref{constructions-lemma-ample-on-proj}.
\end{proof}

\begin{proposition}
\label{proposition-characterize-ample}
Let $X$ be a quasi-compact scheme.
Let $\mathcal{L}$ be an invertible sheaf on $X$.
Set $S = \Gamma_*(X, \mathcal{L})$.
The following are equivalent:
\begin{enumerate}
\item
\label{item-ample}
$\mathcal{L}$ is ample,
\item
\label{item-immersion}
the open sets $X_s$, with $s \in S_{+}$ homogeneous,
cover $X$ and the associated morphism $X \to \text{Proj}(S)$
is an open immersion,
\item
\label{item-s-basis}
the open sets $X_s$, with $s \in S_{+}$ homogeneous,
form a basis for the topology of $X$,
\item
\label{item-s-affine-basis}
the open sets $X_s$, with $s \in S_{+}$ homogeneous,
which are affine form a basis for the topology of $X$,
\item
\label{item-qc-gg}
for every quasi-coherent sheaf $\mathcal{F}$ on $X$
the sum of the images of the canonical maps
$$
\Gamma(X, \mathcal{F} \otimes_{\mathcal{O}_X} \mathcal{L}^{\otimes n})
\otimes_{\mathbf{Z}} \mathcal{L}^{\otimes -n}
\longrightarrow
\mathcal{F}
$$
with $n \geq 1$ equals $\mathcal{F}$,
\item
\label{item-qc-i-gg}
same property as (\ref{item-qc-gg}) with $\mathcal{F}$
ranging over all quasi-coherent sheaves of ideals,
\item
\label{item-c-gg}
$X$ is quasi-separated and
for every quasi-coherent sheaf $\mathcal{F}$ of finite type on $X$
there exists an integer $n_0$ such that
$\mathcal{F} \otimes_{\mathcal{O}_X} \mathcal{L}^{\otimes n}$
is globally generated for all $n \geq n_0$,
\item
\label{item-c-q}
$X$ is quasi-separated and
for every quasi-coherent sheaf $\mathcal{F}$ of finite type on $X$
there exist integers $n > 0$, $k \geq 0$ such that
$\mathcal{F}$ is a quotient of a direct sum of $k$ copies of
$\mathcal{L}^{\otimes - n}$, and
\item
\label{item-c-i-q}
same as in (\ref{item-c-q}) with $\mathcal{F}$ ranging over all
sheaves of ideals of finite type on $X$.
\end{enumerate}
\end{proposition}

\begin{proof}
Lemma \ref{lemma-ample-immersion-into-proj} is
(\ref{item-ample}) $\Rightarrow$ (\ref{item-immersion}).
Lemmas \ref{lemma-ample-power-ample} and \ref{lemma-open-in-proj-ample}
provide the implication
(\ref{item-ample}) $\Leftarrow$ (\ref{item-immersion}).
The implications (\ref{item-immersion}) $\Rightarrow$
(\ref{item-s-affine-basis}) $\Rightarrow$ (\ref{item-s-basis})
are clear from Constructions, Section \ref{constructions-section-proj}.
Lemma \ref{lemma-affine-s-opens} is
(\ref{item-s-basis}) $\Rightarrow$ (\ref{item-ample}).
Thus we see that the first 4 conditions are all equivalent.

\medskip\noindent
Assume the equivalent conditions (1) -- (4).
Note that in particular $X$ is separated (as an open
subscheme of the separated scheme $\text{Proj}(S)$).
Let $\mathcal{F}$ be a quasi-coherent sheaf on $X$.
Choose $s \in S_{+}$ homogeneous such that $X_s$ is affine.
We claim that any section $m \in \Gamma(X_s, \mathcal{F})$
is in the image of one of the maps displayed in
(\ref{item-qc-gg}) above. This will imply (\ref{item-qc-gg})
since these affines $X_s$ cover $X$.
Namely, by Lemma \ref{lemma-invert-s-sections} we may write
$m$ as the image of $m' \otimes s^{-n}$ for some
$n \geq 1$, some
$m' \in \Gamma(X, \mathcal{F} \otimes \mathcal{L}^{\otimes n})$.
This proves the claim.

\medskip\noindent
Clearly (\ref{item-qc-gg}) $\Rightarrow$ (\ref{item-qc-i-gg}).
Let us assume (\ref{item-qc-i-gg}) and prove $\mathcal{L}$ is
ample. Pick $x \in X$. Let $U \subset X$ be an affine open
which contains $x$. Set $Z = X \setminus U$. We may think of
$Z$ as a reduced closed subscheme, see
Schemes, Section \ref{schemes-section-reduced}.
Let $\mathcal{I} \subset \mathcal{O}_X$ be the quasi-coherent
sheaf of ideals corresponding to the closed subscheme $Z$.
By assumption (\ref{item-qc-i-gg}), there exists an $n \geq 1$ and a section
$s \in \Gamma(X, \mathcal{I} \otimes \mathcal{L}^{\otimes n})$
such that $s$ does not vanish at $x$ (more precisely such that
$s \not \in \mathfrak m_x \mathcal{I}_x \otimes \mathcal{L}_x^{\otimes n}$).
We may think of $s$ as a section of $\mathcal{L}^{\otimes n}$.
Since it clearly vanishes along $Z$ we see that
$X_s \subset U$. Hence $X_s$ is affine, see
Lemma \ref{lemma-affine-cap-s-open}.
This proves that $\mathcal{L}$ is ample.
At this point we have proved that (1) -- (6) are equivalent.

\medskip\noindent
Assume the equivalent conditions (1) -- (6). In the following
we will use the fact that the tensor product of two sheaves of
modules which are globally generated is globally generated without
further mention (see
Modules, Lemma \ref{modules-lemma-tensor-product-globally-generated}).
By (1) we can find elements $s_i \in S_{d_i}$ with $d_i \geq 1$
such that $X = \bigcup_{i = 1, \ldots, n} X_{s_i}$.
Set $d = d_1\ldots d_n$. It follows that $\mathcal{L}^{\otimes d}$
is globally generated by
$$
s_1^{d/d_1}, \ldots, s_n^{d/d_n}.
$$
This means that if $\mathcal{L}^{\otimes j}$ is globally generated
then so is $\mathcal{L}^{\otimes j + dn}$ for all $n \geq 0$.
Fix a $j \in \{0, \ldots, d - 1\}$. For any point $x \in X$ there
exists an $n \geq 1$ and a global section $s$ of $\mathcal{L}^{j + dn}$
which does not vanish at $x$, as follows from (\ref{item-qc-gg}) applied
to $\mathcal{F} = \mathcal{L}^{\otimes j}$ and ample invertible
sheaf $\mathcal{L}^{\otimes d}$. Since $X$ is quasi-compact there
we may find a finite list of integers $n_i$ and global sections
$s_i$ of $\mathcal{L}^{\otimes j + dn_i}$ which do not vanish at any point
of $X$. Since $\mathcal{L}^{\otimes d}$ is globally generated this means that
$\mathcal{L}^{\otimes j + dn}$ is globally generated where $n = \max\{n_i\}$.
Since we proved this for every congruence class mod $d$ we
conclude that there exists an $n_0 = n_0(\mathcal{L})$ such that
$\mathcal{L}^{\otimes n}$ is globally generated for all $n \geq n_0$.
At this point we see that if $\mathcal{F}$ is globally generated then
so is $\mathcal{F} \otimes \mathcal{L}^{\otimes n}$ for all
$n \geq n_0$.

\medskip\noindent
We continue to assume the equivalent conditions (1) -- (6).
Let $\mathcal{F}$ be a quasi-coherent
sheaf of $\mathcal{O}_X$-modules of finite type.
Denote $\mathcal{F}_n \subset \mathcal{F}$ the image of the canonical
map of (\ref{item-qc-gg}). By construction
$\mathcal{F}_n \otimes \mathcal{L}^{\otimes n}$ is
globally generated. By (\ref{item-qc-gg}) we see
$\mathcal{F}$ is the sum of the subsheaves $\mathcal{F}_n$,
$n \geq 1$. By
Modules, Lemma \ref{modules-lemma-finite-type-quasi-compact-colimit}
we see that $\mathcal{F} = \sum_{n = 1, \ldots, N} \mathcal{F}_n$
for some $N \geq 1$. It follows that
$\mathcal{F} \otimes \mathcal{L}^{\otimes n}$ is globally
generated whenever $n \geq N + n_0(\mathcal{L})$ with $n_0(\mathcal{L})$
as above. We conclude that (1) -- (6) implies (\ref{item-c-gg}).

\medskip\noindent
Assume (\ref{item-c-gg}). Let $\mathcal{F}$ be a quasi-coherent
sheaf of $\mathcal{O}_X$-modules of finite type.
By (\ref{item-c-gg}) there exists an integer $n \geq 1$ such that
the canonical map
$$
\Gamma(X, \mathcal{F} \otimes_{\mathcal{O}_X} \mathcal{L}^{\otimes n})
\otimes_{\mathbf{Z}} \mathcal{L}^{\otimes -n}
\longrightarrow
\mathcal{F}
$$
is surjective. Let $I$ be the set of finite subsets of
$\Gamma(X, \mathcal{F} \otimes_{\mathcal{O}_X} \mathcal{L}^{\otimes n})$
partially ordered by inclusion. Then $I$ is a directed partially ordered set.
For $i = \{s_1, \ldots, s_{r(i)}\}$ let $\mathcal{F}_i \subset \mathcal{F}$
be the image of the map
$$
\bigoplus\nolimits_{j = 1, \ldots, r(i)} \mathcal{L}^{\otimes -n}
\longrightarrow
\mathcal{F}
$$
which is multiplication by $s_j$ on the $j$th factor. The surjectivity above
implies that $\mathcal{F} = \colim_{i \in I} \mathcal{F}_i$.
Hence Modules, Lemma \ref{modules-lemma-finite-type-quasi-compact-colimit}
applies and we conclude that
$\mathcal{F} = \mathcal{F}_i$ for some $i$.
Hence we have proved (\ref{item-c-q}). In other words,
(\ref{item-c-gg}) $\Rightarrow$ (\ref{item-c-q}).

\medskip\noindent
The implication (\ref{item-c-q}) $\Rightarrow$ (\ref{item-c-i-q}) is trivial.

\medskip\noindent
Finally, assume (\ref{item-c-i-q}).
Let $\mathcal{I} \subset \mathcal{O}_X$ be a quasi-coherent sheaf
of ideals. By Lemma \ref{lemma-quasi-coherent-colimit-finite-type}
(this is where we use the condition that $X$ be quasi-separated)
we see that $\mathcal{I} = \colim_\alpha I_\alpha$ with
each $I_\alpha$ quasi-coherent of finite type. Since by assumption each of
the $I_\alpha$ is a quotient of negative tensor powers of
$\mathcal{L}$ we conclude the same for $\mathcal{I}$ (but of course
without the finiteness or boundedness of the powers). Hence
we conclude that (\ref{item-c-i-q}) implies (\ref{item-qc-i-gg}).
This ends the proof of the proposition.
\end{proof}

The following proofs were used for Exercise 
\ref{complex-geometry-exercise-L-ample-implies-Lotimes2-globally-generated}.

\begin{lemma}
\label{lemma-divisor-has-sections-implies-deg-geq-0}
Let $D$ be a divisor on a complete, nonsingular curve $X$. 
If  $h^0(D)\neq 0$ then $\text{deg}D\geq0$.
\end{lemma}

\begin{proof}
If $D$ has sections, we can take the zero locus of any of its sections to
produce an effective divisor linearly equivalent to $D$. Since degree depends
only on linear equivalence and effective divisors have non negative degree.
\end{proof}

\begin{proposition}
\label{proposition-divisor-is-base-point-free-iff-dimension-condition}
\begin{reference}
\cite[IV, Proposition 3.1(a)]{hart}
\end{reference}
Let $D$ be a divisor on a complete, nonsingular curve $X$. 
Then the complete linear system has no base
points if and only if for every point $P\in X$,
$$
\dim|D-P|=\dim|D|-1
$$
\end{proposition}

\begin{proof}
To show that $D$ has no base points amounts to showing that not every section of
$D$. That is, that the injective map $0\to H^{0}(D-p)\to H^{0}(D)$ is not 
surjective.
\end{proof}

\begin{lemma}
\label{lemma-degree-of-divisor-geq-2g-implies-base-point-free}
\begin{reference}
\cite[IV, Corollary 3.2(a), Examples 1.3.3, 1.3.4]{hart}
\end{reference}
Let $D$ be a divisor on a curve $X$ of genus $g$. 
If $\text{deg}D\geq 2g$, then
$|D|$ has no base points.
\end{lemma}

\begin{proof}
First we prove that $\text{deg}D\geq 2g$ implies that $D$ and $D-P$ are  
{\it nonspecial}, i.e. that
$h^0(K-D)=0=h^0(K-(D-P))$. Then we apply Riemann-Roch to both $D$ and $D-P$ and
the fact that  $\text{deg}(D-P)=\text{deg}(D)-1$ to find that
$\dim|D-P|=\dim|D|-1$ and apply Proposition 
\ref{proposition-divisor-is-base-point-free-iff-dimension-condition}.

To prove that $D$ is nonspecial first apply Riemann-Roch to $K$ to obtain that
$\text{deg}K=2g-2$. Indeed, $h^0(0)=1$ and $h^0(K)=p_g$ by Serre duality on
$H^1(\mathcal{O}_X)$ recalling definition of genus as $h^1(\mathcal{O}_X)$. 
Then apply Riemann-Roch to both $D$ and  $K-D$ to prove
that $\text{deg}D>2g-2$ implies $\text{deg}(K-D)<0$; start with
$$
h^0(K-D)-h^0(K-(K-D))=-(h^0(D)-h^0(K-D))
$$
An analogous result will be valid for $D-P$ since its degree is also greater
than $2g-2$.

Then apply Lemma
\ref{lemma-divisor-has-sections-implies-deg-geq-0}.
\end{proof}

\section{Closed immersions of locally ringed spaces}
\label{section-closed-immersion}

\noindent
We follow our conventions introduced in
Modules, Definition \ref{modules-definition-closed-immersion}.

\begin{definition}
\label{definition-closed-immersion-locally-ringed-spaces}
Let $i : Z \to X$ be a morphism of locally ringed spaces.
We say that $i$ is a {\it closed immersion} if:
\begin{enumerate}
\item The map $i$ is a homeomorphism of $Z$ onto a closed subset of $X$.
\item The map $\mathcal{O}_X \to i_*\mathcal{O}_Z$ is surjective;
let $\mathcal{I}$ denote the kernel.
\item The $\mathcal{O}_X$-module $\mathcal{I}$
is locally generated by sections.
\end{enumerate}
\end{definition}

\begin{lemma}
\label{lemma-closed-local-target}
Let $f : Z \to X$ be a morphism of locally ringed spaces.
In order for $f$ to be a closed immersion it suffices
that there exists an open covering $X = \bigcup U_i$ such
that each $f : f^{-1}U_i \to U_i$ is a closed immersion.
\end{lemma}

\begin{proof}
Omitted.
\end{proof}

\begin{example}
\label{example-closed-subspace}
Let $X$ be a locally ringed space.
Let $\mathcal{I} \subset \mathcal{O}_X$ be a sheaf
of ideals which is locally generated by sections as a sheaf
of $\mathcal{O}_X$-modules. Let $Z$ be the support of
the sheaf of rings $\mathcal{O}_X/\mathcal{I}$.
This is a closed subset of $X$, by
Modules, Lemma \ref{modules-lemma-support-sheaf-rings-closed}.
Denote $i : Z \to X$ the inclusion map.
By Modules, Lemma \ref{modules-lemma-i-star-exact}
there is a unique sheaf of rings $\mathcal{O}_Z$ on $Z$
with $i_*\mathcal{O}_Z = \mathcal{O}_X/\mathcal{I}$.
For any $z \in Z$ the stalk $\mathcal{O}_{Z, z}$
is equal to a quotient $\mathcal{O}_{X, i(z)}/\mathcal{I}_{i(z)}$
of a local ring and nonzero, hence a local ring.
Thus $i : (Z, \mathcal{O}_Z) \to (X, \mathcal{O}_X)$ is
a closed immersion of locally ringed spaces.
\end{example}

\begin{definition}
\label{definition-closed-subspace}
Let $X$ be a locally ringed space.
Let $\mathcal{I}$ be a sheaf of ideals on $X$
which is locally generated by sections.
The locally ringed space $(Z, \mathcal{O}_Z)$
of Example \ref{example-closed-subspace} above
is the {\it closed subspace of $X$ associated to
the sheaf of ideals $\mathcal{I}$}.
\end{definition}

\begin{lemma}
\label{lemma-closed-immersion}
Let $f : X \to Y$ be a closed immersion of
locally ringed spaces. Let $\mathcal{I}$ be the
kernel of the map $\mathcal{O}_Y \to f_*\mathcal{O}_X$.
Let $i : Z \to Y$ be the closed subspace of $Y$
associated to $\mathcal{I}$.
There is a unique isomorphism $f' : X \cong Z$ of
locally ringed spaces such that $f = i \circ f'$.
\end{lemma}

\begin{proof}
Omitted.
\end{proof}

\begin{lemma}
\label{lemma-characterize-closed-subspace}
Let $X$, $Y$ be locally ringed spaces. Let
$\mathcal{I} \subset \mathcal{O}_X$ be a sheaf of ideals locally generated
by sections. Let $i : Z \to X$ be the associated closed subspace.
A morphism $f : Y \to X$ factors through $Z$ if and only if the map
$f^*\mathcal{I} \to f^*\mathcal{O}_X = \mathcal{O}_Y$
is zero. If this is the case the morphism $g : Y \to Z$
such that $f = i \circ g$ is unique.
\end{lemma}

\begin{proof}
Clearly if $f$ factors as $Y \to Z \to X$ then the map
$f^*\mathcal{I} \to \mathcal{O}_Y$ is zero. Conversely
suppose that $f^*\mathcal{I} \to \mathcal{O}_Y$ is zero.
Pick any $y \in Y$, and consider the ring map
$f^\sharp_y : \mathcal{O}_{X, f(y)} \to \mathcal{O}_{Y, y}$.
Since the composition
$\mathcal{I}_{f(y)} \to \mathcal{O}_{X, f(y)} \to \mathcal{O}_{Y, y}$
is zero by assumption and since $f^\sharp_y(1) = 1$
we see that $1 \not \in \mathcal{I}_{f(y)}$, i.e.,
$\mathcal{I}_{f(y)} \not = \mathcal{O}_{X, f(y)}$. We conclude that
$f(Y) \subset Z = \text{Supp}(\mathcal{O}_X/\mathcal{I})$.
Hence $f = i \circ g$ where $g : Y \to Z$ is continuous.
Consider the map $f^\sharp : \mathcal{O}_X \to f_*\mathcal{O}_Y$.
The assumption $f^*\mathcal{I} \to \mathcal{O}_Y$ is zero implies that
the composition $\mathcal{I} \to \mathcal{O}_X \to f_*\mathcal{O}_Y$ is
zero by adjointness of $f_*$ and $f^*$.
In other words, we obtain a morphism of sheaves of rings
$\overline{f^\sharp} : \mathcal{O}_X/\mathcal{I} \to f_*\mathcal{O}_Y$.
Note that $f_*\mathcal{O}_Y = i_*g_*\mathcal{O}_Y$ and
that $\mathcal{O}_X/\mathcal{I} = i_*\mathcal{O}_Z$.
By Sheaves, Lemma \ref{sheaves-lemma-equivalence-categories-closed-structures}
we obtain a unique morphism of sheaves of rings
$g^\sharp : \mathcal{O}_Z \to g_*\mathcal{O}_Y$ whose pushforward
under $i$ is $\overline{f^\sharp}$. We omit the verification that
$(g, g^\sharp)$ defines a morphism of locally ringed spaces
and that $f = i \circ g$ as a morphism of locally ringed spaces.
The uniqueness of $(g, g^\sharp)$ was pointed out above.
\end{proof}

\begin{lemma}
\label{lemma-restrict-map-to-closed}
Let $f : X \to Y$ be a morphism of locally ringed spaces.
Let $\mathcal{I} \subset \mathcal{O}_Y$ be a sheaf of
ideals which is locally generated by sections.
Let $i : Z \to Y$ be the closed subspace associated to the
sheaf of ideals $\mathcal{I}$.
Let $\mathcal{J}$ be the image of the map
$f^*\mathcal{I} \to f^*\mathcal{O}_Y = \mathcal{O}_X$.
Then this ideal is locally generated by sections.
Moreover, let $i' : Z' \to X$ be the associated closed
subspace of $X$. There exists a unique
morphism of locally ringed spaces $f' : Z' \to Z$ such
that the following diagram is a commutative square of
locally ringed spaces
$$
\xymatrix{
Z' \ar[d]_{f'} \ar[r]_{i'} & X \ar[d]^f \\
Z \ar[r]^{i} & Y
}
$$
Moreover, this diagram is a fibre square in the category of
locally ringed spaces.
\end{lemma}

\begin{proof}
The ideal $\mathcal{J}$ is locally generated by sections
by Modules, Lemma \ref{modules-lemma-pullback-locally-generated}.
The rest of the lemma follows from the characterization,
in Lemma \ref{lemma-characterize-closed-subspace} above,
of what it means for a morphism to factor through a closed
subspace.
\end{proof}

\medskip\noindent
Let me finish with a proof that haunted my
second semester of PhD:

\begin{lemma}
\label{lemma-ideal-sheaf-is-line-bundle-schemes}
The ideal sheaf of an irreducible codimension-1 closed subscheme of a smooth
scheme is a line bundle.
\end{lemma}

\begin{proof}
Since $M$ is codimension-1 irreducible, the ideal sheaf at every affine chart is
a codimension-1 prime ideal. This means that there aren't any nontrivial prime
ideals of $\mathcal{I}_X(\text{Spec}A):=I$. Let $f\in I$.
Since $X$ is smooth, $O_X\text{Spec}A=A$ is a UFD (why?). Then there exists an
irreducible element $g\in I$ such that $gh=f$. The ideal generated by $g$ is
prime (again because $A$ is a UFD) 
and contained in $I$, so that $I$ is principal. This shows that
$\mathcal{I}_M$ is locally principal, i.e. it is a line bundle.
\end{proof}


\section{Adjunction formulas}
\label{section-adjunction formulas}

There are several statements called adjunction formula
in different texts. All of them concern ``subvarieties'',
that is, closed embedded subschemes.

\begin{exercise}[Genus formula for a curve on a surface]
\label{exercise-genus-formula-for-curve-on-surface}
Let $C \to X$ be a closed embedded subscheme 
of dimension $1$ (as a topological space, i.e. pure dimension)
inside a smooth surface $X$.
Then $2p_a-2=(\mathcal{O}_X(C),\mathcal{O}_X(C))$.
\end{exercise}

\begin{proof}
Consider the ideal sheaf exact sequence
$$
\xymatrix{
0\ar[r]&\mathcal{O}_X(-C)\ar[r]&\mathcal{O}_X\ar[r]&\mathcal{O}_C\ar[r]&0
}
$$
This sequence splits since there is an obvious inverse morphism
to the inclusion $\mathcal{O}_X(-C)\to \mathcal{O}_X$, namely
mapping a function $f$ to 
Then $\mathcal{O}_X\cong \mathcal{O}_X(-C) \oplus \mathcal{O}_C$.
 $\chi(\mathcal{O}_X)=\chi$
\end{proof}

\section{Normalization}
\label{section-normalization}

\begin{definition}
\label{definition-normalization-X-in-Y}
Let $f : Y \to X$ be a quasi-compact and quasi-separated morphism of schemes.
Let $\mathcal{O}'$ be the integral closure of $\mathcal{O}_X$ in
$f_*\mathcal{O}_Y$. The {\it normalization of $X$ in $Y$} is the
scheme\footnote{The scheme $X'$ need not be normal, for example if
$Y = X$ and $f = \text{id}_X$, then $X' = X$.}
$$
\nu : X' = \underline{\Spec}_X(\mathcal{O}') \to X
$$
over $X$. It comes equipped with a natural factorization
$$
Y \xrightarrow{f'} X' \xrightarrow{\nu} X
$$
of the initial morphism $f$.
\end{definition}

\noindent
The factorization is the composition of the canonical morphism
$Y \to \underline{\Spec}_X(f_*\mathcal{O}_Y)$ (see
Constructions, Lemma
\ref{constructions-lemma-canonical-morphism})
and the morphism of relative spectra coming from the inclusion map
$\mathcal{O}' \to f_*\mathcal{O}_Y$. We can characterize the
normalization as follows.

\begin{lemma}
\label{lemma-characterize-normalization}
Let $f : Y \to X$ be a quasi-compact and quasi-separated morphism of schemes.
The factorization $f = \nu \circ f'$, where $\nu : X' \to X$ is the
normalization of $X$ in $Y$ is characterized by the following
two properties:
\begin{enumerate}
\item the morphism $\nu$ is integral, and
\item for any factorization $f = \pi \circ g$, with $\pi : Z \to X$
integral, there exists a commutative diagram
$$
\xymatrix{
Y \ar[d]_{f'} \ar[r]_g & Z \ar[d]^\pi \\
X' \ar[ru]^h \ar[r]^\nu & X
}
$$
for some unique morphism $h : X' \to Z$.
\end{enumerate}
Moreover, the morphism $f' : Y \to X'$ is dominant and in (2) the
morphism $h : X' \to Z$ is the normalization of $Z$ in $Y$.
\end{lemma}



\section{Reflexive sheaves}
\label{section-reflexive-sheaves}

\begin{slogan}
These are vector bundles except for a small locus.
\end{slogan}

\begin{definition}
\label{definition-reflexive}
Let $X$ be an integral locally Noetherian scheme. Let $\mathcal{F}$
be a coherent $\mathcal{O}_X$-module. The {\it reflexive hull}
of $\mathcal{F}$ is the $\mathcal{O}_X$-module
$$
\mathcal{F}^{**} = \SheafHom_{\mathcal{O}_X}(
\SheafHom_{\mathcal{O}_X}(\mathcal{F}, \mathcal{O}_X), \mathcal{O}_X)
$$
We say $\mathcal{F}$ is {\it reflexive} if the natural map
$j : \mathcal{F} \longrightarrow \mathcal{F}^{**}$
is an isomorphism.
\end{definition}

\begin{lemma}
\label{lemma-reflexive-torsion-free}
Let $X$ be an integral locally Noetherian scheme. Let $\mathcal{F}$
be a coherent $\mathcal{O}_X$-module.
\begin{enumerate}
\item If $\mathcal{F}$ is reflexive, then $\mathcal{F}$ is torsion free.
\item The map $j : \mathcal{F} \longrightarrow \mathcal{F}^{**}$
is injective if and only if $\mathcal{F}$ is torsion free.
\end{enumerate}
\end{lemma}

\begin{remark}[Talk at IMPA, 11 June]\leavevmode
\label{remark-reflexive-talk}
Torsion could also be defined so that the sheaf can inject onto its dual. In
this talk we discussed the moduli space of reflexive/torsion-free sheaves, which
turned out to be parametrized by $c_1, c_2$ and $c_3$. This was denoted by 
$R(c_1,c_2,c_3)$. Actually I think it may have been Manolache that proved the
existence of this moduli space.


Alan Muniz

Nesta palestra discutiremos a classificação de feixes reflexivos de posto dois e
seus espaços de módulos. Apresentaremos algumas ferramentas básicas usadas na
construção e determinação de tais feixes. Aplicaremos estas técnicas para o caso
de feixes com segunda classe de Chern igual a quatro, obtido recentemente em
colaboração com Marcos Jardim.
\end{remark}

\subsection{Distributions on manifolds}
\label{subsection-distributions-on-manifolds}

Here's the abstract from a talk by Marcos Jardim at Geometric Structures:

``I will revise the work done over the past 10 years with various collaborators
on distributions and foliations on 3-folds, especially on the projective space,
with a focus on properties of the tangent sheaf and singular scheme."

Here are two key ideas: if the distribution is codimension 1 we can write:
$$
\xymatrix{
0\ar[r]&F\ar[r]&TX\ar[r]^{\omega}&I_Z \otimes L\ar[r]&0
}
$$
where $L$ is a line bundle and $\omega \in H^{0}(\Omega_X \otimes L)$, and
$Z=\{p:\omega(p)=0\}$.

When codimension is 2 then $\mathcal{D}$ is given by a holomorphic vector field 
$\nu$: $T_p=\left<\nu(p)\right>$.

It can be encoded as an exact sequence
$$
\xymatrix{
	0\ar[r]&L\ar[r]^{\nu}&TX\ar[r]&N\ar[r]&0
}
$$
where $L$ is a line bundle and $\nu \in H^{0}(TX \otimes L^\vee)$;
  $Z=\{p | \nu(p) = 0\}$.

\begin{remark}
\label{remark-stauration}
Saturation means that $Z \subset X$ is a union of curves and points.
\end{remark}

And again, distributions are parametrized by Chern classes.

Two interesting open questions:
\begin{enumerate}
\item {\bf Conjeture.} if $\mathcal{D}$ is a codimension 1 foliation of degree
$d$ on $\mathbb{P}^3$, then $c_2(F)\leq d^2-d+1$ and bound is attained
by rational foliations of type $(1,d+1)$. (True for $d \leq 2$.)
\item {\bf Conjecture (with Pepe Seade).} $\mathcal{D}$ is a codimension 1 
foliation on a smooth projective 3-fold, then $\text{Sing}\mathcal{D}$ is connected.
\end{enumerate}

\begin{theorem}[Jardim-Muniz]
\label{theorem-jardim-muniz}
Conditions on Chern classes used to understand moduli space $R(c_1,c_2,c_3)$.
$c_2=4$ gives (?). For $c_3\leq 6$, possible ``spectrum" exists…
\end{theorem}

\section{Stability}
\label{section-stability}

{\bf Question.} What is stability?

\begin{enumerate}
\item Stable objects in an abelian category are the ``building blocks":
we can reconstruct the whole category from them.
\item  An abelian subcategory (hart) $\subset$ a triangulated
\item stability defined via stability function on $\mathcal{A}$.
\item Q. Can we reconstruct $\mathcal{T}$ from the semistable elements of
$\mathcal{A}$
\item {\bf Example.} $\mathcal{A}=\text{Coh}X$ is heart of $D^b(X)$
w/ funny function.
\item Stability condition is hart + stability function.
\item Bridgeland Stabl:= the stability conditions are a complex manifold
of complex codimension $\text{rk}\Lambda$:
$$
 \mathcal{Z}:\text{Stab}(\mathcal{T})
\longrightarrow \text{Hom}(\Lambda,\mathbb{C})
$$
\item There's a chamber structure; moduli space changes across chambers.
\item I think we typically think of vectors in $\text{Hom}(\Lambda,\mathbb{C})$ 
as Chern classes, to characterize the moduli spaces.
\item Existence: given a projective variety $X$, are there stability
conditions on $D^\text{b}(X)$? Yes for fano 3fold pic rk 1.
\item Moduli spaces: is $M_\sigma(v)$ a projective scheme? Cannot use
usual git techniques to study. A stack!
\item Picture: blue + black are walls. Q. What are $\beta$ and $\alpha$?
\item thm: bridgeland stable = gieseker stable ?
\item Q. slope stability = bridgeland stability? A. Not always.
\item DT/PT correspondance: only one wall between PT and G chambers
\medskip
\item Polynomial stability function. This is an asymptotic version of BS.
\item There are some $\rho$'s. Arrangements of $\rho_i$ are polynomial 
stability conditions on a threefold.
\item Pata-Thomas introduced stability for rank 1 objects.
Bayer compares the---wall. Q. Same for Bridge S---only one wall?
\item Recall Gieseker stability.
\medskip
\item Def. A {\it stable triple}: when
$\text{gcd}(\text{ch}_0,\text{ch}_2,\text{ch}_3)=1$, every PT stable object
comes from three conditions (missing).
\item What happens when you cross the blue wall? 
Both $\mathcal{G}$ and $\mathcal{T}$ are projective. What happens at the blue
wall?
\medskip
\item For $X$ smooth threefold with $\text{rk}\text{Pic}=1$, $\mathcal{G}(v)$,
$v=(r,0,0,-n)$, $\mathcal{G}(v)$ is a known sheaf object and
$\mathcal{T}=\emptyset$.
\item $X$ sm 3 rkpic1, Fake wall; $\mathcal{G}=\mathcal{T}$.
\item (Extra.) red circle is a wall for a weaker form of stability.
\end{enumerate}
\begin{definition}[Talk at impa]
\label{definition-slope-stability}
A rank-2 sheaf $\mathcal{F}$ is {\it semistable (stable)}if
$H^{0}(\mathcal{F}(t))=0$ for $-t \geq(>) \frac{c_1F}{2}$
\end{definition}

Compare with

\begin{definition}[moduli-curves.tex]
\label{definition-semistable}
Let $f : X \to S$ be a family of curves.
We say $f$ is a {\it semistable family of curves} if
\begin{enumerate}
\item $X \to S$ is a prestable family of curves, and
\item $X_s$ has genus $\geq 1$ and
does not have a rational tail for all $s \in S$.
\end{enumerate}
\end{definition}

\begin{itemize}
\item The twistor diagram.
$$
\xymatrix{
&  \ell_\infty \ar@{.>}[dr]\ar[dl]\\
\mathbb{C}P^{2} \ar@{^{(}->}[rr]& &
\mathbb{C}P^{3}\ar[dd]_{\substack{\text{twistor} \\ \text{map}}}\\ \\
\mathbb{C}^2\ar@{^{(}->}[u]
\mathbb{R}^4 \ar[u]\ar[rr]& &S^4
}
$$
\item Instantons are solutions to some Yang-Mills solution.
\item Expository paper of Donaldson arXiv:2205.08639
\item ADHM construction, 1978. The first appearance of Algebraic Geometry in
Mathematical Physics. See Hitchin-Kobayashi correspondence.
\item Donaldson: ``unashamedly computational''.
\item Expository paper by Simon.
\item Take bundle $(E,\nabla)$ with an anti-self dual (ADS) 
connection on $S^4$ and pullback to $\mathbb{C}P^{3}$ via the
twistor map
$$
\tau[x:y:z:w]=[x+jy:z+jw] \qquad \text{note: $\tau^{-1}(p)=\mathbb{C}P^{1}$}
$$
Then:
\begin{itemize}
\item Restriction to fibres are trivial.
\item Invariant under anti holomorphic involution (check this formula!) 
$[x:y:z:w] \mapsto [-y:x:-w:z]$.
\item $\mathcal{E}$ is also an instanton bundle.
\item Penrose Transform: $H^1(\mathcal{E}(-2)) \cong \Ker \Delta=0$ where
$\Delta$ is a Laplacian.
\item Definition of instanton sheaf on $\mathbb{P}^3$ via $c_1(E)=0$ and some
vanishing of cohomologies.
\item Passage from [differential equations? algebraic geometry?] 
to linear algebra: via {\it monads}. The point is that instanton sheaf is
equivalent to  ``$E$ being the cohomology of a linear monad''; theorem by Horps
in the 60's, and is the main tool used by ADHM. Indeed, ADHM equations come from
the cohomology sequences of the so-called monads.
\item Mathamatical inst bund:= locally free instanton sheaves.
\end{itemize}
\end{itemize}

\noindent
{\bf Properties.}
\begin{itemize}
\item The only instanton of rank 1 on $\mathbb{P}^3$ is the structure sheaf.
\item non-trivial rank 2 locally free instanton sheaf ois  $\mu_0stable$
\item double dual is locally free and also instanton
\item non-trivial rank 2 instanton sheaaf is Fieseker stable
therefore it makes sance fo define moduls space of instanton sheaves as an open
subset of $\mathcal{M}(c)=\mathcal{G}(k,0,2,0)$.
\end{itemize}

Then studied the irreducibility (Tikhomirov) and smoothness (Jardim-Verbitsky,
2014. Uses ``3rd hyperkähler quotient'') of $\mathcal{I}(c)$, the moduli
space of rank 2 locally gree instanton sheaves of charge $c.$ But nobody likes
this results; want new proofs.

In contrast, $\mathcal{M}(c)$ of rank 2-instanton sheaves of charge $c$ is not
irreducible in general!
$\mathcal{M}(1)$ and $\mathcal{M}(2)$ are irreducible, $\mathcal{M}(3)$ has
exactly 2 irreducible components of dimension 21;
 $\mathcal{M}(4)$ has 4 irredicuble components: 
the locally free is irreducible, and the other 3 that intersect the closure of
the locally
free, $\overline{\mathcal{I}(c)}$. 3 components of dimension 29 and one of dimension 32

\begin{remark}
\label{remark-}
In general the $\mu$ moduli space is not projective, but the Gieseker is.
\end{remark}


\noindent
{\bf Is $\mathcal{M}(c)$ connected?} Use $\mathbb{C}^*$ action. True for $c \leq
4$; every component intersects $\overline{\mathcal{I}(c)}$ in this range!

\begin{definition}[Elementary transformation]
\label{definition-elementary-transformation}
$F$ of rank 2 locally free instanton, $Q$ of rank 0 instanton with 1 dimensional
sheaf $h^p Q(-2)$ for $p = 0,1$. So in this conditions if we have an epimorphism
 $$
F \overset{\varphi,\text{ surj}}{\to}\} Q
$$
we get that $\Ker \varphi$ is an instanton.
\end{definition}

So  we might be interested in
$$
E \hookrightarrow E^{* *}\xrightarrow{\text{surj.}} Q_E
$$
Let's have a look at $\mathcal{M}(3)=\overline{\mathcal{I}(3)}\cup
\overline{C(0,1,3}$. Consider a generic line bundle of degree 0 on a planar
cubic (cwhich is encoded in that we have intersection of something of cimension
1 and something of simension 3), $L inn \text{Pic}^0(C)$ so we have an
epimorphism
$$
\xymatrix{
0\ar[r]&E\ar[r]&\mathcal{O}_{\mathbb{P}^3}\oplus
\mathcal{O}_{\mathbb{P}^3}\ar[r]&(i_* L)\ar[r]&0
}
$$
yielding a bundle $E$ as previously outlined.

So we are studying the components via pushforwards of sheaves on complete
intersection curves inside $\mathbb{P}^3$ 

``when we do alementary transformation of rational (not rationl?) we get
something on the boundary of locally free.''

\medskip\noindent
{\bf Perverse instanton sheaves.} Like an instnaton sheaf in monad description
but with some restrictions on the cohomologies. This leads to definition of {\it
$0$-rimensional instanton}, a perverse instanton such that $\mathcal{H}^0=0$.

\medskip\noindent
{\bf Instantons and quivers.}

\medskip\noindent
{\bf Framed instantons.} Fix a line $j:L\to \mathbb{P}^3$ and $E$ a perverse
sheaf; a {\it framing} at $L$ is an isomorphis [missing].

Apply GIT to the ADHM data to construct a moduli space 
$$
\mathcal{P}(r,c)=\mathcal{V}(r,c)^{\text{st}}/\!/\text{GL}(V_c)
$$
and it will follow that $\mathbb{P}(?,?)$ is connected.

\medskip\noindent
Considerations on quaternionic spaces lead to generalization of what has been
discussed so far to higher dimensions. I.e. an {\it instanton sheaf} on
$\mathbb{P}^n$ is… [other cohomological conditions]
sheaf

Why are instantons interesting? They are the simplest; may provide examples for
Bridgeland stability.

In Kuznetsov (2012) and Faenzi (2014) introduced {\it rank 2 instanton bundles
on Fano 3-folds}. An {\it instanton bundle} on $X$ is a $\mu$-stable … and some
Chern class is called the {\it charge}. An {\it
instanton sheaf} (introduced by Marcos-Gaia) is ….

 ``Since we are imposing $\mu$-stability on the defintion we can consider the
moduli $\mathcal{I}_X(c) \subset \mathcal{G}_X(2,-r_X,c,0)$''.

There's also monad representations as an ingredient.


\medskip\noindent
Here are two questions that invite us to join the instanton fever:

\medskip\noindent
{\bf Task 1.} Construct rank 2-instanton sheaves that do not deform into locally
free ones, and obtain the new irreducible components of
$\mathcal{G}(2,-r_X,c,0)$.

\medskip\noindent
{\bf Task 2.} Nonlocally instanton sheaves that can be deformed into non-locally
free ones: the {\it instanton boundary}
$\overline{\mathcal{I}(c)}/\mathcal{I}(X)$[formula right?]


\medskip\noindent
{\bf Recipe to construct your own instanton.}
\begin{enumerate}
\item (Make a bunch of instantons.) Find an appropriate curve to 
use Serre correspondence to find some rank 2 instanton sheaf:
$$
\xymatrix{
0\ar[r]&\mathcal{O}_{\mathbb{P}^3}(-1)\ar[r]&\underbrace{E}_{\exists
}\ar[r]&\mathcal{I}_{\sqcup \text{lines}}\ar[r]&0
}
$$
where $E$ is a locally free instanton of charge = the number of lines $-1$.

\item (Is your family of instantons generic?) You look at the $\text{Ext}$s.
``Therefore, the family of instantons only defines a locally closed subset
within a generically smooth irreducible component of $\mathcal{G}$''.

\item ``Find suitable rank 0 intranton sheaves to perform an elementary
transformations on the examples obtained in Step 1.'' But they are non-locally
free.

\item Now I have my instantons, I know they are locally free: but how to prove
that the elementary transformations deform to locally free ones? Looks like
the challenge is to prove that the deformation is locally free.
\end{enumerate}

In the papers by the group there are several particular cases when the
deformations are locally free. But they don't have a general result that would
work for 3-folds.

\medskip\noindent
{\bf What you need to call a thing an instanton.}
\begin{itemize}
\item Minimal cohomology possible; try to kill as much cohomology as you can.
\item Fixing $c_1$ (which may determine other Chern classes).
\item Some stability condition like $\mu$-stability or quiver stability. Here is
an example that is not $\mu$-semistable: 
$T\mathbb{P}^3(-1)\oplus\Omega_{\mathbb{P}^3}(1)$
\item Whenever possible, look for a monadic representation. (The monadic
representation comes from ADHM --- the beginnings of this theory. And it's still
here!)
\end{itemize}

\section{Coherent sheaves}
\label{section-coherent-sheaves}

\begin{lemma}
\label{lemma-quasi-coherent-affine-cohomology-zero}
\begin{slogan}
Serre vanishing: Higher cohomology vanishes on affine schemes
for quasi-coherent modules.
\end{slogan}
Let $X$ be a scheme.
Let $\mathcal{F}$ be a quasi-coherent $\mathcal{O}_X$-module.
For any affine open $U \subset X$ we have
$H^p(U, \mathcal{F}) = 0$ for all $p > 0$.
\end{lemma}

\begin{proof}
We are going to apply
Cohomology, Lemma \ref{cohomology-lemma-cech-vanish-basis}.
As our basis $\mathcal{B}$ for the topology of $X$ we are going to use
the affine opens of $X$.
As our set $\text{Cov}$ of open coverings we are going to use the standard
open coverings of affine opens of $X$.
Next we check that conditions (1), (2) and (3) of
Cohomology, Lemma \ref{cohomology-lemma-cech-vanish-basis}
hold. Note that the intersection of standard opens in an affine is
another standard open. Hence property (1) holds.
The coverings form a cofinal system of open coverings of any element
of $\mathcal{B}$, see
Schemes, Lemma \ref{schemes-lemma-standard-open}.
Hence (2) holds.
Finally, condition (3) of the lemma follows from
Lemma \ref{lemma-cech-cohomology-quasi-coherent-trivial}.
\end{proof}

\section{Hilbert polynomial}
\label{section-Hilbert-polynomial}

\noindent
The following lemma will be made obsolete by the more general
Lemma \ref{lemma-numerical-polynomial-from-euler}.

\begin{lemma}
\label{lemma-hilbert-polynomial}
Let $k$ be a field. Let $n \geq 0$. Let $\mathcal{F}$ be a coherent sheaf
on $\mathbf{P}^n_k$. The function
$$
d \longmapsto \chi(\mathbf{P}^n_k, \mathcal{F}(d))
$$
is a polynomial.
\end{lemma}

\begin{proof}
We prove this by induction on $n$. If $n = 0$, then
$\mathbf{P}^n_k = \Spec(k)$ and $\mathcal{F}(d) = \mathcal{F}$.
Hence in this case the function is constant, i.e., a polynomial
of degree $0$. Assume $n > 0$. By
Lemma \ref{lemma-euler-characteristic-extend-base-field}
we may assume $k$ is infinite. Apply
Lemma \ref{lemma-exact-sequence-induction}.
Applying Lemma \ref{lemma-euler-characteristic-additive}
to the twisted sequences
$0 \to \mathcal{F}(d - 1) \to \mathcal{F}(d) \to i_*\mathcal{G}(d) \to 0$
we obtain
$$
\chi(\mathbf{P}^n_k, \mathcal{F}(d)) -
\chi(\mathbf{P}^n_k, \mathcal{F}(d - 1)) =
\chi(H, \mathcal{G}(d))
$$
See Remark \ref{remark-exact-sequence-induction-cohomology}.
Since $H \cong \mathbf{P}^{n - 1}_k$
by induction the right hand side is a polynomial.
The lemma is finished by noting that any function
$f : \mathbf{Z} \to \mathbf{Z}$ with the property that the map
$d \mapsto f(d) - f(d - 1)$ is a polynomial, is itself a polynomial.
We omit the proof of this fact (hint: compare with
Algebra, Lemma \ref{algebra-lemma-numerical-polynomial}).
\end{proof}

\begin{definition}
\label{definition-hilbert-polynomial}
Let $k$ be a field. Let $n \geq 0$. Let $\mathcal{F}$ be a coherent sheaf
on $\mathbf{P}^n_k$. The function
$d \mapsto \chi(\mathbf{P}^n_k, \mathcal{F}(d))$ is called the
{\it Hilbert polynomial} of $\mathcal{F}$.
\end{definition}

\noindent
The Hilbert polynomial has coefficients in $\mathbf{Q}$ and not
in general in $\mathbf{Z}$. For example the Hilbert polynomial
of $\mathcal{O}_{\mathbf{P}^n_k}$ is
$$
d \longmapsto {d + n \choose n} = \frac{d^n}{n!} + \ldots
$$
This follows from the following lemma and the fact that
$$
H^0(\mathbf{P}^n_k, \mathcal{O}_{\mathbf{P}^n_k}(d)) = k[T_0, \ldots, T_n]_d
$$
(degree $d$ part) whose dimension over $k$ is ${d + n \choose n}$.

\begin{lemma}
\label{lemma-hilbert-polynomial-H0}
Let $k$ be a field. Let $n \geq 0$. Let $\mathcal{F}$ be a coherent sheaf
on $\mathbf{P}^n_k$ with Hilbert polynomial $P \in \mathbf{Q}[t]$.
Then
$$
P(d) = \dim_k H^0(\mathbf{P}^n_k, \mathcal{F}(d))
$$
for all $d \gg 0$.
\end{lemma}

\begin{proof}
This follows from the vanishing of cohomology of high enough twists
of $\mathcal{F}$. See
Cohomology of Schemes,
Lemma \ref{coherent-lemma-coherent-projective}.
\end{proof}


\medskip\noindent
For completeness I include earlier notes
from \cite{har} on the matter.

The fact that $M$ is finitely generated is what makes the following two
definitions make sense.

\begin{definition}
\label{definition-Hilbert-function}
The {\it Hilbert function} of a finitely generated graded $S=k[x_0,\ldots,x_r]$ 
-module $M$ is
$$
H_M(d)=\dim_kM_d
$$
\end{definition}

\begin{definition}
\label{definition-sysygy}
Define $F_0$ to be the free $S$-module on the generators of $M$. Elements in the
 kernel $M_1$ of the inclusion are called {\it sysygies}. By Hilbert's basis
theorem, $M_1$ is also finitely generated, so we may choose a set of generators
and repeat this process.
\end{definition}

\begin{theorem}[Hilbert Syzygy Theorem]
\label{theorem-Hilbert-syzygy}
\begin{reference}
\cite[Theorem 1.1]{sys}
\end{reference}
Any finitely generated $S$-module $M$ has a finite graded free resolution
$$
\xymatrix{
0\ar[r]&F_m\ar[r]^{\varphi_m}&\ar[r]&F_{m-q}\ar[r]&\cdots\ar[r]&
F_1\ar[r]^{\varphi_1}&F_0
}
$$
Moreover, we may take $m\leq r+1$, the number of variables in $S$.
\end{theorem}

\begin{lemma}
\label{lemma-Hilbert-function}
Suppose that $S=k[x_0,\ldots,x_r]$ is a polynomial ring. If the graded
$S$-module $M$ has finite free resolution
$$
\xymatrix{
0\ar[r]&F_m\ar[r]^{\varphi_m}&F_{m-1}\ar\cdots[r]&F_1\ar[r]^{\varphi_1}&
\ar[r]&F_0
}
$$
with each $F_i$ a finitely generated free module,
$F_i=\bigoplus_{j}S(-a_{i,j})$, then
\begin{equation}
\label{equation-Hilbert-function}
H_M(d)=\sum_{i=0}(-1)^i\sum_{j}\binom{r+d-a_{i,j}}{r}
\end{equation}
\end{lemma}

\begin{lemma}
\label{lemma-Hilbert-function-becomes-polynomial}
There is a polynomial $P_M(d)$ called the {\it Hilbert polynomial} such that, if
$M$ has free resolution as above, then $P_M(d)=H_M(d)$ for 
$d\geq\text{max}_{i,j}\{a_{i,j}-r\}$.
\end{lemma}

\begin{proof}
When $d$ satisfies this bound then the binomial coefficients in Eq.
\ref{equation-Hilbert-function} are polynomials of degree $r$ in $d$.
\end{proof}

\begin{theorem}[Hilbert-Serre]
\label{theorem-Hilbert-Serre}
\begin{reference}
\cite[I, Theorem 7.5]{hart}
\end{reference}
Let $M$ be a finitely generated graded $S=k[x_0,\ldots,x_n]$. Then there exists
a unique polynomial $p_M$ such that $p_M(\ell)=\dim S_\ell$ for large enough
$\ell$.
\end{theorem}

\begin{definition}
\label{definition-Hilbert-polynomial}
\begin{reference}
\cite[I, p. 52]{har}
\end{reference}
The polynomial $P_M$ of Hilbert-Serre Theorem \cite{Hilbert-Serre} is the {\it
Hilbert polynomial} of the finitely generated $k[x_0,\ldots,x_n]$-module $M$.
\end{definition}

\begin{definition}
\label{definition-degree-of-projective-variety}
\begin{reference}
\cite[p. 52]{hart}
\end{reference}
If $Y\subset \mathbb{P}^n$ is an algebraic set of dimension $r$, we define the
{\it degree of $Y$} to be $r!$ times the leading coefficient of the Hilbert
polynomial of the homogeneous coordinate ring $S(Y)$.
\end{definition}

\begin{exercise}
\label{exercise-very-ample-bundle-self-intersection-is-degree-of-surface}
\begin{reference}
\cite[V, Exercise 1.2]{har}
\end{reference}
Let $H$ be a very ample divisor on the surface $X$, corresponding to a
projective embedding $X\subseteq\mathbb{P}^N$. If we write the Hilbert
polynomial of $X$ as $P(z)=\frac{1}{2}az^2+bz+c$, show that $a=H^2$,
$b=\frac{1}{2}H^2+1-\pi$, where $\pi$ is the genus of a nonsingular curve
representing $H$, and $c=1+p_a$.
\end{exercise}

\section{Nakai-Moishezon Criterion}
\label{section-Nakai-Moishezon-criterion}

\begin{theorem}[Nakai-Moishezon Criterion]
\label{theorem-Nakai-Moishezon-criterion}
\begin{reference}
\cite[V, Theorem 1.10]{hart}
\end{reference}
A divisor $D$ on the surface $X$ is ample if and only if $D^2>0$ and $D.C>0$ for
all irreducible curves $C$ in $X$.
\end{theorem}

\begin{proof}
The direct implication is easy: since $D$ is ample,  $mD$ is very ample for some
$m$, so that $m^2D^2$ is the self-intersection number of $mD$. By exercise
\ref{exercise-very-ample-bundle-self-intersection-is-degree-of-surface},
 $D^2$ is the leading coefficient of the Hilbert polynomial of $X$ as a 
subscheme of $\mathbb{P}^n$. This means that $D^2$ is twice the leading
coefficient of the Hilbert polynomial of a projective variety for large enough
$m$, so that it must be a positive number (it's the dimension of one of the
graded components of the coordinate ring of the surface).
\end{proof}

\section{Hilbert scheme}
\label{section-Hilbert-scheme}

{\bf Upshot \cite[p. 6]{HarrMorr}.} 
We wish to parametrize subschemes of a projective space (or
perhaps a more general scheme?). Since there are too many such subschemes we
restrict ourselves to schemes with a given Hilbert polynomial, since the latter
``encodes the most important numerical invariants of schemes''. The Hilbert
scheme is introduced via a theorem by Grothendieck 
as the object that represents the functor $\mathbf{Hilb}_{P,r}$ 
that maps a reduced scheme $B$ to the
set of proper flat families
$$
\xymatrix{
\mathcal{X}\ar@{^{(}->}[r]^i\ar[rd]&\mathbb{P}^r \times B \ar[d]^{\pi_B}\\
& B
}
$$
with $\mathcal{X}$ having Hilbert polynomial $P$.

\begin{theorem}[Grothendieck, '66]
\label{theorem-Grothendieck}
The functor $\mathbf{Hilb}_{P,r}$ is representable by a projective scheme
$\mathcal{H}_{P,r}$.
\end{theorem}

SEE Hilbert schemes of subschemes.

\section{Deformation theory}
\label{section-deformation-theory}

\noindent
I start by reading Stacks Project.

\medskip\noindent
The first notion is {\it thickening of ringed spaces},
which I ultimately think of as a closed subscheme
$(X,\mathcal{O}_X) \to (X',\mathcal{O}_{X'})$
with a nilpotent ideal sheaf,
which in an imprecise way means
that $\mathcal{I}_X^n=0$ for some $n$,
and in a precise way its given on sections.

\medskip\noindent
The following definition is from [lucas-defos], which in turn comes from
\cite{Sernesi-deformations}

\begin{definition}
\label{definition-deformation}
Let $X$ be an algebraic $\mathbb{C}$-scheme.
\begin{enumerate}
\item A {\it deformation} of $X$ is a Cartesian diagram $\xi$
$$
\xymatrix{
X\ar[r]\ar[d]&\mathcal{X}\ar[d]^{\pi}\\
\text{Spec}\mathbb{C}\ar[r]^s&S
}
$$
where $\pi$ is a flat surjective morphism of algebraic $\mathbb{C}$-schemes and
$S$ is connected. (Recall that flatness accounts for ``continuity''.)
\item A {\it local deformation} of $X$ is a deformation $\xi$ where
$S=\text{Spec}A$ for $A$ a noetherian local $\mathbb{C}$ algebra with residue
field $\mathbb{C}$.
\item An {\it infinitesimal deformation of $X$} is a local deformation with $A$
an artinian local $\mathbb{C}$-algebra with residue field $\mathbb{C}$. $X$ is
called {\it rigid} if all infinitesimal deformations are trivial.
\item An {\it inifinitesimal deformation of order $n$} is an infinitesimal
deformation when $S=\text{Spec}(\mathbb{C}[t]/(t^{n+1})$.
\end{enumerate}
\end{definition}

{\bf Upshot.} An interpretation of the so-called {\it dual numbers}
$k[t]/(t^2)$ (see \cite{Hartshorne-deformation}) as the tangent space of
something is thinking of Taylor polynomials: after quotienting by $t^2$ we loose
the tails of the polynomials and are left with the first derivative information
only.

So what is the deformation space? Is it a moduli space of curves, that is,
points are curves obtained by deforming the curve? Or is it the fibration whose
fibers are curves and there is a central fiber that is the original curve?

There is the following interpretation in [continued-fractions] p. 39: the space
of first order deformation classes of $X$ is $D(\mathbb{C}[t]/(t^2)$. This is
said to ````represent'' the tangent space $\mathbb{T}^1_X$ of the hypothetical
deformation space of $X$''. (I put double quotations because the word
``represent'' is quoted in the original text.) Further, if $X$ is nonsingular
and compact, then $\mathbb{T}^1_X=H^{1}(X,T_X)$.

Which basically I interpret as: the dimension of the deformation space of a
smooth compact variety is $H^{1}(X,T_X)$.

\begin{example}
\label{example-moduli-space-of-nonsingular-Riemann-surfaces-of-genus-g}
\begin{reference}
\cite[Example 3.1]{continued-fractions}
\end{reference}
Fix $g \geq 2$. The dimension of the deformation space of a nonsingular
projective curve $X$ is $3g-3$. This is ``the dimension of the moduli space of
curves of genus $g$''.
 
We can compute this number by Riemann-Roch formula on the bundle $-K_X$. 
Indeed, since $X$ is a curve, $\Omega_X^1=K_X$ and thus $-K_X=T_X$. We get
\begin{align*}
h^0(-K)-h^0(K-(-K))&=\text{deg}(-K)-g+1\\
&=-2g+2-g+1\qquad \text{degree additive and $\text{deg}K=2g-2$}\\
&=-3g+3
\end{align*}
Now by Serre duality $h^0(K-(-K))=h^1(-K)=h^1(T_X)$, and it turns out that a
Riemann surface of genus $g \geq 2$ has no holomorphic vector fields, so that
$h^0(-K)=h^0(T_X)=0$.
\end{example}

\begin{exercise}
\label{exercise-deformations-of-curves-in-K3}
Let $C$ be a smooth genus $g$ curve which can be embedded in a K3 surface  $M$,
and $X$ the family of all deformations of $C$ in $M$.
\begin{enumerate}
\item Prove that $\dim X \leq  g$.
\item Let $\mathcal{X}_g$ be the space of all curves of genus $g$ (smooth?)
which can be possibly embedded to a K3 surface. Prove that each irreducible
component $Z$ of $\mathcal{X}_g$ satisfies $\dim_\mathbb{C} \leq  g+19$. Deduce
that there exists a compact complex curve which cannot be embedded in a K3
surface.
\end{enumerate}
\end{exercise}

\begin{exercise}
\label{exercise-}
Let $C$ be a smooth curve embedded in a K3 surface $X$. Show that the dimension
of the deformation space of $C$ is $\leq g$.
\end{exercise}

\begin{proof}
%\begin{enumerate}
%\item 
The deformation space of a variety is the space of isomorphism classes of
deformations as explained above. It turns out that there is a way to associate
1-cocyles of the tangent sheaf to deformations, so that in fact the deformation
space $\text{Def}_1$ is isomorphic to $H^{1}(X,\mathcal{T}_X)$ for any variety
$X$.

For our curve $C$ we thus know that the dimension of the space of deformations
(deformations not necessarily contained in $X$) is $h^1(\mathcal{T}_C)$.  The
family of deformations of $C$ that are contained in $X$ is the Hilbert space of
curves with fixed Hilbert polynomial $P(t)$ after quotienting by $\mathbb{C}s$,
where $s$ is the section whose vanishing locus is  $C$. 
This says that the number we are looking for is $h^{0}(X,\mathcal{O}(C))-1$.

Now I will show that $h^0(X,\mathcal{O}(C))=1+g$ (see 
\cite[Lemma 1.2.1, Remark 1.2.2]{huk}).

Consider the ideal sheaf exact sequence twisted by
$\mathcal{O}(C)$:
$$
\xymatrix{
0\ar[r]&\mathcal{O}_X\ar[r]&\mathcal{O}(C)\ar[r]
&\mathcal{O}_X(C) \otimes \mathcal{O}_C=\mathcal{O}(C)|_{C}\ar[r]&0
}
$$
By $X$ being a K3 we know that $H^{1}(\mathcal{O}_X)=0$, so that we have the
short exact sequence in cohomology
$$
\xymatrix{
0\ar[r]&H^{0}(\mathcal{O}_X)\ar[r]&H^{0}(\mathcal{O}(C))\ar[r]
&H^{0}(\mathcal{O}(C)|_{C})\ar[r]&0
}
$$
so that $h^0(\mathcal{O}(C))=1+h^1(\mathcal{O}(C)|_{C})$. 
A version of adjunction formula says
$\omega_C \cong (K_X \otimes\mathcal{O}(C)|_{C}$, and using that 
$K_X=\mathcal{O}_X$ we obtain $h^0(\mathcal{O}(C)|_{C})=h^0(\omega_C)=g$.

To 

\medskip\noindent
For the record I put other thoughts I went through in solving this exercise.

Recall from \cite[p. 146]{gri} that the normal bundle
$\mathcal{N}$ of a hypersurface of a smooth variety $X$ satisfies
$\mathcal{N}^\vee\cong\mathcal{O}_X(-C)|_{C}$. Taking duals we get that
$\mathcal{N}\cong\mathcal{O}_X(C)|_{C}$.

By adjunction formula $2g-2=(\mathcal{O}(C),\mathcal{O}(C))$. 
Applying Riemann-Roch
to $\mathcal{O}(C)$ (which is by definition the dual of the ideal sheaf of $C$,
which is a line bundle on $X$), we obtain that
$\chi(\mathcal{O}(C))=2+\frac{1}{2}(\mathcal{O}(C),\mathcal{O}(C))$. Then
$\chi(\mathcal{O}(C))=g+1$.

Now recall that $\chi(\mathcal{O}(C))=h^0(\mathcal{O}(C)
-h^1(\mathcal{O}(C))+h^2(\mathcal{O}(C))$. By Serre duality and $X$ being a K3
surface we see that $h^2(\mathcal{O}(C))\cong h^{0}(\mathcal{O}(-C))$, which is
the ideal sheaf of $C$. Any section of such a sheaf would vanish along $C$, and
since $X$ is compact we conclude there cannot be any such section.

Now we show that also $h^1(\mathcal{O}(C))=0$ to conclude that
$h^0(\mathcal{O}(C))=g+1$. 

We thus conclude that $h^0(\mathcal{O}(C))$




Since $C$ is smooth we can use the normal exact sequence
$$
\xymatrix{
0\ar[r]&\mathcal{T}_C\ar[r]&\mathcal{T}_X|_{C}\ar[r]&\mathcal{N}\ar[r]&0
}
$$
Taking Euler characteristic we see that
$\chi(\mathcal{T}_C)+\chi(\mathcal{N})=\chi(\mathcal{T}_X|_{C})$.


This means that we should be done once we compute $\chi(\mathcal{T}_X|_{C})$. 
For this we can use Riemman-Roch formula for coherent sheaves on a curve, which
tells us that
$$
\text{deg}(\mathcal{T}_X|_{C})=\chi(\mathcal{T}_X|_{C})
-\text{rk}(\mathcal{T}_X|_{C})\cdot\chi(\mathcal{O}_C)
$$
But of course we know that since $C$ is a curve, by Serre duality we get that
$\chi(\mathcal{O}_C)=1-g$. (Indeed:
$h^1(\mathcal{O}_C)=h^0(\Omega^1_C):=p_a(C)$.) And now the question is what is
the degree. Apparently this is just the first Chern class $c_1$ of the
restricted tangent bundle. So what is it? And what is $h^0(\mathcal{T}_C)$ if
the genus is 0 or 1, and that's it.

I
know that $h^0(\mathcal{T}_X)=0$ and $h^1(\mathcal{T}_X)=20$ by $X$ being a K3
surface, but I'm not sure what happens when we restrict to $C$.

If $g \geq 2$ we know that $h^0(\mathcal{T}_C)=0$, so that
$\boxed{\chi(\mathcal{T}_C)=h^1(\mathcal{T}_C)}$.
\bigskip
To compute the
latter Euler characteristic, which is given by definition by
$\chi(\mathcal{T}_X|_{C})=h^0(\mathcal{T}_X|_{C})-h^1(\mathcal{T}_X|_{C})$, we
first note that $h^0(\mathcal{T}_X|_{C})=0$, because this is the dimension of
the global holomorphic vector fields on $X$
restricted to $C$, which is constant along $C$ since $X$ is
smooth. And then this number is actually zero by Hodge numbers of a K3 surface.

and taking cohomology long exact sequence we obtain
$$
\xymatrix{
\cdots\ar[r]&H^{0}(T_X)\ar[r]&H^{0}(\mathcal{N})\ar[r]&H^{1}(T_C)\ar[r]&\\
H^{1}(T_X)\ar[r]&H^{1}(\mathcal{N})\ar[r]& \cdots
}
$$
\medskip\noindent
Or is it $h^1(\mathcal{N})$? See \cite{HarrMorr}. If this was the case, then we
can use adjunction formula as above to get that
$2g-2=(\mathcal{N},\mathcal{N})=\text{deg}_C \mathcal{N}$. Then we may find
$h^0(\mathcal{N})$ via Riemann-Roch:
$$
h^0(\mathcal{N})-h^0(K_C-\mathcal{N})=\text{deg}\mathcal{N}-g+1
$$
Note that $h^0(N_C-\mathcal{N})=h^1(-K_C+N+K_C)=h^1(\mathcal{N})$ via Serre
duality, and by Riemann-Roch on a surface as above we see that
$$
g+1=\chi(\mathcal{N})=h^0(\mathcal{N})-h^1(\mathcal{N})
\implies h^1(\mathcal{N})=-g-1+h^0(\mathcal{N})
$$
so that
\begin{align*}
h^0(\mathcal{N})-(-g-1+h^0(\mathcal{N}))&=\text{deg}\mathcal{N}-g+1\\
\implies \text{oops! I lost $h^0(\mathcal{N})$ in this operation…}
\end{align*}
Maybe if I just use normal exact sequence and realise that
$$
h^0(\mathcal{N})=h^0(\mathcal{T}_X|_{C})-h^0(\mathcal{T}_C)
$$
I know that $h^0(\mathcal{T}_C)=0$ for $g>1$, so the question is how to compute
the restricted holomorphic vector fields.
%\end{enumerate}
\end{proof}

\section{Continued fractions}
\label{section-continued-fractions}

Definition of HJ continued fraction. For $i>2$ they are in bijection with
$\mathbb{Q}_{>1}$.

The basic diagram of this course starts with a surface $S$ (eg. Hirzebruch
surface $S=\mathbb{F}_m$). Blowing up leads to $X$, and contracting Wahl chains
on $X$ leads to $W$, a normal projective surface that has only Wahl
singularities. Then we construct $\mathbb{Q}$-Gorenstein smoothings $W_t$.
(These  $\mathbb{Q}$-Gorenstein smoothings have Milnor number =0.)

Continued fractions have minimal models:
\begin{itemize}
\item $[1,1]$ means a 0 curve,  $\mathbb{P}^1$.
\item $[1]$ means a  $-1$ curve,  $\mathbb{P}^1$.
\item For $\frac{m}{q}\in\mathbb{Q}_{>1}$, the continued fraction
$[e_1,\ldots,e_r]$ means a chain, which is a sequence of lines that intersect
transversally with $-e_1,\ldots,-e_r$. This is mapped to $\frac{1}{m}(1,q)$.
\end{itemize}

\medskip\noindent

{\bf Third lecture.}

Here's some slogans/recap:
\begin{enumerate}
\item The most important cyclic quotient singularities (c.q.s.) are Wahl
$\frac{1}{n^2}(na-1)$. There is a model to deal with this kind o singularities
using continued fractions. This is very silly but what I picked up is that ``you
add a 2 in the end and add +1 to the first number'', so for example
$[4]\rightsquigarrow[5,2]\rightsquigarrow[6,2,2]$. But on the second step the
$[5,2]$ also goes to $[2,5,3]$ in a way I don't understand. This is called the
Wahl algorithm.
\item (See [KSB88]) There is a notion of $M$-resolution, which is a drawing of 
several curves $\Gamma_i$ intersecting at points $P_i$ that may be Wahl 
singularities or 
smooth points with the key property that $\Gamma_i\cdot K\geq 0$. We have
``toric boundary for $P_i$''. These $M$-resolutions are in 1-1 correspondence
with smoothings of $\frac{1}{m}(1,q)$, and in turn in 1-1 correspondence with
continued fractions $K\left(\frac{m}{m-q}\right)=\{k_1,\ldots,k_s]:
1\leq k_i\leq b_i\;\forall i\}$ where $\frac{m}{m-q}=[b_1,\ldots,b_s]$.
\end{enumerate}

\medskip\noindent

Today we consider the fibers to be $W_t=\mathbb{P}^2$ and
try to find $W$. Set $m_1,m_2,m_3\in\mathbb{Z}_{>0}$. Define
$$
\mathbb{P}(m_1,m_2,m_3):=
\mathbb{P}^2/(\mathbb{Z}/m_1\oplus\mathbb{Z}/m_2\oplus\mathbb{Z}/m_3)
=\mathbb{C}^3\setminus\{0\}/(\lambda\in\mathbb{C}^*\lambda(x,y,z)
=(\lambda^{m_1}x,\lambda^{m_2}y,\lambda^{m_3}z))
$$
For $\text{gdc}(d,m_i)=1$ we have
$\mathbb{P}(dm_1,dm_2,dm_3)=\mathbb{P}(m_1,m_2,m_3)$.

For a triangle $xyz=0$ given by three lines $\Gamma_i$ we have cqs singularities
of the kind $\frac{1}{m_1}(m_2,m_3)$. In this case
$K_W=-(m_1+m_2+m_3)\xi=-\Gamma_1-\Gamma_2-\Gamma_3$ for
$\xi^2=\frac{1}{m_1m_2m_3}$, and $\text{Cl}(W)=\mathbb{Z}\left<\xi\right>$.
Since these are Wahl singularities, we must have that the $m_i$ are squares,
i.e. $m_i=n_i^2$ for some $n_i$. We must have:
\begin{align*}
K_W^2=(m_1+m_2+m_3)^2\frac{1}{m_1m_2m_3}=9&=K^2_{\mathbb{P}^2}\\
\implies (n_1^2+n_2^2+n_3^2)-9n_1^2n_2^2n_3^2&=0\\
\implies (n_1^2+n_2^2+n_3^2-2n_1n_2n_3)\cdot(\text{positive factor})&=0\\
\implies n_1^2+n_2^2+n_3^2&=3n_1n_2n_3
\end{align*}
The last equation is known as {\it Markov equation}.

\begin{example}
\label{example-Hirzebruch-surface}
For $\mathbb{P}(1,1,4)=W$, a triangle with a Wahl singularity
$\frac{1}{4}(1,1)$ in one vertex. Blowing up gives the Hirzebruch surface
$\mathbb{F}_4$, so that a minimal resolution is the triangle. Compare with
[\href{https://arxiv.org/pdf/2504.19929}{Hacking-Prokhorov-2010}]. This example
satisfies the Markov equation for $n_1=1,n_2=1,n_3=2$.
\end{example}

\begin{theorem}[[HP2010]]
\label{theorem-HP2010}
If $\mathbb{P}^2\rightsquigarrow W$ with only log terminal singularities then
$W$ is a partial $\mathbb{Q}$-Gorenstein smoothing of $\mathbb{P}(a^2,b^2,c^2)$
where $a^2+b^2+c^2=3abc$.
\end{theorem}
By the Markov equation condition all the singularities must be Wahl. The triple
$(a,b,c)$ is called {\it Markov triple}. Any permutation of a Markov triple is
another Markov triple. Is $(a,b,c)$ is Markov then so is $(a,b,3ab-c)$. This
allows to construct a {\it Markov tree}. There is so-called Markov conjecture
(due to Frobenius) still unsolved.

\section{Stanley Reisner}
\label{section-}

Antes de introduzir matroides, Os conjuntos $f_s$, que são os conjuntos de
tamanho $s$, eles tem um significado geométrico?



\section{Fano varieties}
\label{section-Fano-varieties}

\begin{definition}
\label{definition-Fano-variety}
A {\it Fano variety} is a projective variety with $-K_X$ ample.
\end{definition}

\begin{definition}
\label{definition-Fano-index}
$$
r(X):=\text{min}\{r:\frac{c_1(X)}{r}\in H^{2}(X,\mathbb{Z})\}
$$
\end{definition}

\begin{exercise}
\label{exercise-Fano-vanishing-higher-cohomology}
By Kodaira vanishing theorem \ref{theorem-Kodaira-vanishing}, 
you can show that the cohomology $H^{i}(X,L)$ for
a Fano variety $X$ vanishes. You just have to put $L=\mathcal{O}(k)$ with $k\geq
-r$, where $r$ is the Fano index.
\end{exercise}

\begin{exercise}
\label{exercise-Pic-H2-Fano}
Show that  $\text{Pic}(X)\cong H^{2}(X,\mathbb{Z})$ holds for Fano varieties.
\end{exercise}

\begin{remark}
\label{remark-derived-category-of-Fano-3-folds-with-vanishing-simplicial
-cohomology}
\begin{reference}
Marcos Jardim, CIMPA 2025 Florianópolis, Lecture 2.
\end{reference}
If $H^3(X,\mathbb{Z})=0$ of a Fano 3-fold, then its derived category is
generated by 4 elements.
\end{remark}


\section{Quivers}
\label{section-quivers}

\begin{definition}
\label{definition-quiver}
A {\it quiver} is a set of vertices $Q_0$, a set of arrows $Q_1$ equipped with
the maps of source $s$ and target $t$ that to each arrow they assign the point
that is source or target of the arrow.
\end{definition}

\begin{definition}
\label{definition-representation-of-quiver}
A {\it representation} of a quiver is a set of finite dimensional vector spaces
equipped with maps between them realising a given quiver (incomplete…).
\end{definition}

There is a notion of projective representation, which I missed to write. But it
is analogous to the injective representation:

\begin{definition}
\label{definition-injective-representation-of-quivers}
Given a quiver $Q$, the {\it injective representation} of $Q_0$ is given by, for
$i \in Q_0$,
$$
I(i)_j=\begin{cases}
k\qquad &i=j \\
k^{d'}\qquad &j \neq i
\end{cases}
$$
where $d'$ is the number of paths from $j$ to $i$.
\end{definition}

\section{Stacks}
\label{section-stacks}

My first definition of stack can be extracted
from

\begin{definition}
\label{definition-superstack}
A {\it superstack} is a stack over
the étale site $\text{SSch}$ of superschemes,
i.e. it is a category fibered in groupoids
over the category of superschemes,
the latter equipped with the 
étale topology, 
satisfying the descent condition.
\end{definition}

Here are some other definitions:

\begin{definition}
\label{definition-algebraic-stacks}
Let  $\mathfrak{X}$ be a stack over $\text{Sch}_{\text{ét}}$.
An {\it algebraic space} is
such that there exists morphism
$\mathcal{U} \to \mathfrak{X}$
where $\mathcal{U}$ is a scheme, that is
schematic, étale and injective (check this one).

$\mathfrak{X} \to y$ is {\it representable} if
there exists a scheme $\mathcal{U}$ and a map
$\mathcal{U} \to y$ such that the 
fibered product
$$
\xymatrix{
\mathcal{U} \times_y \mathfrak{X}\ar[r]\ar[d]\ar@{}[dr]|-{\lrcorner}&\mathfrak{X}\ar[d]\\
\mathcal{U}\ar[r]&y
}
$$
is an algebraic space.

Finally, a  stack is {\it algebraic} (resp. {\it Deligne-Mumford})
is there exists a 
representable surjective morphism  $\mathcal{U} \to \mathfrak{X}$ 
that is smooth (resp. étale).

A {\it stable map} over a projective
variety $X$ is an element of the first
Chow group $\beta \in A_1$, where
 $(C,g)$ is an algebraic curve and
$f:C \to X$ with $[f(C)]=\beta$.
\end{definition}

\noindent
The curves that are points under this map
(contractible) are {\bf stable}.

\bibliography{my}
\bibliographystyle{amsalpha}




\end{document}

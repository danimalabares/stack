\input{preamble}

\begin{document}

\title{Algebraic Geometry}
\maketitle

\phantomsection
\label{section-phantom}

\tableofcontents

\section{Dominant morphisms}
\label{section-dominant}

\noindent
The definition of a morphism of schemes being dominant is a little
different from what you might expect if you are used to the notion
of a dominant morphism of varieties.

\begin{definition}
\label{definition-dominant}
A morphism $f : X \to S$ of schemes is called {\it dominant} if the
image of $f$ is a dense subset of $S$.
\end{definition}

\section{Morphisms of finite type}
\label{section-finite-type}

\noindent
Recall that a ring map $R \to A$ is said to be of finite type if
$A$ is isomorphic to a quotient of $R[x_1, \ldots, x_n]$ as an $R$-algebra, see
Algebra, Definition \ref{algebra-definition-finite-type}.

\begin{definition}
\label{definition-finite-type}
Let $f : X \to S$ be a morphism of schemes.
\begin{enumerate}
\item We say that $f$ is of {\it finite type at $x \in X$} if
there exists an affine open neighbourhood $\Spec(A) = U \subset X$
of $x$ and an affine open $\Spec(R) = V \subset S$
with $f(U) \subset V$ such that the induced ring map
$R \to A$ is of finite type.
\item We say that $f$ is {\it locally of finite type} if it is
of finite type at every point of $X$.
\item We say that $f$ is of {\it finite type} if it is locally of
finite type and quasi-compact.
\end{enumerate}
\end{definition}

\section{Normalization}
\label{section-normalization}

\begin{definition}
\label{definition-normalization-X-in-Y}
Let $f : Y \to X$ be a quasi-compact and quasi-separated morphism of schemes.
Let $\mathcal{O}'$ be the integral closure of $\mathcal{O}_X$ in
$f_*\mathcal{O}_Y$. The {\it normalization of $X$ in $Y$} is the
scheme\footnote{The scheme $X'$ need not be normal, for example if
$Y = X$ and $f = \text{id}_X$, then $X' = X$.}
$$
\nu : X' = \underline{\Spec}_X(\mathcal{O}') \to X
$$
over $X$. It comes equipped with a natural factorization
$$
Y \xrightarrow{f'} X' \xrightarrow{\nu} X
$$
of the initial morphism $f$.
\end{definition}

\noindent
The factorization is the composition of the canonical morphism
$Y \to \underline{\Spec}_X(f_*\mathcal{O}_Y)$ (see
Constructions, Lemma
\ref{constructions-lemma-canonical-morphism})
and the morphism of relative spectra coming from the inclusion map
$\mathcal{O}' \to f_*\mathcal{O}_Y$. We can characterize the
normalization as follows.

\begin{lemma}
\label{lemma-characterize-normalization}
Let $f : Y \to X$ be a quasi-compact and quasi-separated morphism of schemes.
The factorization $f = \nu \circ f'$, where $\nu : X' \to X$ is the
normalization of $X$ in $Y$ is characterized by the following
two properties:
\begin{enumerate}
\item the morphism $\nu$ is integral, and
\item for any factorization $f = \pi \circ g$, with $\pi : Z \to X$
integral, there exists a commutative diagram
$$
\xymatrix{
Y \ar[d]_{f'} \ar[r]_g & Z \ar[d]^\pi \\
X' \ar[ru]^h \ar[r]^\nu & X
}
$$
for some unique morphism $h : X' \to Z$.
\end{enumerate}
Moreover, the morphism $f' : Y \to X'$ is dominant and in (2) the
morphism $h : X' \to Z$ is the normalization of $Z$ in $Y$.
\end{lemma}



\section{Reflexive sheaves}
\label{section-reflexive-sheaves}

\begin{slogan}
These are vector bundles except for a small locus.
\end{slogan}

\begin{definition}
\label{definition-reflexive}
Let $X$ be an integral locally Noetherian scheme. Let $\mathcal{F}$
be a coherent $\mathcal{O}_X$-module. The {\it reflexive hull}
of $\mathcal{F}$ is the $\mathcal{O}_X$-module
$$
\mathcal{F}^{**} = \SheafHom_{\mathcal{O}_X}(
\SheafHom_{\mathcal{O}_X}(\mathcal{F}, \mathcal{O}_X), \mathcal{O}_X)
$$
We say $\mathcal{F}$ is {\it reflexive} if the natural map
$j : \mathcal{F} \longrightarrow \mathcal{F}^{**}$
is an isomorphism.
\end{definition}

\begin{lemma}
\label{lemma-reflexive-torsion-free}
Let $X$ be an integral locally Noetherian scheme. Let $\mathcal{F}$
be a coherent $\mathcal{O}_X$-module.
\begin{enumerate}
\item If $\mathcal{F}$ is reflexive, then $\mathcal{F}$ is torsion free.
\item The map $j : \mathcal{F} \longrightarrow \mathcal{F}^{**}$
is injective if and only if $\mathcal{F}$ is torsion free.
\end{enumerate}
\end{lemma}

\begin{remark}[Talk at IMPA, 11 June]\leavevmode
\label{remark-reflexive-talk}
Torsion could also be defined so that the sheaf can inject onto its dual. In
this talk we discussed the moduli space of reflexive/torsion-free sheaves, which
turned out to be parametrized by $c_1, c_2$ and $c_3$. This was denoted by 
$R(c_1,c_2,c_3)$. Actually I think it may have been Manolache that proved the
existence of this moduli space.


Alan Muniz

Nesta palestra discutiremos a classificação de feixes reflexivos de posto dois e
seus espaços de módulos. Apresentaremos algumas ferramentas básicas usadas na
construção e determinação de tais feixes. Aplicaremos estas técnicas para o caso
de feixes com segunda classe de Chern igual a quatro, obtido recentemente em
colaboração com Marcos Jardim.
\end{remark}

\subsection{Distributions on manifolds}
\label{subsection-distributions-on-manifolds}

Here's the abstract from a talk by Marcos Jardim at Geometric Structures:

``I will revise the work done over the past 10 years with various collaborators
on distributions and foliations on 3-folds, especially on the projective space,
with a focus on properties of the tangent sheaf and singular scheme."

Here are two key ideas: if the distribution is codimension 1 we can write:
$$
\xymatrix{
0\ar[r]&F\ar[r]&TX\arrow[r]^{\omega}&I_Z \otimes L\ar[r]&0
}
$$
where $L$ is a line bundle and $\omega \in H^{0}(\Omega_X \otimes L)$, and
$Z=\{p:\omega(p)=0\}$.

When codimension is 2 then $\mathcal{D}$ is given by a holomorphic vector field 
$\nu$: $T_p=\left<\nu(p)\right>$.

It can be encoded as an exact sequence
$$
\xymatrix{
	0\ar[r]&L\ar[r]^{\nu}&TX\arrow[r]&N\ar[r]&0
}
$$
where $L$ is a line bundle and $\nu \in H^{0}(TX \otimes L^\vee)$;
  $Z=\{p | \nu(p) = 0\}$.

\begin{remark}
\label{remark-stauration}
Saturation means that $Z \subset X$ is a union of curves and points.
\end{remark}

And again, distributions are parametrized by Chern classes.

Two interesting open questions:
\begin{enumerate}
\item {\bf Conjeture.} if $\mathcal{D}$ is a codimension 1 foliation of degree
$d$ on $\mathbb{P}^3$, then $c_2(F)\leq d^2-d+1$ and bound is attained
by rational foliations of type $(1,d+1)$. (True for $d \leq 2$.)
\item {\bf Conjecture (with Pepe Seade).} $\mathcal{D}$ is a codimension 1 
foliation on a smooth projective 3-fold, then $\text{Sing}\mathcal{D}$ is connected.
\end{enumerate}

\begin{theorem}[Jardim-Muniz]
\label{theorem-jardim-muniz}
Conditions on Chern classes used to understand moduli space $R(c_1,c_2,c_3)$.
$c_2=4$ gives (?). For $c_3\leq 6$, possible ``spectrum" exists…
\end{theorem}

\section{Stability}
\label{section-stability}

{\bf Question.} What is stability?

\begin{enumerate}
\item Stable objects in an abelian category are the ``building blocks":
we can reconstruct the whole category from them.
\item  An abelian subcategory (hart) $\subset$ a triangulated
\item stability defined via stability function on $\mathcal{A}$.
\item Q. Can we reconstruct $\mathcal{T}$ from the semistable elements of
$\mathcal{A}$
\item {\bf Example.} $\mathcal{A}=\text{Coh}X$ is heart of $D^b(X)$
w/ funny function.
\item Stability condition is hart + stability function.
\item Bridgeland Stabl:= the stability conditions are a complex manifold
of complex codimension $\text{rk}\Lambda$:
$$
 \mathcal{Z}:\text{Stab}(\mathcal{T})
\longrightarrow \text{Hom}(\Lambda,\mathbb{C})
$$
\item There's a chamber structure; moduli space changes across chambers.
\item I think we typically think of vectors in $\text{Hom}(\Lambda,\mathbb{C})$ 
as Chern classes, to characterize the moduli spaces.
\item Existence: given a projective variety $X$, are there stability
conditions on $D^\text{b}(X)$? Yes for fano 3fold pic rk 1.
\item Moduli spaces: is $M_\sigma(v)$ a projective scheme? Cannot use
usual git techniques to study. A stack!
\item Picture: blue + black are walls. Q. What are $\beta$ and $\alpha$?
\item thm: bridgeland stable = gieseker stable ?
\item Q. slope stability = bridgeland stability? A. Not always.
\item DT/PT correspondance: only one wall between PT and G chambers
\medskip
\item Polynomial stability function. This is an asymptotic version of BS.
\item There are some $\rho$'s. Arrangements of $\rho_i$ are polynomial 
stability conditions on a threefold.
\item Pata-Thomas introduced stability for rank 1 objects.
Bayer compares the---wall. Q. Same for Bridge S---only one wall?
\item Recall Gieseker stability.
\medskip
\item Def. A {\it stable triple}: when
$\text{gcd}(\text{ch}_0,\text{ch}_2,\text{ch}_3)=1$, every PT stable object
comes from three conditions (missing).
\item What happens when you cross the blue wall? 
Both $\mathcal{G}$ and $\mathcal{T}$ are projective. What happens at the blue
wall?
\medskip
\item For $X$ smooth threefold with $\text{rk}\text{Pic}=1$, $\mathcal{G}(v)$,
$v=(r,0,0,-n)$, $\mathcal{G}(v)$ is a known sheaf object and
$\mathcal{T}=\emptyset$.
\item $X$ sm 3 rkpic1, Fake wall; $\mathcal{G}=\mathcal{T}$.
\item (Extra.) red circle is a wall for a weaker form of stability.
\end{enumerate}
\begin{definition}[Talk at impa]
\label{definition-slope-stability}
A rank-2 sheaf $\mathcal{F}$ is {\it semistable (stable)}if
$H^{0}(\mathcal{F}(t))=0$ for $-t \geq(>) \frac{c_1F}{2}$
\end{definition}

Compare with

\begin{definition}[moduli-curves.tex]
\label{definition-semistable}
Let $f : X \to S$ be a family of curves.
We say $f$ is a {\it semistable family of curves} if
\begin{enumerate}
\item $X \to S$ is a prestable family of curves, and
\item $X_s$ has genus $\geq 1$ and
does not have a rational tail for all $s \in S$.
\end{enumerate}
\end{definition}

\section{Coherent sheaves}
\label{section-coherent-sheaves}

\begin{lemma}
\label{lemma-quasi-coherent-affine-cohomology-zero}
\begin{slogan}
Serre vanishing: Higher cohomology vanishes on affine schemes
for quasi-coherent modules.
\end{slogan}
Let $X$ be a scheme.
Let $\mathcal{F}$ be a quasi-coherent $\mathcal{O}_X$-module.
For any affine open $U \subset X$ we have
$H^p(U, \mathcal{F}) = 0$ for all $p > 0$.
\end{lemma}

\begin{proof}
We are going to apply
Cohomology, Lemma \ref{cohomology-lemma-cech-vanish-basis}.
As our basis $\mathcal{B}$ for the topology of $X$ we are going to use
the affine opens of $X$.
As our set $\text{Cov}$ of open coverings we are going to use the standard
open coverings of affine opens of $X$.
Next we check that conditions (1), (2) and (3) of
Cohomology, Lemma \ref{cohomology-lemma-cech-vanish-basis}
hold. Note that the intersection of standard opens in an affine is
another standard open. Hence property (1) holds.
The coverings form a cofinal system of open coverings of any element
of $\mathcal{B}$, see
Schemes, Lemma \ref{schemes-lemma-standard-open}.
Hence (2) holds.
Finally, condition (3) of the lemma follows from
Lemma \ref{lemma-cech-cohomology-quasi-coherent-trivial}.
\end{proof}

\bibliography{my}
\bibliographystyle{amsalpha}

\end{document}


\input{preamble}

\begin{document}

\title{Lista 8}
\maketitle

\phantomsection
\label{section-phantom}

%\tableofcontents

%\section{Lista 8}
%\label{l8-rg}

\noindent

\begin{exercise}
Prop. 2.12 do capítulo XIII, \cite{doc}. Seja $p \in M$. Suponha exista um ponto $q \in C_m(p)$ que realiza a distância de $p$ a $C_m(p)$. Então:
\begin{enumerate}
\item ou existe uma geodésica minimizante $\gamma$ de $p$ a $q$ ao longo da qual $q$ é conjungado a p,
\item ou existem exatamente duas geodésicas minimizantes $\gamma$ e $\sigma$ de $p$ a $q$; além disto, $\gamma'(\ell)$, $\ell=d(p,q)$.
\end{enumerate}
\end{exercise}


\bibliography{my}
\bibliographystyle{amsalpha}

\end{document}


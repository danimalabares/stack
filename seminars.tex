\IfFileExists{stacks-project.cls}{%
\documentclass{stacks-project}
}{%
\documentclass{amsart}
}

% For dealing with references we use the comment environment
\usepackage{verbatim}
\newenvironment{reference}{\comment}{\endcomment}
%\newenvironment{reference}{}{}
\newenvironment{slogan}{\comment}{\endcomment}
\newenvironment{history}{\comment}{\endcomment}

% For commutative diagrams we use Xy-pic
\usepackage[all]{xy}

% We use 2cell for 2-commutative diagrams.
\xyoption{2cell}
\UseAllTwocells

% We use multicol for the list of chapters between chapters
\usepackage{multicol}

% This is generally recommended for better output
\usepackage{lmodern}
\usepackage[T1]{fontenc}

% For cross-file-references
\usepackage{xr-hyper}

% Package for hypertext links:
\usepackage{hyperref}

% For any local file, say "hello.tex" you want to link to please
% use \externaldocument[hello-]{hello}
\externaldocument[introduction-]{introduction}
\externaldocument[conventions-]{conventions}
\externaldocument[sets-]{sets}
\externaldocument[categories-]{categories}
\externaldocument[topology-]{topology}
\externaldocument[sheaves-]{sheaves}
\externaldocument[sites-]{sites}
\externaldocument[stacks-]{stacks}
\externaldocument[fields-]{fields}
\externaldocument[algebra-]{algebra}
\externaldocument[brauer-]{brauer}
\externaldocument[homology-]{homology}
\externaldocument[derived-]{derived}
\externaldocument[simplicial-]{simplicial}
\externaldocument[more-algebra-]{more-algebra}
\externaldocument[smoothing-]{smoothing}
\externaldocument[modules-]{modules}
\externaldocument[sites-modules-]{sites-modules}
\externaldocument[injectives-]{injectives}
\externaldocument[cohomology-]{cohomology}
\externaldocument[sites-cohomology-]{sites-cohomology}
\externaldocument[dga-]{dga}
\externaldocument[dpa-]{dpa}
\externaldocument[sdga-]{sdga}
\externaldocument[hypercovering-]{hypercovering}
\externaldocument[schemes-]{schemes}
\externaldocument[constructions-]{constructions}
\externaldocument[properties-]{properties}
\externaldocument[morphisms-]{morphisms}
\externaldocument[coherent-]{coherent}
\externaldocument[divisors-]{divisors}
\externaldocument[limits-]{limits}
\externaldocument[varieties-]{varieties}
\externaldocument[topologies-]{topologies}
\externaldocument[descent-]{descent}
\externaldocument[perfect-]{perfect}
\externaldocument[more-morphisms-]{more-morphisms}
\externaldocument[flat-]{flat}
\externaldocument[groupoids-]{groupoids}
\externaldocument[more-groupoids-]{more-groupoids}
\externaldocument[etale-]{etale}
\externaldocument[chow-]{chow}
\externaldocument[intersection-]{intersection}
\externaldocument[pic-]{pic}
\externaldocument[weil-]{weil}
\externaldocument[adequate-]{adequate}
\externaldocument[dualizing-]{dualizing}
\externaldocument[duality-]{duality}
\externaldocument[discriminant-]{discriminant}
\externaldocument[derham-]{derham}
\externaldocument[local-cohomology-]{local-cohomology}
\externaldocument[algebraization-]{algebraization}
\externaldocument[curves-]{curves}
\externaldocument[resolve-]{resolve}
\externaldocument[models-]{models}
\externaldocument[functors-]{functors}
\externaldocument[equiv-]{equiv}
\externaldocument[pione-]{pione}
\externaldocument[etale-cohomology-]{etale-cohomology}
\externaldocument[proetale-]{proetale}
\externaldocument[relative-cycles-]{relative-cycles}
\externaldocument[more-etale-]{more-etale}
\externaldocument[trace-]{trace}
\externaldocument[crystalline-]{crystalline}
\externaldocument[spaces-]{spaces}
\externaldocument[spaces-properties-]{spaces-properties}
\externaldocument[spaces-morphisms-]{spaces-morphisms}
\externaldocument[decent-spaces-]{decent-spaces}
\externaldocument[spaces-cohomology-]{spaces-cohomology}
\externaldocument[spaces-limits-]{spaces-limits}
\externaldocument[spaces-divisors-]{spaces-divisors}
\externaldocument[spaces-over-fields-]{spaces-over-fields}
\externaldocument[spaces-topologies-]{spaces-topologies}
\externaldocument[spaces-descent-]{spaces-descent}
\externaldocument[spaces-perfect-]{spaces-perfect}
\externaldocument[spaces-more-morphisms-]{spaces-more-morphisms}
\externaldocument[spaces-flat-]{spaces-flat}
\externaldocument[spaces-groupoids-]{spaces-groupoids}
\externaldocument[spaces-more-groupoids-]{spaces-more-groupoids}
\externaldocument[bootstrap-]{bootstrap}
\externaldocument[spaces-pushouts-]{spaces-pushouts}
\externaldocument[spaces-chow-]{spaces-chow}
\externaldocument[groupoids-quotients-]{groupoids-quotients}
\externaldocument[spaces-more-cohomology-]{spaces-more-cohomology}
\externaldocument[spaces-simplicial-]{spaces-simplicial}
\externaldocument[spaces-duality-]{spaces-duality}
\externaldocument[formal-spaces-]{formal-spaces}
\externaldocument[restricted-]{restricted}
\externaldocument[spaces-resolve-]{spaces-resolve}
\externaldocument[formal-defos-]{formal-defos}
\externaldocument[defos-]{defos}
\externaldocument[cotangent-]{cotangent}
\externaldocument[examples-defos-]{examples-defos}
\externaldocument[algebraic-]{algebraic}
\externaldocument[examples-stacks-]{examples-stacks}
\externaldocument[stacks-sheaves-]{stacks-sheaves}
\externaldocument[criteria-]{criteria}
\externaldocument[artin-]{artin}
\externaldocument[quot-]{quot}
\externaldocument[stacks-properties-]{stacks-properties}
\externaldocument[stacks-morphisms-]{stacks-morphisms}
\externaldocument[stacks-limits-]{stacks-limits}
\externaldocument[stacks-cohomology-]{stacks-cohomology}
\externaldocument[stacks-perfect-]{stacks-perfect}
\externaldocument[stacks-introduction-]{stacks-introduction}
\externaldocument[stacks-more-morphisms-]{stacks-more-morphisms}
\externaldocument[stacks-geometry-]{stacks-geometry}
\externaldocument[moduli-]{moduli}
\externaldocument[moduli-curves-]{moduli-curves}
\externaldocument[examples-]{examples}
\externaldocument[exercises-]{exercises}
\externaldocument[guide-]{guide}
\externaldocument[desirables-]{desirables}
\externaldocument[coding-]{coding}
\externaldocument[obsolete-]{obsolete}
\externaldocument[fdl-]{fdl}
\externaldocument[index-]{index}

% Theorem environments.
%
\theoremstyle{plain}
\newtheorem{theorem}[subsection]{Theorem}
\newtheorem{proposition}[subsection]{Proposition}
\newtheorem{lemma}[subsection]{Lemma}

\theoremstyle{definition}
\newtheorem{definition}[subsection]{Definition}
\newtheorem{example}[subsection]{Example}
\newtheorem{exercise}[subsection]{Exercise}
\newtheorem{situation}[subsection]{Situation}

\theoremstyle{remark}
\newtheorem{remark}[subsection]{Remark}
\newtheorem{remarks}[subsection]{Remarks}

\numberwithin{equation}{subsection}

% Macros
%
\def\lim{\mathop{\mathrm{lim}}\nolimits}
\def\colim{\mathop{\mathrm{colim}}\nolimits}
\def\Spec{\mathop{\mathrm{Spec}}}
\def\Hom{\mathop{\mathrm{Hom}}\nolimits}
\def\Ext{\mathop{\mathrm{Ext}}\nolimits}
\def\SheafHom{\mathop{\mathcal{H}\!\mathit{om}}\nolimits}
\def\SheafExt{\mathop{\mathcal{E}\!\mathit{xt}}\nolimits}
\def\Sch{\mathit{Sch}}
\def\Mor{\mathop{\mathrm{Mor}}\nolimits}
\def\Ob{\mathop{\mathrm{Ob}}\nolimits}
\def\Sh{\mathop{\mathit{Sh}}\nolimits}
\def\NL{\mathop{N\!L}\nolimits}
\def\CH{\mathop{\mathrm{CH}}\nolimits}
\def\proetale{{pro\text{-}\acute{e}tale}}
\def\etale{{\acute{e}tale}}
\def\QCoh{\mathit{QCoh}}
\def\Ker{\mathop{\mathrm{Ker}}}
\def\Im{\mathop{\mathrm{Im}}}
\def\Coker{\mathop{\mathrm{Coker}}}
\def\Coim{\mathop{\mathrm{Coim}}}

% Boxtimes
%
\DeclareMathSymbol{\boxtimes}{\mathbin}{AMSa}{"02}

%
% Macros for moduli stacks/spaces
%
\def\QCohstack{\mathcal{QC}\!\mathit{oh}}
\def\Cohstack{\mathcal{C}\!\mathit{oh}}
\def\Spacesstack{\mathcal{S}\!\mathit{paces}}
\def\Quotfunctor{\mathrm{Quot}}
\def\Hilbfunctor{\mathrm{Hilb}}
\def\Curvesstack{\mathcal{C}\!\mathit{urves}}
\def\Polarizedstack{\mathcal{P}\!\mathit{olarized}}
\def\Complexesstack{\mathcal{C}\!\mathit{omplexes}}
% \Pic is the operator that assigns to X its picard group, usage \Pic(X)
% \Picardstack_{X/B} denotes the Picard stack of X over B
% \Picardfunctor_{X/B} denotes the Picard functor of X over B
\def\Pic{\mathop{\mathrm{Pic}}\nolimits}
\def\Picardstack{\mathcal{P}\!\mathit{ic}}
\def\Picardfunctor{\mathrm{Pic}}
\def\Deformationcategory{\mathcal{D}\!\mathit{ef}}

%Dani's additions
\usepackage{graphicx} %figures


\begin{document}

\title{Seminars}
\maketitle

\phantomsection
\label{section-phantom}
\hfill
\href{http://github.com/danimalabares/stack}{github.com/danimalabares/stack}

\tableofcontents

\section{Friday Seminar, PUC-Rio}
\label{section-Friday-seminar}

\subsection{Neutrinos}
\label{subsection-neutrinos}

\noindent\bigskip
Hiroshi Nunokawas, PUC-Rio. June 27, 2025.

{\bf Abstract.} Hiroshi will come and tell us everything we (not) wanted to know
about these mysterious particles, and are not going to be afraid to ask. In
particular, about the neutrino oscillation, and the great matrices.

\bigskip\noindent

Protons and neutrons have very similar mass of $m_p\approx940$ MeV, while
electrons have mass of $m_e \approx 0.5$ MeV. MeV is $10^{6}$ electronvolts,
where one eV is approximately $1.6\times 10^{-19}$ J. This is standard in high
energy physics, they use electronvolts instead of Joules. Recall that
2J=1N$\times$1m.

Most of the things we see are protons since they are so much larger than
electrons. But protons nor neutrons are elementary particles.

Here's the standard model:

\medskip

\begin{tabular}{c c c c}
Quarks & $\begin{bmatrix} u\\d \end{bmatrix}_L $ & $\begin{bmatrix} c\\s
\end{bmatrix}_L$ & $\begin{bmatrix} t\\b \end{bmatrix}_L$\\
Leptons & $\begin{bmatrix} \nu_e\\e^- \end{bmatrix}_L $ & $\begin{bmatrix}
\nu_\mu\\ \mu^-
\end{bmatrix}_L$ & $\begin{bmatrix} \nu_\tau\\ \tau^- \end{bmatrix}_L$\\
Generation & 1st & 2nd & 3rd\\
Bosons & $g$, $\gamma$ , $\omega^\pm$, $z$ \\
Higgs Bosson & $H$
\end{tabular}

\medskip

It is very particular that nature repeats itself three times. The $L$ in those
matrix actually means left-handed, and accounts for chirality. Only left-handed
fermions have weak interaction. Right-handed have electromagnetic interaction,
gravitational interaction, but not weak interaction.

And then there's neutrinos. They have negative helicity (chirality). Being
left-handed, mathematically, means to have helicity $-1$. I think this means
that the spin is left-handed. But chirality and helicity are not the same:
helicity is observer-dependent, and chirality is not. Almost all neutrinos we
can see (\% 99.99999…) have negative helicity, but not all of them.

Consider the following:
$$
n+\nu_e\leftrightarrow p+e^-
$$
But it's not completely correct: we'd better put $d$ instead of $n$, and $u$
instead of $p$: the $d$ and $u$ quarks, instead of the neutrons and protons.

Now consider the following reaction: a neutron decays into a proton, an electron
and an antineutrino:
$$
n\to p+e^+\overline{\nu}_e
$$
Protons is very stable, that's why we are here. But neutron decays in only 15
minutes.

By experimental data, we can conclude that neutrinos' mass is consistent with
zero. But if they have mass, it should be much smaller than the electron's $m_v
\leq 0.5$ eV. And the electron is already the lightest fermion!

If the mass of the neutrino was zero, i.e. $m_\nu=0$, then $v_\nu=c$ in vacuum,
which would imply that 
$$
\xymatrix{
\nu_e\ar[r]^{L}&\nu_e\ar[r]&\nu_e\\
0:00&0:00&0:00
}
$$
meaning: time doesn't pass! And this means the state of the particle cannot
change.

\section{Differential Geometry Seminar, IMPA}
\label{section-differential-geometry-seminar-impa}

\subsection{Spheres with minimal equators}
\label{subsection-spheres-with-minimal-equators}

\noindent\bigskip
Lucas Ambrozio, IMPA. June 24, 2025.

{\bf Abstract.} We will discuss the connection between Riemannian metrics on the
sphere with respect to which all equators are minimal hypersurfaces, and
algebraic curvature tensors with positive sectional curvatures.

\bigskip\noindent

\begin{definition}
\label{definition-equator}
An {\it $(n-k)$-equator} orthogonal to $\Pi$ is
$$
\Sigma_\Pi:=\{p\in\mathbb{S}^n:\left<p,x\right>=0\forall x\in\Pi\}
$$
for $\Pi$ a $k$-dimensional linear subspace of $\mathbb{R}^{n+1}$.
\end{definition}

\begin{remark}
\label{remark-equators-are-totally-geodesic-with-usual-metric}
Equators are totally geodesic hypersurfaces with the usual sphere metric, which
implies they are minimal hypersurfaces.
\end{remark}

{\bf Problem.} Characterize the set $\mathcal{M}_k(U)$ of metrics $g$ on an 
open set $U\subset\mathbb{S}^n$ such that all $k$-equators $\Sigma_\Pi$ with
 $\Sigma\cap U\neq\emptyset$ yield are minimal hypersurfaces $\Sigma\cap U$ 
on $(U,g)$.

\begin{remark}
\label{remark-why-not-Rn}
This problem can be thought of as a problem of finding metrics on $\mathbb{R}^n$
such that $k$-planes are minimal. To see why project the $k$-equators to
$T_p\mathbb{S}^n$ and pullback those metrics to the sphere.
\end{remark}

\medskip\noindent

Let $g\in\mathcal{M}_k(U)$ for $U\subset\mathbb{S}^n$ open and $n\geq 2$.

\begin{theorem}[Beltrami, Schäfli]
\label{theorem-Beltrami-Schafli}
If $k=1$ then $g$ has constant sectional curvature.
\end{theorem}

\begin{theorem}[Hongan]
\label{theorem-Hongan}
If $1<k<n-1$ then $g$ has constant sectional curvature.
\end{theorem}

Then Hongan also managed to produce a classification of these metrics for
$k=n-1$.

\begin{remark}
\label{remark-linear-invertible-preserving-equators-is-in-Mg}
If $T\in \text{GL}(n+1,\mathbb{R})$, then
\begin{align*}
\varphi: \mathbb{S}^n &\longrightarrow \mathbb{S}^n \\
x &\longmapsto \frac{Tx}{|Tx|}
\end{align*}
is a diffeomorphism that maps $k$-equators into $k$-equators. Thus if
$g\in\mathcal{M}_k(\mathbb{S}^n)$ then so is $\varphi(T)^*g$.
\end{remark}

\begin{theorem}
\label{theorem-equivariant-bijection}
There exists a $\text{GL}(n+1,\mathbb{R})$ equivariant bijection
$$
\mathcal{M}_{n-1}(\mathbb{S}^n)\leftrightarrow \text{Curv}_+(\mathbb{R}^{n+1})
$$
where the set on the right-hand-side is the set of algebraic curvature tensors
(also called curvature-like, i.e. with the same symmetries as the Riemannian
curvature tensor) on $\mathbb{R}^{n+1}$ with positive sectional curvature.

The group action is given as follows for $T\in\text{GL}(n+1,\mathbb{R})$:
$$
(R\cdot T)(x,y,z,w)=\frac{1}{|\det(T)|^{\frac{1}{n+1}}}R(Tx,Ty,Tz,Tw)
$$
\end{theorem}

The point is that $\text{Curv}_+(\mathbb{R}^{n+1})$ is an open cone on a linear
space. Here are two simple corollaries:

\begin{lemma}
\label{lemma-corollaries}
\begin{enumerate}
\item $\mathcal{M}_{n+1}(\mathbb{S}^n)$ is in bijection with an open positive
cone of an $\frac{n(n+2)(n+1)^2}{12}$-dimensional real vector space.
\item Every metric on $\mathcal{M}_{n-1}(\mathbb{S}^n)$ is invariant by the
antipodal map.
\end{enumerate}
\end{lemma}

{\bf Algorithm.} From any $R\in \text{Curv}_p(\mathbb{R}^{n+1})$ we obtain a
symmetric positive definite (positive-definitiness comes from the positiveness
of the curvature of $R$) 2-tensor  $k_R$ satisfying
$$
(k_R)_p(v,v)=R(pv,pv)>0
$$
Also, $k_R$ has the  {\it Killing property},
i.e. that $\overline{\nabla}k(X,X,X)=0$ for all
$X\in\mathfrak{X}(\mathbb{S}^n)$.

Then we define a positive function on
 $\mathbb{S}^n$ by
\begin{equation}
\label{equation-def-DR}
D_R:=\left(\frac{d\text{Vol}_{k_R}}{dV_g}\right)^{\frac{4}{n-1}}
\end{equation}
and finally a Riemannian metric on $\mathbb{S}^n$ in
$\mathcal{M}_{n-1}\mathbb{S}^n$ by
$$
g_R=\frac{1}{D_R}k_R
$$
And to go back, for $g \in \mathcal{M}_{n-1}(\mathbb{S}^n)$ define a positive 
function on $\mathbb{S}^n$
$$
F_g:=\left(\frac{dV_g}{dV_{\overline{g}}}\right)^{\frac{4}{n-1}}
$$
Then let $k_g:=\frac{1}{F_g}g>0$, which is a positive definite Killing 2-tensor,
from which we may define $R_g\in\text{Curv}_+(\mathbb{R}^{n+1})$ with
$R_g(pv,pv)=(k_g)_p(v,v)$ for all $p,v\in T\mathbb{S}^n$.

More corollaries:
\begin{lemma}
\label{lemma-corollaries2}
\begin{enumerate}
\item $g\in\mathcal{M}_{n-1}(\mathbb{S}^n)$ is analytic because it is a Killing
tensor on $\mathbb{S}^n$, which are well-known.
\item If $g$ is left-invatiant on $\mathbb{S}^3$, seen as unit quaternions, then
$g\in\mathcal{M}_2(\mathbb{S}^3$. Moreover, for $a\geq b\geq c>0$,
$$
aL_i\odot L_i+bL_j\odot L_j+cL_k\odot L_k=k
$$
is Killing, $k>0$,  $D_k$ constant and thus $g=\frac{1}{\text{const.}}k
\in\mathcal{M}_2(\mathbb{S}^3)$.
\item $R$ curvature tensor of $(\mathbb{C}P^{2},g_{FS})$. We may not remember
what's the curvature tensor, but we know the sectional curvature is $1\leq
\text{sec}(R)\leq 4$,
$$
(k_R)_p(v,w)=\overline{g}(v,w)+3\overline{g}(Jp,v)\overline{g}(Jp,w)
$$
and $D_R=4^{\frac{4}{3-1}}=4$, so that by \ref{equation-def-DR} we obtain 
$g_R=\frac{1}{4}k_R$, which is a Berger
metric on $\mathbb{S}^3$ with scalar curvature 0.
\end{enumerate}
\end{lemma}

\bigskip
Now define
$$
\Sigma_V=\{p\in\mathbb{S}^n:\left<p,v\right>=0\}=V^{-1}(0)
$$
where $V(x):=\left<x,v\right>$ for all $x\in\mathbb{S}^n$. Then the normal
vector field is $\nabla V/|\nabla V|_g$, and the second fundamental form is
given by
$$
A=\frac{1}{|\nabla V|_g}\text{Hess}_gV
$$
and its mean curvature by
\begin{equation}
\label{equation-mean-curvature-for-level-sets-V}
H=\frac{1}{|\nabla V|_g}\left(\Delta_gV-\text{Hess}_gV\left(\frac{\nabla
V}{|\nabla V|},\frac{\nabla V}{|\nabla V|}\right)\right)
\end{equation}
For every $v \in \mathbb{S}^n$ and $p\in\Sigma_V$, we see that $H_{\Sigma_V}=0$
iff
$$
|\nabla V|^2_g(p)\Delta_gV(p)-\text{Hess}_gV(\nabla V(p),\nabla V(p))=0
$$
And for $\overline{g}$,
$$
\text{Hess}_{\overline{g}}V+V\overline{g}=0\implies
\text{Hess}_{\overline{g}}V(X,X)=0
$$
for all $X\in T_p\mathbb{S}^n$ and $p\in\Sigma_v$. Then
$$
J_g(X,Y,Z)=g(\nabla_XY-\overline{\nabla}_XY,Z)
$$
$$
J_g(X,Y,\nabla V)=\text{Hess}_{\overline{g}}-\text{Hess}
$$
{\bf Problems.}
\begin{enumerate}
\item Similar story for $\mathbb{C}P^{n},\mathbb{H}P^n$?
\item Complete metrics on $\mathbb{R}^n$ with minimal hyperplanes.
\item Find geometric invariants of metrics on $\mathcal{M}_{n-1}(\mathbb{S}^n)$
(may be useful to study $(M^n,g)$, $n\geq 4$, $\text{sec}>0$.
\end{enumerate}
\bibliography{my}
\bibliographystyle{amsalpha}

\end{document}


\IfFileExists{stacks-project.cls}{%
\documentclass{stacks-project}
}{%
\documentclass{amsart}
}

% For dealing with references we use the comment environment
\usepackage{verbatim}
\newenvironment{reference}{\comment}{\endcomment}
%\newenvironment{reference}{}{}
\newenvironment{slogan}{\comment}{\endcomment}
\newenvironment{history}{\comment}{\endcomment}

% For commutative diagrams we use Xy-pic
\usepackage[all]{xy}

% We use 2cell for 2-commutative diagrams.
\xyoption{2cell}
\UseAllTwocells

% We use multicol for the list of chapters between chapters
\usepackage{multicol}

% This is generally recommended for better output
\usepackage{lmodern}
\usepackage[T1]{fontenc}

% For cross-file-references
\usepackage{xr-hyper}

% Package for hypertext links:
\usepackage{hyperref}

% For any local file, say "hello.tex" you want to link to please
% use \externaldocument[hello-]{hello}
\externaldocument[introduction-]{introduction}
\externaldocument[conventions-]{conventions}
\externaldocument[sets-]{sets}
\externaldocument[categories-]{categories}
\externaldocument[topology-]{topology}
\externaldocument[sheaves-]{sheaves}
\externaldocument[sites-]{sites}
\externaldocument[stacks-]{stacks}
\externaldocument[fields-]{fields}
\externaldocument[algebra-]{algebra}
\externaldocument[brauer-]{brauer}
\externaldocument[homology-]{homology}
\externaldocument[derived-]{derived}
\externaldocument[simplicial-]{simplicial}
\externaldocument[more-algebra-]{more-algebra}
\externaldocument[smoothing-]{smoothing}
\externaldocument[modules-]{modules}
\externaldocument[sites-modules-]{sites-modules}
\externaldocument[injectives-]{injectives}
\externaldocument[cohomology-]{cohomology}
\externaldocument[sites-cohomology-]{sites-cohomology}
\externaldocument[dga-]{dga}
\externaldocument[dpa-]{dpa}
\externaldocument[sdga-]{sdga}
\externaldocument[hypercovering-]{hypercovering}
\externaldocument[schemes-]{schemes}
\externaldocument[constructions-]{constructions}
\externaldocument[properties-]{properties}
\externaldocument[morphisms-]{morphisms}
\externaldocument[coherent-]{coherent}
\externaldocument[divisors-]{divisors}
\externaldocument[limits-]{limits}
\externaldocument[varieties-]{varieties}
\externaldocument[topologies-]{topologies}
\externaldocument[descent-]{descent}
\externaldocument[perfect-]{perfect}
\externaldocument[more-morphisms-]{more-morphisms}
\externaldocument[flat-]{flat}
\externaldocument[groupoids-]{groupoids}
\externaldocument[more-groupoids-]{more-groupoids}
\externaldocument[etale-]{etale}
\externaldocument[chow-]{chow}
\externaldocument[intersection-]{intersection}
\externaldocument[pic-]{pic}
\externaldocument[weil-]{weil}
\externaldocument[adequate-]{adequate}
\externaldocument[dualizing-]{dualizing}
\externaldocument[duality-]{duality}
\externaldocument[discriminant-]{discriminant}
\externaldocument[derham-]{derham}
\externaldocument[local-cohomology-]{local-cohomology}
\externaldocument[algebraization-]{algebraization}
\externaldocument[curves-]{curves}
\externaldocument[resolve-]{resolve}
\externaldocument[models-]{models}
\externaldocument[functors-]{functors}
\externaldocument[equiv-]{equiv}
\externaldocument[pione-]{pione}
\externaldocument[etale-cohomology-]{etale-cohomology}
\externaldocument[proetale-]{proetale}
\externaldocument[relative-cycles-]{relative-cycles}
\externaldocument[more-etale-]{more-etale}
\externaldocument[trace-]{trace}
\externaldocument[crystalline-]{crystalline}
\externaldocument[spaces-]{spaces}
\externaldocument[spaces-properties-]{spaces-properties}
\externaldocument[spaces-morphisms-]{spaces-morphisms}
\externaldocument[decent-spaces-]{decent-spaces}
\externaldocument[spaces-cohomology-]{spaces-cohomology}
\externaldocument[spaces-limits-]{spaces-limits}
\externaldocument[spaces-divisors-]{spaces-divisors}
\externaldocument[spaces-over-fields-]{spaces-over-fields}
\externaldocument[spaces-topologies-]{spaces-topologies}
\externaldocument[spaces-descent-]{spaces-descent}
\externaldocument[spaces-perfect-]{spaces-perfect}
\externaldocument[spaces-more-morphisms-]{spaces-more-morphisms}
\externaldocument[spaces-flat-]{spaces-flat}
\externaldocument[spaces-groupoids-]{spaces-groupoids}
\externaldocument[spaces-more-groupoids-]{spaces-more-groupoids}
\externaldocument[bootstrap-]{bootstrap}
\externaldocument[spaces-pushouts-]{spaces-pushouts}
\externaldocument[spaces-chow-]{spaces-chow}
\externaldocument[groupoids-quotients-]{groupoids-quotients}
\externaldocument[spaces-more-cohomology-]{spaces-more-cohomology}
\externaldocument[spaces-simplicial-]{spaces-simplicial}
\externaldocument[spaces-duality-]{spaces-duality}
\externaldocument[formal-spaces-]{formal-spaces}
\externaldocument[restricted-]{restricted}
\externaldocument[spaces-resolve-]{spaces-resolve}
\externaldocument[formal-defos-]{formal-defos}
\externaldocument[defos-]{defos}
\externaldocument[cotangent-]{cotangent}
\externaldocument[examples-defos-]{examples-defos}
\externaldocument[algebraic-]{algebraic}
\externaldocument[examples-stacks-]{examples-stacks}
\externaldocument[stacks-sheaves-]{stacks-sheaves}
\externaldocument[criteria-]{criteria}
\externaldocument[artin-]{artin}
\externaldocument[quot-]{quot}
\externaldocument[stacks-properties-]{stacks-properties}
\externaldocument[stacks-morphisms-]{stacks-morphisms}
\externaldocument[stacks-limits-]{stacks-limits}
\externaldocument[stacks-cohomology-]{stacks-cohomology}
\externaldocument[stacks-perfect-]{stacks-perfect}
\externaldocument[stacks-introduction-]{stacks-introduction}
\externaldocument[stacks-more-morphisms-]{stacks-more-morphisms}
\externaldocument[stacks-geometry-]{stacks-geometry}
\externaldocument[moduli-]{moduli}
\externaldocument[moduli-curves-]{moduli-curves}
\externaldocument[examples-]{examples}
\externaldocument[exercises-]{exercises}
\externaldocument[guide-]{guide}
\externaldocument[desirables-]{desirables}
\externaldocument[coding-]{coding}
\externaldocument[obsolete-]{obsolete}
\externaldocument[fdl-]{fdl}
\externaldocument[index-]{index}

% Theorem environments.
%
\theoremstyle{plain}
\newtheorem{theorem}[subsection]{Theorem}
\newtheorem{proposition}[subsection]{Proposition}
\newtheorem{lemma}[subsection]{Lemma}

\theoremstyle{definition}
\newtheorem{definition}[subsection]{Definition}
\newtheorem{example}[subsection]{Example}
\newtheorem{exercise}[subsection]{Exercise}
\newtheorem{situation}[subsection]{Situation}

\theoremstyle{remark}
\newtheorem{remark}[subsection]{Remark}
\newtheorem{remarks}[subsection]{Remarks}

\numberwithin{equation}{subsection}

% Macros
%
\def\lim{\mathop{\mathrm{lim}}\nolimits}
\def\colim{\mathop{\mathrm{colim}}\nolimits}
\def\Spec{\mathop{\mathrm{Spec}}}
\def\Hom{\mathop{\mathrm{Hom}}\nolimits}
\def\Ext{\mathop{\mathrm{Ext}}\nolimits}
\def\SheafHom{\mathop{\mathcal{H}\!\mathit{om}}\nolimits}
\def\SheafExt{\mathop{\mathcal{E}\!\mathit{xt}}\nolimits}
\def\Sch{\mathit{Sch}}
\def\Mor{\mathop{\mathrm{Mor}}\nolimits}
\def\Ob{\mathop{\mathrm{Ob}}\nolimits}
\def\Sh{\mathop{\mathit{Sh}}\nolimits}
\def\NL{\mathop{N\!L}\nolimits}
\def\CH{\mathop{\mathrm{CH}}\nolimits}
\def\proetale{{pro\text{-}\acute{e}tale}}
\def\etale{{\acute{e}tale}}
\def\QCoh{\mathit{QCoh}}
\def\Ker{\mathop{\mathrm{Ker}}}
\def\Im{\mathop{\mathrm{Im}}}
\def\Coker{\mathop{\mathrm{Coker}}}
\def\Coim{\mathop{\mathrm{Coim}}}

% Boxtimes
%
\DeclareMathSymbol{\boxtimes}{\mathbin}{AMSa}{"02}

%
% Macros for moduli stacks/spaces
%
\def\QCohstack{\mathcal{QC}\!\mathit{oh}}
\def\Cohstack{\mathcal{C}\!\mathit{oh}}
\def\Spacesstack{\mathcal{S}\!\mathit{paces}}
\def\Quotfunctor{\mathrm{Quot}}
\def\Hilbfunctor{\mathrm{Hilb}}
\def\Curvesstack{\mathcal{C}\!\mathit{urves}}
\def\Polarizedstack{\mathcal{P}\!\mathit{olarized}}
\def\Complexesstack{\mathcal{C}\!\mathit{omplexes}}
% \Pic is the operator that assigns to X its picard group, usage \Pic(X)
% \Picardstack_{X/B} denotes the Picard stack of X over B
% \Picardfunctor_{X/B} denotes the Picard functor of X over B
\def\Pic{\mathop{\mathrm{Pic}}\nolimits}
\def\Picardstack{\mathcal{P}\!\mathit{ic}}
\def\Picardfunctor{\mathrm{Pic}}
\def\Deformationcategory{\mathcal{D}\!\mathit{ef}}

%Dani's additions
\usepackage{graphicx} %figures


\begin{document}

\title{Seminars}
\maketitle

\phantomsection
\label{section-phantom}
\hfill
\href{http://github.com/danimalabares/stack}{github.com/danimalabares/stack}

\tableofcontents

\section{Neutrinos}
\label{section-neutrinos}

\noindent
Hiroshi Nunokawas, PUC-Rio.
Friday Seminar, Seminar Name. 
June 27, 2025.

\medskip
{\bf Abstract.} Hiroshi will come and tell us everything we (not) wanted to know
about these mysterious particles, and are not going to be afraid to ask. In
particular, about the neutrino oscillation, and the great matrices.

\bigskip\noindent

Protons and neutrons have very similar mass of $m_p\approx 940$ MeV, while
electrons have mass of $m_e \approx 0.5$ MeV. MeV is $10^{6}$ electronvolts,
where one eV is approximately $1.6\times 10^{-19}$ J. This is standard in high
energy physics, they use electronvolts instead of Joules. Recall that
2J=1N$\times$1m.

Most of the things we see are protons since they are so much larger than
electrons. But protons nor neutrons are elementary particles.

Here's the standard model:

\medskip

\begin{tabular}{c c c c}
Quarks & $\begin{bmatrix} u\\d \end{bmatrix}_L $ & $\begin{bmatrix} c\\s
\end{bmatrix}_L$ & $\begin{bmatrix} t\\b \end{bmatrix}_L$\\
Leptons & $\begin{bmatrix} \nu_e\\e^- \end{bmatrix}_L $ & $\begin{bmatrix}
\nu_\mu\\ \mu^-
\end{bmatrix}_L$ & $\begin{bmatrix} \nu_\tau\\ \tau^- \end{bmatrix}_L$\\
Generation & 1st & 2nd & 3rd\\
Bosons & $g$, $\gamma$ , $\omega^\pm$, $z$ \\
Higgs Bosson & $H$
\end{tabular}

\medskip

It is very particular that nature repeats itself three times. The $L$ in those
matrix actually means left-handed, and accounts for chirality. Only left-handed
fermions have weak interaction. Right-handed have electromagnetic interaction,
gravitational interaction, but not weak interaction.

And then there's neutrinos. They have negative helicity (chirality). Being
left-handed, mathematically, means to have helicity $-1$. I think this means
that the spin is left-handed. But chirality and helicity are not the same:
helicity is observer-dependent, and chirality is not. Almost all neutrinos we
can see (\% 99.99999…) have negative helicity, but not all of them.

Consider the following:
$$
n+\nu_e\leftrightarrow p+e^-
$$
But it's not completely correct: we'd better put $d$ instead of $n$, and $u$
instead of $p$: the $d$ and $u$ quarks, instead of the neutrons and protons.

Now consider the following reaction: a neutron decays into a proton, an electron
and an antineutrino:
$$
n\to p+e^+\overline{\nu}_e
$$
Protons is very stable, that's why we are here. But neutron decays in only 15
minutes.

By experimental data, we can conclude that neutrinos' mass is consistent with
zero. But if they have mass, it should be much smaller than the electron's $m_v
\leq 0.5$ eV. And the electron is already the lightest fermion!

If the mass of the neutrino was zero, i.e. $m_\nu=0$, then $v_\nu=c$ in vacuum,
which would imply that 
$$
\xymatrix{
\nu_e\ar[r]^{L}&\nu_e\ar[r]&\nu_e\\
0:00&0:00&0:00
}
$$
meaning: time doesn't pass! And this means the state of the particle cannot
change.

\section{Spheres with minimal equators}
\label{section-spheres-with-minimal-equators}

\noindent
Lucas Ambrozio, IMPA.
Differential Geometry Seminar, IMPA. 
June 24, 2025.

\medskip
{\bf Abstract.} We will discuss the connection between Riemannian metrics on the
sphere with respect to which all equators are minimal hypersurfaces, and
algebraic curvature tensors with positive sectional curvatures.

\bigskip\noindent

\begin{definition}
\label{definition-equator}
An {\it $(n-k)$-equator} orthogonal to $\Pi$ is
$$
\Sigma_\Pi:=\{p\in\mathbb{S}^n:\left<p,x\right>=0\forall x\in\Pi\}
$$
for $\Pi$ a $k$-dimensional linear subspace of $\mathbb{R}^{n+1}$.
\end{definition}

\begin{remark}
\label{remark-equators-are-totally-geodesic-with-usual-metric}
Equators are totally geodesic hypersurfaces with the usual sphere metric, which
implies they are minimal hypersurfaces.
\end{remark}

{\bf Problem.} Characterize the set $\mathcal{M}_k(U)$ of metrics $g$ on an 
open set $U\subset\mathbb{S}^n$ such that all $k$-equators $\Sigma_\Pi$ with
 $\Sigma\cap U\neq\emptyset$ yield are minimal hypersurfaces $\Sigma\cap U$ 
on $(U,g)$.

\begin{remark}
\label{remark-why-not-Rn}
This problem can be thought of as a problem of finding metrics on $\mathbb{R}^n$
such that $k$-planes are minimal. To see why project the $k$-equators to
$T_p\mathbb{S}^n$ and pullback those metrics to the sphere.
\end{remark}

\medskip\noindent

Let $g\in\mathcal{M}_k(U)$ for $U\subset\mathbb{S}^n$ open and $n\geq 2$.

\begin{theorem}[Beltrami, Schäfli]
\label{theorem-Beltrami-Schafli}
If $k=1$ then $g$ has constant sectional curvature.
\end{theorem}

\begin{theorem}[Hongan]
\label{theorem-Hongan}
If $1<k<n-1$ then $g$ has constant sectional curvature.
\end{theorem}

Then Hongan also managed to produce a classification of these metrics for
$k=n-1$.

\begin{remark}
\label{remark-linear-invertible-preserving-equators-is-in-Mg}
If $T\in \text{GL}(n+1,\mathbb{R})$, then
\begin{align*}
\varphi: \mathbb{S}^n &\longrightarrow \mathbb{S}^n \\
x &\longmapsto \frac{Tx}{|Tx|}
\end{align*}
is a diffeomorphism that maps $k$-equators into $k$-equators. Thus if
$g\in\mathcal{M}_k(\mathbb{S}^n)$ then so is $\varphi(T)^*g$.
\end{remark}

\begin{theorem}
\label{theorem-equivariant-bijection}
There exists a $\text{GL}(n+1,\mathbb{R})$ equivariant bijection
$$
\mathcal{M}_{n-1}(\mathbb{S}^n)\leftrightarrow \text{Curv}_+(\mathbb{R}^{n+1})
$$
where the set on the right-hand-side is the set of algebraic curvature tensors
(also called curvature-like, i.e. with the same symmetries as the Riemannian
curvature tensor) on $\mathbb{R}^{n+1}$ with positive sectional curvature.

The group action is given as follows for $T\in\text{GL}(n+1,\mathbb{R})$:
$$
(R\cdot T)(x,y,z,w)=\frac{1}{|\det(T)|^{\frac{1}{n+1}}}R(Tx,Ty,Tz,Tw)
$$
\end{theorem}

The point is that $\text{Curv}_+(\mathbb{R}^{n+1})$ is an open cone on a linear
space. Here are two simple corollaries:

\begin{lemma}
\label{lemma-corollaries}
\begin{enumerate}
\item $\mathcal{M}_{n+1}(\mathbb{S}^n)$ is in bijection with an open positive
cone of an $\frac{n(n+2)(n+1)^2}{12}$-dimensional real vector space.
\item Every metric on $\mathcal{M}_{n-1}(\mathbb{S}^n)$ is invariant by the
antipodal map.
\end{enumerate}
\end{lemma}

{\bf Algorithm.} From any $R\in \text{Curv}_p(\mathbb{R}^{n+1})$ we obtain a
symmetric positive definite (positive-definitiness comes from the positiveness
of the curvature of $R$) 2-tensor  $k_R$ satisfying
$$
(k_R)_p(v,v)=R(pv,pv)>0
$$
Also, $k_R$ has the  {\it Killing property},
i.e. that $\overline{\nabla}k(X,X,X)=0$ for all
$X\in\mathfrak{X}(\mathbb{S}^n)$.

Then we define a positive function on
 $\mathbb{S}^n$ by
\begin{equation}
\label{equation-def-DR}
D_R:=\left(\frac{d\text{Vol}_{k_R}}{dV_g}\right)^{\frac{4}{n-1}}
\end{equation}
and finally a Riemannian metric on $\mathbb{S}^n$ in
$\mathcal{M}_{n-1}\mathbb{S}^n$ by
$$
g_R=\frac{1}{D_R}k_R
$$
And to go back, for $g \in \mathcal{M}_{n-1}(\mathbb{S}^n)$ define a positive 
function on $\mathbb{S}^n$
$$
F_g:=\left(\frac{dV_g}{dV_{\overline{g}}}\right)^{\frac{4}{n-1}}
$$
Then let $k_g:=\frac{1}{F_g}g>0$, which is a positive definite Killing 2-tensor,
from which we may define $R_g\in\text{Curv}_+(\mathbb{R}^{n+1})$ with
$R_g(pv,pv)=(k_g)_p(v,v)$ for all $p,v\in T\mathbb{S}^n$.

More corollaries:
\begin{lemma}
\label{lemma-corollaries2}
\begin{enumerate}
\item $g\in\mathcal{M}_{n-1}(\mathbb{S}^n)$ is analytic because it is a Killing
tensor on $\mathbb{S}^n$, which are well-known.
\item If $g$ is left-invatiant on $\mathbb{S}^3$, seen as unit quaternions, then
$g\in\mathcal{M}_2(\mathbb{S}^3$. Moreover, for $a\geq b\geq c>0$,
$$
aL_i\odot L_i+bL_j\odot L_j+cL_k\odot L_k=k
$$
is Killing, $k>0$,  $D_k$ constant and thus $g=\frac{1}{\text{const.}}k
\in\mathcal{M}_2(\mathbb{S}^3)$.
\item $R$ curvature tensor of $(\mathbb{C}P^{2},g_{FS})$. We may not remember
what's the curvature tensor, but we know the sectional curvature is $1\leq
\text{sec}(R)\leq 4$,
$$
(k_R)_p(v,w)=\overline{g}(v,w)+3\overline{g}(Jp,v)\overline{g}(Jp,w)
$$
and $D_R=4^{\frac{4}{3-1}}=4$, so that by \ref{equation-def-DR} we obtain 
$g_R=\frac{1}{4}k_R$, which is a Berger
metric on $\mathbb{S}^3$ with scalar curvature 0.
\end{enumerate}
\end{lemma}

\bigskip
Now define
$$
\Sigma_V=\{p\in\mathbb{S}^n:\left<p,v\right>=0\}=V^{-1}(0)
$$
where $V(x):=\left<x,v\right>$ for all $x\in\mathbb{S}^n$. Then the normal
vector field is $\nabla V/|\nabla V|_g$, and the second fundamental form is
given by
$$
A=\frac{1}{|\nabla V|_g}\text{Hess}_gV
$$
and its mean curvature by
\begin{equation}
\label{equation-mean-curvature-for-level-sets-V}
H=\frac{1}{|\nabla V|_g}\left(\Delta_gV-\text{Hess}_gV\left(\frac{\nabla
V}{|\nabla V|},\frac{\nabla V}{|\nabla V|}\right)\right)
\end{equation}
For every $v \in \mathbb{S}^n$ and $p\in\Sigma_V$, we see that $H_{\Sigma_V}=0$
iff
$$
|\nabla V|^2_g(p)\Delta_gV(p)-\text{Hess}_gV(\nabla V(p),\nabla V(p))=0
$$
And for $\overline{g}$,
$$
\text{Hess}_{\overline{g}}V+V\overline{g}=0\implies
\text{Hess}_{\overline{g}}V(X,X)=0
$$
for all $X\in T_p\mathbb{S}^n$ and $p\in\Sigma_v$. Then
$$
J_g(X,Y,Z)=g(\nabla_XY-\overline{\nabla}_XY,Z)
$$
$$
J_g(X,Y,\nabla V)=\text{Hess}_{\overline{g}}-\text{Hess}
$$
{\bf Problems.}
\begin{enumerate}
\item Similar story for $\mathbb{C}P^{n},\mathbb{H}P^n$?
\item Complete metrics on $\mathbb{R}^n$ with minimal hyperplanes.
\item Find geometric invariants of metrics on $\mathcal{M}_{n-1}(\mathbb{S}^n)$
(may be useful to study $(M^n,g)$, $n\geq 4$, $\text{sec}>0$.
\end{enumerate}

\section{Smoothable compactified Jacobians of nodal curves}
\label{section-smoothable-compactified-Jacobians-of-nodal-curves}

\noindent
Nicola Pagani, University of Liverpool and Bologna. 
Seminar of Algebraic Geometry UFF. 
August 20, 2025.

\medskip
{\bf Abstract.} Building from examples, we introduce an abstract notion of a
'compactified Jacobian' of a nodal curve. We then define a compactified Jacobian
to be 'smoothable' whenever it arises as the limit of Jacobians of smooth
curves. We give a complete combinatorial characterization of smoothable
compactified Jacobians in terms of some 'vine stability conditions', which we
will also introduce. This is a joint work with Fava and Viviani.

\medskip\noindent



Let $C$ be a smooth curve and $d \in \mathbb{Z}$. Define
$$
J^d_C=\{L:\text{$L$ is a line bundle of degree $d$}\}/\sim
$$
which is a smooth projective variety of dimension $g(C)$.

If $C$ is nodal we still can consider $J^d_C$.
\begin{enumerate}
\item One connected component. Then the Jacobian is $\mathbb{P}^1$ minus two
points. This is not universally closed, so it is not proper.
\item Two components intersecting at one point. 
The pullback of the normalization splits the degree in
intintely many ways, givieng that $J^{-1}_C$ is an infinite set of points. This
is not of finite type, so it is not proper.
\item The curve has two components intersecting at two points. 
This gives $J^{-2}_C$,
 which is a mixture of the two former items. (Probably not proper too.)
\end{enumerate}

\medskip\noindent
Now consider
$$
\text{TF}_C^d=\{\mathcal{F}:
\text{coherent on $C$, torsion-free, rank-1 on $C$}\}/\sim
$$
This satisfies the existence oart of the valu point of properness.

Now we consider the moduli. Now we consider the ideal sheaf of the (singular?)
point(s?):
\begin{enumerate}
\item One component. The stack is proper!
\item Two components intersecting once. Now we get stacky points, 
$x=[\bullet/\mathbb{G}_m]$. These points have generic stabilizer. The resulting
stack is not separated because a morphism of a curve, say $\mathbb{P}^1$ minus a
point … there are infinitely many ways to extend a morphism from this thing to a
line bundle. So you cannot include any of these stacky points. Recall that a
sheaf is {\it simple} if its automorphism group is $\mathbb{G}_m$.
\item  The ideal sheaf of both nodes 
$\mathcal{I}(N_1,N_2)$ has a positive dimensional automorphism
group. The stack is not proper.
\end{enumerate}

\begin{definition}
\label{definition-fine-compactified-Jacobian}
A {\it fined compactified Jacobian} of $C$ is an open connected substack of
$\text{TF}^d(C)$ that is also proper.
\end{definition}

\begin{remark}
\label{remark-algebraic-space}
This thing is automatically an algebraic space.
\end{remark}

\begin{definition}
\label{definition-compactified-Jacobian}
A {\it compactified Jacobian} is an open connected of $\text{TF}^d(C)$ that
admits a proper, good moduli space.
\end{definition}

Consider the Artin stack $\mathfrak{X} \xrightarrow{\Gamma}X$ […]
 is a {\it good
moduli space} if 
\begin{enumerate}
\item Every moduli factors
$$
\xymatrix{
\mathfrak{X} \ar[r]\ar[rd]&\mathcal{I}\text{ (ACC. space)}\\
& X
}
$$
\item $\pi_*\mathcal{O}_{\mathfrak{X}}=\mathcal{O}_X$.
\end{enumerate}

\medskip\noindent
We expect to find a notion of stability condition to produce these things 
[…] $[\bullet/ \mathbb{G}_m]$ would be the polystable representative.

\begin{definition}
\label{definition-smoothable-compactified-Jacobian}
A compactified Jacobian $\overline{J_C}$ is {\it smoothable} if all smoothings 
$\mathcal{C}\to \Delta=\{0,\eta\}$ (with $\mathcal{C}_0=C$), 
$$
J^d_{\mathcal{C}_\eta}\cup C \to \overline{J_C}
$$
is proper.
\end{definition}

\medskip\noindent

\begin{definition}
\label{definition-connected}
Let $X$ be a curve.
$$
\text{BCON}(X)=\{Y \subseteq X\text{s.t. }Y,Y^c \text{ are connected}\}
$$
\end{definition}

\begin{definition}
\label{definition-v-satbility}
A $v$-curve is a generalization of items (2) and (3) in the lists above 
[it looks like two long snakes $\sim$ that intersect several times, and $t$ is
the number of nodes]. 
A {\it $v$-condition} is a pair $n=(n_1,n_2)$ such that 
$$
n_1+n_2=\begin{cases}
d+1-t\qquad &\text{we say the s.c. is nondegenerate} \\
d-t\qquad &\text{degenerate}
\end{cases}
$$
$\mathcal{F}$ on $X$ is  {\it $n$-(semi)stable} if 
$\text{deg}\mathcal{F}_{X_i}>n_i$ ($\text{deg}\mathcal{F}_{X_i}\geq n_i$) 
for $i=1,2$.
\end{definition}
$\mathcal{F}_{X_i}=\mathcal{F}|_{X_i}$ torsion.
$$
\text{deg}(\mathcal{F}_{X_i})+\text{deg}(\mathcal{F}_{X_2})
=d-|\text{sing}(F)|.
$$
Then 
$$
\overline{J_C}(n)=\{\mathcal{F}\text{ is semistable}\},
$$
a smooth compact Jacobian.

\begin{definition}
\label{definition-n-degeneracy}
A {\it degeneratation} of $v$-stab. on $X$ is $n:\text{BCON}(X) \to \mathbb{Z}$
such that
\begin{enumerate}
\item 
$$
n_Y+n_{Y^c}+|Y \cap Y^c|=
\begin{cases}
d+1\qquad &\text{ we say $Y$ is $n$-nondegenerate} \\
d\qquad &\text{$Y$ is $n$-degenerate}
\end{cases}
$$
\item $Y_i$ no pa. common component $n_{Y_1}+n_{Y_2}+\ldots$
$$
\xymatrix{
&Y_1\ar[dr]\ar[dl]\\
Y_2\ar[ur]\ar[rr]&&Y_3\ar[ll]\ar[ul]
}
$$
\end{enumerate}
\end{definition}

\begin{theorem}[-, et al]
\label{theorem-bijection-stability-conditions-and-nodal-curves}
(bijection between stability conditions and nodal curves) The map
$$
\left\{ \substack{\text{sm. comp.} \\ \text{Jac of $X$}} \right\} \to
\left\{ \substack{\text{$v$-stab.} \\ \text{cond. of $X$}} \right\} 
$$
$$
n\mapsto \overline{J_X}(n)=\{n\text{-semistable sheaves}\}
$$
is a bijection. (The arrow should be from right to left!)
\end{theorem}

F. Viviani had proved it for fine compact Jacobians.

\section{Equivariant spaces of matrices of constant rank}
\label{section-equivariant-spaces-of-matrices-of-constant-rank}

\noindent
Ada Boralevi, France.
Algebraic Geometry Seminar, IMPA. 
August 27, 2025.

\medskip
{\bf Abstract.} A space of matrices of constant rank is a vector subspace V, say
of dimension n+1, ofthe set of matrices of size axb over a field k, such that
any nonzero element of V has fixed rank r. It is a classical problem to look for
different ways to construct such spaces of matrices. In this talk I will give an
introduction up to the state of the art of the topic, and report on my latest
joint project with D. Faenzi and D. Fratila, where we give a classification of
all spaces of matrices of constant corank one associated to irreducible
representation of a reductive group.

\medskip\noindent

We are interested in vector spaces $U \subset \text{Mat}_{m,n}(\mathbb{C})$,
with $m \leq n$, of {\it constant rank, i.e. such that for all $f \in U$,
$r:=\text{rank}f$ is the same.

Let $\ell(r,m,n):=\text{max}\dim U:U$ is of rank $r$.

\medskip\noindent
{\bf Questions.}
\begin{enumerate}
\item $\ell(r,m,n)=$? In general not known.
\item Find relations among $\ell, r,m$ and $n$.
\item Construction of examples and classification.
\end{enumerate}

\begin{example}
\label{example-what-we-know}
\begin{enumerate}
\item $\ell(1,m,n)=n$,
$$
\begin{pmatrix}
x_1&x_2&\cdots &x_n\\
\\
\\
\\
\end{pmatrix}
$$
\item $\text{rank}=2$? There are two cases ([Atkinson '83], [Eisenbud-Haus '88])
\begin{itemize}
\item Compression spaces,
$$
\begin{pmatrix}
*&*&\cdots &*\\
*&0&\cdots & 0\\
\vdots &\\
*&0&\cdots&0
\end{pmatrix}
$$
\item Skew-symmetric matrices of $3 \times 3$.
\end{itemize}
\item $\ell(r,m,n) \geq n-r+1$. Because you can put a matrix of
$m\times n$ (with the first $r$ rows that can have nonzero entries):
$$
\begin{pmatrix}
x_1 &  x_2 &  \cdots & & x_{n-r+1}\\
& x_1 & x_2 & \cdots &\\
\\
\\
\\
\end{pmatrix}
$$
\end{enumerate}
\end{example}

\begin{theorem}[Westwick '86]
\label{theorem-Westwick}
\begin{enumerate}
\item $n-r+1 \leq \ell(r,m,n) \leq 2n-2r+1$.
\item If $n-r+1 \not| \frac{(m-1)!}{(r-1)!}\implies \ell$
\end{enumerate}
\end{theorem}

We can see these spaces as (subvarieties?) of determinantal varieties
$M_n=\{f \in \text{Mat}_{m,n}(\mathbb{C}):\text{rank}(f) \leq  r\}$. 
[Interpretation via secant varieties inside Segre embedding.]

\medskip\noindent
Consider a map $\varphi: U \to \Hom(V,W)$.
Then $\varphi \in U^* \otimes  V^* \otimes  W$.
We get that
$\varphi$ is of constant rank if and only if
some kernel, image and cokernel are vector bundles.

\medskip\noindent
{\bf Focus of today.} What happens when $U,V,W$ are irreducible
representations of a complex reductive group  $G$?

\medskip\noindent
{\bf Question.} What is the natural equivariant morphism
$$
U \to \Hom(V,W)=V^* \otimes  W
$$
of constant rank?

\medskip\noindent
Consider the case of $G=\text{SL}_2$. All the 
irreducible representations (which are self-dual) are given by
$V(m-1)\cong \mathbb{C}[x,y]_{\text{deg}=m-1}$.

Recall the Clebsh-Gordon decomposition ($m \leq  n$)
$$
V(m-1)\otimes  V(n-1)=\bigoplus_{i=0}^{m-1}V(n-m+2i)
$$

\begin{theorem}[B-Faenzi-Lella '22]
\label{theorem-BFL22}
$$
V(n+m-2)\hookrightarrow \Hom(V(m-1),V(n-1))
$$
is of constant rank (corank 1) if and only if
$$
n-m+2i|m-1
$$
\end{theorem}

\begin{theorem}[B. Faenzi, Fratila '25]
\label{theorem-BFF25}
Let $V(\nu)$, $V(\mu)$ and $V(\lambda)$ be irreducible representations
of a comple reductive group $G$, with 
$$
\dim(V(\mu)) \leq \dim(V(\lambda))_-
$$
If there exists a morphism of representations
$$
\varphi: V (\nu) to \Hom (V (\lambda),V(\mu))
$$
then $\varphi$ is of constant corank 1 if and only if 
there exists a simple root $\alpha_i$ such that
\begin{enumerate}
\item $\lambda=\mu+\nu-\alpha_i$,
\item $\nu$ is a multiple of $\nu$,
\item  $\nu$ is a multple of $\omega_i$
\end{enumerate}
\end{theorem} 

\section{On wrapped Floer homology barcode entropy and hyperbolic sets}
\label{section-wrapped-Floer-homology-barcode-entropy-and-hyperbolic-sets}

\noindent
Rafael Fernandes, UC Santa Cruz.
Differential Geometry Seminar, IMPA. 
September 4, 2025.

\medskip {\bf Abstract.} In this talk, we will discuss the interplay between the
wrapped Floer homology barcode and topological entropy. The concept of barcode
entropy was introduced by Çineli, Ginzburg, and Gürel and has been shown to be
related to the topological entropy of the underlying dynamical system in various
settings. Specifically, we will explore how, in the presence of a topologically
transitive, locally maximal hyperbolic set for the Reeb flow on the boundary of
a Liouville domain, barcode entropy is bounded below by the topological entropy
restricted to the hyperbolic set.

\medskip\noindent

Let $M^n$ be a manifold.
$\omega \in \Omega^2(M)$ is {\it symplectic} if $d \omega=0$ and it is
nondegenerate.

\begin{example}
\label{example-Rn-is-symplectic}
$\mathbb{R}^n$ is symplectic with canonical Darboux form.
\end{example}

Recall the definition of Hamiltonian vector field
associated to a function $H \in C^\infty(M)$.

\begin{definition}
\label{definition-nondegenerate-diffeomorphism}
A diffeomorphism $\varphi:M \to M$ is called
{\it non-degenerate} if $\Phi(\varphi)\cap\Delta \subset M\times M$
(pitchfork, i.e. transversal intersection!).
\end{definition}

Let $M^{2n}$ be a closed symplectic manifold. Arnold's conjecture says
\begin{enumerate}
\item If $\varphi=\varphi_H$ (Hamiltonian flow) is nondegenerate, then
$$
\# \text{Fix}(\varphi_H) \geq \sum_{i=0}^{2n}\dim H_i(M,k)=\dim H_*(M,k)
$$
\item If $\varphi=\varphi_H$ is degenerate, then
$$
\# \text{Fix}(\varphi)\geq \text{Cl}(M)+1
$$
where $\text{Cl}(M)$ is the maximum number of homology classes
we can add before getting to zero.
\end{enumerate}

Why do we care? Because
$$
\# \text{Fix}(\varphi_H)\leftrightarrow\{\text{1-periodic 
orbits of $X_H$}\}
$$

Idea by Floer. Construct an invariant that would 
say something about periodic orbits.

\medskip\noindent
{\bf Question.} Can Floer theory capture other ``dynamical information''? 
(Other than the periodic orbits.)

\medskip\noindent
A {\it persistence module} is a pair $(V,\Pi)$, where $V=\{V_t\}_{t \in
\mathbb{R}}$ is a family of $\mathbb{F}$-vector spaces
and $\Pi=\{\Pi_{st}\}_{s \leq  t}$ is a family of maps such that
\begin{enumerate}
\item $\Pi_{ss}=, \Pi_{ts}\circ \Pi_{rt}=\Pi_{rs}$.
\item $\exists  s \subset \mathbb{R}$ such that
$\Pi_{s t}$ is an isomorphism for $s,t$ in the same connected component of
$R\setminus S$.
\item $\Pi_{s t}$ have finite rank.
\item $\exists s_0$ $V_s=\{0\}$, $s \leq s_0$.
\item $V_t= \lim_{s \to t} V_s$ (lower limit!!)
\end{enumerate}

\begin{theorem}
\label{theorem-persistence-module-is-sum-of-integral}
Any persistence module is a sum of integral persistence modules,
$$
(V,\Pi) \cong \bigoplus_{I \in B(V)}F(I).
$$
\end{theorem}

\begin{example}
\label{example-hart-and-sphere}
Heart and sphere. There is a noise in the persistence module
of the heart due to an unnecessary critical point.
\end{example}

$(M^{2n},\omega)$ a {\it Liouville domain} is a compact
symplectic manifold and $X \in \mathfrak{X}(M)$ 
with $X \cap\partial M$ (pitchfork, i.e. transversal intersection!)
pointing outwards
and preserved by the symplectic form, i.e. $\mathcal{L}_X \omega=\omega$
($\omega=d \alpha$).

When we restrict $\omega$ to the boundary,
we obtain a contact form and get some
interesting dynamics.

A Lagrangian $(L,\partial L) \subset (M,\partial M)$ 
is {\it asympotically conical} if
\begin{enumerate}
\item $\partial L \subset \partial M$ is Legendrian.
\item $L \cap [1-\varepsilon,1] \times \partial M=
[1-\varepsilon,1] \times \partial L$.
\end{enumerate}

\begin{remark}
\label{remark-Hamiltonian}
Take a Hamiltonian $H: \hat{M}\to \mathbb{R}$ such that
$$
\begin{cases}
H(r,x)=h(r)\qquad &r=1 \\
H(r,x)=rT-B\qquad &
\end{cases}
$$
then $X_H=h'(r)R_\alpha$.

For $L_0,L,A\subset$ Lagrangians, $H$ linear at infinite,
then $A_{H}^{L_0 \to L_1}$, 
\begin{align*}
A_{H}^{L_0 \to L_1}: P_{L_0\to L_1}  &\longrightarrow \mathbb{R} \\
\gamma &\longmapsto \int_0^1 \gamma^* \alpha- \int_0^1H(x(t))dt
\end{align*}
where $P_{L_0\to L} =\{\gamma:[0,1] \to \hat{M}:\gamma(0) \in L_0,
\gamma(1) \in L_1\}
$ is the set of chords.
\end{remark}

\begin{remark}
\label{remark-1-chords-are-crticial-points}
$\text{crit}(A_{H}^{L_0 \to L_1})=\{\text{1-chords of $X_H$ from 
 $L_0$ to $L_1$}\}$.
\end{remark}

Putting a metric on $P_{L_0 \to L_1}$ we can consider $\varphi:\mathbb{R} \times
[0,1] \to \hat{M}$, solutions of some PDE which is some kind
of generalization of a gradient, $- \nabla A_{H}^{L_0 \to L_1}$.
These solutions can be put in a moduli space
$$
\tilde{\mathcal{M}}(x_-,x_+,H,J)=\{\varphi\text{ solutions s.t. …}\}
$$
Then we define a boundary operator $\partial$.
\begin{theorem}
\label{theorem-homology}
$\partial^2=0$
\end{theorem}

So that we have a homology, called {\it wrapped Floer homology}
$HW^t(H,L_0,L_1,J)$

\begin{remark}
\label{remark-embedding}
We have $H \leq  K \rightsquigarrow HW^t(H,L_0,L_1,J) \to
HW^t(K,L_0,L_1,K)$.
\end{remark}

\begin{definition}
\label{definition-direct-limit}
For $t \geq 0$
$$
HW^t(M,L_0,L_1)=\underline{\lim }_H HW^t(H,L_0,L_1,J)
$$
\end{definition}

(Where we have taken direct limit.) 

Taking direct limit of the homology,
we make sure the homology theory is independent of the
choice of objects (I think, complex structure and Hamiltonian)
we used to construct it.


\begin{proposition}
\label{proposition-persistence-module}
$t \to HW^t(M,L_0,L_1)$ is a peristence module $B(M,L_0,L_1).$
\end{proposition}

Finally we can define {\it barcode entropy}. Fix $\varepsilon>0$,
$t \geq 0$,
$$
b_\varepsilon(M,L_0,L_1,t)=
\# \{\text{ of bars in }B(M,L_0,L_1) \text{with length $\geq \varepsilon$ 
and start before $t$}\}
$$
Then
$$
\bar{h}^{HW}(M_0,L_0,L_1)=
\lim_{\varepsilon \to 0} \text{lim sup}_{t \to \infty} 
\frac{\text{log}^+(b_\varepsilon(M,L_0,L_1,t)}{t}
$$

\medskip\noindent
Consider a contact manifold $(\Sigma,\lambda,L_0,L_1)$ 
and $A$ the Lagrangians. (Example raising question of filling.)

\begin{theorem}[M '24]
\label{theorem-entropy-is-independent-of-filling}
$\bar{h}^{HW}$ is independent of the filling.
\end{theorem}

\begin{theorem}[M '24]
\label{theorem-bound}
$\bar{h}^{HW}(M,L_0,L_1) \leq  h_{\text{top}}(\alpha)$.
\end{theorem}

\begin{theorem}[M '25]
\label{theorem-restriction}
Consider $(M,L_0,L_1)$. 
Let $K$ be a compact topologically transitive hyperbolic set
for the Reeb flow $\alpha$. Assume $W_\delta ^s (q) \subset \partial L_0$,
$W_\delta^s(p) \subset \partial L_1$. Then
$$
\bar{h}^{HW}(M,L_0,L_1)\geq  h_{\text{top}}(\alpha |_{K})>0.
$$
\end{theorem}
Which says it captures dynamics beyond unconditional phenomena.
In lower dimensions these tend to coincide, but in higher dimension we don't
know. This is related to
 ``the sup over hyperbolic sets […]''

Here's a conjecture:
$$
\text{sup}_{L_0,L_1}\bar{h}^{HW}(M,L_0,L_1)=h_{\text{top}}(\alpha).
$$
\medskip\noindent
{\bf Extra comments.} 
One of the aims is to describe topological entropy $h_\text{top}$ 
using Floer theory. Theorems by Çineli-Ginzburg-Gürel show bounds
of topological entropy and barcode entropy (one of which is for
hyperbolic sets).

There is a notion of  {\it admissible Hamiltonian and Reeb vector fields}
which is related to some asymptotical behaviour  ``linear at infinity''.
I understand that admissible vector fields give 
the interesting chords for the Floer homology construction.

\section{Revisiting cotangent bundles}
\label{section-revisiting-cotangent-bundles}

\noindent
Mieugel Cueca, KU Leuven.
Symplectic Geometry Joint Seminar, IMPA. 
September 5, 2025.

\medskip {\bf Abstract.} Cotangent bundles provide key examples of symplectic
manifolds. On the other hand, one can think of Lie groupoids as generalizations
of manifolds. In this context, Alan Weinstein constructed their cotangent
bundles and proved that they are so-called symplectic groupoids. In this talk, I
will recall this construction and explain what happens when one replaces a Lie
groupoid with a Lie 2 (or n)-groupoid. If time permits, I will exhibit some of
their main applications. This is joint work with Stefano Ronchi.

\medskip\noindent

Recall the basic properties of the cotangent
bundle $T^*M$ for a symplectic manifold:
\begin{enumerate}
\item It's a vector bundle.
\item $\left<\cdot,\cdot\right>:TM \otimes T^*M \to \mathbb{R}_M$ 
is the dual pairing.
\item $\omega_{\text{can}}\in \Omega^2(T^*M)$ is symplectic.
\item $\mathcal{L}_\varepsilon \omega_{\text{can}}=\omega_{\text{can}}$,
$\varepsilon$ Euler vector field.
\end{enumerate}

\medskip\noindent
{\bf Main goal.} Reproduce the above for Lie $n$-groupoids.

For $n=0$ we get the above situation. For $n=1$ [Duzud-Weinstein], [Prezun],
for $n\geq 2$ ? and we care about $n=2$.

\begin{definition}
\label{definition-groupoid}
A {\it Lie $n$-groupoid} $\mathcal{G}:\Delta^{\text{op}}\to \mathsf{Man}$ 
such that
$$
P_{e,j}:\mathcal{G}_{\ell}\to \Lambda_j^\ell \mathcal{G}
$$
are surjective submersions $\forall  \ell,j$ 
are differomorphisms for $\ell > n$.
\end{definition}

\begin{remark}
\label{remark-generators}
This sort of manifolds-valued presheaf category is generated by
\begin{align*}
d_i^\ell:\mathcal{G}_\ell \to \mathcal{G}_{\ell-1}&\text{face maps}
0 \leq  i,j \leq  \ell\\
s_j^\ell:\mathcal{G}_\ell \to \mathcal{G}_{\ell+1}&  \qquad \text{degeneracies}
\end{align*}
\end{remark}

\medskip\noindent
The tangent is a functor, it satisfies
$$
T_\bullet(\mathcal{G})=T_k(\mathcal{G})=T\mathcal{G}_k
$$
(It looks like $T$ preserves diagrams.)

\medskip\noindent
{\bf Dold-Kan.} The category $\mathsf{SVect}$ of simplicial vector spaces
has objects
$$
\xymatrix{
\mathbb{V}_\bullet&  \mathbb{V}_n\ar[r]&\cdots
\ar[r]&\mathbb{V}_2\ar[r]^{\text{3 arrows}}&
\mathbb{V}_1\ar[r]^{\text{2 arrows}}&\mathbb{V}_0
}
$$
where the $\mathbb{V}_i$ are vector spaces.

There is a functor
$$
\mathsf{SVect}\overset{N}{\to}\{\text{chain complexes}\geq 0\}
$$
$$
\mathbb{V}_{\bullet}\to N\mathbb{V}=
\left(N_\ell \mathbb{V}\Ker P_{\ell,\ell},\partial=d_\ell\right)
$$

\begin{theorem}[Dold-Kan]
\label{theorem-Dold-Kan}
That's an equivalence of categories. [Confirm this!]
\end{theorem}

Those categories are monodial:
$$
(\mathsf{SVect},\otimes),\qquad  (\mathbb{V}_\bullet \otimes \mathbb{W})_\ell
=\mathbb{V}_\ell \otimes \mathbb{W}_\ell
$$
$$
(\mathsf{ch}_{\geq 0},\otimes),\qquad (V \otimes W)_i=
\bigoplus_{\ell +k=1}V_\ell \otimes W_k
$$
And $N$ is Lax monoidal with Lax structure given by the
Eilenberg-Zilber map, though we won't explain the details of this.

There are duals given by internal Hom:
\begin{align*}
\mathbb{V}^{n*}&=\underline{\Hom}(\mathbb{V},B^n\mathbb{R})
\end{align*}
Where the internal Hom is given by
$$
\underline{\Hom}(\mathbb{V},B^n\mathbb{R})_\ell
=\Hom_{\mathsf{SVect}}(\mathbb{V}\otimes\Delta_n[\ell],B^n\mathbb{R})
$$
for an object $\Delta[\ell]=\mathbb{R}[\Delta[\ell]]$.


\medskip\noindent
{\bf Properties.}
\begin{enumerate}
\item $\mathbb{V}^{n*}$ isa simplicial vector bundle.
\item $N(V^{n*})$ and $N(\mathbb{V})^*[n]$ is a quasi isomorphism.
\item $\left<\cdot,\cdot\right>:\mathbb{V}\otimes \mathbb{V}^{n*}\to
B^n\mathbb{R}$ is non-degenerate on homology.
\item $\mathbb{V}\hookrightarrow (\mathbb{V}^{n*})^{n*}$ 
Mont. a eq.
\end{enumerate}

\medskip\noindent
{\bf The vector bundle case.}
$$
\text{Maps}(\Delta[i],\mathcal{G})_k=
\Hom_{\mathsf{SSet}}(\Delta[i]\times\Delta[k],\mathcal{G})
$$
\begin{proposition}
\label{proposition-on-an-n-Lie-groupoid}
Let $\mathcal{G}$ be a Lie $n$-groupoid.
\begin{enumerate}
\item $\text{Maps}(\Delta[i],\mathcal{G})$ Lie $n$-groupoids ME $\mathcal{G}$.
\item $\text{Maps}(\Delta[i],\mathcal{G})_0=\mathcal{G}_i$.
\item $\text{ev}:\Delta[i]\times \text{Maps}(\Delta[i],\mathcal{G})\to
\mathcal{G}$.
\end{enumerate}
\end{proposition}

[Staircase looking diagram.]

\begin{definition}
\label{definition-what-is-this}
$\mathcal{G}_\bullet$.
$$
T_i^{n*}\mathcal{G}=\Hom_{\mathsf{SVect}}(1^*  \Pi_{\Delta[i]}(T\mathcal{G}),
B^n\mathbb{R}_{\mathcal{G}_i})
$$
$$
(d_j,F)_{K|d_j\mathcal{G}}(x^a)=(F_k)|_{\mathcal{G}}(x^{\delta_ja}).
$$
\end{definition}

\begin{proposition}
\label{proposition-properties-of-that}
$\mathcal{G}$ Lie $n$-groupoid, then $T^{n*}\mathcal{G}$ satisfy
\begin{enumerate}
\item is a vector bundle $n$-groupoid
\item dual to $T\mathcal{G}$
$$
\left<\cdot,\cdot\right>:T\mathcal{G} \otimes T^{n*}\mathcal{G}
\to B^n\mathbb{R}_{\mathcal{G}}
$$
non-degenerate on homology.
\item $n$-shifted symplectic
$$
T^{n*}_n\mathcal{G}\overset{p}{\to}T^*\mathcal{G}_n
$$
and $p^*\omega_{\text{can}}$.
\end{enumerate}
\end{proposition}

[More computations I missed]
\bibliography{my}
\bibliographystyle{amsalpha}

\end{document}


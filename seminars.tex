\IfFileExists{stacks-project.cls}{%
\documentclass{stacks-project}
}{%
\documentclass{amsart}
}

% For dealing with references we use the comment environment
\usepackage{verbatim}
\newenvironment{reference}{\comment}{\endcomment}
%\newenvironment{reference}{}{}
\newenvironment{slogan}{\comment}{\endcomment}
\newenvironment{history}{\comment}{\endcomment}

% For commutative diagrams we use Xy-pic
\usepackage[all]{xy}

% We use 2cell for 2-commutative diagrams.
\xyoption{2cell}
\UseAllTwocells

% We use multicol for the list of chapters between chapters
\usepackage{multicol}

% This is generally recommended for better output
\usepackage{lmodern}
\usepackage[T1]{fontenc}

% For cross-file-references
\usepackage{xr-hyper}

% Package for hypertext links:
\usepackage{hyperref}

% For any local file, say "hello.tex" you want to link to please
% use \externaldocument[hello-]{hello}
\externaldocument[introduction-]{introduction}
\externaldocument[conventions-]{conventions}
\externaldocument[sets-]{sets}
\externaldocument[categories-]{categories}
\externaldocument[topology-]{topology}
\externaldocument[sheaves-]{sheaves}
\externaldocument[sites-]{sites}
\externaldocument[stacks-]{stacks}
\externaldocument[fields-]{fields}
\externaldocument[algebra-]{algebra}
\externaldocument[brauer-]{brauer}
\externaldocument[homology-]{homology}
\externaldocument[derived-]{derived}
\externaldocument[simplicial-]{simplicial}
\externaldocument[more-algebra-]{more-algebra}
\externaldocument[smoothing-]{smoothing}
\externaldocument[modules-]{modules}
\externaldocument[sites-modules-]{sites-modules}
\externaldocument[injectives-]{injectives}
\externaldocument[cohomology-]{cohomology}
\externaldocument[sites-cohomology-]{sites-cohomology}
\externaldocument[dga-]{dga}
\externaldocument[dpa-]{dpa}
\externaldocument[sdga-]{sdga}
\externaldocument[hypercovering-]{hypercovering}
\externaldocument[schemes-]{schemes}
\externaldocument[constructions-]{constructions}
\externaldocument[properties-]{properties}
\externaldocument[morphisms-]{morphisms}
\externaldocument[coherent-]{coherent}
\externaldocument[divisors-]{divisors}
\externaldocument[limits-]{limits}
\externaldocument[varieties-]{varieties}
\externaldocument[topologies-]{topologies}
\externaldocument[descent-]{descent}
\externaldocument[perfect-]{perfect}
\externaldocument[more-morphisms-]{more-morphisms}
\externaldocument[flat-]{flat}
\externaldocument[groupoids-]{groupoids}
\externaldocument[more-groupoids-]{more-groupoids}
\externaldocument[etale-]{etale}
\externaldocument[chow-]{chow}
\externaldocument[intersection-]{intersection}
\externaldocument[pic-]{pic}
\externaldocument[weil-]{weil}
\externaldocument[adequate-]{adequate}
\externaldocument[dualizing-]{dualizing}
\externaldocument[duality-]{duality}
\externaldocument[discriminant-]{discriminant}
\externaldocument[derham-]{derham}
\externaldocument[local-cohomology-]{local-cohomology}
\externaldocument[algebraization-]{algebraization}
\externaldocument[curves-]{curves}
\externaldocument[resolve-]{resolve}
\externaldocument[models-]{models}
\externaldocument[functors-]{functors}
\externaldocument[equiv-]{equiv}
\externaldocument[pione-]{pione}
\externaldocument[etale-cohomology-]{etale-cohomology}
\externaldocument[proetale-]{proetale}
\externaldocument[relative-cycles-]{relative-cycles}
\externaldocument[more-etale-]{more-etale}
\externaldocument[trace-]{trace}
\externaldocument[crystalline-]{crystalline}
\externaldocument[spaces-]{spaces}
\externaldocument[spaces-properties-]{spaces-properties}
\externaldocument[spaces-morphisms-]{spaces-morphisms}
\externaldocument[decent-spaces-]{decent-spaces}
\externaldocument[spaces-cohomology-]{spaces-cohomology}
\externaldocument[spaces-limits-]{spaces-limits}
\externaldocument[spaces-divisors-]{spaces-divisors}
\externaldocument[spaces-over-fields-]{spaces-over-fields}
\externaldocument[spaces-topologies-]{spaces-topologies}
\externaldocument[spaces-descent-]{spaces-descent}
\externaldocument[spaces-perfect-]{spaces-perfect}
\externaldocument[spaces-more-morphisms-]{spaces-more-morphisms}
\externaldocument[spaces-flat-]{spaces-flat}
\externaldocument[spaces-groupoids-]{spaces-groupoids}
\externaldocument[spaces-more-groupoids-]{spaces-more-groupoids}
\externaldocument[bootstrap-]{bootstrap}
\externaldocument[spaces-pushouts-]{spaces-pushouts}
\externaldocument[spaces-chow-]{spaces-chow}
\externaldocument[groupoids-quotients-]{groupoids-quotients}
\externaldocument[spaces-more-cohomology-]{spaces-more-cohomology}
\externaldocument[spaces-simplicial-]{spaces-simplicial}
\externaldocument[spaces-duality-]{spaces-duality}
\externaldocument[formal-spaces-]{formal-spaces}
\externaldocument[restricted-]{restricted}
\externaldocument[spaces-resolve-]{spaces-resolve}
\externaldocument[formal-defos-]{formal-defos}
\externaldocument[defos-]{defos}
\externaldocument[cotangent-]{cotangent}
\externaldocument[examples-defos-]{examples-defos}
\externaldocument[algebraic-]{algebraic}
\externaldocument[examples-stacks-]{examples-stacks}
\externaldocument[stacks-sheaves-]{stacks-sheaves}
\externaldocument[criteria-]{criteria}
\externaldocument[artin-]{artin}
\externaldocument[quot-]{quot}
\externaldocument[stacks-properties-]{stacks-properties}
\externaldocument[stacks-morphisms-]{stacks-morphisms}
\externaldocument[stacks-limits-]{stacks-limits}
\externaldocument[stacks-cohomology-]{stacks-cohomology}
\externaldocument[stacks-perfect-]{stacks-perfect}
\externaldocument[stacks-introduction-]{stacks-introduction}
\externaldocument[stacks-more-morphisms-]{stacks-more-morphisms}
\externaldocument[stacks-geometry-]{stacks-geometry}
\externaldocument[moduli-]{moduli}
\externaldocument[moduli-curves-]{moduli-curves}
\externaldocument[examples-]{examples}
\externaldocument[exercises-]{exercises}
\externaldocument[guide-]{guide}
\externaldocument[desirables-]{desirables}
\externaldocument[coding-]{coding}
\externaldocument[obsolete-]{obsolete}
\externaldocument[fdl-]{fdl}
\externaldocument[index-]{index}

% Theorem environments.
%
\theoremstyle{plain}
\newtheorem{theorem}[subsection]{Theorem}
\newtheorem{proposition}[subsection]{Proposition}
\newtheorem{lemma}[subsection]{Lemma}

\theoremstyle{definition}
\newtheorem{definition}[subsection]{Definition}
\newtheorem{example}[subsection]{Example}
\newtheorem{exercise}[subsection]{Exercise}
\newtheorem{situation}[subsection]{Situation}

\theoremstyle{remark}
\newtheorem{remark}[subsection]{Remark}
\newtheorem{remarks}[subsection]{Remarks}

\numberwithin{equation}{subsection}

% Macros
%
\def\lim{\mathop{\mathrm{lim}}\nolimits}
\def\colim{\mathop{\mathrm{colim}}\nolimits}
\def\Spec{\mathop{\mathrm{Spec}}}
\def\Hom{\mathop{\mathrm{Hom}}\nolimits}
\def\Ext{\mathop{\mathrm{Ext}}\nolimits}
\def\SheafHom{\mathop{\mathcal{H}\!\mathit{om}}\nolimits}
\def\SheafExt{\mathop{\mathcal{E}\!\mathit{xt}}\nolimits}
\def\Sch{\mathit{Sch}}
\def\Mor{\mathop{\mathrm{Mor}}\nolimits}
\def\Ob{\mathop{\mathrm{Ob}}\nolimits}
\def\Sh{\mathop{\mathit{Sh}}\nolimits}
\def\NL{\mathop{N\!L}\nolimits}
\def\CH{\mathop{\mathrm{CH}}\nolimits}
\def\proetale{{pro\text{-}\acute{e}tale}}
\def\etale{{\acute{e}tale}}
\def\QCoh{\mathit{QCoh}}
\def\Ker{\mathop{\mathrm{Ker}}}
\def\Im{\mathop{\mathrm{Im}}}
\def\Coker{\mathop{\mathrm{Coker}}}
\def\Coim{\mathop{\mathrm{Coim}}}

% Boxtimes
%
\DeclareMathSymbol{\boxtimes}{\mathbin}{AMSa}{"02}

%
% Macros for moduli stacks/spaces
%
\def\QCohstack{\mathcal{QC}\!\mathit{oh}}
\def\Cohstack{\mathcal{C}\!\mathit{oh}}
\def\Spacesstack{\mathcal{S}\!\mathit{paces}}
\def\Quotfunctor{\mathrm{Quot}}
\def\Hilbfunctor{\mathrm{Hilb}}
\def\Curvesstack{\mathcal{C}\!\mathit{urves}}
\def\Polarizedstack{\mathcal{P}\!\mathit{olarized}}
\def\Complexesstack{\mathcal{C}\!\mathit{omplexes}}
% \Pic is the operator that assigns to X its picard group, usage \Pic(X)
% \Picardstack_{X/B} denotes the Picard stack of X over B
% \Picardfunctor_{X/B} denotes the Picard functor of X over B
\def\Pic{\mathop{\mathrm{Pic}}\nolimits}
\def\Picardstack{\mathcal{P}\!\mathit{ic}}
\def\Picardfunctor{\mathrm{Pic}}
\def\Deformationcategory{\mathcal{D}\!\mathit{ef}}

%Dani's additions
\usepackage{graphicx} %figures


\begin{document}

\title{Seminars}
\maketitle

\phantomsection
\label{section-phantom}
\hfill
\href{http://github.com/danimalabares/stack}{github.com/danimalabares/stack}

\tableofcontents

\section{A primer on symplectic groupoids}
\label{section-a-primer-on-symplectic-groupoids}

\noindent
Camilo Angulo, Jilin University.
Geometric Structures Seminar, IMPA. 
February 13, 2025.

\medskip
{\bf Abstract.} 
In the late 17th century, Simeon Denis Poisson discovered an operation that
helped encoding and producing conserved quantities. This operation is what we
now know as a Lie bracket, an infinitesimal symmetry, but what is its global
counterpart? Symplectic groupoids are one possible answer to this question. In
this talk, we will introduce all the basic concepts to define symplectic
groupoids, and their role in Poisson geometry. We will discuss key examples, and
applications. The talk will be accessible to those familiar with differential
geometry, but no prior knowledge of groupoids will be assumed.  


\medskip\noindent
{\bf Part 1. Poisson geometry.}

Hamiltonian formalism. Recall that being a conserved quantity \(f \in
C^\infty(X)\) is the same thing as \(\{H,f\}=0\).

\begin{itemize} \item We have seen that it is always possible to take quotient
of a symplectic manifold with a group action to obtain a  Poisson
manifold.  \item Then we have found a way to produce a symplectic foliation
from a 2-vector \(\pi \in \mathfrak{X}^2(M):=\Lambda^{2}(TM)\).  \item

\begin{remark}
\[\left\{ \text{Lie algebra on \(\mathfrak{g}\)}
\right\} \xrightarrow{1-1}\left\{ \text{Linear Poisson bracket on
\(\mathfrak{g}^*\)}  \right\}  \]
\end{remark} 

\item We saw very nice examples
of foliation that have to do with Lie algebras. So \(\mathfrak{b}^*_3\) which
gives the ``open book foliation", and \(\mathfrak{e}^*\) that gives a foliation
by cylinders.
\end{itemize}

\medskip\noindent
{\bf Part 2. Symplectic realizations.}

Consider
\[(\Sigma,\omega) \xrightarrow{\mu}(M,\pi)\]


\iffalse\begin{tikzcd}
	T_p\Sigma \arrow[r,"d_p\mu"]\arrow[d,"\omega^\flat"]&T_xM\\
	T_p^*\Sigma &  T^*_pM\arrow[l,"(d\mu)^*",swap]\arrow[u,"\pi ^\sharp"]
\end{tikzcd}\]\fi
So that 
\[\pi ^\sharp=d_p\mu \circ \omega^{-1} \circ(d\mu)^*\]



\begin{lemma}
\(\dim (\Sigma) \geq  2 \dim (M) - \operatorname{rk}(\pi_x)\) for all \(x \in M\).
\end{lemma}

\begin{proof}
Done in seminar.
\end{proof}

\begin{example}
\((\mathbb{R}^2,0)\). So the map
\begin{align*}
	(\mathbb{R}^4,dx \wedge du + dy \wedge dv) &\longrightarrow  \mathbb{R}^2\\
	(x,y,u,v) &\longmapsto (x,y)
\end{align*}

\end{example}

\begin{exercise}
Find the symplectic realization \(\omega\) in \((\mathbb{R}^4, \omega ) \xrightarrow{\mu}(\mathbb{R}^2,(x^2+y^2)\partial_x \wedge \partial_y\)

\begin{align*}
	(\mathbb{R}^4, \omega) &\longrightarrow (\mathbb{R}^2,(x^2+y^2) \partial_x \wedge \partial _y) \\
	(x,y,z,w) &\longmapsto (x,y)
\end{align*}

Also find the symplectic realization of \(\operatorname{ a f f }^*\) with \(\{x,y\}=x\).
\end{exercise}

\medskip\noindent
{\bf Part 3: Groupoids}

\medskip\noindent
{\bf Motivation}
\begin{enumerate}
\item Fundamental grupoid: objects are points in the manifold and arrows are paths.
\item \(S^1 \mathbb{y} S^2\) by rotation does not give a nice quotient because there are two singular points. Consider the groupoid  \(S^1 \times S^2\) of orbits. These are the arrows. The points are just the points of \(S^2\).
\item Consider a foliation (like Möbius foliation of circles; where there is a singular circle, the soul). You can do the same thing as in fundamental groupid  leafwise. Arrows then are equivalence classes of paths that live inside leaves. This is called monodromy of a foliation. Again objects are points.
\item You can take a quotient of monodromy using a connection given by the foliation. This allows to identify certain paths between the leaves. This is called {\it holonomy (of a foliatiation)}. (So you can make this notion match the usual holonomy given by riemmanian connection.)

{\bf Upshot.}
	So the point is taking some sort of function space on these groupoids you can gather the information given by the non-smooth quotient (like in the case of the sphere rotating). So this groupoid motivation says how to get some structure that resembles a non smooth quotient.
\item Last motivation: the grid of squares has a tone of symmetries. If you restrict to just a few squares you loose so many symmetries. But there's a grupoid hidden in there that tells you what you intuition knows about this finite grid of squares.
\end{enumerate}

\begin{definition}
A {\it groupoid} is a category where all morphisms are invertible.
\end{definition}

So there is a kind of product among the objects, given by composition but:
 not every two pair of objects can be multiplied!---only those whose
source and target match. So that's the lance about grupoids.

Just so you make sure you understand: the groupoid \(G\) is the morphisms of the category. The objects are points (of a manifold).

\begin{definition}
{\it Lie grupoid} is when the following diagram is inside category of smooth manifolds and \(s,t\) (source and target maps) are submersions:
$$
\xymatrix{
G^{(2)}\ar[r]^m& G\ar[r]^ s_t & M\ar[r]^u&G
}
$$
\end{definition}

\begin{proposition}[Properties of Lie groupoids]
\begin{itemize}
\item \(m\) is also a submersion.
\item \(i\) (inversion) is a diffeomorphism.
\item  \(u\) (unit=identity) is an embbeding.
\end{itemize}
\end{proposition}

\begin{definition}
Consider \(x \in M\) and the inverse image of source map: \(s^{-1}(x)=\{\text{arrows that start at \(x\)} \}\). Now if you act with \(t\) on this set you get {\it the orbit} of \(x\): \(\{\text{\(y \in M\) such that there is an arrow from \(x\) to \(y\)} \}\).

And there also {\it an isotropy} \(G_x=\{g \in G: \text{\(g\) goes from \(x\) to \(x\)} \}\)
\end{definition}

\begin{example}
\begin{enumerate}
\item \(G=M\),  \(M=M\).
 \item Lie groups.
	\item Lie group bundles.
\item \(G=M \times M\), \(M =M\).
 \item Fundamental groupoid. Isotropy group is fundamental group! And orbit is…\clearpage universal cover!
\item Subgroupoids.
\item  Foliations.
\item If you have a normal group action \(G \mathbb{y} M\) you construct a groupoid action with groupoid \(G \times M\) and objects \(M\), with product given on the group part of the product. Orbits are orbits. Isotropy group is isotropy group.
 \item Principal bundles.
\end{enumerate}
\end{example}

{\bf Back to Poisson.}

There's also a notion of Lie algebroid. Which is strange. But the point is that to every Poisson manifold there is a Lie algebroid.

So the question is whether there is a Lie groupoid associated to that Lie algebroid. Not always.

\medskip\noindent
{\bf Big question}[Fernandez and ?]
When a symplectic manifold is integrable?


(Remember that integrating means go from algebra(oid) to group(oid).

And the point is that:

\medskip\noindent
When you {\it can} go back, you get a \textit{ symplectic groupoid}.

\begin{remark}
Look for Kontsevich's notes on Weinstein!
\end{remark}

\begin{remark}
History: Weinstein did this intending to do quantization (geometric?) on Poisson manifolds. (That involves a \(C^*\)algebra coming from the symplectic groupoid.)
\end{remark}

\begin{definition}
A {\it symplectic groupoid} is a groupoid \(G, M\) together with  \(\omega \in \Omega^2(M)\) such that \(\omega\) is symplectic and multiplicative, meaning that \(\partial  \omega =0\), that is, \(\iff m^*\omega=\operatorname{pr}_1^*\omega+\operatorname{pr}^*_2\omega\in \Omega^{2}(G^{(2)})\iff\) take two vectors \( X_k,Y_k \in TG\), and \(\omega(X_0 \star Y_0,X_1\star Y_1) =  \omega(X_0,Y_0) + \omega (X_1,Y_1)\)
\end{definition}

\begin{theorem}
If \((G, \omega)\) is a symplectic groupoid, then
\begin{enumerate}
\item \(\exists !\) poisson structure on \(M\) 
\item for which \(t: G \to M\) is a symplectic realization,
\item  Leaves are connected components of orbits,
\item \(\mathsf{Lie}(G)\cong T^*M\) via \(X \mapsto -u ^*(i_X \omega\).
\end{enumerate}
\end{theorem}

\begin{remark}
Look for Alejandro Cabrera, Kontsevich. There are two things one is de Rham and the other… from the future: simplicial?
\end{remark}

\medskip\noindent
{\bf Upshot.} The obstruction to knowing when symplectic groupoid exists is ``variation of symplectic form \(\omega=(1+t^2) \omega_{S^2}\)". So how does the symplectic group vary from leaf to leaf. So there are two situations in which the thing doesn't work.



\section{Neutrinos}
\label{section-neutrinos}

\noindent
Hiroshi Nunokawas, PUC-Rio.
Friday Seminar, Seminar Name. 
June 27, 2025.

\medskip
{\bf Abstract.} Hiroshi will come and tell us everything we (not) wanted to know
about these mysterious particles, and are not going to be afraid to ask. In
particular, about the neutrino oscillation, and the great matrices.

\bigskip\noindent

Protons and neutrons have very similar mass of $m_p\approx 940$ MeV, while
electrons have mass of $m_e \approx 0.5$ MeV. MeV is $10^{6}$ electronvolts,
where one eV is approximately $1.6\times 10^{-19}$ J. This is standard in high
energy physics, they use electronvolts instead of Joules. Recall that
2J=1N$\times$1m.

Most of the things we see are protons since they are so much larger than
electrons. But protons nor neutrons are elementary particles.

Here's the standard model:

\medskip

\begin{tabular}{c c c c}
Quarks & $\begin{bmatrix} u\\d \end{bmatrix}_L $ & $\begin{bmatrix} c\\s
\end{bmatrix}_L$ & $\begin{bmatrix} t\\b \end{bmatrix}_L$\\
Leptons & $\begin{bmatrix} \nu_e\\e^- \end{bmatrix}_L $ & $\begin{bmatrix}
\nu_\mu\\ \mu^-
\end{bmatrix}_L$ & $\begin{bmatrix} \nu_\tau\\ \tau^- \end{bmatrix}_L$\\
Generation & 1st & 2nd & 3rd\\
Bosons & $g$, $\gamma$ , $\omega^\pm$, $z$ \\
Higgs Bosson & $H$
\end{tabular}

\medskip

It is very particular that nature repeats itself three times. The $L$ in those
matrix actually means left-handed, and accounts for chirality. Only left-handed
fermions have weak interaction. Right-handed have electromagnetic interaction,
gravitational interaction, but not weak interaction.

And then there's neutrinos. They have negative helicity (chirality). Being
left-handed, mathematically, means to have helicity $-1$. I think this means
that the spin is left-handed. But chirality and helicity are not the same:
helicity is observer-dependent, and chirality is not. Almost all neutrinos we
can see (\% 99.99999…) have negative helicity, but not all of them.

Consider the following:
$$
n+\nu_e\leftrightarrow p+e^-
$$
But it's not completely correct: we'd better put $d$ instead of $n$, and $u$
instead of $p$: the $d$ and $u$ quarks, instead of the neutrons and protons.

Now consider the following reaction: a neutron decays into a proton, an electron
and an antineutrino:
$$
n\to p+e^+\overline{\nu}_e
$$
Protons is very stable, that's why we are here. But neutron decays in only 15
minutes.

By experimental data, we can conclude that neutrinos' mass is consistent with
zero. But if they have mass, it should be much smaller than the electron's $m_v
\leq 0.5$ eV. And the electron is already the lightest fermion!

If the mass of the neutrino was zero, i.e. $m_\nu=0$, then $v_\nu=c$ in vacuum,
which would imply that 
$$
\xymatrix{
\nu_e\ar[r]^{L}&\nu_e\ar[r]&\nu_e\\
0:00&0:00&0:00
}
$$
meaning: time doesn't pass! And this means the state of the particle cannot
change.



\section{Spheres with minimal equators}
\label{section-spheres-with-minimal-equators}

\noindent
Lucas Ambrozio, IMPA.
Differential Geometry Seminar, IMPA. 
June 24, 2025.

\medskip
{\bf Abstract.} We will discuss the connection between Riemannian metrics on the
sphere with respect to which all equators are minimal hypersurfaces, and
algebraic curvature tensors with positive sectional curvatures.

\bigskip\noindent

\begin{definition}
\label{definition-equator}
An {\it $(n-k)$-equator} orthogonal to $\Pi$ is
$$
\Sigma_\Pi:=\{p\in\mathbb{S}^n:\left<p,x\right>=0\forall x\in\Pi\}
$$
for $\Pi$ a $k$-dimensional linear subspace of $\mathbb{R}^{n+1}$.
\end{definition}

\begin{remark}
\label{remark-equators-are-totally-geodesic-with-usual-metric}
Equators are totally geodesic hypersurfaces with the usual sphere metric, which
implies they are minimal hypersurfaces.
\end{remark}

{\bf Problem.} Characterize the set $\mathcal{M}_k(U)$ of metrics $g$ on an 
open set $U\subset\mathbb{S}^n$ such that all $k$-equators $\Sigma_\Pi$ with
 $\Sigma\cap U\neq\emptyset$ yield are minimal hypersurfaces $\Sigma\cap U$ 
on $(U,g)$.

\begin{remark}
\label{remark-why-not-Rn}
This problem can be thought of as a problem of finding metrics on $\mathbb{R}^n$
such that $k$-planes are minimal. To see why project the $k$-equators to
$T_p\mathbb{S}^n$ and pullback those metrics to the sphere.
\end{remark}

\medskip\noindent

Let $g\in\mathcal{M}_k(U)$ for $U\subset\mathbb{S}^n$ open and $n\geq 2$.

\begin{theorem}[Beltrami, Schäfli]
\label{theorem-Beltrami-Schafli}
If $k=1$ then $g$ has constant sectional curvature.
\end{theorem}

\begin{theorem}[Hongan]
\label{theorem-Hongan}
If $1<k<n-1$ then $g$ has constant sectional curvature.
\end{theorem}

Then Hongan also managed to produce a classification of these metrics for
$k=n-1$.

\begin{remark}
\label{remark-linear-invertible-preserving-equators-is-in-Mg}
If $T\in \text{GL}(n+1,\mathbb{R})$, then
\begin{align*}
\varphi: \mathbb{S}^n &\longrightarrow \mathbb{S}^n \\
x &\longmapsto \frac{Tx}{|Tx|}
\end{align*}
is a diffeomorphism that maps $k$-equators into $k$-equators. Thus if
$g\in\mathcal{M}_k(\mathbb{S}^n)$ then so is $\varphi(T)^*g$.
\end{remark}

\begin{theorem}
\label{theorem-equivariant-bijection}
There exists a $\text{GL}(n+1,\mathbb{R})$ equivariant bijection
$$
\mathcal{M}_{n-1}(\mathbb{S}^n)\leftrightarrow \text{Curv}_+(\mathbb{R}^{n+1})
$$
where the set on the right-hand-side is the set of algebraic curvature tensors
(also called curvature-like, i.e. with the same symmetries as the Riemannian
curvature tensor) on $\mathbb{R}^{n+1}$ with positive sectional curvature.

The group action is given as follows for $T\in\text{GL}(n+1,\mathbb{R})$:
$$
(R\cdot T)(x,y,z,w)=\frac{1}{|\det(T)|^{\frac{1}{n+1}}}R(Tx,Ty,Tz,Tw)
$$
\end{theorem}

The point is that $\text{Curv}_+(\mathbb{R}^{n+1})$ is an open cone on a linear
space. Here are two simple corollaries:

\begin{lemma}
\label{lemma-corollaries}
\begin{enumerate}
\item $\mathcal{M}_{n+1}(\mathbb{S}^n)$ is in bijection with an open positive
cone of an $\frac{n(n+2)(n+1)^2}{12}$-dimensional real vector space.
\item Every metric on $\mathcal{M}_{n-1}(\mathbb{S}^n)$ is invariant by the
antipodal map.
\end{enumerate}
\end{lemma}

{\bf Algorithm.} From any $R\in \text{Curv}_p(\mathbb{R}^{n+1})$ we obtain a
symmetric positive definite (positive-definitiness comes from the positiveness
of the curvature of $R$) 2-tensor  $k_R$ satisfying
$$
(k_R)_p(v,v)=R(pv,pv)>0
$$
Also, $k_R$ has the  {\it Killing property},
i.e. that $\overline{\nabla}k(X,X,X)=0$ for all
$X\in\mathfrak{X}(\mathbb{S}^n)$.

Then we define a positive function on
 $\mathbb{S}^n$ by
\begin{equation}
\label{equation-def-DR}
D_R:=\left(\frac{d\text{Vol}_{k_R}}{dV_g}\right)^{\frac{4}{n-1}}
\end{equation}
and finally a Riemannian metric on $\mathbb{S}^n$ in
$\mathcal{M}_{n-1}\mathbb{S}^n$ by
$$
g_R=\frac{1}{D_R}k_R
$$
And to go back, for $g \in \mathcal{M}_{n-1}(\mathbb{S}^n)$ define a positive 
function on $\mathbb{S}^n$
$$
F_g:=\left(\frac{dV_g}{dV_{\overline{g}}}\right)^{\frac{4}{n-1}}
$$
Then let $k_g:=\frac{1}{F_g}g>0$, which is a positive definite Killing 2-tensor,
from which we may define $R_g\in\text{Curv}_+(\mathbb{R}^{n+1})$ with
$R_g(pv,pv)=(k_g)_p(v,v)$ for all $p,v\in T\mathbb{S}^n$.

More corollaries:
\begin{lemma}
\label{lemma-corollaries2}
\begin{enumerate}
\item $g\in\mathcal{M}_{n-1}(\mathbb{S}^n)$ is analytic because it is a Killing
tensor on $\mathbb{S}^n$, which are well-known.
\item If $g$ is left-invatiant on $\mathbb{S}^3$, seen as unit quaternions, then
$g\in\mathcal{M}_2(\mathbb{S}^3$. Moreover, for $a\geq b\geq c>0$,
$$
aL_i\odot L_i+bL_j\odot L_j+cL_k\odot L_k=k
$$
is Killing, $k>0$,  $D_k$ constant and thus $g=\frac{1}{\text{const.}}k
\in\mathcal{M}_2(\mathbb{S}^3)$.
\item $R$ curvature tensor of $(\mathbb{C}P^{2},g_{FS})$. We may not remember
what's the curvature tensor, but we know the sectional curvature is $1\leq
\text{sec}(R)\leq 4$,
$$
(k_R)_p(v,w)=\overline{g}(v,w)+3\overline{g}(Jp,v)\overline{g}(Jp,w)
$$
and $D_R=4^{\frac{4}{3-1}}=4$, so that by \ref{equation-def-DR} we obtain 
$g_R=\frac{1}{4}k_R$, which is a Berger
metric on $\mathbb{S}^3$ with scalar curvature 0.
\end{enumerate}
\end{lemma}

\bigskip
Now define
$$
\Sigma_V=\{p\in\mathbb{S}^n:\left<p,v\right>=0\}=V^{-1}(0)
$$
where $V(x):=\left<x,v\right>$ for all $x\in\mathbb{S}^n$. Then the normal
vector field is $\nabla V/|\nabla V|_g$, and the second fundamental form is
given by
$$
A=\frac{1}{|\nabla V|_g}\text{Hess}_gV
$$
and its mean curvature by
\begin{equation}
\label{equation-mean-curvature-for-level-sets-V}
H=\frac{1}{|\nabla V|_g}\left(\Delta_gV-\text{Hess}_gV\left(\frac{\nabla
V}{|\nabla V|},\frac{\nabla V}{|\nabla V|}\right)\right)
\end{equation}
For every $v \in \mathbb{S}^n$ and $p\in\Sigma_V$, we see that $H_{\Sigma_V}=0$
iff
$$
|\nabla V|^2_g(p)\Delta_gV(p)-\text{Hess}_gV(\nabla V(p),\nabla V(p))=0
$$
And for $\overline{g}$,
$$
\text{Hess}_{\overline{g}}V+V\overline{g}=0\implies
\text{Hess}_{\overline{g}}V(X,X)=0
$$
for all $X\in T_p\mathbb{S}^n$ and $p\in\Sigma_v$. Then
$$
J_g(X,Y,Z)=g(\nabla_XY-\overline{\nabla}_XY,Z)
$$
$$
J_g(X,Y,\nabla V)=\text{Hess}_{\overline{g}}-\text{Hess}
$$
{\bf Problems.}
\begin{enumerate}
\item Similar story for $\mathbb{C}P^{n},\mathbb{H}P^n$?
\item Complete metrics on $\mathbb{R}^n$ with minimal hyperplanes.
\item Find geometric invariants of metrics on $\mathcal{M}_{n-1}(\mathbb{S}^n)$
(may be useful to study $(M^n,g)$, $n\geq 4$, $\text{sec}>0$.
\end{enumerate}

\section{Smoothable compactified Jacobians of nodal curves}
\label{section-smoothable-compactified-Jacobians-of-nodal-curves}

\noindent
Nicola Pagani, University of Liverpool and Bologna. 
Seminar of Algebraic Geometry UFF. 
August 20, 2025.

\medskip
{\bf Abstract.} Building from examples, we introduce an abstract notion of a
'compactified Jacobian' of a nodal curve. We then define a compactified Jacobian
to be 'smoothable' whenever it arises as the limit of Jacobians of smooth
curves. We give a complete combinatorial characterization of smoothable
compactified Jacobians in terms of some 'vine stability conditions', which we
will also introduce. This is a joint work with Fava and Viviani.

\medskip\noindent



Let $C$ be a smooth curve and $d \in \mathbb{Z}$. Define
$$
J^d_C=\{L:\text{$L$ is a line bundle of degree $d$}\}/\sim
$$
which is a smooth projective variety of dimension $g(C)$.

If $C$ is nodal we still can consider $J^d_C$.
\begin{enumerate}
\item One connected component. Then the Jacobian is $\mathbb{P}^1$ minus two
points. This is not universally closed, so it is not proper.
\item Two components intersecting at one point. 
The pullback of the normalization splits the degree in
intintely many ways, givieng that $J^{-1}_C$ is an infinite set of points. This
is not of finite type, so it is not proper.
\item The curve has two components intersecting at two points. 
This gives $J^{-2}_C$,
 which is a mixture of the two former items. (Probably not proper too.)
\end{enumerate}

\medskip\noindent
Now consider
$$
\text{TF}_C^d=\{\mathcal{F}:
\text{coherent on $C$, torsion-free, rank-1 on $C$}\}/\sim
$$
This satisfies the existence oart of the valu point of properness.

Now we consider the moduli. Now we consider the ideal sheaf of the (singular?)
point(s?):
\begin{enumerate}
\item One component. The stack is proper!
\item Two components intersecting once. Now we get stacky points, 
$x=[\bullet/\mathbb{G}_m]$. These points have generic stabilizer. The resulting
stack is not separated because a morphism of a curve, say $\mathbb{P}^1$ minus a
point … there are infinitely many ways to extend a morphism from this thing to a
line bundle. So you cannot include any of these stacky points. Recall that a
sheaf is {\it simple} if its automorphism group is $\mathbb{G}_m$.
\item  The ideal sheaf of both nodes 
$\mathcal{I}(N_1,N_2)$ has a positive dimensional automorphism
group. The stack is not proper.
\end{enumerate}

\begin{definition}
\label{definition-fine-compactified-Jacobian}
A {\it fined compactified Jacobian} of $C$ is an open connected substack of
$\text{TF}^d(C)$ that is also proper.
\end{definition}

\begin{remark}
\label{remark-algebraic-space}
This thing is automatically an algebraic space.
\end{remark}

\begin{definition}
\label{definition-compactified-Jacobian}
A {\it compactified Jacobian} is an open connected of $\text{TF}^d(C)$ that
admits a proper, good moduli space.
\end{definition}

Consider the Artin stack $\mathfrak{X} \xrightarrow{\Gamma}X$ […]
 is a {\it good
moduli space} if 
\begin{enumerate}
\item Every moduli factors
$$
\xymatrix{
\mathfrak{X} \ar[r]\ar[rd]&\mathcal{I}\text{ (ACC. space)}\\
& X
}
$$
\item $\pi_*\mathcal{O}_{\mathfrak{X}}=\mathcal{O}_X$.
\end{enumerate}

\medskip\noindent
We expect to find a notion of stability condition to produce these things 
[…] $[\bullet/ \mathbb{G}_m]$ would be the polystable representative.

\begin{definition}
\label{definition-smoothable-compactified-Jacobian}
A compactified Jacobian $\overline{J_C}$ is {\it smoothable} if all smoothings 
$\mathcal{C}\to \Delta=\{0,\eta\}$ (with $\mathcal{C}_0=C$), 
$$
J^d_{\mathcal{C}_\eta}\cup C \to \overline{J_C}
$$
is proper.
\end{definition}

\medskip\noindent

\begin{definition}
\label{definition-connected}
Let $X$ be a curve.
$$
\text{BCON}(X)=\{Y \subseteq X\text{s.t. }Y,Y^c \text{ are connected}\}
$$
\end{definition}

\begin{definition}
\label{definition-v-satbility}
A $v$-curve is a generalization of items (2) and (3) in the lists above 
[it looks like two long snakes $\sim$ that intersect several times, and $t$ is
the number of nodes]. 
A {\it $v$-condition} is a pair $n=(n_1,n_2)$ such that 
$$
n_1+n_2=\begin{cases}
d+1-t\qquad &\text{we say the s.c. is nondegenerate} \\
d-t\qquad &\text{degenerate}
\end{cases}
$$
$\mathcal{F}$ on $X$ is  {\it $n$-(semi)stable} if 
$\text{deg}\mathcal{F}_{X_i}>n_i$ ($\text{deg}\mathcal{F}_{X_i}\geq n_i$) 
for $i=1,2$.
\end{definition}
$\mathcal{F}_{X_i}=\mathcal{F}|_{X_i}$ torsion.
$$
\text{deg}(\mathcal{F}_{X_i})+\text{deg}(\mathcal{F}_{X_2})
=d-|\text{sing}(F)|.
$$
Then 
$$
\overline{J_C}(n)=\{\mathcal{F}\text{ is semistable}\},
$$
a smooth compact Jacobian.

\begin{definition}
\label{definition-n-degeneracy}
A {\it degeneratation} of $v$-stab. on $X$ is $n:\text{BCON}(X) \to \mathbb{Z}$
such that
\begin{enumerate}
\item 
$$
n_Y+n_{Y^c}+|Y \cap Y^c|=
\begin{cases}
d+1\qquad &\text{ we say $Y$ is $n$-nondegenerate} \\
d\qquad &\text{$Y$ is $n$-degenerate}
\end{cases}
$$
\item $Y_i$ no pa. common component $n_{Y_1}+n_{Y_2}+\ldots$
$$
\xymatrix{
&Y_1\ar[dr]\ar[dl]\\
Y_2\ar[ur]\ar[rr]&&Y_3\ar[ll]\ar[ul]
}
$$
\end{enumerate}
\end{definition}

\begin{theorem}[-, et al]
\label{theorem-bijection-stability-conditions-and-nodal-curves}
(bijection between stability conditions and nodal curves) The map
$$
\left\{ \substack{\text{sm. comp.} \\ \text{Jac of $X$}} \right\} \to
\left\{ \substack{\text{$v$-stab.} \\ \text{cond. of $X$}} \right\} 
$$
$$
n\mapsto \overline{J_X}(n)=\{n\text{-semistable sheaves}\}
$$
is a bijection. (The arrow should be from right to left!)
\end{theorem}

F. Viviani had proved it for fine compact Jacobians.

\section{Equivariant spaces of matrices of constant rank}
\label{section-equivariant-spaces-of-matrices-of-constant-rank}

\noindent
Ada Boralevi, France.
Algebraic Geometry Seminar, IMPA. 
August 27, 2025.

\medskip
{\bf Abstract.} A space of matrices of constant rank is a vector subspace V, say
of dimension n+1, ofthe set of matrices of size axb over a field k, such that
any nonzero element of V has fixed rank r. It is a classical problem to look for
different ways to construct such spaces of matrices. In this talk I will give an
introduction up to the state of the art of the topic, and report on my latest
joint project with D. Faenzi and D. Fratila, where we give a classification of
all spaces of matrices of constant corank one associated to irreducible
representation of a reductive group.

\medskip\noindent

We are interested in vector spaces $U \subset \text{Mat}_{m,n}(\mathbb{C})$,
with $m \leq n$, of {\it constant rank}, i.e. such that for all $f \in U$,
$r:=\text{rank}f$ is the same.

Let $\ell(r,m,n):=\text{max}\dim U:U$ is of rank $r$.

\medskip\noindent
{\bf Questions.}
\begin{enumerate}
\item $\ell(r,m,n)=$? In general not known.
\item Find relations among $\ell, r,m$ and $n$.
\item Construction of examples and classification.
\end{enumerate}

\begin{example}
\label{example-what-we-know}
\begin{enumerate}
\item $\ell(1,m,n)=n$,
$$
\begin{pmatrix}
x_1&x_2&\cdots &x_n\\
\\
\\
\\
\end{pmatrix}
$$
\item $\text{rank}=2$? There are two cases ([Atkinson '83], [Eisenbud-Haus '88])
\begin{itemize}
\item Compression spaces,
$$
\begin{pmatrix}
*&*&\cdots &*\\
*&0&\cdots & 0\\
\vdots &\\
*&0&\cdots&0
\end{pmatrix}
$$
\item Skew-symmetric matrices of $3 \times 3$.
\end{itemize}
\item $\ell(r,m,n) \geq n-r+1$. Because you can put a matrix of
$m\times n$ (with the first $r$ rows that can have nonzero entries):
$$
\begin{pmatrix}
x_1 &  x_2 &  \cdots & & x_{n-r+1}\\
& x_1 & x_2 & \cdots &\\
\\
\\
\\
\end{pmatrix}
$$
\end{enumerate}
\end{example}

\begin{theorem}[Westwick '86]
\label{theorem-Westwick}
\begin{enumerate}
\item $n-r+1 \leq \ell(r,m,n) \leq 2n-2r+1$.
\item If $n-r+1 \not| \frac{(m-1)!}{(r-1)!}\implies \ell$
\end{enumerate}
\end{theorem}

We can see these spaces as (subvarieties?) of determinantal varieties
$M_n=\{f \in \text{Mat}_{m,n}(\mathbb{C}):\text{rank}(f) \leq  r\}$. 
[Interpretation via secant varieties inside Segre embedding.]

\medskip\noindent
Consider a map $\varphi: U \to \Hom(V,W)$.
Then $\varphi \in U^* \otimes  V^* \otimes  W$.
We get that
$\varphi$ is of constant rank if and only if
some kernel, image and cokernel are vector bundles.

\medskip\noindent
{\bf Focus of today.} What happens when $U,V,W$ are irreducible
representations of a complex reductive group  $G$?

\medskip\noindent
{\bf Question.} What is the natural equivariant morphism
$$
U \to \Hom(V,W)=V^* \otimes  W
$$
of constant rank?

\medskip\noindent
Consider the case of $G=\text{SL}_2$. All the 
irreducible representations (which are self-dual) are given by
$V(m-1)\cong \mathbb{C}[x,y]_{\text{deg}=m-1}$.

Recall the Clebsh-Gordon decomposition ($m \leq  n$)
$$
V(m-1)\otimes  V(n-1)=\bigoplus_{i=0}^{m-1}V(n-m+2i)
$$

\begin{theorem}[B-Faenzi-Lella '22]
\label{theorem-BFL22}
$$
V(n+m-2)\hookrightarrow \Hom(V(m-1),V(n-1))
$$
is of constant rank (corank 1) if and only if
$$
n-m+2i|m-1
$$
\end{theorem}

\begin{theorem}[B. Faenzi, Fratila '25]
\label{theorem-BFF25}
Let $V(\nu)$, $V(\mu)$ and $V(\lambda)$ be irreducible representations
of a comple reductive group $G$, with 
$$
\dim(V(\mu)) \leq \dim(V(\lambda))_-
$$
If there exists a morphism of representations
$$
\varphi: V (\nu) to \Hom (V (\lambda),V(\mu))
$$
then $\varphi$ is of constant corank 1 if and only if 
there exists a simple root $\alpha_i$ such that
\begin{enumerate}
\item $\lambda=\mu+\nu-\alpha_i$,
\item $\nu$ is a multiple of $\nu$,
\item  $\nu$ is a multple of $\omega_i$
\end{enumerate}
\end{theorem} 

\section{On wrapped Floer homology barcode entropy and hyperbolic sets}
\label{section-wrapped-Floer-homology-barcode-entropy-and-hyperbolic-sets}

\noindent
Rafael Fernandes, UC Santa Cruz.
Differential Geometry Seminar, IMPA. 
September 4, 2025.

\medskip {\bf Abstract.} In this talk, we will discuss the interplay between the
wrapped Floer homology barcode and topological entropy. The concept of barcode
entropy was introduced by Çineli, Ginzburg, and Gürel and has been shown to be
related to the topological entropy of the underlying dynamical system in various
settings. Specifically, we will explore how, in the presence of a topologically
transitive, locally maximal hyperbolic set for the Reeb flow on the boundary of
a Liouville domain, barcode entropy is bounded below by the topological entropy
restricted to the hyperbolic set.

\medskip\noindent

Let $M^n$ be a manifold.
$\omega \in \Omega^2(M)$ is {\it symplectic} if $d \omega=0$ and it is
nondegenerate.

\begin{example}
\label{example-Rn-is-symplectic}
$\mathbb{R}^n$ is symplectic with canonical Darboux form.
\end{example}

Recall the definition of Hamiltonian vector field
associated to a function $H \in C^\infty(M)$.

\begin{definition}
\label{definition-nondegenerate-diffeomorphism}
A diffeomorphism $\varphi:M \to M$ is called
{\it non-degenerate} if $\Phi(\varphi)\cap\Delta \subset M\times M$
(pitchfork, i.e. transversal intersection!).
\end{definition}

Let $M^{2n}$ be a closed symplectic manifold. Arnold's conjecture says
\begin{enumerate}
\item If $\varphi=\varphi_H$ (Hamiltonian flow) is nondegenerate, then
$$
\# \text{Fix}(\varphi_H) \geq \sum_{i=0}^{2n}\dim H_i(M,k)=\dim H_*(M,k)
$$
\item If $\varphi=\varphi_H$ is degenerate, then
$$
\# \text{Fix}(\varphi)\geq \text{Cl}(M)+1
$$
where $\text{Cl}(M)$ is the maximum number of homology classes
we can add before getting to zero.
\end{enumerate}

Why do we care? Because
$$
\# \text{Fix}(\varphi_H)\leftrightarrow\{\text{1-periodic 
orbits of $X_H$}\}
$$

Idea by Floer. Construct an invariant that would 
say something about periodic orbits.

\medskip\noindent
{\bf Question.} Can Floer theory capture other ``dynamical information''? 
(Other than the periodic orbits.)

\medskip\noindent
A {\it persistence module} is a pair $(V,\Pi)$, where $V=\{V_t\}_{t \in
\mathbb{R}}$ is a family of $\mathbb{F}$-vector spaces
and $\Pi=\{\Pi_{st}\}_{s \leq  t}$ is a family of maps such that
\begin{enumerate}
\item $\Pi_{ss}=, \Pi_{ts}\circ \Pi_{rt}=\Pi_{rs}$.
\item $\exists  s \subset \mathbb{R}$ such that
$\Pi_{s t}$ is an isomorphism for $s,t$ in the same connected component of
$R\setminus S$.
\item $\Pi_{s t}$ have finite rank.
\item $\exists s_0$ $V_s=\{0\}$, $s \leq s_0$.
\item $V_t= \lim_{s \to t} V_s$ (lower limit!!)
\end{enumerate}

\begin{theorem}
\label{theorem-persistence-module-is-sum-of-integral}
Any persistence module is a sum of integral persistence modules,
$$
(V,\Pi) \cong \bigoplus_{I \in B(V)}F(I).
$$
\end{theorem}

\begin{example}
\label{example-hart-and-sphere}
Heart and sphere. There is a noise in the persistence module
of the heart due to an unnecessary critical point.
\end{example}

$(M^{2n},\omega)$ a {\it Liouville domain} is a compact
symplectic manifold and $X \in \mathfrak{X}(M)$ 
with $X \cap\partial M$ (pitchfork, i.e. transversal intersection!)
pointing outwards
and preserved by the symplectic form, i.e. $\mathcal{L}_X \omega=\omega$
($\omega=d \alpha$).

When we restrict $\omega$ to the boundary,
we obtain a contact form and get some
interesting dynamics.

A Lagrangian $(L,\partial L) \subset (M,\partial M)$ 
is {\it asympotically conical} if
\begin{enumerate}
\item $\partial L \subset \partial M$ is Legendrian.
\item $L \cap [1-\varepsilon,1] \times \partial M=
[1-\varepsilon,1] \times \partial L$.
\end{enumerate}

\begin{remark}
\label{remark-Hamiltonian}
Take a Hamiltonian $H: \hat{M}\to \mathbb{R}$ such that
$$
\begin{cases}
H(r,x)=h(r)\qquad &r=1 \\
H(r,x)=rT-B\qquad &
\end{cases}
$$
then $X_H=h'(r)R_\alpha$.

For $L_0,L,A\subset$ Lagrangians, $H$ linear at infinite,
then $A_{H}^{L_0 \to L_1}$, 
\begin{align*}
A_{H}^{L_0 \to L_1}: P_{L_0\to L_1}  &\longrightarrow \mathbb{R} \\
\gamma &\longmapsto \int_0^1 \gamma^* \alpha- \int_0^1H(x(t))dt
\end{align*}
where $P_{L_0\to L} =\{\gamma:[0,1] \to \hat{M}:\gamma(0) \in L_0,
\gamma(1) \in L_1\}
$ is the set of chords.
\end{remark}

\begin{remark}
\label{remark-1-chords-are-crticial-points}
$\text{crit}(A_{H}^{L_0 \to L_1})=\{\text{1-chords of $X_H$ from 
 $L_0$ to $L_1$}\}$.
\end{remark}

Putting a metric on $P_{L_0 \to L_1}$ we can consider $\varphi:\mathbb{R} \times
[0,1] \to \hat{M}$, solutions of some PDE which is some kind
of generalization of a gradient, $- \nabla A_{H}^{L_0 \to L_1}$.
These solutions can be put in a moduli space
$$
\tilde{\mathcal{M}}(x_-,x_+,H,J)=\{\varphi\text{ solutions s.t. …}\}
$$
Then we define a boundary operator $\partial$.
\begin{theorem}
\label{theorem-homology}
$\partial^2=0$
\end{theorem}

So that we have a homology, called {\it wrapped Floer homology}
$HW^t(H,L_0,L_1,J)$

\begin{remark}
\label{remark-embedding}
We have $H \leq  K \rightsquigarrow HW^t(H,L_0,L_1,J) \to
HW^t(K,L_0,L_1,K)$.
\end{remark}

\begin{definition}
\label{definition-direct-limit}
For $t \geq 0$
$$
HW^t(M,L_0,L_1)=\underline{\lim }_H HW^t(H,L_0,L_1,J)
$$
\end{definition}

(Where we have taken direct limit.) 

Taking direct limit of the homology,
we make sure the homology theory is independent of the
choice of objects (I think, complex structure and Hamiltonian)
we used to construct it.


\begin{proposition}
\label{proposition-persistence-module}
$t \to HW^t(M,L_0,L_1)$ is a peristence module $B(M,L_0,L_1).$
\end{proposition}

Finally we can define {\it barcode entropy}. Fix $\varepsilon>0$,
$t \geq 0$,
$$
b_\varepsilon(M,L_0,L_1,t)=
\# \{\text{ of bars in }B(M,L_0,L_1) \text{with length $\geq \varepsilon$ 
and start before $t$}\}
$$
Then
$$
\bar{h}^{HW}(M_0,L_0,L_1)=
\lim_{\varepsilon \to 0} \text{lim sup}_{t \to \infty} 
\frac{\text{log}^+(b_\varepsilon(M,L_0,L_1,t)}{t}
$$

\medskip\noindent
Consider a contact manifold $(\Sigma,\lambda,L_0,L_1)$ 
and $A$ the Lagrangians. (Example raising question of filling.)

\begin{theorem}[M '24]
\label{theorem-entropy-is-independent-of-filling}
$\bar{h}^{HW}$ is independent of the filling.
\end{theorem}

\begin{theorem}[M '24]
\label{theorem-bound}
$\bar{h}^{HW}(M,L_0,L_1) \leq  h_{\text{top}}(\alpha)$.
\end{theorem}

\begin{theorem}[M '25]
\label{theorem-restriction}
Consider $(M,L_0,L_1)$. 
Let $K$ be a compact topologically transitive hyperbolic set
for the Reeb flow $\alpha$. Assume $W_\delta ^s (q) \subset \partial L_0$,
$W_\delta^s(p) \subset \partial L_1$. Then
$$
\bar{h}^{HW}(M,L_0,L_1)\geq  h_{\text{top}}(\alpha |_{K})>0.
$$
\end{theorem}
Which says it captures dynamics beyond unconditional phenomena.
In lower dimensions these tend to coincide, but in higher dimension we don't
know. This is related to
 ``the sup over hyperbolic sets […]''

Here's a conjecture:
$$
\text{sup}_{L_0,L_1}\bar{h}^{HW}(M,L_0,L_1)=h_{\text{top}}(\alpha).
$$
\medskip\noindent
{\bf Extra comments.} 
One of the aims is to describe topological entropy $h_\text{top}$ 
using Floer theory. Theorems by Çineli-Ginzburg-Gürel show bounds
of topological entropy and barcode entropy (one of which is for
hyperbolic sets).

There is a notion of  {\it admissible Hamiltonian and Reeb vector fields}
which is related to some asymptotical behaviour  ``linear at infinity''.
I understand that admissible vector fields give 
the interesting chords for the Floer homology construction.

\section{Revisiting cotangent bundles}
\label{section-revisiting-cotangent-bundles}

\noindent
Mieugel Cueca, KU Leuven.
Symplectic Geometry Joint Seminar, IMPA. 
September 5, 2025.

\medskip {\bf Abstract.} Cotangent bundles provide key examples of symplectic
manifolds. On the other hand, one can think of Lie groupoids as generalizations
of manifolds. In this context, Alan Weinstein constructed their cotangent
bundles and proved that they are so-called symplectic groupoids. In this talk, I
will recall this construction and explain what happens when one replaces a Lie
groupoid with a Lie 2 (or n)-groupoid. If time permits, I will exhibit some of
their main applications. This is joint work with Stefano Ronchi.

\medskip\noindent

Recall the basic properties of the cotangent
bundle $T^*M$ for a symplectic manifold:
\begin{enumerate}
\item It's a vector bundle.
\item $\left<\cdot,\cdot\right>:TM \otimes T^*M \to \mathbb{R}_M$ 
is the dual pairing.
\item $\omega_{\text{can}}\in \Omega^2(T^*M)$ is symplectic.
\item $\mathcal{L}_\varepsilon \omega_{\text{can}}=\omega_{\text{can}}$,
$\varepsilon$ Euler vector field.
\end{enumerate}

\medskip\noindent
{\bf Main goal.} Reproduce the above for Lie $n$-groupoids.

For $n=0$ we get the above situation. For $n=1$ [Duzud-Weinstein], [Prezun],
for $n\geq 2$ ? and we care about $n=2$.

\begin{definition}
\label{definition-groupoid}
A {\it Lie $n$-groupoid} $\mathcal{G}:\Delta^{\text{op}}\to \mathsf{Man}$ 
such that
$$
P_{e,j}:\mathcal{G}_{\ell}\to \Lambda_j^\ell \mathcal{G}
$$
are surjective submersions $\forall  \ell,j$ 
are differomorphisms for $\ell > n$.
\end{definition}

\begin{remark}
\label{remark-generators}
This sort of manifolds-valued presheaf category is generated by
\begin{align*}
d_i^\ell:\mathcal{G}_\ell \to \mathcal{G}_{\ell-1}&\text{face maps}
0 \leq  i,j \leq  \ell\\
s_j^\ell:\mathcal{G}_\ell \to \mathcal{G}_{\ell+1}&  \qquad \text{degeneracies}
\end{align*}
\end{remark}

\medskip\noindent
The tangent is a functor, it satisfies
$$
T_\bullet(\mathcal{G})=T_k(\mathcal{G})=T\mathcal{G}_k
$$
(It looks like $T$ preserves diagrams.)

\medskip\noindent
{\bf Dold-Kan.} The category $\mathsf{SVect}$ of simplicial vector spaces
has objects
$$
\xymatrix{
\mathbb{V}_\bullet&  \mathbb{V}_n\ar[r]&\cdots
\ar[r]&\mathbb{V}_2\ar[r]^{\text{3 arrows}}&
\mathbb{V}_1\ar[r]^{\text{2 arrows}}&\mathbb{V}_0
}
$$
where the $\mathbb{V}_i$ are vector spaces.

There is a functor
$$
\mathsf{SVect}\overset{N}{\to}\{\text{chain complexes}\geq 0\}
$$
$$
\mathbb{V}_{\bullet}\to N\mathbb{V}=
\left(N_\ell \mathbb{V}\Ker P_{\ell,\ell},\partial=d_\ell\right)
$$

\begin{theorem}[Dold-Kan]
\label{theorem-Dold-Kan}
That's an equivalence of categories. [Confirm this!]
\end{theorem}

Those categories are monodial:
$$
(\mathsf{SVect},\otimes),\qquad  (\mathbb{V}_\bullet \otimes \mathbb{W})_\ell
=\mathbb{V}_\ell \otimes \mathbb{W}_\ell
$$
$$
(\mathsf{ch}_{\geq 0},\otimes),\qquad (V \otimes W)_i=
\bigoplus_{\ell +k=1}V_\ell \otimes W_k
$$
And $N$ is Lax monoidal with Lax structure given by the
Eilenberg-Zilber map, though we won't explain the details of this.

There are duals given by internal Hom:
\begin{align*}
\mathbb{V}^{n*}&=\underline{\Hom}(\mathbb{V},B^n\mathbb{R})
\end{align*}
Where the internal Hom is given by
$$
\underline{\Hom}(\mathbb{V},B^n\mathbb{R})_\ell
=\Hom_{\mathsf{SVect}}(\mathbb{V}\otimes\Delta_n[\ell],B^n\mathbb{R})
$$
for an object $\Delta[\ell]=\mathbb{R}[\Delta[\ell]]$.


\medskip\noindent
{\bf Properties.}
\begin{enumerate}
\item $\mathbb{V}^{n*}$ isa simplicial vector bundle.
\item $N(V^{n*})$ and $N(\mathbb{V})^*[n]$ is a quasi isomorphism.
\item $\left<\cdot,\cdot\right>:\mathbb{V}\otimes \mathbb{V}^{n*}\to
B^n\mathbb{R}$ is non-degenerate on homology.
\item $\mathbb{V}\hookrightarrow (\mathbb{V}^{n*})^{n*}$ 
Mont. a eq.
\end{enumerate}

\medskip\noindent
{\bf The vector bundle case.}
$$
\text{Maps}(\Delta[i],\mathcal{G})_k=
\Hom_{\mathsf{SSet}}(\Delta[i]\times\Delta[k],\mathcal{G})
$$
\begin{proposition}
\label{proposition-on-an-n-Lie-groupoid}
Let $\mathcal{G}$ be a Lie $n$-groupoid.
\begin{enumerate}
\item $\text{Maps}(\Delta[i],\mathcal{G})$ Lie $n$-groupoids ME $\mathcal{G}$.
\item $\text{Maps}(\Delta[i],\mathcal{G})_0=\mathcal{G}_i$.
\item $\text{ev}:\Delta[i]\times \text{Maps}(\Delta[i],\mathcal{G})\to
\mathcal{G}$.
\end{enumerate}
\end{proposition}

[Staircase looking diagram.]

\begin{definition}
\label{definition-what-is-this}
$\mathcal{G}_\bullet$.
$$
T_i^{n*}\mathcal{G}=\Hom_{\mathsf{SVect}}(1^*  \Pi_{\Delta[i]}(T\mathcal{G}),
B^n\mathbb{R}_{\mathcal{G}_i})
$$
$$
(d_j,F)_{K|d_j\mathcal{G}}(x^a)=(F_k)|_{\mathcal{G}}(x^{\delta_ja}).
$$
\end{definition}

\begin{proposition}
\label{proposition-properties-of-that}
$\mathcal{G}$ Lie $n$-groupoid, then $T^{n*}\mathcal{G}$ satisfy
\begin{enumerate}
\item is a vector bundle $n$-groupoid
\item dual to $T\mathcal{G}$
$$
\left<\cdot,\cdot\right>:T\mathcal{G} \otimes T^{n*}\mathcal{G}
\to B^n\mathbb{R}_{\mathcal{G}}
$$
non-degenerate on homology.
\item $n$-shifted symplectic
$$
T^{n*}_n\mathcal{G}\overset{p}{\to}T^*\mathcal{G}_n
$$
and $p^*\omega_{\text{can}}$.
\end{enumerate}
\end{proposition}

[More computations I missed]

\section{A theorem on complexifications of Lie groups}

\section{Holomorphic extensions of s-proper Lie groupoids}
\label{section-holomorphic-extensions-of-s-proper-Lie-groupoids}

\noindent
Rui L. Fernandes, ?.
Symplectic Geometry Seminar, IMPa. 
September 17, 2025.

\medskip {\bf Abstract.} Every smooth manifold admits a compatible analytic
structure, and a classical result of Whitney–Bruhat states that any analytic
manifold has a holomorphic extension. Lie groups also admit compatible analytic
structures, and another classical result, due to C. Chevalley, shows that any
compact Lie group has a holomorphic extension to a complex Lie group. D.
Martínez Torres has shown that any proper Lie groupoid admits a compatible
analytic structure. I will discuss an extension of the classical results of
Whitney–Bruhat and Chevalley, establishing that any s-proper Lie groupoid has a
holomorphic extension. This talk is based on recent joint work with Ning Jiang
(arXiv:2508.18036).

\medskip\noindent

\begin{theorem}[Whitney-Beuhat]
\label{theorem-Whitney-Beuhat}
If $M$ is analytic there exists a complex manifold  $M_\mathbb{C}$
together with an analytic map $i: M \to M_\mathbb{C}$ 
which is totally real ($T_HM_{\mathbb{C}}=TM \oplus J(TM)$)
and
\begin{enumerate}
\item For every complex manifold $X$ and every
analytic map $\phi:N \to X$ there exists $\phi^*:U \to X$ 
holomorphic contained open $M \subset U \subset M_\mathbb{C}$.
\item If $\psi:V \to X$ holomorphic on $M \subset V \subset M_\mathbb{C}$ then
$\psi=\phi^*$ where $\phi=\psi \circ i$ on a possibly
smaller open contained in $U \cap V$.
\end{enumerate}
\end{theorem}

\begin{theorem}[Chevalley]
\label{theorem-Chevalley}
If $G$ is a Lie group, there exists a complex
Lie group $G_{\mathbb{C}}$ and a morphism $i:G \to G_\mathbb{C}$ 
satisfying the following universal property.
For every complex Lie group $H$ and morphism
$\phi:G \to H$ there exists a unique
holomorphic map $\phi^*:G_\mathbb{C} \to H$ such that
$\phi=\phi^* \circ i$.
\end{theorem}

Here's the construction of $G_\mathbb{C}$:

$$
\xymatrix{
N\subset \tilde{G}\ar[r]\ar[d]&G^*\ar[d]\\
\tilde{G}/N \simeq G\ar[r]_{i}&G_\mathbb{C}=G^* /\overline{i^*N}
}
$$
where $\tilde{G}$ is the universal cover group of $G$,
 $G^*$ is the group that integrates to 
the complexification of the Lie algebra of $G$,
i.e. $\text{Lie}(G^*)=\mathfrak{g}_\mathbb{C}$
and $\overline{i^*(N)}$ is the smallest closed
normal complex Lie subgroup of $G^*$
containing $i^*(N)$.

\begin{example}
\label{example-complexification-groupoid}
$$
\xymatrix{
\widehat{\text{SL}_2(\mathbb{R})}\ar[d]\ar[dr]\\
\text{SL}_2(\mathbb{R})\ar@{^{(}->}[r]&\text{SL}_2(\mathbb{C})
}
$$
which is a simple case,
but consider instead $\tilde{\text{SL}_2}(\mathbb{R})\times \mathbb{R}$ 
and the normal subgroup $N=\left<a\right>\times \left<\lambda\right>$ 
for irrational $\lambda$. 
Then we obtain
$$
\xymatrix{
\widehat{\text{SL}_2(\mathbb{R})}\times\mathbb{R}\ar[r]\ar[d]
&\text{SL}_2(\mathbb{C})\times\mathbb{C}\ar[d]\\
\widehat{\text{SL}_2(\mathbb{R})}\times\mathbb{R}/N\simeq G\ar[r]&
G_\mathbb{C}
}
$$
and $G_\mathbb{C}$ is 3 complex dimensions! It is not
$\text{SL}_2(\mathbb{C})\times\mathbb{C}$.
\end{example}

\section{Everything you always wanted to know about polygons but were too
afraid to ask}
\label{section-everything-you-awlways-wanted-to-know}

\noindent
Alessia Mandini, UFF.
GAAG, IMPA. 
September 22 and 23, 2025.

\medskip
{\bf Abstract.} Moduli spaces of polygons form a family of Kähler
manifolds that can be constructed as a Kähler reduction of coadjoint orbits.
These spaces have deep connections to various areas of mathematics, including
symplectic and algebraic geometry as well as representation theory. In this
talk, I will define these spaces and explore their connections. I will then
discuss wall-crossing phenomena in these spaces and demonstrate how it can be
used to determine their cohomology rings. Finally, I will introduce the
hyperkähler analogue of these spaces, known as hyperpolygon spaces, and describe
some of their generalizations.

\medskip\noindent



Plan of the talk
\begin{enumerate}
\item Polygons in $\mathbb{R}^3$ and relations with
other moduli spaces.
\item Polygons in other spaces.
\end{enumerate}

Let $\alpha=(\alpha_1,\ldots,\alpha_n)\in \mathbb{R}^n_{>0}$.
Consider $S^2$ with its usual symplectic, Kähler form.
Also consider the product of several spheres.
There's a Hamiltonian action of $\text{SO}(3)$ by rotations.
The moduli space of polygons is
the symplectic reduction obtained with this
Hamilatonian action,
$$
M(\alpha)=\prod S^2_{\alpha_i}/\!/\text{SO}(3).
$$

Why is it called the space of polygons?
We think that
$$
[v_1,\ldots,v_n] \in M(\alpha) \iff \sum_{i=1}^nv_i=0
$$
are polygons.

When smooth, $M(\alpha)$ is a $(n-3)$-complex-dimensional
Kähler manifold. Consider
$$
\varepsilon_I(\alpha)=\sum_{i \in I}\alpha_i
-\sum_{j \in I^c}\alpha_j
$$
for some index set $I \subseteq \{1,\ldots, n\}$.
$M(\alpha)$ is {\it smooth} if $\varepsilon_I(\alpha)
\neq 0$ for all $I\subseteq \{1,\ldots,n\}$.
If so, we say $\alpha$ is {\it generic}.

\begin{theorem}[Kapovich-Millson]
\label{theorem-Kapovich-Millson}
$M(\alpha)$ is a complex analytic space
with (eventually) isolated singularity
(homogeneous quadratic cones).
\end{theorem}

\begin{example}
\label{examples-polygon-moduli-spaces}
\begin{enumerate}
\item $(n=3.)$ Then $M(\alpha)$ is either empty or a point.

\item $(n=4.)$ $M(\alpha)$ is either empty or a sphere.
\end{enumerate}
\end{example}

\begin{remark}[Hausmann-Knutson]
\label{remark-Hausmann-Knutson}
Let $M_n$ be the space of all $n$-gons
modulo rigid motions. Then $M_n$ 
can be equipped with a Poisson structure
for which $M(\alpha)$ are the symplectic
leaves. 

\medskip\noindent
{\bf Polygons as quiver varieties.}
Consider the star-shaped quiver,
which is a distinguished point with some points around it,
and arrows from every point to the distinguished one.
Let the distinguished point be $V_0=\mathbb{C}^2$ and
the rest $\mathbb{C}$. Then a representation of this quiver
is
$$
\text{Rep}Q=\bigoplus_{i=1}^n \Hom(\mathbb{C},\mathbb{C}^2)=\mathbb{C}^{2n}.
$$
Now put
$$
K=(U(2) \times U(1)^n)/\Delta
$$
where $\Delta$ is the diagonal $S^1$.
This gives a Hamiltonian action on $\mathbb{C}^{2n}$
as follows. For
$$
(A,\lambda_1,\ldots,\lambda_n)\cdot (q_1,\ldots,q_n)
$$
we map
$q_i \mapsto  A^{-1}q_i \lambda_i$.
That is
\begin{align*}
\mu: \mathbb{C}^{2n} &\longrightarrow K^* \\
(q_1,\ldots,q_n) &\longmapsto 
\left(\sum_{n=1}^n (q_iq_i^* ),\ldots,\frac{1}{2}|q_i|^2\ldots\right)
\end{align*}

\begin{theorem}[Hausmann-Knutson]
\label{theorem-Hausmann-Knutson}
$$
\mathbb{C}^{2n}/\!/_{(0,\alpha)}K=M(\alpha)
$$
\end{theorem}

\begin{proof}
Uses 
\begin{align*}
\mathbb{R}^3  &\xrightarrow{\simeq} \mathfrak{su}(2)^* \\
v_i &\longmapsto (q_iq_i^*)_0
\end{align*}
\end{proof}
\end{remark}

$$
\xymatrix{
&\mathbb{C}^{2n}\ar[dl]_{/\!/_\alpha U(1)^n}
\ar[dr]^{/\!/U(2)}\\
\prod S^2 \ar[dr]_{/\!/_\alpha \text{SO}(3)}
 &  &  \text{Gr}(2,\mathbb{C}^n)\ar[dl]^{/\!/_\alpha U(1)^n}\\
&M(\alpha).
}
$$
So we have (I think former symplectic reduction)
$$
\mu_{U(1)^n}:\text{Gr}(2,\mathbb{C}^n)\to \mathbb{R}^n
$$
\medskip\noindent
{\bf Walls and wall crossing.}
The walls are
$$
W_I=\{\alpha \in \mathbb{R}^n_{>0}:\varepsilon_I(\alpha)=0\}
$$
for $I \subseteq \{1,\ldots,n\}$.

To understand wall crossing suppose we have
a wall $W_I$, with $a^c$ in the wall,
$\alpha^+$ on one side and $\alpha^-$ on the other.

Not that
$$
\varepsilon_I(\alpha^+)>0 \iff
\sum_{k \in I}\alpha_i^+> 
\sum_{ j \in I^c}\alpha_j^+
$$
Define
\begin{align*}
M_{I^c}(\alpha^+)
&=\{(v_1,\ldots,v_n) \in M(\alpha^+):
v_i=\lambda v_j \forall i,j \in I^c, \lambda>0\}\\
&=M(\tilde{\alpha}),\qquad \tilde{\alpha}=
\left(\alpha_{i_1},\ldots,\alpha_{i_k},\sum_{j \in I^c}\alpha_j\right)
\end{align*}
Then $\varepsilon_I(\alpha^+)<0$ 
implies $M_I(\alpha^-) \subseteq M(\alpha^-)$.

$$
\xymatrix{
&\tilde{M} \subseteq E=M_I \times M_{I^c}\ar[dd]^{\text{blow up}}\\
M(\alpha^+)\ar[dr]&  &  M(\alpha^-)\ar[dl]\\
&M(\alpha^c)
}
$$
\begin{remark}
\label{remark-moduli-spaces-of-polygons-proportional}
\begin{itemize}
\item $M(\alpha)\cong M(\sigma(\alpha))$
for $\sigma$ a permutation on the order of sets.
\item $M(\alpha)$ is conformally symplectomorphic
to $M(\lambda \alpha)$ for all $\lambda>0$.
\end{itemize}
\end{remark}

\begin{definition}
\label{definition-short-and-long}
Let $\alpha \in \mathbb{R}^n_{>0}$.
$I \subseteq \{1,\ldots,n\}$ is {\it short} 
if $\varepsilon_I(\alpha)<0$ and {\it long}
if  $\varepsilon_I(\alpha)>0$.
\end{definition}

\begin{example}
\label{example-wall-crossing}
We did an example of wall crossing.
The relevant manifolds $M_{I^c}(\alpha)$ and $M_I(\alpha)$ 
were projective spaces.
\end{example}

\medskip\noindent
Now recall our first quotient
$$
\xymatrix{
\mu_{U(1)}^{-1}(\alpha)\subseteq \text{Gr}(2,\mathbb{C}^n)
\ar[d]^{/\!/U(1)^n}\\
M(\alpha)
}
$$
Let $c_i$ be the first Chern
classes associated to the
$n$ $S^1$-bundles above
(by taking reduction in stages).

\begin{theorem}[Haussmann-Knutson,M.]
\label{theorem-Haussmann-Knutson-M}
The $c_i$ generate the cohomology of $M(\alpha)$.
\end{theorem}

\begin{theorem}[Guillemin-Stenberg]
\label{theorem-Guillemin-Stenberg}
In this setup, for $\alpha$ generic,
$$
H(M)=\mathbb{C}[c_1,\ldots,c_n]/\text{Ann}(\text{Vol}(M(\alpha)))
$$
i.e., $Q(c_1,\ldots,c_n) \in \text{Ann}(\text{Vol}M(\alpha))$ 
$\iff$ $Q\left(\frac{\partial }{\partial \alpha_i},\ldots,
\frac{\partial }{\partial \alpha_1}\right)
\text{Vol}(M(\alpha))=0$
\end{theorem}

\begin{theorem}[Takakura, The Koi]
\label{theorem-Takakura-The-Koi}
$$
\text{Vol}(M(\alpha))
=-\frac{(2\pi)^{n-3}}{(n-3)!}
\sum_{I\text{ long}}(-1)^{n-|I|}\varepsilon_I(\alpha)^{n-3}.
$$
\end{theorem}

\begin{example}
\label{example-Delta1}
According to Example \ref{example-wall-crossing} (which I
did not copy) we find that
$\alpha_1 \in \Delta_1$ gives $\text{Vol}(M(\alpha_1))=2\pi^3(1-2\alpha_3)^2$
and
$$
H(M(\alpha_1))=\mathbb{C}[c_3]/(c_3^3)
$$
\end{example}

\medskip\noindent
{\bf Polygon game.}
Let $G$ be a Lie group and $\mathfrak{g}$ its Lie algebra.
Then the co-adjoint orbits
$\mathfrak{g}^*  \supseteq \mathcal{O}_{\xi_i}$ 
are symplectic manifolds with the KKS form.
Then
\begin{align*}
G\mathbb{y} \Pi \mathcal{O}_{\xi_i}  &\longrightarrow \mathfrak{g}^* \\
(A_i,\ldots\alpha_n) &\longmapsto \sum A_i
\end{align*}
Then the quotient
$$
M(\xi)=\Pi \mathcal{O}_{\xi_i}/\!/_0 G
$$
generalizes the previous construction,
which we get with $G=\text{SU}(2)$.

It could be interesting to investigate
which of the next constructions can be generalized
to other Lie groups.

\begin{theorem}[Sotillo-Floentins-Gadihl]
\label{theorem-Sotillo}
Wall-crossing for $\text{SU}(m)$.
\end{theorem}

\medskip\noindent
{\bf Bending action.}
We consider a polygon of $n$ sides
and put some diagonals that don't intersect.
We introduce some
notion of ``bending'' that allows
to define a Hamiltonian function.
In turn, this defines a torus action on $M(\alpha)$
and a moment map. 

For $n=2$ we obtain the moment map
\begin{align*}
\mu: M(\alpha) &\longrightarrow \mathbb{R}^2 \\
p &\longmapsto (\ell_1(p),\ell(p))
\end{align*}
and we can see the moment polytope in $\mathbb{R}^2$.

\begin{remark}
\label{remark-nonvanishing-and-nonintersecting-diagonals}
Any system of $(n-3)$ non-vanishing
and non-intersecting diagonals determine
a torus action $T^{n-3}\mathbb{y} M(\alpha)$.
\end{remark}

\medskip\noindent
{\bf Relations of polygon spaces to other moduli spaces.}
Representations of the fundamental group
of the punctured sphere in $SU(2)$, i.e.
$$
\text{Rep}(\pi_1(S^2\setminus \{p_1,\ldots,p_n\},\text{SU}(2))/\text{SU}(2) 
\cong
M(\alpha)
$$ 
This can be put very explicitly:
\begin{align*}
&\text{Rep}(\pi_1(S^2\setminus \{p_1,\ldots,p_n\},\text{SU}(2))\\
&=\{(g_1,\ldots,g_n)\in \text{SU}(2)^n:g_1\cdot\ldots\cdot g_n=\text{Id},
t_rg_i=2\cos\pi \alpha_i\}
\end{align*}
See [Agapito-Godinho]. That's all we
will say about this example.

\medskip\noindent
Now consider the moduli space of parabolic bundles.
Consider a holomorphic bundle of rank 2 over $\mathbb{C}P^{2}$.
Choose some points $D=\{x_1,\ldots,x_n\}$ in $\mathbb{C}P^{2}$ 
and a flag
$E_{x_i}=E^{x_i,1}\supseteq E^{x_i,2}\supseteq \{0\}$ 
for every $x_i \in D$. Each of the
$E_{x_i,j}$ is isomorphic to $\mathbb{C}$
(1-dimensional).
A {\it quasi-parabolic bundle} 
is an holomorphic bundle $E$
with such a choice of flag.
A {\it parabolic bundle} is a quasi-parabolic
bundle with a choice of parabolic weights
$0< \beta_1(x_i)< \beta_2(x_i)<1$.

There is also a notion of stability:
\begin{align*}
\text{pdeg}E&=\text{deg}E+ \sum_{i=1}^n (\beta_1(x_i)+\beta_2(x_i))\\
\mu(E)&=\frac{\text{pdeg}(E)}{\text{rank}(E)}
\end{align*}
We say $E$ is {\it (semi)stable} if
$µ(E)> \mu(L)$ ($\mu(E) \geq \mu(L)$)
for any $L \subseteq E$ parabolic
subbundle.

Then we obtain that $\mathcal{M}_{\pi,d}(\beta)$
is the moduli space of (semi)-stable
parabolic bundles on $\mathbb{P}^1$ 
of rank $r$ and degree $d$.
In particular $\mathcal{M}_{2,0}(\beta)$
is the moduli space of (semi)-stable parabolic
bundles on $\mathbb{P}^1$ of rank 2 and degree 0
holomorphically trivial.

\begin{theorem}[Jeffrey, Godinho, M.]
\label{theorem-Jeffrey-Godinho}
For generic $ \alpha$,
$M(\alpha)$ is diffeomorphic to
$\mathcal{M}_{2,0}(\beta)$ 
whenever $\alpha_i=\beta_2(x_i)-\beta_1(x_i)$.
\end{theorem}

\begin{proof}[Idea of proof]
The correspondence can be made quite explicit
as a map
\begin{align*}
M(\alpha) &\longrightarrow \mathcal{M}_{2,0}(\beta) \\
[q_1,\ldots,q_n] &\longmapsto \substack{E=\mathbb{C}P^{1}\times\mathbb{C}^2
 \\ E_{x_i}\supset E_{x_i,1}\supset X_{x_i,2}\supset \{0\}}\end{align*}

\end{proof}

\medskip\noindent
Now consider the quiver variety we described above:
$$
\text{Rep}Q=\bigoplus_{i=1}^n \Hom(\mathbb{C},\mathbb{C}^2)=\mathbb{C}^{2n}.
$$
Now put
$$
K=(U(2) \times U(1)^n)/\Delta
$$
where $\Delta$ is the diagonal $S^1$.

But this time put
$$
\text{Rep}\tilde{Q}=\bigoplus_{i=1}^n \Hom(\mathbb{C},\mathbb{C}^2)
\oplus \bigoplus_{i=1}^n \Hom(\mathbb{C}^2,\mathbb{C})
$$
We also have an action of $K$, and we get
$$
K \mathbb{y} \text{Rep}\tilde{Q} = T^*  \mathbb{C}^{2n},
$$
a so-called {\it hyperHamiltonian action}.
The quotient
$$
X(\alpha)=/\!/\!/_{\substack{(0,\alpha) \\ (0,0)}}K=
\frac{\mu_{\mathbb{R}}1(0,\alpha) \wedge \mu^{-1}_{\mathbb{C}}(0,0)}{K}
$$
called {\it hyperpolygon space}.

Here
\begin{align*}
\mu_\mathbb{R}:T^*\mathbb{C}^{2n}&\to \mathcal{K}^* \\
\mu_{\mathbb{C}}:T^*\mathbb{C}^{2n}&  \to \mathcal{K}_{\mathbb{C}}^* \\
(q_1,\ldots,q_n,p_1,\ldots,p_n)&  \mapsto 
(\sum_{i=1}^n(p_i,q_i)_0,\ldots)
\end{align*}
It turns out that $X(\alpha)$ is smooth
if and only if $\varepsilon_1(\alpha)=0$ for all
$I \subseteq \{ 1,\ldots,n\}$.
When smooth,  $X(\alpha)$ is a hyperkähler manifold
(non-compact, unfortunately),
$M(\alpha)=\{[0_{1i},0] \in X(\alpha)\}$.

\begin{theorem}[Boalch]
\label{theorem-Boalch}
$X(\alpha)$ is the moduli space of polygons for $\text{GL}(2,\mathbb{C})$.
\end{theorem}

\medskip\noindent
{\bf Parabolic Higgs bundles.}
Let $E$ be a parabolic bundle as before, i.e.
$E \in \mathcal{M}_{0,2}(\beta)$.
A {\it Higgs field on $E$} is
$$
\phi \in H^{0}(\mathbb{P}^1,\text{SPEnd}(E) \otimes K_{\mathbb{P}^1}(D)))
$$
where a {\it strongly parabolic endomorphism} is  $f:E \to E$
such that $f(E_{x_i})\subseteq E_{x_{i+1}}$ for all $i$.
(Fix details!)

A {\it parabolic Higgs bundle} is (probably the pair $(E,\phi)$).
A parabolic Higgs bundle $(E,\phi)$ is {\it (semi)stable}
if  $\mu(E) > \mu(L)$ ($\mu(E) \geq \mu(L)$) for all 
$L$ parabolic Higgs subbundle.

Then $\mathcal{N}_{r,a}^{0,1}(\beta)$ is the moduli
space of parabolic Higgs bundles over $\mathbb{P}^1$ 
with rank $r$ and degree $d$ (with fixed determinant,
and traceless; two notions that we will not define here).


\begin{theorem}[Goldinho, M.; Biswar, Florentino, Godinho, M.]
\label{theorem-GBSFG}
Let $\alpha$ be generic, let $\beta$ 
be such that $\alpha_i=\beta_2(x_i)-\beta_1(x_i)$.
Then the hyperpolygon space $X(\alpha)$ 
is symplectomorphic to the moduli space
of parabolic HIggs bundles over $\mathbb{P}^1$, $\beta$-stable,
holomorphically trivial (with fixed determinant and trace-free).


\begin{proof}[Idea of proof]
We can define a correspondence
\begin{align*}
X(\alpha) &\longrightarrow \mathcal{H}(\beta) \\
[p,q] &\longmapsto \substack{E=\mathbb{C}P^{1}\times \mathbb{C}^2
 \\ E_{x_i}\simeq E_{x_i,1}\supset E_{x_i,2\supset \{0\}}}
\end{align*}
No we can use the moment map condition
that the sum of the residues is zero,
where $\text{Res}_{x_i}\phi=(p_i,q_i)$
where $\phi$ is the unique function
that satisfies the last equality by
the residue theorem.
\end{proof}
\end{theorem}

\medskip\noindent
{\bf Wall-crossing for moduli spaces of parabolic bundles.}
See ``Translation''from Thaddeus.
What happens is that there is an $S^1$-action on $X(\alpha)$ given by
$$
\lambda\cdot[p,q]=[\lambda p,q].
$$
(This can be paralleled with the $S^1$-action on $\mathcal{H}(\beta)$,
given by $\lambda\cdot[E,\phi]=[E,\lambda\phi]$.)

This action has a moment map
$$
\mu_{S^1}([p,q])=\frac{1}{2}\sum_{i}|p_i|^2
$$
Fixed points are [Konno]
$$
X_S=\{[p,q] \in X(\alpha):
S,S^c \text{ are straight, }p_j=0 \forall j \in S\}
$$
for all $|S|\geq 2$ short. Here $S\subseteq \{1,\ldots,n\}$ 
is {\it straight} if $q_i=\lambda_j q_j$
for $\lambda_j>0$ $\forall  i,j \in S$ 
(which in the case of polygons means the edges are aligned).

Let $U_S$ be the flow down from $X_S$
Then, crossing a wall  $W_I$ as defined above,
i.e. $W_I=\{\alpha \in \mathbb{R}^n_{>0}:\varepsilon_S(\alpha)=0\}$,
``replaces'' $U_S$ by $U_{S^c}$.

Let $\alpha$ and $\tilde{\alpha}$ be generic,
then $X(\alpha)$ is diffeomorphic to $X(\tilde{\alpha})$.
The wall crossing for $M(\alpha)$
involves
\begin{align*}
M_S(\alpha^+)&=U_S \cap M(\alpha^+)\\
M_{S^c}(\alpha^-)&=U_{S^c}\cap M(\alpha)
\end{align*}
This is thought of as a Mukai transform.

\medskip\noindent
{\bf Generalization.}
See [Florentino, Godinho, Sotillo] for wall crossing.
See also [Fisher] PhD thesis, [Fisher, Rayan].
[Hausel et al], [Rayan-Schopnick].

\section{Vertex algebras and special holonomy on quadratic Lie algebras}
\label{section-vertex-algebras-and-special-holonomy-on-quadratic-Lie-algebras}

\noindent
Mario García Fernandez, ICMAT.
GAAG, IMPA. 
September 24, 2025.

\medskip
{\bf Abstract.} 

The chiral de Rham complex (CDR) is a sheaf of vertex algebras on any smooth
manifold, introduced by Malikov, Schechtman and Vaintrob, which provides a
formal quantization of the non-linear sigma model in mathematical physics.
Motivated by the algebra of chiral symmetries in two-dimensional superconformal
field theories, vertex algebra embeddings on the CDR have been studied for
special holonomy Riemannian manifolds, thanks mainly to the work of Heluani,
Zabzine, and collaborators, with interesting applications to the elliptic genus.
In these lectures, we will discuss extensions of some of these results to the
case of special holonomy manifolds with skew-torsion. The presence of torsion
typically allows for continuous symmetries in the geometry, with an enhanced
interplay with Lie theory and algebra, as well as the application of techniques
from generalized geometry.


\medskip\noindent
Based on joint work with Luis Álvarez Cónsul,
Andoni De Arriba de la Hera. arXiv:2012.01851 (IMRN '24).

\medskip\noindent
{\bf Motivation.}
Let $(M^n,g)$ be a Riemannian spin manifold with
parallel spinor $\nabla^g \varphi=0$.
Then $\text{hol}(g) \subset G_\varphi \subseteq \text{SO}(n)$,
and $\text{Ric}(g)=0$.

There is a construction by Markov-S-V that puts a sheaf
of vertex algebras $\mathcal{V} \to M^n$.
How to construct special embeddings of trivial vertex algebras
in the cohomology of $\mathcal{V}$, i.e.
$\mathcal{V}_\varphi \hookrightarrow H^*(\mathcal{V})$
by Heluani et al.

\medskip\noindent
Applications. Construction of topological invariants.
Elliptic genus, [Borisov-L].
How to understand mirror symmetry using vertex algebras [Borisov].
Holography [Witten].

\medskip\noindent
{\bf Geometry in algebra.}
We shall do geometry in quadratic Lie algebras.
Recall that a {\it quadratic Lie algebra} is
 $(\mathfrak{g},[\cdot,\cdot], (\cdot,\cdot))$ where
$(\mathfrak{g},[\cdot,\cdot])$ is a Lie algebra
(in this course we use $\mathbb{R}$ as base field)
and $(\cdot,\cdot)$ is a symmetric bilinear invariant form.

\begin{example}
\label{example-quadratic-Lie-algebra}
Let $R$ be a Lie algebra with
$\left<\cdot,\cdot\right>_R: R \otimes R \to \mathbb{R}$.
Take $\mathfrak{g}=R \oplus R^*$,
and pick $H \in \Lambda^{3}R^*$.
Put $H(a,b,c)=\left<[a,b],c\right>$, and define
$$
[v+\alpha,w+\varphi]=[v,w]-\varphi([v,-])+\alpha([v,-])+H(v,w,-)
$$
for $v,w \in R$ and $\varphi,\alpha \in R ^*$. This turns out to be a bracket.

If you like geometry you can pick $K$ compact, $\text{Lie}K=R$,
this is ``Generalized geometry on $T K \oplus T^* K$'' 
à la Hitchin.
\end{example}

\medskip\noindent
QLA:
\begin{itemize}
\item Courant algebroids /$\{*\}$.
\item  Symplectic supermanifolds.
\end{itemize}

\medskip\noindent
\begin{definition}
\label{definition-generalizaed}
\begin{enumerate}
\item A {\it (generalized) metric} on $(\mathfrak{g},(\cdot,\cdot))$ is
$G \in \text{End}(\mathfrak{g})$ with $G^2=\text{Id}$, $(G,G)=(\cdot,\cdot)$,

This gives $\mathfrak{g}=V_+ \oplus V_-$ where $G$ acts as identity on the
first term and as $-\text{Id}$ on the second one.
$(G_-,-)$ is a non-degenerate pairing.
$(-,-)_{V_I}$ non-degenerate tensor.

\item A divergence $\varphi \in \mathfrak{g} ^*$.

\item A {\it connection} is 
$$
D: \mathfrak{g} \to \mathfrak{g}^* \otimes \mathfrak{g}
$$
such that $(D_ab,c)+(b,D_ac)=0$.
So that $D \in \mathfrak{g}^*  \otimes \Lambda^{2}\mathfrak{g}$ 
(where we identify $\mathfrak{g}$ with its dual using the pairing).

$D$ is  {\it compatible} with $G$ if
$[D_a,G]=0$. This condition says that $D$ splits into four operators:
$$
\xymatrix{
D^+_+&  &  D_+^-\\
&  D\ar[ul]\ar[ur]\ar[dl]\ar[dr]\\
D^+_-& & D_-^-
}
$$
\item  Let $a \in \mathfrak{g}$, $a=a_++a_-$.
A {\it generalized connection} satisfies
$D_a^+=[a_-,b_+]_+ +[a_+,b_-]_-$. (This will happen to things
living in $\mathcal{D}^0$, see below.)
\end{enumerate}
\end{definition}

\begin{definition}
\label{definition-}
Given $D$ define
\begin{enumerate}
\item $\varphi_D(a)=-\text{tr}Da$.
\item $\text{Torsion}(AX):T_D \in \Lambda^{2}\mathfrak{g}^*$,
$$
T_D(a,b,c)=(D_ab-D_ba-[a,b],c)+(D_ca,b).
$$
(Just copying the definition of torsion.)
\end{enumerate}
\end{definition}

\begin{lemma}
\label{lemma-}
Define $\mathcal{D}^0(G,\varphi)=\{D:G\text{-compatible }, \varphi_D=\varphi\}$.
Then this is non-empty, but is not a point.
Furthermore, $\forall  D \in \mathcal{D}^0(G,\varphi)$,
$$
D_{a_-}b_+=[a_-,b_+]_+,\qquad 
D_{a_+}b_-=[a_+,b_-]_-.
$$
\end{lemma}

(Checking non-emptyness is just writing out the equations.)

\medskip\noindent
Given $G$ consider $Cl (V_+)$,
 $a_+\cdot a_+=(a_+,a_+)$ for $a_+ \in V_+$.

Fix an irreducible representation for $Cl(V_+)$: $S_+$.
Given  $D \in \mathcal{D}^0(G,\varphi)$:
\begin{enumerate}
\item $D_-^{\delta_+} \in V_-^* \otimes \text{End}((S_+)\supset
V_-^*\otimes \Lambda^{2}V_+ \ni D_-^+$.
\item An operator like Dirac operator:
\begin{align*}
\underline{D}^+: S_+ &\longrightarrow S_+ \\
\zeta &\longmapsto \sum_{j=1}^{\dim V}e_j^+\cdot D_{e_j^+}\zeta.
\end{align*}
\end{enumerate}

\begin{lemma}
\label{lemma-independendant}
$D^{S_+}$ and $\underline{D}^+$ are independnt of
$D \in \mathcal{D}^0(G,\varphi)$.
\end{lemma}

\begin{definition}
\label{definition-Killing-spinor-equations}
$(G,\varphi,\zeta)$, $\varphi \in S_+$ satisfies
{\it Killing spinor equations} if
$$
KSE\qquad \underbrace{D_-^{S_+}\zeta=0,}_{\text{Gravitino Eq.}}\qquad 
\underbrace{\underline{D}^+\varphi=0
}_{\text{Dilatino Eq.}}$$
\end{definition}

Expectation.
\begin{enumerate}
\item $\mathcal{M}=\{(G,\varphi,\zeta):KSE\}/\mathfrak{g}$ special metric.
\item Given a solution $(G,\varphi,\zeta)$
then 
$$
V_{(g,\varphi,\zeta)}\hookrightarrow V^k(\mathfrak{g}).
$$
(Where I think $V^k(\mathfrak{g})$ is the Kac-Moody affinization.)
\end{enumerate}

\begin{remark}
\label{remark-Lie-algebra}
In the case $T K \oplus T^*K: \mathcal{M}$ 2-stack. See [Bursztyn].
\end{remark}

\begin{lemma}
\label{lemma-Ricci-tensors}
Associated to $(G,\varphi)$ there are
well-defined ``Ricci tensors''
$\text{Ric}^+ \in V_- \otimes V_+$,
$\text{Ric}^- \in V_+ \otimes V_-$.
$$
\text{Ric}^+(a_-,b_+)=\text{Tr}(C_+ \to R_D(c_+,a_-)b_+)
$$
$$
R_D(a,b)=[D_a,D_b]_c-D_{[a,b]}c,\qquad D \in \mathcal{D}^0(G,\varphi).
$$
\begin{remark}
\label{remark-in-geometric-setup}
In geometric setup, the vanishing of the Ricci corresponds to
the motion equations of some physical supersymmetry theory.
\end{remark}

\begin{proposition}
\label{proposition-KSE}
If $(G,\varphi,\zeta)$ is solution of KSE, then
 $\text{Ric}^+_{G,\phi}=0$.
If $[\varphi,G]=0 \implies \text{Ric}^-_{G,\varphi}=0$.
\end{proposition}
\end{lemma}

\begin{proof}
$\text{Ric}^+(a-,-)\cdot \varphi=[\underline{D}^+,D_{a_-}^{S_+}\zeta
-D_{D_+a_-}^{S_+}\zeta=0$.
$[\varphi,G]=0 \implies  \text{Ric}^+(a_-,b_+)=\text{Ric}^-(b_+,a_-)$.
\end{proof}

\medskip\noindent
{\bf Generalized Ricci flow.}
Finding solutions to these equations.
Evaluating by
$$
G_t^{-1}\partial_tG_t=-2(\text{Ric}^+-\text{Ric}^-)
$$
$\Hom(V_+,V_-) \oplus \Hom(V_-,V_+)$.
$G G+G - G=0$, $G=\text{Id}_{V_+}-\text{Id}_{V_-}$.

\begin{exercise}
\label{exercise-}
\begin{enumerate}
\item Prove STE for GRF.
\item Assuming there exists a solution of KSE, prove long
time existence and convergence. (See theorem by Streets, Jordan, GF;
and a book by Streets, GF.)
\end{enumerate}
\end{exercise}

\begin{remark}
\label{remark-sigma-model}
For physicists the generalized Ricci flow (GRF) is the GRF of a $2d$
$\sigma$-model with target a compact Lie group 
$K$,  $\mathfrak{g}=R \oplus R^*$
\end{remark}

Let's explain something in this situation.
consider a compact Lie group $K$ and $\mathfrak{g}=R \oplus R^*$.
A generalized metric: $\mathfrak{g}=V_+ \oplus V_-$,
$(-,-)|_{V_\pm}$ non-degenerate,
$(-,-)|_{V_+}>0$.
$V_+=\mathcal{C}^b\{X+g(X)\}$,
$X \in R$, $g \in S^2(R^*)$, $b \in \Lambda^{2}R^*$,
$H=H_0+\overline{\partial}$.

\medskip\noindent
{\bf Case $\dim V_+=2n$.}
Complex pure spinor $\varphi$ on $V_+$ is
equivalent $V_+ \otimes \mathbb{C}=\ell \oplus \overline{\ell}$,
where $\ell=\{0 \in V_+ \oplus \mathbb{C}:0 \cdot\zeta=0\}$.
$(\ell,\ell)=0$, $(\overline{\ell},\overline{\ell})=0$.

This gives a complex structure
$J_\zeta:V_+ \to V_+$, $J_\zeta=i \text{Id}_\ell-i \text{Id}_{\overline{\ell}}$.

\begin{lemma}
\label{lemma-solutions-KSE}
$(G,\varphi,\zeta)$ with $\zeta$ pure satisfies KSE if and only if
\begin{enumerate}
\item $[\ell,\ell]\subset \ell$.
\item Take bars $\{\varepsilon_j,\overline{\varepsilon}_j\}^n_{j=0}$,
$\varepsilon_j \in \ell$, $\overline{\varepsilon}_j\in \overline{\ell}$
$(\varepsilon_j,\overline{\varepsilon}_j) = s_{j,k}$.
$\sum_{j=1}^n[\varepsilon_j,\overline{\varepsilon}_j=-J_\zeta \varphi_+$.
\end{enumerate}
\end{lemma}
(This sum is a moment map condition.)

Consider $\mathcal{L}=\{\ell \subset \mathfrak{g} \otimes \mathbb{C}:
\dim\ell=n,\ell\cap\overline{0},(-,-)|_{\ell \oplus \overline{\ell}}
\text{non-degenerate}\}$.
This is a complex manifold.
$T_\ell \mathcal{L} 
\cong \Hom(\ell,\overline{\ell}\oplus V_- \oplus \mathbb{C})$,
$V_- \otimes \mathbb{C}=(\ell \oplus \overline{\ell})^\perp$
Pick a vector $\dot \ell = \dot J + \dot G \in 
\Hom(\ell,\overline{\ell}\oplus V_- \oplus \mathbb{C})$
That is, $\dot J \in \Hom(\ell, \overline{\ell})$,
$\dot J :V_+ \to V_+$,
$\dot J J+ J \dot J=0$. (Deformations of the complex structure.)
$G \in \Hom(V_+,V_-) \oplus \Hom(V_-,V_+)$.

\begin{proposition}[Romero]
\label{proposition-Romero}
$\mathcal{L}$ has a pseudo-Kähler structure
preserved by the $\mathfrak{g}$-action and there exists
a moment map
\begin{align*}
\mu: \mathcal{L} &\longrightarrow \mathfrak{g}^* \\
\ell &\longmapsto
\frac{i}{2}\sum_{j=1}^n[\varepsilon_j,\overline{\varepsilon}_j],-),
\end{align*}
which is the quantity we mentioned in Lemma above.
\end{proposition}

\begin{proof}
Use the natural complex structure on the space of complex structures
(found by Fujiki),
$$
\text{tr}_{V_+}(J \dot G_2,\dot G_1)-\text{tr}_{V_+}(J \dot J_{\varepsilon}
\dot J_1).
$$ 
\end{proof}

Problem: prove that 
$\mathcal{M}=\{(G,\varphi,\zeta):\zeta\text{ puse}\}/\mathfrak{g}$
is a pseudo-Kähler manifold.

\begin{enumerate}
\item Pseudo-Kähler.
\item Shifter symplectic stuff.
\end{enumerate}

\begin{example}
\label{example-SU2xSU1}
Take $K=\text{SU}(2)\times\text{U}(1)\cong S^3\times S^1$
and $\mathfrak{g}=R \oplus R^*$.
Take generators $v_1,v_2,v_3$ of $\text{SU}(2)$ and $v_4$ of $\text{U}(1)$.
We have $[v_2,v_3]=-v_1$, $[v_3,v_1]=-v_2$, $[v_1,v_2]=-v_3$.
Put $H_\ell=\ell v^{123}$. $\ell \in \mathbb{R}$. $x,a \in \mathbb{R}_{>0}$.
$$
g_{x,a}=\frac{a}{x}\left(\sum_{i=1}^3\omega^{\otimes 2}
+x^2 (v^4)^{\otimes 2}\right)
$$
$$
V_+=\{ x+g_{xa}\}\subset \mathfrak{g}
$$
$$
I_X v_4=xv_1,\qquad U_X v_2=v_3, \qquad \varphi=-xv_4
$$
\end{example}

\begin{exercise}
\label{exercise-solution-of-KSE}
If $\ell=\frac{a}{x} \implies $ solution of KSE.
\end{exercise}

Furthermore $[\varphi_+,\ell] \subset \ell$
(holomorphic divergence).

The $a$ parameter is naturally complexified by  $b=yv^{23}$.
$y+ia=z$. 
$$
V=\text{log} \left(\frac{\omega^2_{x,n}}{v^{1234}}\right)=\text{log}a.
$$
KSV hyperbolic metric, metric on $\mathbb{H}$.

\medskip\noindent
{\bf $\dim V_+=7$}.
A real spinor on $V_+$ is equivalent to $\phi \in (\Lambda^{3}V_+^* )_{>0}$.
The space $\text{GL}(\mathbb{R}^7)\mathbb{y}\Lambda^{3}(V_+^*)$.
The space of spinors $S_+=\mathbb{R}^8
=\mathbb{R}^7\oplus\mathbb{R}\left<\varphi\right>$.
$\phi(x,y,z)=\left<x\cdot y\cdot z\cdot \zeta,\zeta\right>$.

\begin{remark}
\label{remark-7}
$v^1,\ldots,v^7 \cong \mathbb{R}^6 \oplus \mathbb{R}$.
$\phi=(v^{12}+v^{34}+v^{56})\wedge v^7+
\text{Re}((v^1+i v^2)\wedge(v^3+iv^4)\wedge(v^5+iv^6))$.
\end{remark}

\begin{example}
\label{example-SU2+SU2+R}
Take $R=\text{SU}(2)\oplus \text{SU}(2)\oplus \mathbb{R}$.
$\mathfrak{g}=R \oplus R^*$.
$e,s \in \mathbb{R}$, $H=sv^{123}+\ell v^{456}$.
$\phi=\omega \wedge \zeta+\Omega^+$,
$\zeta=\sqrt{\varepsilon/\ell}v^7$,
$\omega=\sqrt{s\ell}(v^{14}+v^{25}-v^{36})$,
$\Omega^+=\sqrt{s^3}v^{123}+\ell \sqrt{s}v^{156}-\ell \sqrt{s}v^{345}$.
\end{example}

\section{Non-Kähler Hodge Lefschetz theory
and the Bianchi identity}
\label{section-non-kahler-hodge-lefschetz-theory-and-bianchi-identity}

\noindent
Arpan Saha, UNICAMP.
Geometric Structures Seminar, IMPA. 
September 25, 2025.

\medskip
{\bf Abstract.} 

Being Kähler imposes severe constraints on the cohomology of compact complex
manifolds such as the Hard Lefschetz property, and the question of how far this
generalises beyond the class of Kähler manifolds has been of great interest for
a while. In this talk, I shall report on ongoing joint work with Mario García
Fernández and Raúl González Molina that abstracts out the definition of a
variation of Hodge--Lefschetz structure and provides evidence that, under
certain natural assumptions, such a structure exists more generally on
distinguished subspaces within moduli spaces of Bismut--Ricci-flat metrics that
are pluriclosed up to source terms. In particular, these distinguished subspaces
may be regarded as replacements for the Kähler cone, with affine structure
modelled on a subspace of the (1,1) Aeppli cohomology of the compact complex
manifold.

\medskip\noindent
{\bf Broad motivation.}
As we move on the parameter space associated
to a CY manifold we encounter walls.
The Kähler cone is contained in this
parameter space, and we may cross 
its walls. When we cross a wall
we obtain a birational transformation
(I think this means that the moduli
on either side are birational).
But the Kähler condition is not a 
birational invariant.

\medskip\noindent
{\bf Hodge-Lefschetz theory.}
For $(X,J,\omega)$ Kähler,
the cohomology $H^{\bullet}(X)$ comes with:
\begin{itemize}
\item the Lefschetz operator $L$,
given by $[\omega]\wedge-$.

\item Hodge star operator.
\item $\Lambda=* L *$ satisfying
the $\mathfrak{sl}_2$ relations:
$$
[L,\Lambda]=H,\qquad [H,L]=2L,\qquad [H,\Lambda]=-2\Lambda.
$$
These relations are equivalent to 
the Hard Lefschetz property that
$$
L^q:H^{d-q} \to H^{d+q}.
$$
\item The Poincaré pairing
(which is a different pairing from Lefschetz)
of the form
$$
H^{d-q}\cong (H^{d+q})
$$
given by integration.

\item Hodge-Riemann bilinear relations
which is a positive definite form given by
$\int_X\cdot \wedge *\cdot$.

\end{itemize}

\noindent
Moving about on the Kähler cone gives 
a variation of Hodge-Lefschetz structure.

\begin{definition}
\label{definition-variation-of-hodge-lefschetz-structure}
A {\it variation of Hodge Lefschetz structure (VHLS)} 
of weight $d$ over a manifold $K$ consists of
\begin{itemize}
\item a graded real vector bundle
$E=\bigoplus_{q=0}^d E^q \to K$
with $\text{rk}_\mathbb{R}E^0=1$.

\item An isomorphism $E^0 \otimes TK \xrightarrow{\simeq}E^1$
(this is reminiscent of the usual definition
of variations of Hodge structures, and 
we will not use it in this talk).

\item Endomorphism fields
$L, \Lambda,H \in \mathcal{A}^0(K,\text{End}(E))$
satisfying the $\mathfrak{sl}_2$ relations
that
$$
[L,\Lambda]=H,\qquad [H,L]=2L,\qquad [H,\Lambda]=-2\Lambda.
$$
and that
$$
H|_{E_q}=(2q-d)|_{\text{id}_{E^q}}.
$$
\item An involution $* \in \mathcal{A}^0(K,\text{End}(E))$
such that $* L *=\Lambda$.

\item A nondegenerate symmetric pairing
$P \in \mathcal{A}^0(\mathcal{K},E^* \otimes E^*)$
w.r.t. $L$ and $*$ are self-adjoint.

\item (like the Gauss-Manin connection)
a flat connection $D: \mathcal{A}^0(K,E)\to\mathcal{A}^1(K,E)$
such that $DH=0=DP$.
\end{itemize}
The VHLS is said to be {\it positive} if $P(*-,-)$ 
is positive definite.
\end{definition}

\noindent
{\bf Calabi's dream beyond Kähler manifolds.}
Let $(X,\Omega)$ be a compact Calabi-Yau manifold,
where $\Omega$ is the volume form.
In general such a manifold is not Kähler.

\begin{example}[Non-Kähler manifolds]
\label{example-non-kahler}
These manifolds are not Kähler:
Heisenberg algebra quotiented by some lattice,
torus bundle, Hopf surface.
\end{example}

\noindent
By [Yau], the Kähler cone is the moduli
space of Kähler Ricci-flat metrics.

We look for PDEs on  $X$ such that
\begin{itemize}
\item Assumption C (Calabi). Moduli
space of alutions is an open set $K$ 
of an affine space modelled on
a subspace $\mathbb{V}$ of
$(1,1)$-Aeppli cohomology group,
which is given by
$$
H^{1,1}_A(X)=\frac{\Ker \partial\overline{\partial}|_{\Omega^{1,1}}}
{\text{Im} \partial+\text{Im}\overline{\partial}}
$$

\item Assumtion D (dilaton).
A volume function $v:K\to \mathbb{R}_{>0}$
such that $\underline{d}\text{log} V$ 
is nowhere zero (where $\underline{d}$ denotes
Aeppli differential), and the bilinear form
$g_K:=-Dd \text{log}V$
is nondegenerate (basically the Hessian metric).
(Ideally this would be positive definite.)

\item Assumption V (vector field).
$Z=g_K^{-1}d \text{log}V$
is such that $DZ$ is invertible
(when thought of as an endomorphism)
$g_K(Z,Z)=d \in \mathbb{Z}_{>0}$,
 $$
(D_{(DZ)^{-1}})^{d+1}V=0,
$$
\end{itemize}

\begin{remark}
\label{remark-pairing}
$$
(D_{(DZ)^{-1}})^dV=\int_X \wedge \wedge \wedge.
$$
\end{remark}

\begin{theorem}[García Fernández-González Molina-AS]
\label{theorem-GGA}
If assumptions C, D and V all hold with  $d \leq 3$,
then $K$ admits a VHLS of weight $d$.
\end{theorem}

\begin{proof}[Idea of proof for $d=3$]
$$
E=\underbrace{\mathbb{R}_K}_{E^0} \oplus \underbrace{TK}_{E^1}
\oplus \underbrace{T^*K}_{E^2} \oplus \underbrace{\mathbb{R}_K^*}_{E^3}.
$$
Define $L$ in each of the terms of the direct sum as
$$
L1=Z,\qquad Lv=D^2_vV,\qquad L_\alpha=i_Z\alpha
$$
and $*$ as
$$
*1=V 1^\vee,\qquad  *v=VD^2_0 \text{log}V=Vg_K(v,-).
$$
\end{proof}

\medskip\noindent
Recall the Hull-Strominger
system, which is given by two conditions
on the connection and the metric
$$
d\left(\|\Omega\|_\omega \frac{\omega^{n-1}}{(n-1)!}\right)=0,\qquad 
F_\theta \wedge \frac{\omega^{n-1}}{(n-1)!}=0.
$$
This determines $d d^c\omega=\alpha\left<F_\theta \wedge F_\theta\right>$.



\section{Irreducibility of the Hilbert scheme of points and the class of 2-step
ideals}
\label{section-irreducibility-of-hilbert-scheme-of-points}

\noindent
Michele Graffeo, SISSA.
Algebraic Geometry Seminar, UFF. 
September 29, 2025.

\medskip
{\bf Abstract.}
Hilbert schemes of points on a quasi-projective variety X are classical objects
in algebraic geometry. Roughly speaking, they parametrise ideals of a polynomial
ring with complex coefficients having finite colength. Although Hilbert schemes
always have a distinguished component called the smoothable component, their
geometry is quite pathological and one of the main open problems around them
concerns their irreducibility. In a joint work with Giovenzana, Giovenzana,
Lella we introduce the notion of 2-step ideals. We show that the loci
parametrising these ideals are in general not contained in the smoothable
component of the Hilbert scheme, thus  providing new examples of
extra-components.  In my seminar I will discuss our class of ideals and relate
them to the compressed algebras considered by Iarrobbino in the eighties.
Finally, I will show how to extend our result to the nested setting.
 

\medskip\noindent

As usual, you start with a functor and
prove representativity.

\begin{definition}
\label{definition-hilbert-scheme}
Let $X$ be a smooth
quasi-projective variety and a nonegative integer $d$.
Define
\begin{align*}
\underline{Hilb}^d: \text{Sch}_\mathbb{C} &\longrightarrow \mathsf{Set} \\
B &\longmapsto \left\{Z \hookrightarrow B \times X:
\substack{B\text{-flat} \\ B\text{-finite}\\ \text{length }d}\right\}
\end{align*}
\end{definition}

\begin{theorem}[Grothendieck]
\label{theorem-grothendieck}
$\underline{Hilb}^d(X)$ is represented by a quasi-projective
scheme $Hilb^d(X)$
\end{theorem}

Idea: $\mathbb{C}$-points of $Hilb^d(X)$
are in correspondence with $Z \overset{\subset}{\text{finite}}X$
of length.

A fat point $Z= \text{Spec}(A)$ $(A, \mathfrak{m})$ is a
local artinian $\mathbb{C}$-algebra of finite type.

If $X$ smooth can consider $X=\mathbb{A}^n$, and it's the same to consider
\begin{itemize}
\item $Z \hookrightarrow \mathbb{A}^n$ of length $d$.
\item $I \subset \mathbb{C}[x_1,\ldots,x_n]$ of 
colength $d$.
\item $A= \mathbb{C}[x_1,\ldots,x_n]/I$ with $\dim_\mathbb{C}=d$.
\end{itemize}

\begin{definition}
\label{definition-nested-Hilbert-scheme}
$X$, $0 \leq d_1 \leq \ldots d_r=\underline{d}$.
$$
Hilb^{\underline{d}}(X)=\{Z_1\not\hookrightarrow \ldots
\not\hookrightarrow Z_r  \not\hookrightarrow X:
\text{len}(Z_i)=d_i\}
$$
\end{definition}

\begin{theorem}[Hartshorne (r=0), Fogarty, Kalpan (r>1)]
\label{theorem-HK}
$Hilb^d(X)$ is connected.
\end{theorem}

For $r=1$, $Hilb^d(\mathbb{A}^n)$ is smooth iff
$n \leq 2$, $d \leq 3$.
If $r>1$, $n=2$, it is smooth iff $r=2$ and $d_1=d_2=-1$.
If $r>1$ and $n>2$, it is smooth iff $r=2$, $(d_1,d_2)=\{(1,2),(2,3)\}$.

Iror components.
\begin{itemize}
\item $(r=1)$ $n\geq 4$, $Hilb^d(\mathbb{A}^3)$ 
is irreducible iff $d\leq 7$ ($\impliedby$ [E-I], $\implies $ [Mozzola].
$n=3$ irredicuble if $d \leq  11$ (8,9 Sivic; 10 Jardim et al; 11, Jelisievv et
al.)
\item $(r=2)$, $n=2$, irreducible [G-Rosul-Sebastian].
\item $(r \geq 3)$ $n=2$, there exist $d_1 \leq \ldots \leq d_5$
such that $Hilb^{\underline{d}}(\mathbb{A}^i)$ is irreducible [S-R].
\end{itemize}

There is an analogy between classical in dimension 3
and nested in dimension 2. 

Schematic structure.
\begin{itemize}
\item [I] $Hilb^{21}(\mathbb{A}^4)$ has at least a generically
non-reduced component.
\item Problem: what about $n=3$?
\item Theorem. If $Hilb^{d_1,\ldots d_r}(\mathbb{A}^i)$ is
irreducible $\implies $ $Hilb^{(1,d_1,\ldots,d_r)}(\mathbb{A}^n$
has a generically non reduced component.
\end{itemize}

Other results.
\begin{itemize}
\item …
\end{itemize}
Hilbert scheme function.
For a local artinian $\mathbb{C}$-algebra of
finite type there is an associated graded
$\text{Gr}_\mathfrak{m}(A)
=\bigoplus_{i \geq 0}\mathfrak{m}^1/\mathfrak{m}^{i+1}$ 
which allows us to define the Hilbert function
\begin{align*}
h_A: \mathbb{Z} &\longrightarrow \mathbb{N} \\
i &\longmapsto \dim_\mathbb{C} \mathfrak{m}_i/\mathfrak{m}_{i+1}
\end{align*}
sistability for graded $A$-modules.
The socol is the annihilator of
the ideal which gives the algebra
e.g. $\mathbb{C}[x,y](x^2,xy,y^2)$ then
$\text{Soc}(A)=\left<x,y^2\right>_\mathbb{C}$.

\medskip\noindent
\begin{definition}
\label{definition-smoothable-component}
The {\it smoothable component} is 
 $$
V_{sm}=\overline{\{[(z_i)_{i=1}^n \in Hilb^{\underline{d}}(\mathbb{A}^n):
Z_r\text{ is reduced}\}}
$$
\end{definition}

\begin{definition}
\label{definition-elementary-componenet}
$V \subset Hilb^{\underline{d}}(\mathbb{A}^n)$ is
an {\it elementary component} if
$\forall [Z_1,\ldots,Z_r] \in V$
then $Z_r$ is a fat point.
\end{definition}

Instead of looking for all components of the Hilbert scheme
we look for elementary components. This may not be some simple;
sometimes there's infinitely many. There
is no method to find them.

\begin{theorem}[Irrabaro]
\label{theorem-I}
Every irreducible component $V \subset Hilb^d(\mathbb{A}^n)$ 
is generically étale locally product of elementary components.
\end{theorem}

\begin{definition}
\label{definition-}
Let $h:\mathbb{Z} \to \mathbb{N}$ be a function
with finite support, $|h|=\sum_{ i \in \mathbb{Z}}h(i)=d$. Define
\begin{itemize}
\item $H_n=\{[A] \in Hilb^d(\mathbb{A}^n): h_A \equiv h\}$.
This is closed by semicontinuity of the function; but can be expressed
a by a representable functor.
\item The following also has a canonical schematic structure
\begin{align*}
\pi_n:H_n  &\longrightarrow \mathcal{H}_n=\{[A] \in H_n: A\text{ is graded}\} \\
A &\longmapsto A=\text{Gr}_{\mathfrak{m}}(A)
\end{align*}
Where the first is $\not\hookrightarrow $ in the second.
\end{itemize}
\end{definition}

Recall that
$$
T_{[I]}Hilb^d(\mathbb{A}^n)=\Hom_R(I,R/I)
$$
\begin{theorem}[B-B,J,G G G L]
\label{theorem-B-B-J-G-G-G-L}
There is a graded $[I] \in \mathcal{H}_n$.
Then
\begin{align*}
T_{[I]}\mathcal{H}_n&=\Hom_R(I,R/I)_{=0}\\
T_{[I]}\pi_n([I])&='>0\\
T_{[I]}H_n&=' \geq 0.
\end{align*}
\end{theorem}

\begin{definition}
\label{definition-compressed}
$(A,\mathfrak{m}_A)$ is {\it compressed} is for all  $(A',\mathfrak{m}')$ 
with $e(A)=e(A')$ and $len(A) \geq  len(A') $.
\end{definition}

\begin{theorem}
\label{theorem-E}
$E_h \subset H_n$ is $H$-compressed 
$E_h \neq  \emptyset$
\end{theorem}

\section{Normal and Restricted Tangent Bundles of Rational
Curves in Hypersurfaces}
\label{section-restricted-tangent-bundles}

\noindent
Lucas Mioranci, IMPA.
Algebra Seminar, IMPA. 
October 8, 2025.

\medskip
{\bf Abstract.} 

Let $X\subset \mathbb{P}^n$ be a degree $d$ hypersurface containing a smooth
rational curve $C$ of degree $e$. The normal bundle $N_{C/X}$ and the restricted
tangent bundle $T_X|_C$ split as direct sums of line bundles of the form
$\bigoplus_i \mathcal{O}_{\mathbb{P}^1}(a_i)\qquad$ called their splitting type.
The splitting type of $N_{C/X}$ and ${T_X}|_C\;\;$ controls the local structure
of the space of rational curves and determines how many general points of $X$ we
can interpolate by deforming the curve $C$. By combining explicit computations
of $T_X|_C\;\;$ and an induction argument, I classify all triples $(e,d,n)$ such
that a general degree $d$ hypersurface $X\subset \mathbb{P}^n$ contains a
rational curve $C$ of degree $e$ whose restricted tangent bundle ${T_X}|_C\;\;$
is balanced. The case of quadrics is particular, in which we show that
odd-degree rational curves do not interpolate the expected number of points on
quadric hypersurfaces.  For the normal bundle, I compute explicit examples of
hypersurfaces $X$ for all possible splitting types of $N_{C/X}$ when $C$ is the
rational normal curve. Additionally, for $d\ge 3$, we compute the dimension of
the space of hypersurfaces $X$ such that $N_{C/X}$ has a given splitting type.



\medskip\noindent

$\mathcal{O}$ with no subindex means $\mathcal{O}_{\mathbb{P}^1}$.
$$
\xymatrix{
&0\ar[d]&0\ar[d]\\
&T_{\mathbb{P}^1}=\mathcal{O}_(2)\ar[d]\ar[r]^=
&T_{\mathbb{P}^1}=\mathcal{O}(2)\ar[d]\\
0\ar[r]&T_{X}|_{C}\ar[d]\ar[r]&T_{\mathbb{P}^n}|_{C}\ar[d]\ar[r]&
N_{X/\mathbb{P}^n}|_{C}\cong \mathcal{O}(de)\ar[d]_=\ar[r]&0\\
0\ar[r]& N_{C/X}\ar[d]\ar[r]&N_{C/\mathbb{P}^n}\ar[d]\ar[r]&
N_{X/\mathbb{P}^n}|_{C}\cong \mathcal{O}(de)\ar[r]&0\\
&0&0
}
$$
Throughout this talk
$X$ is a degree $d$ hypersurface in $\mathbb{P}^n$ 
and $C \subset X$ is a smooth rational curve of degree $e$.

Since $C$ is rational we can think of it
as a map $f:\mathbb{P}^1 \to X$.
Then we can pullback any bundle over $C$ 
back to $\mathbb{P}^1$,
and by Birkhoff-? Theorem
we know that vector bundles over $\mathbb{P}^1$ 
split as
$$
E\cong \bigoplus_{i=1}^r \mathcal{O}_{\mathbb{P}^1}(a_i),
\qquad a_1\leq \ldots \leq a_n.
$$
This decomposition is called the
{\it splitting type} of $E$.
We say $E$ is {\it balanced} if
$|a_i-a_j|\leq 1$ for all $i,j$.

Balancedness is an open condition,
i.e. if you have a family of bundles over $\mathbb{P}^1$ 
and one of them is balanced, then all of them are,
(also you can think it's a generic condition).

We know the normal bundle $N_{C/X}$ has rank $n-2$
and degree  $e(n-d+1)-2$.
And the tangent bundle $T_{X/C}$ 
has rank $n-1$ and degree $e(n-d+1)$.

\begin{example}
\label{example-conic}
A conic $C \subset \mathbb{P}^3$.
Since a conic in $\mathbb{P}^3$ is contained in a plane,
this obliges to have a ``distinguished normal direction'',
$\mathcal{O}(2)$. Indeed,
\begin{align*}
N_{C/\mathbb{P}^3}&\cong(N_{Q/\mathbb{P}^3}\oplus
N_{\mathbb{P}^2/\mathbb{P}^3}|_{C}\\
&\cong(\mathcal{O}(2) \oplus \mathcal{O}(1))|_{C}\\
&\cong\mathcal{O}(1)\oplus \mathcal{O}(2).
\end{align*}
\end{example}

\medskip\noindent
Now we discuss interpolation of general points.
Let $p_1,\ldots,p_n \in \mathbb{P}^1$ 
and $x_1,\ldots, x_n \in X$ 
be general points.
Let $f:\mathbb{P}^1 \to X$ be such that
$f(p_i)=x_i$.

Consider the space of morphisms mapping
the  $p_i$ to $x_i$. It's known that it's tangent space
at $f$
is given by
$$
T_{[f]}\Mor(\mathbb{P}^1,X,p_i \mapsto x_i)
\cong H^0(\mathbb{P}^1,f^*T_X(-n)).
$$
\noindent
Deformations of $f$ (mapping $p_i \mapsto x_i$)
dominate $X$ if $a_i-n_i \geq 0$.
Also, deformations of $f$ interpolate up
to $a_1+f$ general points in $X$.

\medskip\noindent
How many points can deformations of the curve interpolate?
This means (I think) that the deformations of the curve
are still such that $p_i \mapsto  x_i$.
``And for the normal bundle, interpolations
means that the curves contain the points''.

If $T_X|_{C}$ is balanced, $C$ can interpolate up to
$$
\text{floor function}(\frac{e(n+1-d)}{n_1}) +1
$$
general points. The case of $N_{C/X}$ is similar.

\medskip\noindent
Next we survey some relevant results.

\begin{theorem}[Coskun-Riedl, 2018]
\label{theorem-CR18}
A general Fano $X$ of degree $d \geq 2$ 
contains a degree $c$ rational smooth curve
 $C$ with balanced $N_{C/X}$ 
for every $1 \leq e \leq  n$.
\end{theorem}

The idea is to observe that the space of curves has
balanced normal bundle.

\begin{theorem}[Ran, 2021]
\label{theorem-R21}
Extends the above for
$1 \leq  e \leq 2n-2$ 
and $d \geq 4$.
\end{theorem}

Idea: general hypersurface contains a curve of balanced normal bundle.

\begin{theorem}[Ran, 2024]
\label{theorem-R24}
$X$ general Fano hypersurface.
There exist smooth rational curve with  $C$ 
with balanced tangent bundle $T_{X}|_{C}$ 
for $e$ in some arithmetic progressions.
\end{theorem}

$e$ is in general very large.
They show there are curves of arbitrarily large
degrees with balanced tangent bundle.

\medskip\noindent
Now we describe the new results.

\begin{theorem}
\label{theorem-Lucas}
Let $X \subseteq \mathbb{P}^n$ be Fano
hypersurface of degree $3 \leq d \leq n$.

\begin{enumerate}
\item If $C \subset X$ is a rational smooth curve
of degree $e$ with 
$e \leq  \frac{n-1}{n+1-d}$,
then $T_X|_{C}$ is not balanced.

\item A general hypersurface $X$ 
contains rational curves $C$ 
of degree $e$ with balanced
$T_X|_{C}$ for every
$e> \frac{n-1}{n+1-d}$.
\end{enumerate}

\begin{theorem}
\label{theorem-lucas2}
Let $X \subseteq \mathbb{P}^n$ be a smooth quadric hypersurface.

\begin{enumerate}
\item If $e \leq 2$ is even,
then exists a curve $C$ of degree $e$ 
with $T_X|_{C}\cong \mathcal{O}(e)^{n-1}$.
(I.e. we can interpolate $e+2$ points
if $e$ is even.)

\item If $e \geq 1$ is odd,
then there is no curve $C$ of degree $e$ 
with balanced $T_X|_C$.
The best we can do is the most balanced bundle
before balancedness, namely
$T_X|_C \cong \mathcal{O}(e-1)\oplus\mathcal{O}(e)\oplus\mathcal{O}(e+1)$. 
(I.e. we can only interpolate $e$ points if $e$ is odd.)
\end{enumerate}

Next we give the main ideas in the proof.
Let $C$ be the rational normal curve
of degree $e$ in $\mathbb{P}^1$,
\begin{align*}
C:\mathbb{P}^1  &\longrightarrow \mathbb{P}^n \\
(s:t) &\longmapsto (s^2:s^{e-1}t:\ldots:t^e:0:\ldots:0).
\end{align*}
$X=V(F)$.

Then find some examples. Then,

\begin{proposition}
\label{proposition-proposition}
If $\mu(T_X|_{C})=\frac{e(n+1-d)}{n-1}\leq 1$
then $T_X|_{C}$ is not balanced.
\end{proposition}

\noindent
Use the proposition to show that $T_X|_C$
is not balanced if $d \leq  n$ 
and $e le \frac{ n-1}{n+1-d}$.

This implies Theorem \ref{theorem-Lucas}.

\begin{lemma}
\label{lemma-split-normal-bundle}
If $N_{C/X}\cong \bigoplus_i \mathcal{O}(a_i)$ 
with $a_i < 4$ for all $i$,
then $T_{X}|_C \cong N_{C/X} \oplus \mathcal{O}(2)$.
\end{lemma}

\noindent
This implies that $\text{Ext}^1(N_{C/X},\mathcal{O}(2))=0$.

The most important step is the following,
used as the induction step:

\begin{proposition}
\label{proposition-induction}
If $T_Y|_C$ is balanced for some degree $d$ 
hypersurface $Y \subseteq \mathbb{P}^{n-1}$,
then there exists a degree $d$ hypersurface
$C \subseteq \mathbb{P}^n$ with balanced
$T_X|_C$.
\end{proposition}

\noindent
Combining the examples, Lemma \ref{lemma-split-normal-bundle}
and Proposition \ref{proposition-induction},
we settle that for $e \leq  \text{max}\{2d-2,n\}$.

The following argument is used to show that 
actually what we showed so far is enough for all degrees.

\begin{lemma}
\label{lemma-smoothing-argument}
$C=C_1 \cup  C_2$, $C_1 \cong \mathbb{P}^1$.
$E$ vector bundle on $C$ such that
$E|_{C_1}$ is balanced,
$E_{C_2}$ is perfectly balanced
(numerical condition that works when $e=n-1$).
Then if $E$ is the specialization of a family
of vector bundles $E'$ on $\mathbb{P}^1$,
then $E'$ is balanced.
\end{lemma}

\noindent
Basically, glue and smooth low degree curves
(i.e. $e \leq  \text{max}\{2d-2,n\}$)
to get all degrees

\noindent

\end{theorem}

\noindent

\end{theorem}

\medskip\noindent
Now we prove the result for quadrics,
i.e. $X$ is a quadraic.
For even $e$ compute examples and conclude
 $T_X|_C \cong \mathcal{O}(e)^{n-1}$.

For odd $e$, compute examples,
$T_X|_C\cong \mathcal{O}(c-1) \oplus \mathcal{O}(2)^{3}\oplus \mathcal{O}(-1)$.

\begin{proposition}
\label{proposition-interpolation2}
$n\geq 3$, $m\leq  e+1$.
There is a degree $e$ curve $\phi_e$ interpolating
$m$ general points if and only if there is a 
degree $e-2$ curve $\psi_{e-2}$ interpolating
$m-2$ general points in $X$.
\end{proposition}

\noindent
The idea is to do induction.
Suppose you can interpolate $e$ points,
then show you can interpolate $e+2$.

Chose $y_1,\ldots,y_{m-2}$ on the quadric.
Choose a line in $\mathbb{P}^{m-1}$ intersecting the quadric,
and $x_1,\ldots,x_{m-1}$ points on the line.
Join the points on the line to the points on the quadric,
Then you get somem points of intersection with the quadric.
Those points and the two points of intersection of the line and the quadric
let us construct a scroll, which intersects the quadric.
The intersection of the scroll and the quadric
(I guess the scroll contains the lines, and so the points $y_i$)
gives the curve passing through the $y_i$.

\section{Regular but non smooth curves of genus 3}
\label{section-regular-but-non-smooth}

\noindent
Cesar Hilario, .
Algebra Seminar, IMPA. 
October 15, 2025.

\medskip {\bf Abstract.} Regular but non-smooth curves are a unique feature of
geometry in positive characteristic, that results from the fact that over an
imperfect field the notion of regularity is weaker than the notion of
smoothness. In the setting of algebraic geometry over an algebraically closed
field, these curves correspond to fibrations by singular curves, which are
fibrations of relative dimension 1 whose fibers are singular. The most famous
examples are arguably the so-called quasi-elliptic curves and quasi-elliptic
fibrations, which play a key role in the Bombieri-Mumford classification of
algebraic surfaces in characteristics 2 and 3. In this talk I will discuss the
case of genus 3 in characteristic 2, with an eye towards the classification of
regular plane projective quartic curves that become rational after base change.

\medskip\noindent
arXiv 2409.05464.

\medskip\noindent
$C$ will denote a projective geometrically integral curve over
a not necessaily closed field $K$. Geometricallt integral
means that $C \otimes_K \overline{K}$ is integra.
Suppose that $p=\text{char}(K) \geq 0$
and $g=h_1(\mathcal{O}_C)$ is of genus $C$.

We define
\begin{enumerate}
\item $C$ regular iff local rings of $C$ are regular (DVR).
\item $C$ smooth iff $C$ regular and $K \otimes_K \overline{K}$ regular.
\end{enumerate}

\noindent
Thus it's obvious that smoothness implies regularity.
In this talk we explain why the converse is not true.

Note that if $K$ is perfect (e.g. $p=0$ or $K=\overline{K}$) implies
 smooth = regular. Recall that $K$ is imperfect iff $p>0$
the powers of $K$, $K^p$, is a proper subfield of $K$
(it's always a subfield but we ask it's proper).

\medskip\noindent
Assume that $C$ is regular and $p>0$.
The genus of rthe normalization is less than or equal to the genus of the curve,
i.e.
$$
\tilde{g}=h^1(\mathcal{O}_{\widehat{C \otimes_K \overline{K}}})\leq g.
$$
We know that $C$ is smooth iff $\overline{g}=g$.
So $g-\overline{g}$ is a measure of non-smoothness.

\begin{theorem}[Tate]
\label{theorem-tate}
$\frac{p-1}{2}$ divides $g-\overline{g}$.
\end{theorem}

As a corollary, we get that if $C$ is non-smooth,
$g-\overline{g}>0$ and so $p \leq  2g+1$.

\begin{example}
\label{example-attain}
There are cases in which the bound is sharp,
namely $C:y^2=x^p+t$, with $t \in K\setminus K^p$.
Then $g=\frac{p-1}{2}$ and $\bar{g}=0$.
Over $\overline{K}$ rhere exists a singular point,
$(x,y)=(-t^{1/p},0)$ which
{\bf is not visible over $K$}, that is we get that
$t^{1/p}\not \in K$.
\end{example}

\begin{example}
\label{example-elliptic}
$\{\text{elliptic curves}\}=\{g = \overline{g}=1,\text{$\exists K$-rational
point }\}=\{\text{smooth cubic curves with $K$-rational point}\}$.
$y^2=x^3+4g_i x+27$.
\end{example}

\begin{example}
\label{example-quasielliptic}
$g=1$, $\overline{g}=0$, $p=2,3$,
quasielliptic curves. (Important in the classification of surfaces
in positive characteristic.)
\end{example}

\medskip\noindent
Now we make a digression to describe Kodaira-Enriques (over $\mathbb{C}$)
in positive characteristic. Bombieri-Mumford: {\bf new objects} appear,
quasielliptic surfaces.

Let $k=\overline{k}$, $S$ smooth elliptic 
({\it quasielliptic}) surface over $k$. Suppose we have a fibration
where the general fiber is an elliptic curve an elliptic curve (resp. a plane 
cuspidal cubic) and the generic fiber an elliptic (resp. quasielliptic)
curve over $k(B)$.

$$
\xymatrix{
S\ar[d]\\
B
}
$$

\medskip\noindent
Back to our main content, let's assume that $g=3$.
$$
\xymatrix{
\{g=1,\exists K\text{-rat. pt}\}\ar[r]&\ar[l]
\left\{\substack{\text{reg. cubic curves} \\ \text{with $K$-rat pt}}\right\}
}
$$
Recall that genus-degree formula
$$
g=\frac{(d-1)(d-2)}{2}=1,
$$
So then $d=3,4$ gives  $g=3$ (right?).

No we present our main theorem, which is a classification result:

\begin{theorem}
\label{theorem-classification-quartics}
\begin{itemize}
\item $C$ regular non-hyperelliptic curve over $K$.
\item  $g=3$, $\overline{g}=0$.
\item $p=2$.
\end{itemize}
(Not that non-hyperllipticity of $C$ and $g=3$ imply that $C$ is a quartic
over $K$, and that  $\overline{g}=0$ implies $C$ is geometrically rational.)
Then $C$ is isomorphic to one of the following quartics.

\begin{enumerate}
\item $y^4+az^4+xz^3+bx^2z^2+cx^4=0$ where $a,b,c \in K$, $c \not \in K^2$.
(Notice the imperfectness of $K$ is crucial.)

\item $y^4+az^4+bx^2y^2+cx^2z^2+bx^3z+dx^4=0$ for $a,b,c,d, \in K$,
$a \not \in K^2$, $b\not =0$.

\item 
\item 
\item 
\end{enumerate}
\end{theorem}

\noindent
Now we enumerate properties of these families.

\begin{itemize}
\item $C$ is a purely inseparable double cover of a quasielliptic curve.
To understand this consider the induced map from
Frobenius map $K \to K$, $a \mapsto a^p$ and do base change:
$$
\xymatrix{
C^{(p)}\ar[r]\ar[d]\ar@{}[dr]|-{\lrcorner}&C\ar[d]\\
\Spec K\ar[r]^{\substack{\text{induced from} \\ \text{Frobenius}}}&\Spec K
}
$$
This gives a map $C \to C^{(p)}$ by universal property of pullback.

$$
\xymatrix{
C \ar[rrrd]^{\text{Frobenius}} \ar@{-->}[rrd]\ar[rrdd] & & \\
& &  C^{(p)}\ar[d] \ar[r] & C \ar[d] \\
& &  \Spec K\ar[r] & \Spec K
}
$$

Then consider the normalization of $C^{(p)}=C_1$ which
turns out to be quasielliptic; by universal
property of normalization we can lift:
$$
\xymatrix{
& \widehat{C^{(p)}}\ar[d]\\
C\ar@{.>}[ur]\ar[r]&C^{(p)}.
}
$$
Here the lower arrow is purely inseparable of degree 2.

\item There exists a unique non smooth point $x \in C$.

\item Now we iterate the construction of the first item:
$$
\xymatrix{
\underbrace{C}_{g=3}\ar[r]&\underbrace{C_1}_{g=1}\ar[r]&
\underbrace{C_2}_{g_2-9}\ar[r]&\underbrace{C_3}_{g_3=0}\ar[r]&
\underbrace{C_4}_{g_4=0}\ar[r]&\cdots
}
$$
Let $x$ be the non-smooth (non-rational) point of $C$. Map it under
these maps to $x_1$( non-smooth,rational), then  $x_2$ (smooth, rational?),
then $x_3 $ (smooth, rational.), then $x_4$ (smooth, rational), etc.

We have
$$
\begin{matrix}
&\text{$x_2$ rational}& \text{$x$ canoical divisor} & 
E=K(c_2)\\
(i)& Y & Y & Y\\
(ii) & N & Y & N\\
(iii) & Y & N & N\\
(iv) & N & N & Y\\
(v) & N & N & N
\end{matrix}
$$
Where the last column is explained next.

\item Canonical field of $C$ = subfield of $K(C)$ generated
by the quotients of all non-zero holomorphic differentials
=  $K(C)$ ($C$ non-hyperelliptic).

\item Pseudocanonical field of $C$ = … non-zero \underline{exact}
holomorphic differential := $E$. So one has $[K(c),E]=4=p^2$.

$$
\xymatrix{
&K(C)\ar[ld]_{4}\ar[rd]^{4}\\
E&&K(C_2).
}
$$
\end{itemize}

\noindent
In conclusion, we manage to characterize these families.

\begin{example}
\label{example-chain-of-maps}
In case 1, let $a=b=0$. We get
\begin{equation}
\label{equation-curve-pencil}
C: y^4+xz^3+\underbrace{c}_{\in K\setminus K^2}x^4=0
\end{equation}
$$
\underbrace{C_1}_{\text{quasiell.}}:xy^2+z^3+cx^3=0.
$$
Further
$$
\xymatrix{
C\ar[r]&C_1\ar[r]&C_2 \cong\mathbb{P}^1\ar[r]
&C_3 \cong \mathbb{P}^1\ar[r]&C_4\cong \mathbb{P}^1\ar[r]&\cdots.
}
$$
\end{example}

\medskip\noindent
With $C$ one can construct a pencil (fibration over $\mathbb{P}^1$)
of quartics.
Let $k=\overline{k}$ (new base field from what we fixed at the beginning!).
Idea: replace $c$ for $t$ Equation \ref{equation-curve-pencil}.
 $K=k(\mathbb{P}^1)$, $S=V(t_0(y^4+xz^3)+t_1x^4)
\subseteq \mathbb{P}^2_{(x:y:z)}\times \mathbb{P}_{(t_0:t_1)}$.
This has a singular point, so we blow up:
$$
\xymatrix{
\tilde{S}\ar[d]\ar@/^1em/[dd]^f\\ S\ar[d]\\\mathbb{P}^1
}
$$
The generic fiber is the curve $y^4+xz^3+tx^4=0$,
so that $t_1/t_0 \in k(\mathbb{P}^1)=K$.
Fibers are plane rational quartics,
and the singular fiber $f_1(0:1)$ is a configuration of lines
given by a Dynkin diagram
$$
\xymatrix{
E_1 \ar@{-}[r]& E_2\ar@{-}[r]
& E_3 \ar@{-}[r]& E_4\ar@{-}[d]\ar@{-}[r]
& \cdots \ar@{-}[r]& E_{15}\\
& & & E
}
$$
We do the same with $C_1$: define 
$S^1=V(t_0(xy^2+z^3)+tx^3) \subseteq \mathbb{P}^2 \times \mathbb{P}^1$.
Then $S_1$ is also singular, and we blow up to obtain a smooth
quasielliptic fibration ($\tilde{S^1}$ is quasielliptic).
The fibers are plane cuspidal cubics.
The singular fiber an arrangement of curves
$$
\xymatrix{
F_1 \ar@{-}[r]& F_2\ar@{-}[r]& F_3\ar@{-}[d]\ar@{-}[r]& \cdots \ar@{-}[r]& F_8\\
& & F
}
$$

In fact, there is a correspondence between the 
singular fibers of the two fibrations.



\section{Lagrangian blow-up and blow-down for 4-dimensional symplectic manifolds
}
\label{section-blow-up-down}

\noindent
Misha Verbitsky, IMPA.
Geoemtric Structures Seminar, IMPA. 
October 23, 2025.

\medskip {\bf Abstract.} The usual (complex geometric) blow-up has a symplectic
version, called the symplectic cut. I will introduce a variant of this
construction, which is valid only in symplectic category, called ``Lagrangian
blow-up'', and its inverse, called Lagrangian blow-down. Lagrangian blow-down
takes a Lagrangian sphere in a 4-dimensional  symplectic manifold and contracts
it to a symplectic orbifold with a double point. Given a symplectic orbifold
with a double point, Lagrangian blow-up produces a symplectic manifold, and this
construction is inverse to the Lagrangian blow-down. I will explain why these
constructions are functorial, that is, defined on the corresponding symplectic
Teichmuller spaces. This is used to prove an orbifold counterexample to a famous
conjecture, sometimes attributed to Donaldson, who asked whether any symplectic
form on a K3 surface is compatible with a Kahler structure. I will prove that
this is false for orbifolds: some K3 orbifolds admit symplectic forms not
compatible with any Kahler structure. The same argument is used to produce a
countable family of Lagrangian spheres in a K3 surface which are not Lagrangian
isotopic to special Lagrangian spheres. This is a joint work with Michael Entov. 



\medskip\noindent


Consider the total space of the cotangent bundle of $\mathbb{C}P^1$.
It's $\mathcal{O}(-2)$.
Let $\gamma \in H^{0}(\mathcal{O}(2))$.
$\gamma$ defines a fiberwise linear function on $T^*\mathbb{C}P$.

The blow-up of $\mathbb{C}^2/\pm 1$, which is the orbifold $\mathbb{C}^2$ with
 a doble point, is the cotangent bundle $T^* \mathbb{C}P^1$.
That is,
$T^*\mathbb{C}P^1 \to \mathbb{C}^2/\pm 1$.
This is called {\it crepant resolution}.
Consider the $dx \wedge dy$, which is holomorphically symplectic
on $\mathbb{C}^2$ and descends to the quotient $\mathbb{C}^2/\pm 1$
as $\Omega_0$.

\begin{proposition}
\label{proposition-}
$\pi:T^*\mathbb{C}P^1 \to \mathbb{C}^2/\pm 1$.
Then $\pi^*\Omega_0$ is holomorphically symplectic.
\end{proposition}

\begin{proof}
Actually, it gives the standard symplectic form on $T^*\mathbb{C}^2$.
\end{proof}

\medskip\noindent
Now we shall define the Langrangian blow-up.
It's a procedure that starts with a symplectic manifold
with double points and outputs a symplectic manifold
without double points.

\begin{definition}[Lagrangian blow up]
\label{definition-lagrangian-blow-up}
Let $(B,\omega)$ be the standard ball in $\mathbb{R}^n$,
$dz_1 \wedge dz_2 +dz_3 \wedge dz_4$.
Consider the symplectic orbifold $(B,\omega)/\pm 1$.
Consider $M$ be a symplectic orbifold
with double point singularities.
Darboux coordinates in a neighbourhood of $0$ 
with $(B,\omega)/\pm 1$ resolve singularities.
\end{definition}

This definition has the caveat that there ar e lots of choices involved.
We shall later that these choices do not alter the result,
namely Theorem \ref{theorem-lagrangian-blow-up-well-defined}.

To define the Lagrangian blow-down we first recall

\begin{theorem}[Weinstein's Lagrangian neighbourhood]
\label{theorem-weinstein-lagrangian-neighbourhood}
Let $X \subset M$ be a compact Lagrangian submanifold in $(M,\omega)$.
Then there exists a neighbourhood $U$ of $X \subset M$
which is symplectomorphic to a neighbourhood of $X$ 
in $X \subset T^*M$.
\end{theorem}

\noindent
Let $S \subset M$ be a Lagrangian sphere (diffeomorphic to sphere
and Lagrangian).

\begin{definition}
\label{definition-lagrangian-blow-down}
The {\it Lagrangian blow-down} is obtained from $M$ 
by removing $W \subset M$ and gluing $W_0$ in its place,
where $W_0$ is such that

$$
\xymatrix{
W \ar@{^{(}->}[r]\ar[d]&T^*\mathbb{C}P^1\ar[d]\\
W_0\ar@{^{(}->}[r]&\mathbb{C}^2/\pm 1.
}
$$
\end{definition}

\medskip\noindent
Now we recall Teichmüller spaces.

Let $\Gamma(\Lambda^2M)$ be the space of all 2-forms on a manifold $M$.
The space of symplectic forms is a subspace of the closed 2-forms on $M$,
which in turn is a subspace of $\Gamma(\Lambda^2M)$.
Equip this space with the $C^\infty$ topology.
The {\it Teichmüller space} is $\text{Symp}/\text{Diff}_0$.

Two symplectic structures are called {\it isotopic} if they lie
in the same orbit of $\text{Diff}_0$.
The {\it period map} $\text{Per}:\text{Teich}_s \to H^{2}(M,\mathbb{R})$ 
maps a symplectic structure to its cohomology class.

\begin{theorem}[Moser, 1965]
\label{theorem-period-map-is-local-diffeo}
The period map is a local diffeomorphism.
\end{theorem}

\noindent
This implies that $\text{Teich}_s$ is smooth.

\begin{theorem}[Moser's trick]
\label{theorem-mosers-trick}
Let $\omega_t$, $t \in S$ be a smooth family of symplectic
structures, parametrized by a connected manifold $S$.
Assume that the cohomology class  $[\omega_t]$ is constant.
Then all $\omega_t$ are diffeomorphic.
\end{theorem}

\noindent
This means that the fibers of $\text{Symp}\to H^{2}(M)$ 
are the orbits (of the action of $\text{Diff}_0$, I suppose).

\medskip\noindent
Consider a Lagrangian submanifold $L \subset (M,\omega)$.
Let us assume for simplicity (though the results 
can also be obtained without this) that $b_1(L)=0$,
i.e. first cohomology group is zero.

Let $\text{Symp}_L$ be the set of symplectic forms vanishing on $L$,
and $\text{Diff}_0^L$ the connected component of diffeomorphisms
preserving $L$.
Define $\text{Teich}_L=\text{Symp}_L/\text{Diff}_0^L$.

\begin{theorem}
\label{theorem-}
Let $V \subset H^{2}(M,\mathbb{R})$ be the space of all
cohomology classes on $M$ which vanish on $L$,
and $\text{Per}_L: \text{Symp}_L/\text{Diff}^L_0 \to V$
take $(M,\omega)$ to the cohomology class of $\omega$.
Then Per is a local diffeomorphism.
\end{theorem}

\begin{proof}
We used a version of Moser's trick: for $\omega_t$ a family
of symplectic forms with $\omega_t|_{L}=0$,
with $[\omega_t]$ constant, then $\omega_t$ 
are $\text{Diff}_0^L$-isotropic.
\end{proof}

\noindent
As a corollary we obtain that this Teichmüller space
classifies Lagrangian submanifolds:

\begin{lemma}
\label{lemma-techmuller-classifies-lagrangian-submanifolds}
Assuma that $L$ and $L'$ are Lagrangian submanifolds in $(M,\omega)$,
and $\varphi \in \text{Diff}_0(M)$ a smooth isotopy such that
$\varphi(L)=L'$. Then $(\omega,L)$ and $(\varphi^*\omega,L)$ 
represent the same point in $\text{Teich}_L$ if and only if
$L$ and $L'$ are Lagrangian isotopic in $(M,\omega)$.
\end{lemma}

\noindent
Again: points of this Teichmüller space are isotopy classes
of Lagrangian submanifolds.

\medskip\noindent
\begin{theorem}
\label{theorem-blow-down-diffeomorphism}
Let $L$ be a Lagrangian 2-sphere in a K3 surface  $M$
and $\hat{M}$ the Lagrangian blow-down of $M$.
The blow-up/blow-down
is a diffeomorphism $\text{Teich}_L(M) \to \text{Teich}(\hat{M})$.
\end{theorem}

\medskip\noindent
Now we prove the theorem which asserts that the Lagrangian
blow-up is independent of choices.

\begin{theorem}
\label{theorem-theorem-lagrangian-blow-up-well-defined}
Let $(\hat{M},\hat{\omega})$ be a orbifold. Then its blow-up
$(M,L)$ defines a point in $\text{Teich}_L$ independent of the choices.
\end{theorem}

\begin{proof}
By taking local Darboux coordinates, and then Moser's trick.
\end{proof}

\noindent
Actually, the same happens for blow-downs

\begin{theorem}
\label{theorem-same-for-blow-down}
Let $(M,L,\omega)$ be a 4-manifold with a Lagrangian sphere $L$, and
$(\hat{M},\hat{\omega)}$ be its Lagrangian blow-down.

Then the corresponding point in $\text{Teich}_{\hat{M}}$ is
independent from the choices made.
\end{theorem}

\begin{proof}
Similar.
\end{proof}

\medskip\noindent
A conjecture by Donaldson ways that all symplectic structures on a K3
surface are compatible with a Kähler structure. This conjecture
(although it's commonly believed to be true) is still open.
When trying to prove that this would hold for orbifolds,
Misha and collaborators realised it's actually false.

Note: Seiberg-Witten invariants may be used to tell whether a symplectic
form is compatible with a complex structure.

\begin{theorem}
\label{theorem-conjecture-false-for-orbifolds}
There exists an orbifold symplectic K3 surface $M$ with a single
double point not admitting an orbifold Kähler structures.
\end{theorem}

\begin{proof}
\noindent
Step 1. Consider 
$$
\text{Teich}_{\text{Kähler-symplectic}}(\hat{M})
=\{\eta \in H^{2}(\hat{M}):\eta^2>0\}.
$$
As follows from [Amerik-V., 2005], the
Teichmüller space of symplectic structures compatible
with a hyperkähler structure is Hausdorff and connected
(the same argument works for Kähler K3 orbifolds).

Essentially, if there wasn't any orbifold
as required, the Teichmüller space classifying
Lagrangian submanifolds would be connected, which
is impossible by [Seidel 2000].
\end{proof}

\noindent
In fact, Seidel's result is essentially mirror symmetry.

\medskip\noindent
Now we discuss special Lagrangian submanifolds.

\begin{definition}
\label{definition-phase}
Let $(M,I,\omega,\Omega)$ be a Calabi-Yau manifold, where
$\Omega \in \Lambda^{n,0}_I(M)$ is nondegenerate and $\omega$ is Kähler.
Let $L$ be a Lagrangian submanifold.
Define the {\it phase} as
\begin{align*}
\text{phase}: L &\longrightarrow \Lambda^nL \otimes \mathbb{C} \\
x &\longmapsto \Omega|_{T_xL}.
\end{align*}
\end{definition}

\noindent
A better definition is:
\begin{align*}
\text{phase}: L &\longrightarrow S^1=\text{U}(1) \\
x &\longmapsto \frac{\Omega|_{T_xL}}{|\Omega T_xL|}.
\end{align*}

\begin{definition}
\label{definition-special-lagrangian}
A special Lagrangian submanifold is the phase is constant.
\end{definition}

\noindent
Special Lagrangian submanifolds have deformation space of
dimension 1, it's unobstructed. They are calibrated, so minimal.

\begin{theorem}
\label{theorem-lagrangians-of-k3}
$L \subset$ K3 is Lagrangian is isotopic to a special Lagrangian
if and only if its blow-down is Kähler type.
\end{theorem}

\medskip\noindent
One of the basic results about special Lagrangian
submanifolds is

\begin{lemma}[Hitchin]
\label{lemma-hitchin}
If $(M,\Omega)$ is holomorphically symplectic and $X \subset M$ 
is Lagrangian with respect to $\text{Re}\Omega$ and $\text{Im}\Omega$
then $X$ is complex.
\end{lemma}

\noindent
As corollaries:

\begin{lemma}
\label{lemma-corollary-hitchin}
Let $(M,I,J,K,g)$ be a hyperkähler manifold of real dimension $4n$,
considered as a Kähler manifold $(M,I,\omega_I)$ and $\phi:\Omega^n \in
\Lambda^{2n,0}(M,I)$ the corresponding holomorphic volume form.
Consider a Lagrangian submaniold $S \subset (M,\omega_I)$
such that $a \omega_J + b\omega_K|_S=0$ for some nonzero real numbers
$a,b \in \mathbb{R}$ such that $a^2+b^2=1$.
Then $S$ is special Lagrangian, and, moreover,
it is holomorphic with respect to the complex structure
$-a K+b J$.
\end{lemma}

\begin{lemma}
\label{lemma-corollary2-hitchin}
Let $\omega_I,\omega_J$ and $\omega_K$ on $M$, with $\dim M=4$.
Let $S \subset (M,\omega_I)$ be Lagrangian.
Then $S$ is special Lagrangian if and only if $S$ is complex analytic
on $a J + b K$ for two nonzero numbers  $a,b \in \mathbb{R}$ such that
$a^2+b^2=1$.
\end{lemma}

\noindent
Now an auxiliary claim:

\begin{lemma}
\label{lemma-k3-type}
$\Omega$ is holomorphic symplectic 
\end{lemma}

\begin{theorem}
\label{theorem-}
Let $S\subset M$ be a Lagrangian sphere in a K3 surface.
Then  $S$ is isotropic to a special Lagrangian if and only if
$M$ is of Kähler type.
\end{theorem}




\bibliography{my}
\bibliographystyle{amsalpha}




\end{document}

\input{preamble}
\begin{document}

\title{A tour through algebraic surface}
\maketitle

\noindent
Minicourse by 
Aline Zanardini
(EPFL, Switzerland),
IMPA Summer School, 2026.

\medskip\noindent

\hfill Notes at 
\href{http://github.com/danimalabares/stack}
{github.com/danimalabares/stack}

\bigskip\noindent

\tableofcontents

\section{Plan}
\label{section-plan}

\begin{enumerate}
\item Basics.
\item Birational maps and classification.
\item Elliptic surfaces.
\item Halphen surfaces.
\item K3 surfaces and friends.
\item Research questions.
\end{enumerate}

\section{References}
\label{section-references}

\begin{enumerate}
\item A. Beauville. Complex algebraic surfaces.
\item R. Miranda. Overview of algebraic surfaces.
\item M. Reid. Chapters on algebraic surfaces.
\end{enumerate}

\section{Introduction}
\label{section-intro}

\noindent
Everything will be over the complex numbers.
By ``surface'' we mean ``algebraic surface'',
but we drop the term ``algebraic''.
Informally, a surface is a smooth projective algebraic variety of dimension 2.
We may think of surfaces also as
compact connect complex manifolds of dimension 2.

$\mathcal{M}(S)$, meromorphic functions on $S$, has transcendence degree
2 over $\mathbb{C}$.
That is, for $p,q \in S$
there exists $f \in \mathcal{M}(S)$
such that $f(p)\neq f(q)$
and for all $p \in S$ there exists $f,g \in \mathcal{M}(S)$
such that $(f,g)$ give local coordinates at $p$.

\begin{remark}
\label{remark-transendence-implies-surface-def}
The remark on transcendence just made implies that $S$
is smooth algebraic of dimension 2.
Indeed, compactness implies that there exists
meromorphic functions  $\varphi_1,…,\varphi_n \in \mathcal{M}(S)$
such that $S \xymatrix{\ar@{^{(}->}[r]^{alg}&}\mathbb{P}^n$,
$p \mapsto (1:\varphi_1(p):…:\varphi_n(p))$ and Chow theorem
imply that $S = V(f_1,…,f_k)$ with 
$f_i \in \mathbb{C}[x_0,x_1,…,x_n]$.
\end{remark}

\begin{example}
\label{example-surfaces}
\begin{enumerate}
\item $S = V(f) \subset \mathbb{P}^3$
with $f \in \mathbb{C}[x_0,x_1,x_2,x_3]$
homogeneous of degree $d$.
For example, $f=x_0^d+x_1^d+x_2^d+x_3^d$.
Take $d=1$, then we can assume that $f=x_i$,
and then $S\simeq\mathbb{P}^2$.
If $d=2$ we can assume that  $f=x_0x_1-x_2x_3$ and
one can verify that $S \simeq \mathbb{P}^1 \times \mathbb{P}^1$.
If $d=3$, $S \simeq Bl_6\mathbb{P}^2$.
These are all rational surfaces.

If $d=4$, $S$ is a K3 surface.

\item (Complete intersections of type $(d_1,…,d_{n-1})$ in $\mathbb{P}^n$.)
Let $f_1,…,f_{n-2} \in \mathbb{C}[x_0,x_1,…,x_n]$ homogeneous
of degrees $d_1,…,d_{n-2}$.
Then $S=V(f_1,…,f_{n-2})$.

E.g., intersections of type $(2,3)$ in  $\mathbb{P}^4$
or of type $(2,2,2)$ in $\mathbb{P}^5$, which are K3 surfaces.

\item (Ruled surfaces.)
Take any curve $C$ and consider  $C \times \mathbb{P}^1$
(or anything birational to it).
[This is interesting because we can consider the projection to $C$,
and we get a fibration.]
\end{enumerate}
\end{example}

\section{Curves on surfaces (divisors)}
\label{section-curves-on-surfaces}

\noindent
We will use: $\text{Div}(S)/\sim \simeq \text{Pic}(X)=H^1(S,\mathcal{O}_S^*)$
where $\sim$ is linear equivalence.

$D \subset S$ is a divisor if and only if $D$ is one of the following:
\begin{enumerate}
\item (Cartier divisor.)
[I can think of $D$ given locally as the zero locus of some function.]
$D:f=0$ for  $0 \neq  f \in \mathcal{M}(S)$ + gluing condition.
This is equivalent to saying that $D|_f=\text{div}(f)$.

\item (Weil divisors.)
$D=\sum n_iC_i$ for $n_i \in \mathbb{Z}$,
$C_i \subset S$ irreducible curves,
where there are only finitely many nonzero coefficients $n_i$.

\item (Line bundle.)
$D$ effective (i.e. $n_i\geq 0$), we associate
$\mathcal{O}_S(D)$, a holomorphic line bundle on $S$ along
with a nonzero section [whose zero locus defines $D$.
Recall that $U \mapsto  \mathcal{O}_S(D)(U)$ is ``given''
by $g(z,w)$ such that  $fg$ is holomorphic
where $D|_U=\text{div}(f)$.
And we extend by linearity.
\end{enumerate} 

\noindent
[And we say that two divisors are linearly
equivalent if their difference is $div(f)$ for some $f$.]

\begin{example}
\label{example-divisors}
\begin{enumerate}
\item $\text{Pic}(\mathbb{P}^2)=\mathbb{Z}$
generated by $\mathcal{O}_{\mathbb{P}^2}(1)$,
the dual of the tautological bundle, or, equivalently,
the class of a hyperplane $H$,
whose sections are polynomials of degree 1
(in the appropriate number of variables).

\item $\text{Pic}(\mathbb{P}^n)=\mathbb{Z}$ 
is generated by $\mathcal{O}_{\mathbb{P}^n}(1)$.
We have $\mathcal{O}_{\mathbb{P}^n}(d):=\mathcal{O}_{\mathbb{P}^n}^{\otimes d}$.
For $n=2$, the sections of $\mathcal{O}(d)$ are
plane curves of degree 2.
\end{enumerate}
\end{example}

\noindent
Suppose that $\pi:S \xymatrix{\ar@{^{(}->}[r]&} \mathbb{P}^n$.
The {\it hyperplane class} is  $H|_S:= \pi^*\mathcal{O}_{\mathbb{P}^n}(1)$.
We denote $\mathcal{O}_S(1)=\mathcal{O}_S(H|_S)$.
This is always a non-trivial class.

We also have the {\it canonical class},
which is the class of any divisor $K$ 
such $\mathcal{O}_S(K)=\Omega^2_S$,
the sheaf of holomorphic 2-forms,
informally its elements look like $f(z,w)dx\wedge dw$.
We denote this as $K$,  $K_S$ or $\omega_S$.

\begin{example}[Canonical class of projective plane]
\label{example-canonical-class}
Take coordinates $Z=X_1/X_0$ and $W=X_2/X_0$
on $U_0: X_0 \neq 0$.
Likewise put $U=X_0/X_1$, $V=X_2/X_1$ at $U_1$.
Then $dZ\wedge dW=-U^{-3}dU\wedge dV$.
Thus, the canonical class of $\mathbb{P}^2$
is linearly equivalent to $-3H$.
In general, $K_{\mathbb{P}^n}\sim -(n+1)H$.

In practice, we can use this to compute $K_S$
for any  $S \xymatrix{\ar@{^{(}->}[r]&} \mathbb{P}^n$
via $\omega_S=\omega_{\mathbb{P}^n}\otimes \Lambda^{n-2}N_S/\mathbb{P}^n$.
\end{example}


\section{Numerical invariants}
\label{section-numerical-invariants}

\begin{itemize}
\item (Hodge decomposition.)
$$
H^k(S,\mathbb{C})=\bigoplus_{p+q=k}\underbrace{H^q(S,\Omega^p_S)}_{
H^{p,q}(S)}.
$$
Setting $h^{p,q}=\dim H^{p,q}$ + symmetries,
we have the Hodge diamond
$$
\begin{aligned}\xymatrix@C=.5em@R=.5em{
&  &   h^{0,0}\\
&  h^{0,1}&  &  h^{1,0}\\
h^{0,2}&&  h^{1,1}&  &  h^{2,0}\\
&  h^{1,2}&  &  h^{2,1}\\
&  &  h^{2,2}
}\end{aligned}
\qquad =\qquad 
\begin{aligned}\xymatrix@C=.5em@R=.5em{&  &   1\\
&  q&  &  q\\
p_g&&  h^{1,1}&  &  p_g\\
&  q&  &  q\\
&  &  1
}\end{aligned}
$$
where we call $q$ the {\it irregularity}
$p_g$ the {\it geometric genus}
and the {\it Euler characteristic}
is the alternating sum of Betti numbers.
Notice that $h^{1,1}\geq 1$.
\end{itemize}

\section{The Néron-Severi group and Lefschetz (1,1) theorem}
\label{section-neron-severi}

\noindent
There is always a map
\begin{align*}
c_1: \text{Pic}(S) &\longrightarrow H^2(S,\mathbb{Z})
\overset{\substack{\text{Poincaré} \\ \text{dual}}}{\cong}
H_2(S,\mathbb{Z}) \\
\mathcal{L} &\longmapsto c_1(\mathcal{L})
\end{align*}

\noindent
We define $\text{NS}(X):=\text{Im}c_1/\text{torsion}$.
It is a finitely generated abelian group isomorphic to $\mathbb{Z}^{\rho}$
where $\rho$ is the Picard rank.

%\begin{theorem}[Lefschetz (1,1)-classes]
%\label{theorem-lefschetz-11}

\begin{theorem}
\label{theorem-form}
$$
\text{NS}(S)=H^2(S,\mathbb{Z})\cap H^{1,1}(S)
$$
$\text{NS}(S)$, $\text{Div}(S)/\sim$ and $\text{Pic}(S)$
have more structure!
There exists a unique bilinear form 
\begin{align*}
\cdot: \text{Div}(S)\times \text{Div}(S) &\longrightarrow \mathbb{Z} \\
(C,D) &\longmapsto C\cdot D
\end{align*}

\noindent
such that 
\begin{enumerate}
\item If $C $ and $D$ are smooth curves which intersect
transversally, then $C.D=\# C \cap D$.
\item If $C \sim C'$, then $C\cdot D = C' \sim D$.
\end{enumerate}
\end{theorem}

\begin{definition}
\label{definition-form}
$C.D=\mathcal{O}_S(C).\mathcal{O}_S(D)
=\left<c_1(\mathcal{O}_S(D) \cup c_1(\mathcal{O}_S(D)),
[S]_{\text{fund}}\right>$,
and we say that $C.C=\text{deg} \mathcal{N}_{C/S}$.
\end{definition}

\noindent
[An important invariant of a surface
is the self-intersection number of the canonical class.]
Particular case: $K^2=c_1^2$
where $c_1^2$ is the first Chern number.

\begin{example}
\label{example-intersection}
\begin{enumerate}
\item $K_{\mathbb{P}^2}^2=(-3 H).(-3 H)=9$.
\item (Degree of $S$.)
Let $S \xymatrix{\ar@{^{(}->}[r]&} \mathbb{P}^n$.
Recall our notation above for
$\mathcal{O}_S(1).\mathcal{O}_S(1)$.
If $n=3$,
the degree $d$ of $S$ is the degree of $\mathcal{N}_{S/\mathbb{P}^3}$.
Thus the adjunction becomes $\omega_S=\mathcal{O}_S(d-4$

\item $C$ smooth curve, then 
$f:S \xymatrix{\ar@{->>}[r]&}C$,
$F$ fiber, then  $F^2=0$.

\item $g=\tilde{S}\overset{d}{\to}D$,
then $D_1,D_2 \in \text{Div}(S)\implies 
g^*(D_1)\cdot g^*(D_2)=d(D_1.D_2)$.
\end{enumerate}
\end{example}

\begin{definition}
\label{definition-nef-and-ample}
A divisor $D \subset S$ is {\it nef} if and only if
for each $C \subset S$ irreducible,
$D\cdot C \geq 0$.
We say that $D$ is ample if and only if
$D^2>0$ and for each irreducible $C \subset S$
we have $D.C \geq 0$.
\end{definition}

\section{Big theorems}
\label{section-big-theorems}

\begin{theorem}
\label{theorem-}
Let $S$ be a surface.
\begin{enumerate}
\item (Noether's formula.)
$$
12\chi = K^2+e=c_1^2+c_2
$$
where $e$ is the Euler number.

\item (Riemann-Roch.) Given $D \in \text{Div}(S)$,
$$
\chi(\mathcal{O}_S(D))=\frac{D.(D-K)}{2}+\chi(\mathcal{O}_S).
$$

\item (Genus formula.) Given $C \subset S$ smooth,
$$
2g-2=C.(C+K).
$$
\end{enumerate}
\end{theorem}

\begin{example}
\label{example-big-theorems}
\begin{enumerate}
\item For $\mathbb{P}^2$, $K^2=9$ and $e=3$.
Thus $12\chi = 12 \implies \chi =1$.
The irregularity is $q=H^0(\mathbb{P}^2,\Omega^1)=0$.
The arithmetic genus is $p_g=\dim H^0(\mathbb{P}^2,\Omega^2)=0$.
Thus, the Hodge diamond is
$$
\xymatrix@C=.5em@R=.5em{
& & 1\\
& 0 && 0\\
0 && 1 && 0\\
& 0 && 0\\
& & 1
}
$$
\item Any rational surface has a Hodge diamond
$$
\xymatrix@C=.5em@R=.5em{
& & 1\\
& 0 && 0\\
0 && \rho && 0\\
& 0 && 0\\
& & 1
}
$$
The Noether formula gives
$12=K_S^2+\rho+2 \implies K_S^2=10-\rho$.

\item $C \subset \mathbb{P}^2$ smooth of degree $d$,
then $2g-2=d^2-3d$ and
$g=\frac{d^2-3d+2}{2}=\frac{(d-1)(d-2)}{2}
=\binom{d-1}{2}$.
In particular,
if $d=1$ or  $2$,
then  $g=g(C)=0$ 
if $d=3$ then  $g(C)=1$.
\end{enumerate}
\end{example}

\section{Birational maps and classification theorem}
\label{section-birational-maps-and-classification-theorem}

\noindent
Question: given two surfaces $S$ and $S'$,
when are $S$ and $S'$ birational/biregular or isomorphic?

\begin{itemize}
\item Rules surfaces
(all birational to $C\times \mathbb{P}^1$ for some $C$.
\item Classify minimal (models of) surfaces
by looking at the positive of $K$ 
(and related invariants
$q$, $p_g$ and $\kappa$).
\end{itemize}

\begin{center}\begin{tabular}{c|c|c|c|c}
$K$ & $\kappa$ & $p_g$ &  $q$  & Structure\\
\hline\hline
$K^2>0$&2&&&General type\\ \hline
$K^2=0$&1&&&Elliptic*\\\hline
$K=0$&0&
$\substack{1\\0\\1\\0 \\ }$&
$\substack{2 \\1\\0\\0 }$
&$\substack{
\text{abelian} \\ 
\text{hyperelliptic}\\
\text{K3}\\
\text{Enriques}}$\\\hline
&$-\infty$&$\substack{0 \\ 0}$&
$\substack{\geq 1 \\ 0}$&
$\substack{\text{ruled surfaces} \\ \text{rational}}$
\end{tabular}\end{center}

\medskip\noindent
The following proposition is also a definition:

\begin{proposition}
\label{proposition-birational}
Let $S$ and $S'$ be two surfaces, then the following
statements about $S$ and $S'$ are equivalent:
\begin{enumerate}
\item $S$ and $S'$ are birational
\item $\mathcal{M}(S)\simeq \mathcal{M}(S')$
as algebras (over $\mathbb{C}$).
\item There exists $U \subset S$, $V \subset S'$ 
opens such that $U \simeq V$.
\end{enumerate}
\end{proposition}

\section{The blow up of the projective plane at a point}
\label{section-blow-up}

\noindent
Consider $p=(0:0:1)$ and
\begin{align*}
\varphi_p: \mathbb{P}^2\setminus \{p\} &\longrightarrow \mathbb{P}^1 \\
(x_0:x_1:x_2) &\longmapsto (x_0:x_1)\\
q &  \longmapsto L_{pq}
\end{align*}

\noindent
Consider also
$$
\Gamma:=\left\{((x_0:x_1:x_2),(y_0:y_1)) \in \mathbb{P}^2\times \mathbb{P}^1
: \det\begin{pmatrix}x_0&x_1\\ y_0&y_1\end{pmatrix}=0\right\}.
$$

By construction,
$$
\pi:\Gamma \to \mathbb{P}^2
$$
is an isomorphism over $U=\mathbb{P}^2 \setminus \{p\}$
and $\pi^{-1}(p)=\{p\}\times \mathbb{P}^1 \simeq\mathbb{P}^1:=E$.

\begin{definition}
\label{definition-blow-up}
$\pi:\Gamma \to \mathbb{P}^2$ as above is called
the {\it blowup} of $\mathbb{P}^2$ at $p$ 
(also $\Gamma$). The curve $E$ is called
the {\it exceptional divisor}.
\end{definition}

\noindent
Note that we have the following local description
(chart $x_2=1$):
$\{(x,\ell) \in \mathbb{C}^2\times \mathbb{P}^1: x \in \ell\}
\to \mathbb{C}^2$.
That is, the exceptional divisor may be identified with
the zero section of the tautological bundle $\mathcal{O}_{\mathbb{P}^1}(-1)$,
which means that the self intersection of $E$ is 1,
i.e. $E^2=1$.

\begin{remark}
\label{remark-blowup}
\begin{enumerate}
\item The surface $Bl_p\mathbb{P}^2:=\Gamma$
is the Hirzebruch surface $\mathbb{F}_1$ (or $\Sigma_1$).

\item Choosing local coordinates, we can talk about 
$Bl_p S$ for $p \in S$.

\item In fact, $\pi:Bl_pS \to S$ 
is characterized by the following universal property:
if $f:\tilde{S}\to S$ is birational such that $f^{-1}$
is not defined at $p \in S$, then $f$ factorizes
as $f:\tilde{S} \underbrace{\xrightarrow{g}}_{\text{bir}} 
\hat{S}:=Bl_pS \xrightarrow{\varepsilon} S$ 
where $\varepsilon^{-1}(S\setminus \{p\}) \simeq S \setminus \{p\}$ 
and $\varepsilon^{-1}(p) \simeq \mathbb{P}^1$.
\end{enumerate}
\end{remark}

\begin{proposition}
\label{proposition-blowup}
Consider $\pi:\tilde{S}=Bl_pS \to S$
as above, with exceptional divisor $E$,
\begin{enumerate}
\item $E^2=-1$,

\item $E\cdot \pi^*D=0$ for all $D \in \text{Div}(S)$,

\item If $C \subset S$ has mutliplicity $m$ at $p$, then
$\tilde{C}:= \overline{\pi^{-1}(C) \setminus E}\subset\tilde{S}$
satisfies
$\tilde{C}\cdot E=m$ and $\tilde{C}^2=C^2-m^2$

\item $K_{\tilde{S}}=\pi^* K_S+E\implies K_{\tilde{S}}^2=K_S^2-1$ 

\item $q,p_g$ and  $\chi$ are the same for $\tilde{S}$ as for $S$ 

\item $e$,  $h^{1,1}$ and $b_2$ all go up by 1

\item $\text{Pic}(\tilde{S})=\pi^*(\text{Pic}(S))\oplus \mathbb{Z}E$.
\end{enumerate}
\end{proposition}

\section{Castelnuovo's criterion}
\label{section-castelnuovo-criterion}

\noindent
[It turns out that if you find a surface with a $-1$ curve,
then that surface is the blow up of another!]

\begin{theorem}
\label{theorem-castelnuovo}
If on a surface $\tilde{S}$ one finds a curve $E \simeq \mathbb{P}^1$
with $E^2=-1$, then $\tilde{S}=Bl_pS$
for some $p \in S$.
\end{theorem}

\begin{definition}
\label{definition-minimal-surface}
A surface $S$ is {\it minimal} 
if it has no $(-1)$-curves
(i.e. a curve $\simeq \mathbb{P}^1$ with self-intersection -1).

(If and only if
$f:S \xymatrix{\ar@{-->}[r]&}\tilde{S}$ and $S$ minimal,
then $f$ is an isomorphism.)
\end{definition}


\section{Mori's point of view}
\label{section-moris-point-of-view}

\noindent
If $S$ is a surface such that $K_S$ is not nef
[so, by definition of nef there must exist
an irreducible curve that intersects
the canonical divisor negatively, and this curve
in fact is a $(-1)$ curve:]
then there exists a $(-1)$-curve on $S$.




\section{Kodaira dimension}
\label{section-kodaira-dimension}

\noindent
Let $S$ be a surface.

\begin{definition}
\label{definition-kodaira-dimension}
For each $n \geq 0$,
we define the {\it pluri-genera}
$$
P_n:= \dim H^0(S,\mathcal{O}_S(n K))
$$
and the {\it Kodaira dimension}
$$
\kappa:=
\begin{cases}
-\infty&\qquad \text{if }P_n=0\forall n \\
\text{smallest $k$ such that }
P_n=O(n^k)&\qquad \text{otherwise}
\end{cases}
$$
\end{definition}

\begin{remark} \label{remark-kodaira} (By Daniel.) In one of Misha's course we
have that the function $\dim H^0(n K)$ is known to be polynomial for all
projective varieties (Birkar, Cascini, Hacon, McKernan). Conjecturally it is
always polynomial. Thus, the definition of Kodaira dimension is nothing
more than the degree of this polynomial.

Further, in \cite[Definition 2.2.26]{huc} we see
that this should be equivalent to defining it as the
transcendence degree over $\mathbb{C}$
of the fraction field of the ring 
$\bigoplus_{m \geq 0}H^0(X,K_X^{\oplus m})$,
which is endowed with a natural product which, in general for
vector bundles $E$ and $F$, maps
$H^0(X,E) \otimes H^0(X,F) \to H^0(X,E \otimes F)$.
\end{remark}

\section{Results of Mori's theory}
\label{section-results-of-mori}

\begin{theorem}
\label{theorem-one}
Given a surface $S$,
there exists a chain
$$
S=S^0 \xrightarrow{\sigma_1} S^1 \to \cdots \xrightarrow{\sigma_N}S_N=S'
$$
where each $\sigma_i$ is the contraction of a $(-1)$ curve $E_i$
and $S'$ satisfies
\begin{enumerate}
\item $K_{S'}$ is nef or
\item $S'$ is a $\mathbb{P}^1$-bundle
over a curve or 
\item $S' \simeq \mathbb{P}^2$.
\end{enumerate}
\end{theorem}

\begin{theorem}
\label{theorem-two}
If a surface $S$ is such that
$K_S$ is not nef then there exists $\varphi:S \to X$,
with $\dim X=0,1$ or $2$, contracting at least
one curve to a point, such that 
$-K_S\cdot C>0$ for every curve $C$ contained in a fiber of $\varphi$.
\end{theorem}

Note:
$$
\dim X=
\begin{cases}
0\xymatrix{\ar@{~>}[r]&}&-K \text{ is ample}\\
1\xymatrix{\ar@{~>}[r]&}&S \text{ is a conic bundle}\\
2\xymatrix{\ar@{~>}[r]&}& S=Bl_{p_1,…,p_n}.
\end{cases}
$$


\section{Examples and the cubic surface revisited}
\label{section-examples-and-cubic}

\begin{enumerate}
\item Slogan: ``blowups resolve singularities''.
Consider the plane cubic
$$
C=\{(x_0:x_1:x_2) \in \mathbb{P}^2:
\underbrace{x_1^2x_2-x_0^2(x_0+x_2)=}_{:=f_C}0\},
$$
the point $p=(0:0:1)$
and $\pi:Bl_p\mathbb{P}^2 \to \mathbb{P}^2$,
then
$$
\pi^{-1}(C)=
\begin{cases}
f_C=0\qquad & \\
x_0y_0=y_0x_1\qquad &
\end{cases}
$$
[This curve has a simple node.
In the blowup, the curve will intersect
the exceptional divisor in two points ---
we have ``separated'' the singularity into two points]

\item $Bl_1\mathbb{P}^1 \times \mathbb{P}^1
\simeq Bl_1 \mathbb{F}_1
\simeq Bl_2\mathbb{P}^2$.
[Consider two curves $L_1$ and $L_2$ on $\mathbb{P}^1 \times \mathbb{P}^1$.
Their strict transforms along with the exceptional divisor
become $(-1)$ curves.
Then contract $\tilde{L}_1$, this gives $\mathbb{F}_1$,
the Hirzebruch surface, which is just the blowup
at a point of $\mathbb{P}^2$.
$\mu(\tilde{L}_2)=-1$, $\mu(E\setminus \tilde{L}_1)=0$.]

\item Consider two cubics $C_1$ and $C_2$.
They intersect in 9 points.
Do
\begin{align*}
\mathbb{P}^2 &\xymatrix{\ar@{-->}[r]&} \mathbb{P}^1 \\
p &\longmapsto (C_1(p):C_2(p))
\end{align*}

\noindent
Let $S=Bl_{p_1,…,p_9}\mathbb{P}^2$.
Then
\begin{enumerate}
\item $K_S^2=0$ [Because $K^2$ of  $\mathbb{P}^2$ is 9,
and for every blowup we substract 1!]
\item $\text{Pic}(S)$  ``$=$'' $\pi^*\text{Pic}(\mathbb{P}^2)
\oplus\mathbb{Z}E_1 \oplus … \oplus \mathbb{Z}E_q$.
\item $-K_S=-\pi^* (K_{\mathbb{P}^2})
-E_1-…-E_q$ is nef.
(In fact, this is the class of a fiber of $S \to \mathbb{P}^1$.)
\end{enumerate}

\item (Continuation of previous item.)
$-K_S$ nef + genus formula  $\implies $ 
$C^2 \geq -2$ for every $C\subset S$ irreducible and rational.
This is an elliptic surface!

[We have taken a 1-dimensional linear system.
But why not take a higher-dimensional linear system?]

\item Fix $p_1,…,p_6$ in  $\mathbb{P}^2$ in general position
(not three in a line, not six in a conic).
Consider $f_1,f_2,f_3,f_4 \in \mathbb{C}[x_0,x_1,x_2]$ 
homogeneous of degree $3$ vanishing at $p_i$
and consider
\begin{align*}
\mathbb{P}^2 &\xymatrix{\ar@{-->}[r]&}\mathbb{P}^3 \\
p &\longmapsto (f_1(p):f_2(p):f_3(p):f_4(p)).
\end{align*}

\noindent
The resolved morphism $Bl_6\mathbb{P}^2 \to \mathbb{P}^3$ 
is an embedding with image of degree $3$
[a cubic surface on $\mathbb{P}^3$].
\end{enumerate}



\section{Elliptic surfaces}
\label{section-elliptic-surfaces}

\begin{definition}
\label{definition-elliptic-surface}
An {\it elliptic surface} 
is a surface $S$ together with $f:S\to C$ 
holomorphic 
(with connected fibers)
from $S$ to a smooth curve $C$ 
such that the generic fiber 
is a smooth curve of genus 1.

Moreover, we assume that there
do not exist $(-1)$-curves on fibers of $f$.
\end{definition}

Warning: I don't assume the existence
of a section
[a distinguished point on the fibers],
and I'm not saying that
all fibers are smooth.

\medskip\noindent
Singular fibers have been classified by Kodaira:
$I_0,I_1,(m)I_n,II, IV,IV^*=\tilde{E}_6,
III^*=\tilde{E}_7, II^*=\tilde{E}_8$ and $I_n$.

\begin{example}
\label{example-pencils-of-cubics}
$$
\left\{\substack{\text{pencils} \\ 
\text{of cubics}\\
\text{w/ smooth}\\
\text{member}}\right\}
\xymatrix{\ar@{<~>}[r]&}
\left\{\substack{\text{RES} \\ \text{w/ section}}\right\}.
$$
$$
\mathcal{P} \mapsto S_{\mathcal{P}}.
$$
 Question: given $\mathcal{P}$ and $S_{\mathcal{P}}$ 
how many singular fibers does $S_{\mathcal{P}}$ have?
\end{example}

\begin{lemma}
\label{lemma-discriminant-locus}
The discriminant locus in $\mathbb{P}^9$
[this $\mathbb{P}^9$ parametrizes cubics in $\mathbb{P}^2$ ]
has degree 12.
\end{lemma}

\noindent
Note that the Hodge diamond $S_{\mathcal{P}}$ 
is 
$$
\xymatrix@C=.5em@R=.5em{
& & 1\\
& 0 && 0\\
0 && 10 && 0\\
& 0 &&0\\
& & 1
}
$$
(Notice that adding up the center column gives 12.)

\begin{example}
\label{example-pencil-of-conics}
[We take a conic $C$ 
and a triple line $3L$ 
(it has to be triple so that it is a pencil of conics)
which is tangent to $C$ at an inflection point.
Then we blow up the point of intersection 9 times,
each time picking a point in the intersection
of the strict transform of the original cubic
and the triple line.]

Blowing up 9 times give
a surface $S_{\mathcal{P}}$ with
a $II^*$ fiber.
\end{example}

\medskip\noindent
Question: How is Kodaira's classification
obtained?

\begin{enumerate}
\item If $F=\sum n_iC_i$ is singular fiber,
then the intersection form restricts
to 
$\text{span}(\{C_i\})$.
This restriction is negative semi-definite
with a one-dimensional kernel spanned by $F$.

\item If $F$ is irreducible and $p_a(F)=1$,
then $F=$smooth,nodal or cuspidal.
 If $F$ is reducible, then each $C_i$ is
a $(-2)$-curve.
\end{enumerate}

\noindent
This two items combined imply that the
dual graph [vertices are curves,

two vertices joined if they intersect]
is a Dynkin diagram (ADE type)
[by a result of graph theory].




\section{The Weistrass model}
\label{section-weierstrass-model}

\noindent
Let $S \to C$ be an elliptic surface
with a section.
A choice of a section defines
a divisor of degree 1 on the generic fiber $S_\eta$,
i.e. a point $p$.

Looking at $H^0(S_\eta,np)$
for every $n$, we get:
\begin{align*}
H^0(p)&=\left<1\right>,\qquad  H^0(2p)=\left<1,x\right>,
\qquad H^0(3p)=\left<1,x,y\right>,\\
…,&\qquad   H^0(6p)=\left<1,x,y,x^2,xy,x^3,y^2\right>,
\end{align*}

\noindent
so there must exist a relation.
Up to further tricks,
we get to $y^2=x^3+ax+b$.

Making the construction global
gives the Weierstrass model.

\begin{itemize}
\item Any $S \to C$ is birational to a
Weierstrass fibration $W \to C$ 
(where $W$ can be singular, while $C$ is smooth)
where all fibers are irreducible and have $p_a=1$.
\end{itemize}

\noindent
Geometrically, $W$ is obtained from $S$ 
by contracting all fiber components that
do not meet the chosen section

[Once we show that we can do this,
it will suffice to study Weierstrass fibrations.]

[From now on $W$ is a Weierstrass fibration.]
Given $\pi_i:W \to C$ (with a section $\sigma$)
consider the fundamental line bundle
$\mathcal{L}=(\pi_*N_{\sigma/K})^\vee=(R^1\pi_*\mathcal{O}_W)^\vee$.
Then for all $n \geq 2$ 
we have a splitting
$\pi_*\mathcal{O}_{W}(n\sigma)\simeq \mathcal{O}_C \oplus \mathcal{L}^{-2}
\oplus … \oplus \mathcal{L}^{-n}$.
In particular, we have a section
$$
\pi^*\pi_*\mathcal{O}_W(3\sigma) 
\xymatrix{\ar@{->>}[r]&}\mathcal{O}_W(3\sigma)
$$
gives a map $f:W \to \mathbb{P}(\pi_*\mathcal{O}_W(3\sigma))$
that exhibits $W$ as a relative cubic in a $\mathbb{P}^2$-bundle
over $C$.

A global equation is given by
$$
Y^2Z=X^3+AXZ^2+BZ^3
$$
where $(Z,X,Y,A,B)$ are interpreted
as sections of 
$(\mathcal{O}_C,\mathcal{L}^2,\mathcal{L}^3,\mathcal{L}^4,\mathcal{L}^6)$.
$\sigma$ is given by $Z=X=0$.

\begin{remark}
\label{remark-singular-fibers}
The singular fibers of a minimal resolution of $W$ 
are determined by the order of vanishing of $A,B$ and
$\Delta=4A^3+27B^2$.
\end{remark}

\medskip\noindent
Consequences:
\begin{enumerate}
\item $\omega_W=\pi^*(\omega_C \otimes \mathcal{L})$ 
implies that $K_W^2=0$.
$e=R\cdot \text{deg} \mathcal{L}=$
number of singular fibers.

\item $\kappa(W) \leq  1$.

\item $\text{deg} \mathcal{L}=\chi(W)$, $e=R\cdot \text{deg} \mathcal{L}=$
number of singular fibers.

\item If $C=\mathbb{P}^1$,
then $\mathcal{L}=\mathcal{O}_{\mathbb{P}^1}(d)$ 
for some $d$.

$$
\begin{cases}
d=1\xymatrix{\ar@{<~>}[r]&}&\text{(general) RES w/section} \\
d=2\xymatrix{\ar@{<~>}[r]&}&\text{ K3s, }K_W=0\\
3\geq 3\xymatrix{\ar@{<~>}[r]&}& \kappa=1\text{ and }K_W=\alpha F
\text{for some }\alpha>0.
\end{cases}
$$
\end{enumerate}

\medskip\noindent
Invariants. $\pi:W \to C$.
Set $g=g(C)$.

\begin{center}\begin{tabular}{c|c c}
& $q$ & $p_g$\\
\hline
$W$ not a product &  $g$ & $ g+ \text{deg}\mathcal{L}-1$ \\
product& $g+1$ &  $g+\text{deg}\mathcal{L}$.
\end{tabular}\end{center}

\noindent
We can also compute the pluri-genera $P_n$ 
and deduce:
\begin{itemize}
\item $g=0$: done.

\item $g=1$:
 $$
\begin{cases}
\mathcal{L}=\mathcal{O}_C\qquad &E\times E \\
\mathcal{L}\text{ is torsion of order 2,3,4 or 6 hyp}\\
\text{deg}(\mathcal{L})\geq 1\qquad &\kappa=1
\end{cases}
$$

\item $g\geq 2$: $\kappa=1$.
\end{itemize}

\begin{proposition}
\label{proposition-minimal-surface}
Let $S$ be a minimal surface with $\kappa=1$.
Then
\begin{enumerate}
\item $K^2_S=0$ and
\item there exists $f:S \xymatrix{\ar@{->>}[r]&} C$ elliptic fibration.
\end{enumerate}
\end{proposition}

\begin{proof}
\begin{itemize}
\item $S$ contains a curve
$E= \sum n_iC_i$
($n_i \in \{ 0,1\}, \text{supp}(E)$ is connected)
with $E \overset{num}{\not \sim}0$ 
and $E^2=K_S E=0$.

In fact $K_S \overset{num}{\sim} rE$ for some $r \in \mathbb{Q}$.

\item Some multiple of $E$ moves in an elliptic pencil.
Some multiple of $E$ is the movable part of $\tilde{r}K_S$ 
for some $\tilde{r} \in \mathbb{Q}$.
\end{itemize}
\end{proof}

\section{Examples}
\label{section-examples}

\begin{itemize}
\item (The Fermat quartic.)
Consider $S \subset \mathbb{P}^3$ 
given by $x_0^4+x_1^4+x_2^4+x_3^4=0$ 
and for each $(\lambda:\mu) \in \mathbb{P}^1$,
let
$$
C_{(\lambda:\mu)}
=\begin{cases}
\lambda(x_0^2+\zeta^2x_1^2-\mu(x_2^2-\zeta^2x_3^2=0\\
\mu(x_0^2-\zeta^2x_1)+\lambda(x_2^2+\zeta^2x_3^2)=0
\end{cases}
$$
where $\zeta \in \mathcal{M}_8$
(an eight root of unity), primitive.
These curves live in $S$ and
\begin{align*}
\pi: S &\longrightarrow \mathbb{P}^1 \\
(x_0:x_1:x_2:x_3) &\longmapsto 
(x_0^2-\zeta^2x_1^2:-x_2^2-\zeta x_3^2)
\end{align*}

\noindent
is an elliptic surface.

\item Consider a smooth plane quartic $Q$
and choose a point $p \not\in Q$.
$$
\xymatrix{
S \ar[d]^{Bl_{q_1,q_2}Y \text{ where
$\pi^{-1}(p)=\{q_1,q_2\}$}}\\
Y \simeq Bl_7\mathbb{P}^2\ar[d]_\pi^{2:1}\\
\mathbb{P}^2 \supset Q.
}
$$
where $\pi^{-1}(\ell)$ has $g=1$.

[This produces a rational elliptic surface 
with a section]
\end{itemize}

\noindent
Notice that there exist elliptic surfaces with multiple fibers
(hence without a section).

\begin{example}
\label{example-multiple-fibers}
Start with three lines in the plane
intersecting pairwise, and a conic intersecting tangentially
once every line [Fano plane].
We blow up these three points and obtain
a type $2I_3$ fiber.

In general, we take pencils of sextics
with nodes at 9 points. These are in correspondence
with Halpen surfaces.
\end{example}






\end{document}

\input{preamble}

\begin{document}

\title{Research Journal}
\maketitle

\phantomsection
\label{section-phantom}

\tableofcontents

\section{Toy deformations}
\label{section-toy-deformations}

In this section I will do some practice examples deforming simple schemes like
the twisted cubic.

\subsection{Last entry (June 2)}
\label{subsection-june-2}

On Thursday I had a meeting with Sergey where he explained more details about how to do deformations using weights and how this translates to the resolution. It will take me some time to metabolize this discussion and I have to go super RG this month since exam is coming soon. After that RG course is over and I'll be much more available. But still I'll try to start the digestion of how to do deformations using weights. The \textbf{objective} is to get super good practice and acquaintance doing simple examples (the twisted cubic deforms to SR of three lines, this was the first nontrivial example of the deformation of a non-complete intersection, remember? after triangle and pentagon…).

So basically just \textit{really} understand how to get from a twisted cubic to SR of three lines would be good enough for a next meeting---probably not tomorrow.

Here's what happened:
$$
\text{*picture of the board (stacks-project has no figures…)*} 
$$
Here's the summary
\begin{enumerate}
\item The twisted cubic is
	\begin{align*}
		\nu: \mathbb{P}^1 &\longrightarrow \mathbb{P}^3 \\
		[u:v] &\longmapsto [x_0:x_1:x_2:x_3]
	\end{align*}
\item We replace
	\[x_1,x_2,x_3\longmapsto t^a x_1, t^bx_2, t^c x_3\]
\textbf{question 1:} what about \(x_0\)? I think I like more (this is also on the photo)
\[x_i \longmapsto t^{w_i}x_i\]
\item nice but we have to do the actual computations so, recall the equations and realise:
	\begin{align*}
	x_1^2-x_0x_2&\longmapsto t^{w_1}x_1-t^{w_0}x_0t^{w_2}x_2\\
	x_1x_2-x_0x_3&\longmapsto t^{w_1}x_1t^{w_2}x_2-t^{w_0}x_0t^{w_3}x_3\\
	x_2^2-x_1x_3 &\longmapsto t^{w_2}x_2-t^{w_1}x_1t^{w_3}x_3
	\end{align*}
\end{enumerate}


\section{Deforming Grünbraum sphere}
\label{section-grunbaum-sphere}

Here I try to deform Grünbraum sphere.

\section{Twistor spaces}
\label{section-twistor-spaces}

Objectives:
\begin{enumerate}
\item Define twistor space.
\item Define de Sitter space.
\item How is Segre embedding working in this story?
\item What are magnetic monopoles?
\end{enumerate}

\begin{definition}[Twistor space]
\label{definition-twistor-space}
It's the space of compatible quaternionic structures $(I,J,K)$ on a complex
manifold. In some simple case it's $\mathbb{C}P^{1}$.
\end{definition}

The twistor space of Anti-de-Sitter space is
$$
\text{Tw}(\text{AdS}^3)=\mathbb{D}\times \mathbb{D}
$$
There is a natural quadratic form of signature $(2,2)$ on
$\mathbb{R}^4=\mathbb{R}^2 \otimes \mathbb{R}^2$. 

Now projectivize:
$$
\mathbb{P}^3=\mathbb{P}(\mathbb{R}^{2,2},Q)
$$
Now consider the hypersurface in $\mathbb{R}^4$ given by $Q(u)=1$. When we
projectivize we realize that the projectivization of the hypersurface is given
by those vectors where $Q$ is positive as a form restricted to the line
generated by that vector.

Anti de Sitter is the space where $Q>0$ in $\mathbb{R}^4$. Topologically it is a
product of spheres $S^{p-1}\times \mathbb{R}^q$. We can produce a metric by
consider the function:
$$
F(u):=\text{log}Q(u)
$$
then
$$
\mu \mapsto  \text{log}\det\mu
$$
and then it's Hessian $\partial_i \partial_j \text{log}\det \mu$.

Three perspectives… The symmetries of $\mathbb{D}$ are those which preserve the
 hermitian metric $|z|^2+|w|^2$… This allows to see the twistor space
 as a homogeneous space.

\section{Bridgeland Stability}
\label{section-Bridgeland-stability}

{\bf Objectives}
\begin{enumerate}
\item What is the $(\alpha,\beta)$-plane from \cite{wall-crossing}? Knowing this
is crucial for understanding the picture.
\item What is an instanton?
\end{enumerate}

\bibliography{my}
\bibliographystyle{amsalpha}

\end{document}


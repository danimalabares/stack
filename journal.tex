\input{preamble}

\begin{document}

\title{Research Journal}
\maketitle

\phantomsection
\label{section-phantom}

\tableofcontents

\section{Grünbraum sphere}
\label{section-Grunbaum-sphere}

A long-standing quest to find a CY threefold.

\subsection{Gröbner bases}
\label{subsection-Grobner-bases}

Here's a short summary on what are Gröbner bases and related concepts from
\cite{sturmfelds-groebner}.

Let $k$ be any field and $k[\mathbf{x}]:=k[x_1,\ldots,x_n]$. The monomials in
$k[\mathbf{x}]$ are denoted $\mathbf{x}^a=x_1^{a_1}\ldots x_n^{a_n}$ and
identified with lattice points $\mathbf{a}=(a_1,\ldots,a_n)\in \mathbb{N}^n$. A
total order $\prec$ on $\mathbb{N}^n$ is a {\it term order} if the zero vector
is the unique minimal element and $\mathbf{a}\prec \mathbf{b}$ implies
$\mathbf{a}+\mathbf{c}\prec\mathbf{b}+\mathbf{c}$ for all
$\mathbf{a},\mathbf{b},\mathbf{c}\in \mathbb{N}^n$.

Given a term order $\prec$, every nonzero polynomial $f\in k[\mathbf{x}]$ has a
unique {\it initial monomial}, denoted $\text{in}_{\prec}(f)$. If $I$ is an
ideal in $k[\mathbf{x}]$ then its {\it initial ideal} is the monomial ideal
$$
\text{in}_{\prec}(I):=\left<\text{in}_{\prec}(f):f \in I\right>
$$
A finite subset $\mathcal{G} \subset I$ is a {\it Gröbner basis} for $I$ with
respect to $\prec$ if $\text{in}_{\prec}(I)$ is generated by
$\{\text{in}_{\prec}(g):g \in \mathcal{G}\}$. It is called {\it reduced} if for
any two distinct elements $g,g'\in\mathcal{G}$, no term of $g'$ is divisible by
$\text{in}_{\prec}(g)$. 

The reduced Gröbner basis is unique for an ideal and a
term order, provided one requires the coefficient of $\text{in}_{\prec}(g)$ in
$g$ to be $1$ for each $g \in \mathcal{G}$ (why?). Starting with any set of
generators of $I$, the {\it Buchberg algorithm} computes the reduced Gröbner
basis $\mathcal{G}$.

\medskip\noindent

Fix a weight vector $\omega=(\omega_1,\ldots,\omega_n)\in\mathbb{R}^n$. For any
polynomial $f=\sum c_i \mathbf{x}^{\mathbf{a}_i}$ we define the {\it initial
form} $\text{in}_\omega(f)$ to be the sum of all terms
$c_i\mathbf{x}^{\mathbf{a}_i}$ such that the inner product $\omega\mathbf{a}_i$
is minimal. For an ideal $I$ we define the {\it initial ideal} to be the ideal
generated by all initial forms
$$
\text{in}_\omega(I):=\left<\text{in}_\omega(f):f\in I\right>
$$
This initial ideal need not be a monomial ideal (like the initial ideal for a
term order). But is will be whenever $\omega$ is chosen "sufciciently generic".
If in addition $\omega$ is non-negative, then, as we shall se,
$\text{in}_\omega(I)$ is an initial monomial ideal in the earlier sense.

Let $\omega\geq 0$ and let $\prec$ be an arbitrary term order. We define a new
term order $\prec_{\omega}$ as follows: for $\mathbf{a},\mathbf{b}\in
\mathbb{N}^n$, we set
$$
\mathbf{a}\prec_\omega\mathbf{b} :\iff
\omega\cdot\mathbf{a}<\omega\cdot\mathbf{b}
\text{ or }(\omega\cdot\mathbf{a}=\omega\cdot\mathbf{b}\text{ and }
\mathbf{a}\prec \mathbf{b}).
$$
\medskip\noindent

A {\it polyhedron} is a finite intersection of closed half-spaces in
$\mathbb{R}^n$. Thus a polyhedron $P$ can be written as
$P=\{\mathbf{x}\in\mathbb{R}^n:A\mathbf{x}\leq \mathbf{b}\}$ where $A$ is a 
matrix with $n$ columns and as many rows as there are hyperplanes to intersect 
(to visualize it think of the case of equality, which is an affine
plane, then the inequality gives a half-space. Do this for as many
hyperplanes/equations required). If $\mathbf{b}=0$, then there exist vectors
$\mathbf{u}_1,\ldots,\mathbf{u}_m\in \mathbb{R}^n$ such that
\begin{equation}
\label{equation-cone}
P=\text{pos}(\mathbf{u}_1,\ldots,\mathbf{u}_m):=
\{\lambda_1\mathbf{u}_1+\ldots+\lambda_m \mathbf{u}_m
:\lambda_i \in \mathbb{R}_+\}
\end{equation}
A polyhedron of as in Eq. \ref{equation-cone} is called a {\it cone}.
(I think this is the convex hull of $\mathbf{u}_i$ but extending by any positive
constant.)

The faces of a polyhedron in $\mathbb{R}^n$ can be defined as follows. Let
$\omega\in \mathbb{R}^n$, viewed as a linear functional. Define
 $$
\text{face}_{\omega}(P):=\{u \in P:\omega\cdot u\geq \omega v \;\forall v\in P\}
$$
Every subset $F$ of $P$ which has this form, that is, which maximizes some
linear functional $\omega$, is called a {\it face} of $P$.

If $P\subset \mathbb{R}^n$ is a polyhedron and $F$ a face of $P$, then the 
{\it normal cone}  of $P$ at $F$ is
$$
\mathcal{N}_P(F):=\{\omega \in \mathbb{R}^n:\text{face}_\omega(P)=F\}
$$
That is, the set of functionals $\omega$ for which $F$ is a face
$\text{face}_\omega(P)$.

A {\it polyhedral complex} $\Delta$ is a finite collection of polyhedra in
$\mathbb{R}^n$ such that
\begin{enumerate}
\item if $P \in \Delta$ and $F$ is a face of $P$, then $F \in \Delta$.
\item if $P_1,P_2\in\Delta$, then $P_1\cap P_2$ is a face of $P_1$ and of $P_2$.
\end{enumerate}
The support of a complex $\Delta$ is the union of all the elements of $\Delta$.
A complex $\Delta$ which consists of cones is called a {\it fan}. The collection
of normal cones $\mathcal{N}_P(F)$ as $F$ ranges over the faces of $P$ is a fan
called the {\it normal fan} of $P$.

The {\it Newton polytope} is the convex hull
of the indices $\alpha_i \in \mathbb{N}^n$ of the polynomial 
$f=\sum c_ix^{\alpha_i}$, where 
$x^{\alpha_i}:=x_1^{\alpha_i^1}\ldots x_n^{\alpha_i^n}$.

\begin{proposition}
\label{proposition-equivalence-classes-of-weight-vectors}
\begin{reference}
\cite[Proposition 2.3]{sturmfelds-groebner}
\end{reference}
Each equivalence class of weight vectors is a relatively open convex polyhedral
cone. This is implied by the fact that
$$
C[\omega]=\{\omega'\in \mathbb{R}^n:
\text{in}_{\omega'}(g)=\text{in}_\omega(g),\qquad \forall g \in \mathcal{G}\}
$$
for a reduced Gröbner basis $\mathcal{G}$ of $I$ with respect to $<_\omega$.
\end{proposition}

The union of these cones is a fan called the {\it Gröbner fan}.

The state polytope is a polytope whose vertices correspond to
initial ideals with respect to different monomial orderings/weight vectors.

\begin{theorem}
\label{theorem-groebner-fan-and-state-polytope}
\begin{reference}
\cite[Theorem 2.2]{sturmfelds-groebner}
\end{reference}
Let $I$ be a homogeneous ideal in $k[\mathbf{x}]$. There exists a polytope
$\text{State}(I) \subset \mathbb{R}^n$ whose normal fan
$\mathcal{N}(\text{State}(I))$ coincides with the Gröbner fan $\text{GF}(I)$.
\end{theorem}

\medskip\noindent
{\bf Summary.} Let $I$ be an ideal.
\begin{itemize}
\item The Gröbner fan is a union of Gröbner cones.
\item A Gröbner cone is an equivalence class of weight vectors, 
where two weight vectors are identified if they give the same initial form
$\text{in}_\omega(g)$ for all $g$ in a Gröbner basis $\mathcal{G}$ of $I.$
\item An initial form $\text{in}_\omega(g)$ for a polynomial 
$g = \sum c_i\mathbf{x}^{\mathbf{a}_i}$ is the sum of monomials 
$c_i\mathbf{x}^{\mathbf{a}_i}$ such that $\omega \mathbf{a}_i$ is minimal.
\item The initial ideal $\text{in}_\omega(I)$ is the ideal generated by the
initial forms $\text{in}_\omega(f)$ for all $f \in I$.
\end{itemize}

{\bf July 19.} The question is: how is this related to deformations? I have
successfully ran \texttt{gfan\_groebnerfan < twisted-cubic-gfan} on the ideal of
the twisted cubic, but what are the deformations? Why would the basis of
$\bullet-\bullet-\bullet-\bullet$ be in the Gröbner fan of twisted cubic? I have
come to highly suspect that the weight vector $(1,2,2,0)=(a,b,c,d)$ associated
to the deformations I computed below will give me a cone, or an initial ideal,
that will correspond to the ideal of $\bullet-\bullet-\bullet-\bullet$. Maybe
I'm almost there.

{\bf August 18.} Let me think about this again. 
The Gröbner fan is just a union of cones,
each of which is an equivalence class of ``weight vectors'' that yield the same
initial ideal for every element of a given Gröbner basis of the ideal.
If the ideal of $\bullet-\bullet-\bullet-\bullet$ is to be found in 
the Gröbner cone, it will be as one such weight vector.
And a weight vector, in turn, gives me an ideal, the initial ideal given by the
weight vector.

{\bf August 20.} There are two things: the ``weight vector'' I found by
deforming by hand the ideal of the twisted cubic to the ideal of 
$\bullet-\bullet-\bullet-\bullet$, and the ``weight vectors'' encoded in the
Gröbner fan. So it would be great if these things coincided, that is, 
that every weight vector in the Gröbner fan could be interpreted as a
deformation weight vector of the kind I used: it encodes the exponents of the
deformation variable $t$.

\subsection{Twisted cubic example}
\label{subsection-twisted-cubic-example}

As en exercise I will do a deformation of the twisted cubic to the
Stanley-Raisner scheme of $\bullet-\bullet-\bullet-\bullet$.

\subsubsection{Twisted cubic basics}
\label{subsubsection-twisted-cubic-basics}

\begin{definition}
\label{definition-twisted-cubic}
The {\it twisted cubic} is the image of the Veronesse map of degree 3,
\begin{align*}
\nu: \mathbb{P}^1 &\longrightarrow \mathbb{P}^3 \\
[u:v] &\longmapsto [u^3:u^2v:uv^2:v^3]=[x_0:x_1:x_2:x_3]
\end{align*}
\end{definition}

\begin{lemma}
\label{lemma-twisted-cubic-is-variety}
\begin{reference}
\cite[Proposition 6.1]{sys}
\end{reference}
The twisted cubic is an algebraic variety whose ideal is generated by the minors
of the matrix
$$
M_{3,1}=\begin{pmatrix}
x_0 & x_1 & x_2\\
x_1 & x_2 & x_3
\end{pmatrix}
$$
\end{lemma}

\begin{proof}
See \cite[Proposition 6.1]{sys}.
\end{proof}

Thus, the ideal of the twisted cubic is generated by
\begin{equation}
\label{equation-twisted-cubic1}
x_1^2-x_0x_2,\quad x_1x_2-x_0x_3,\quad x_2^2-x_1x_3
\end{equation}

The minimal resolution of the twisted cubic is
\begin{equation}
\label{equation-resolution-of-twisted-cubic}
\xymatrix{
0\ar[r]&\mathcal{O}(-3)^{\oplus 2}\ar[r]^{d_2}&
\mathcal{O}(-2)^{\oplus 3}\ar[r]^{d_1}&
\mathcal{O}_{\mathbb{P}^3}\ar[r]&\mathcal{O}_X\ar[r]&0
}
\end{equation}
where
\begin{equation*}
\label{equation-matrices-twisted-cubic}
d_1=\begin{pmatrix}
x_1^2-x_0x_2&x_1x_2-x_0x_3&x_2^2-x_1x_3
\end{pmatrix},\qquad 
d_2=\begin{pmatrix}
-x_2&x_3\\
x_1&-x_2\\
-x_0&x_1
\end{pmatrix}.
\end{equation*}
It may be checked that $d_1d_2=0$.

\medskip\noindent

\subsubsection{Stanley-Reisner scheme of four points and three lines}
\label{subsubsection-Stanley-Reisner-scheme-of-four-points-and-three-lines}

The twisted cubic should deform to the Stanley-Reisner scheme of 
$\bullet-\bullet-\bullet-\bullet$, which we denote by $X$.
 The ideal of this scheme is 
$(x_0x_2,x_0x_3,x_1x_3)$ inside $S:=k[x_0,x_1,x_2,x_3]$.

It's minimal resolution is
\begin{equation}
\label{equation-resolution-of-three-lines}
\xymatrix{
0\ar[r]&\mathcal{O}(-3)^{\oplus 2}\ar[r]^{d_2}&
\mathcal{O}(-2)^{\oplus 3}\ar[r]^{d_1}&
\mathcal{O}_{\mathbb{P}^3}\ar[r]&\mathcal{O}_X\ar[r]&0
}
\end{equation}
where
\begin{equation*}
\label{equation-matrices-three-lines}
d_1=\begin{pmatrix}
x_0x_2&x_0x_3&x_1x_3
\end{pmatrix},\qquad 
d_2=\begin{pmatrix}
0&-x_3\\
-x_1&x_2\\
x_0&0
\end{pmatrix}.
\end{equation*}

\subsubsection{Deformation of the twisted cubic}
\label{subsubsection-deformation-of-twisted-cubic}

To do the deformation of the twisted cubic we replace
$$
x_0\rightsquigarrow t^ax_0,x_1\rightsquigarrow t^bx_1,
x_2\rightsquigarrow t^cx_2,x_3\rightsquigarrow t^cx_3
$$
So that the equations of the twisted cubic \ref{equation-twisted-cubic1} become
\begin{align*}
\label{equation-twisted-cubic2}
\begin{aligned}
t^{2b}x_1^2&-t^{a+c}x_0x_2\\
t^{b+c}x_1x_2&-t^{a+d}x_0x_3\\
t^{2c}x_2^2&-t^{b+d}x_1x_3
\end{aligned}
\end{align*}
To arrive at the equations of $X$ we must have
\begin{equation}
\label{equation-inequalities}
2b>a+c,\qquad b+c>a+d,\qquad 2c>b+d.
\end{equation}
A valid solution is $a=1,b=2,c=2,d=0$.

We obtain deformation matrices

\begin{equation}
\label{equation-deformation-matrices-twisted-cubic}
\begin{aligned}
d_1'&=\begin{pmatrix}
t^4x_1^2-t^3x_0x_2 &
t^4x_1x_2-tx_0x_3 &
t^4x_2^2-t^2x_1x_3
\end{pmatrix}\\
d_2'&=\begin{pmatrix}
-tx_2 & x_3\\
tx_1 & -t^2x_2\\
-x_0 & t^2x_1
\end{pmatrix}
\end{aligned}
\end{equation}

From \cite[p. 10]{HarrMorr}: ``The {\it twisted cubics} --- rational normal
curves in $\mathbb{P}^3$ that have Hilbert polynomial $P_X(m)=3m+1$''

\section{Twistor spaces}
\label{section-twistor-spaces}

Objectives:
\begin{enumerate}
\item Define twistor space.
\item Define de Sitter space.
\item How is Segre embedding working in this story?
\item What are magnetic monopoles?
\end{enumerate}

\begin{definition}[Twistor space]
\label{definition-twistor-space}
It's the space of compatible quaternionic structures $(I,J,K)$ on a complex
manifold. In some simple case it's $\mathbb{C}P^{1}$.
\end{definition}

The twistor space of Anti-de-Sitter space is
$$
\text{Tw}(\text{AdS}^3)=\mathbb{D}\times \mathbb{D}
$$
There is a natural quadratic form of signature $(2,2)$ on
$\mathbb{R}^4=\mathbb{R}^2 \otimes \mathbb{R}^2$. 

Now projectivize:
$$
\mathbb{P}^3=\mathbb{P}(\mathbb{R}^{2,2},Q)
$$
Now consider the hypersurface in $\mathbb{R}^4$ given by $Q(u)=1$. When we
projectivize we realize that the projectivization of the hypersurface is given
by those vectors where $Q$ is positive as a form restricted to the line
generated by that vector.

Anti de Sitter is the space where $Q>0$ in $\mathbb{R}^4$. Topologically it is a
product of spheres $S^{p-1}\times \mathbb{R}^q$. We can produce a metric by
consider the function:
$$
F(u):=\text{log}Q(u)
$$
then
$$
\mu \mapsto  \text{log}\det\mu
$$
and then it's Hessian $\partial_i \partial_j \text{log}\det \mu$.

Three perspectives… The symmetries of $\mathbb{D}$ are those which preserve the
hermitian metric $|z|^2+|w|^2$… This allows to see the twistor space 
as a homogeneous space.

\section{Chiral}
\label{section-chiral}

This project follows my master's work started with Roli and Isabel in Mexico.

\subsection{Review of master's thesis}
\label{subsection-masters-thesis}

My master's work was mostly based on \cite{petcox}. This paper contains a list
of so-called quiral polyhedra on $\mathbb{R}^4=\mathbb{E}^4$. My master's thesis
is basically using one of them to construct a 4-polytope (in the same ``skeletal
polytope'' sense) that remains quiral; owing the quirality of the 4-polytope to
the quirality of the 3-cell.

\subsection{Riemann surface of genus 7}
\label{subsection-Riemann-surface-of-genus-7}

The Euler characteristic of the quiral polyhedron I used as ``3-cell'' or
``facet'' is $48-72+12=-12$. We are interested in studying the associated
Riemann surface.

We may check that such a Riemann surface exists by
\cite[Construction 1.3.20]{lando}. The two conditions that a graph must satisfy
are that the graph be connected, and 
\begin{equation}
\label{equation-sigma-alpha-phi}
\sigma\alpha\varphi=\text{id}
\end{equation}
where
$\sigma$ is the rotation about the vertex (sommet),
$\alpha$ is the rotation about the edge (arête),
and $\varphi$ is the rotation about the face. 
Note that these are not the three
rotations in a base flag: once we apply a symmetry, the next one should be
conjugated by the first.

When considering the conjugation scheme, the Eq \ref{equation-sigma-alpha-phi}
becomes $\alpha\sigma\varphi=\text{id}$. This is because we first apply
$\sigma$, but to apply $\alpha$ correctly we must conjugate by $\sigma$, that
is, multiply on the left of $\alpha$ by $\sigma^{-1}$ and on its right by
$\sigma$. Visual arguments that are sufficient for discussion right now show
that there is no change in face with these operations, so that no conjugation on
$\varphi$ is necessary.

In terms of $R_0,R_1$ and $R_2$, the generating reflections of the symmetry
group of a regular polyhedron, we have $\sigma_0=R_2R_1$, $\alpha_0=R_0R_2$ and
$\varphi_0=R_1R_0$. Here the $0$ denotes the fact that these are the vertex,
edge and face transpositions about the base flag. Notice that the transposition
about the edge is not simply $R_0$, it is a 180-degree rotation; the three
symmetries are orientation-preserving. Eq. \ref{equation-sigma-alpha-phi}
becomes $(R_0R_2)(R_2R_1)(R_1R_0)=\text{id}$.

This barely shows that indeed
there exists a Riemann surface underlying every regular convex polyhedron.

For our chiral 3-cell we obtain… I failed to prove Eq.
\ref{equation-sigma-alpha-phi} in this case. I made other attempts using
permutation groups in GAP, but still no success. Are there other ways of
constructing the Riemann surface? What kind of result do I expect from this
exercise?

{\bf July 25.} I conclude that it is worth the detour to correctly define a
constellation from the combinatorial data I have from my master's thesis as a
way to find my Riemann surface. Perhaps after this it wouldn't be too hard to
find the corresponding Belyi function and study it.

\bigskip\noindent

The associated genus $7$ Riemann surface may admit
a Belyi function inducing a chiral dessin in the sense that it would not be
definable over the reals (as suggested by Sergey in the annotations of my
tesina). This means that the Riemann surface would not be biholomorphic to its
``conjugate''.

Based on some data I computed back in the day, the Euler
characteristic of the 3-cell of Roli's cube (the first and probably simplest
example) is $16-24+6=2$, so $\mathbb{P}^1$.

I would be mostly interested in knowing whether the Riemann surface is minimal
in $S^3$, just because this is a topic of interest for the group. It appears
this is possible: in \cite[Theorem 2.1]{brendle} we find Lawson's result that
there are minimal surfaces in $S^3$ for every genus.

But how are these surfaces embedded in $S^3$? What method we would use to know
confirm whether they are minimal or not? Is this surface really or the
3-manifold obtained by gluing the facets really interesting?

\subsection{Quaternions}
\label{subsection-quaternions}

\begin{lemma}
\label{lemma-generators-SO3}
$\text{SO}(3)$ is generated by simple rotations, i.e. a rotation that fixes a
vector.
\end{lemma}

\begin{lemma}
\label{lemma-double-cover-of-SO3}
Every element of $\text{SO}(3)$ has the form $x \mapsto q^{-1}xq$ for some
quaternion $q$, and is a simple rotation. The map that takes $q$ to the map
$[q]:x \mapsto q^{-1}xq$ is a 2-to-1 homomorphism from the group of unit
quaternions to $\text{SO}(3)$.
\end{lemma}

Apparently the set of unit quaternions is called {\it spin group}
$\text{Spin}(3)$, and this is surely just $\text{SU}(2)$.

The fact that the covering is 2-to-1 is because opposite quaternions yield the
same rotation.

\begin{definition}
\label{definition-binary-icosahedral-group}
The {\it binary icosahedral group} is the preimage of the icosahedral group
$I\cong A_5$ by
the covering described above.
\end{definition}

As an aside, the manifold $\text{SO}(3)/I$ is the infamous Poincaré homology
sphere, also known as Mukai-Umemura threefold in Algebraic Geometry (double
check this).

\section{Bridgeland Stability}
\label{section-Bridgeland-stability}

{\bf Objectives}
\begin{enumerate}
\item What is the $(\alpha,\beta)$-plane from \cite{wall-crossing}? Knowing this
is crucial for understanding the picture.
\item What is an instanton?
\end{enumerate}

\section{Quantum}
\label{section-quantum}

\subsection{Quantum surfaces}
\label{subsection-quantum-surfaces}

Consider $\varphi:\Sigma\hookrightarrow \mathbb{R}^n$ given by coordinate
functions $x_1,\ldots,x_n$. Define the {\it classical Schild action} by
$$
S^{\text{cl}}_\alpha(x)=\int_\Sigma
\omega\left[\sum_{i<j}\left\{x_i,x_j\right\}^2 \right]^\alpha
$$
Now let $\mathcal{H}$ be a finite-dimensional Hilbert space and $X_1,\ldots,X_n$ are
observables, i.e., self-adjoint operators. Define the {\it quantum Schild
action} by
$$
S_\alpha^q(X)=\text{tr}_{\mathcal{H}}\left(-\sum_{i<j}[X_i,X_j]^2\right)^\alpha
$$
Notice that if the $X_i$ commute then $S_\alpha^q$ vanishes. That is, just as
the classical Schild functional measures areas, the quantum Schild functional
is a measure of commutativity.

\medskip\noindent

The following reference should shed light on how to travel from Riemann surfaces
and graphs (markings) to QFT: Chris Negron, Agustina Czenky.

\subsection{Quantum cohomology: Galkin seminar (June 20 2025)}
\label{subsection-Galkin-June-20}

{\bf Question.} Is it true that quantum cohomology is just simplicial cohomology
classes with coefficients in $\mathbb{C}$ but with a different product?

{\bf A.} (Adapted from Sergey's.) We extend the coefficients from $\mathbb{C}$
to $\mathbb{C}[q]$ where $q$ is ``a formal variable''. So as a
$\mathbb{C}[q]$-module ($HQ^*(X)$) this is the ``usual'' cohomology, but the
product is not the cup product (however it coincides with cup-product module the
ideal $(q)$).

\begin{quotation}
``Then geometrically you take Spec of such algebra and it
fibers over $\text{Spec} \mathbb{C}[q] = A^1_\mathbb{C}$ with coordinate q
(alias, any element of a commutative algebra can be consideredf as a function
from its Spectrum to the affine line), and the you can take a fiber over any
point a, which is the same as substituting q to a value (complex number) a.
Functions on this fiber is the quotient algebra QH(X,a) = QH(X,C[q])/(q-a), the
structure constants are obtained by sybstituting q by a."
\end{quotation}

{\bf Question.} What is that fibered construction good for?


{\bf Theme.} Under what conditions $QH^{\text{ev}}_{\text{big}}(X)$ is
generically semisimple?

{\bf Results.} Odd Betti numbers and odd Hodge numbers will vanish. See
Hertling, Manin, Teleman 2009.

{\bf Beautiful question.} When is the cohomology $H^*(Y)$ of Hodge-Tate type,
where $Y$ is a generic hyperplane section of the complex Grassmanian
$\text{Gr}(n,k)$?

{\bf Reference + remark.} Debarre-Voisin computed the Hodge diamond of $Y
\subset \text{Gr}(3,10)$ when studying Hyper-Kähler fourfolds. Since this is not
Hodge-Tate type, the small quantum-cohomology is not generically semisimple.

\begin{theorem}[Galkin-Leung-L.-Xiong]
\label{theorem-GLLX}
For $\text{Gr}(k,n)$, the cohomology $H^{*}(Y)$ is of Hodge-Type if and only if
\begin{itemize}
\item $Y \subset \text{Gr}(1,n)\cong \text{Gr}(n-1,n)$ where $n\geq 1$.
\item $Y \subset \text{Gr}(2,n) \cong \text{Gr}(n-2,n)$ where $n \geq 4$.
\item $Y \subset\text{Gr}(3,n) \cong\text{Gr}(n-3,n)$ where $n=6,7,8$.
\end{itemize}
\end{theorem}

Theorem \ref{theorem-GLLX} can be viewed as an {\bf ADE classification}. We call
the above $Y$ a {\it Hodge-Tate hyperplane section}.

{\bf Reference.} Try it online
\url{https://cubicbear.github.io/PluckerHodge.html} 

For Fano varieties, we have the Fano index \ref{definition-Fano-index}. In this
case, $QH^*(X)|_{q=1}$ is naturally a $\mathbb{Z}_r$-graded algebra
$\mathcal{A}$.

\section{Miscellaneous}
\label{section-miscellaneous}

Recent citation of a \href{https://www.youtube.com/watch?v=luthVy-H9OI}{seminar
talk} by my dad on Navier-Stokes equation on the paper
\href{https://arxiv.org/abs/2507.02923}{https://arxiv.org/abs/2507.02923}.
 
\bibliography{my}
\bibliographystyle{amsalpha}

\end{document}

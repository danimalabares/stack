\input{preamble}

\begin{document}

\title{Birational maps from the commutative algebra viewpoint}
\maketitle

\noindent
Minicourse by Hamid Hassanzadeh,
OPEGA 2026, UFPE.

\hfill Notes at
\href{http://github.com/danimalabares/stack}{
github.com/danimalabares/stack}


\medskip\noindent
O estudo da equivalência de variedades algébricas a partir da biracionalidade é um dos temas centrais da Geometria Algébrica. Nessa direção, a classificação de variedades pode ser vista como parte da chamada geometria birracional. De modo análogo, podemos concentrar-nos na estrutura dos mapas racionais que definem biracionalidades entre variedades — este será o ponto de partida do minicurso. Apresentarei aplicações de métodos de Álgebra Comutativa, como o uso de syzygies no estudo dos ideais de base de mapas racionais, e discutirei critérios algébricos que fornecem ferramentas computacionais para verificar a biracionalidade de tais mapas. Introduziremos também alguns métodos no Macaulay2 para testar biracionalidade. Além disso, veremos como diferentes teorias de multiplicidades — como mixed multiplicity e j-multiplicity — desempenham um papel importante na abordagem algébrica dos mapas racionais.
\phantomsection
\label{section-phantom}

\section{Plan}
\label{section-plan}

\begin{itemize}
\item A. Simis, A. Doria, - 2012 (Adv. Math) Criterio de Biracionalidade
\item Simis, - 2012: Plane Cremona.
\item Simis, -, 2017. Bounds for degrees of Birational maps with CM graph.
\item Simis, Chardin 2021:
degree of rational map versus syzygy.
\item Mostafazadeh, 2024: loci de $\text{Bir}(X)$.
\item \texttt{RationalMaps.m2}
\end{itemize}

\section{Introdução}
\label{section-intro}

\noindent
Let $R$ be a ring.
We are interested in $Mod_R$ and $R$-algebras categories.
The first one is abelian, while the second one isn't.
There are other diferences.

We are interested in discussing flat $R$-algebras
and flat $R$-modules.

\begin{theorem}[Grothendieck Generic freeness theorem]
\label{theorem-grothendieck}
Let $R$ be noetherian domain, $\varphi:R \to S$,
$S$ a finitely generated $R$-algebra.
Then there exists $f \in R$
such that $\varphi:R_f \to S_{\varphi(f)}$,
$S_{\varphi(f)}$ is a free $R_f$-module.
$S_{\varphi(f)}$ is a flat $R_f$-module.
\end{theorem}

\begin{theorem}[Dimension of fibers theorem (Matsumura 15.1)]
\label{theorem-dimension-of-fibers}
$\varphi:R \to S$, $R$ Noetherian,
$S$ Noetherian, $Q \in \Spec(S)$,
$p=\varphi^{-1}(Q) \in \Spec(R)$,
\begin{enumerate}
\item $ht Q \leq  ht p + \dim \frac{S_Q}{pS_Q}$ 
[Shafarevich section 3].
\item If $\varphi$ is flat then
$\text{ht}Q=\text{ht}p+\dim \frac{S_Q}{pS_Q}$.
\end{enumerate}

\noindent
If $Q \not\ni \varphi(f) \implies $
$S_Q$ is a free $R_p$-module (by Grothendieck's theorem).
$\dim S_Q= \dim R_p+\dim \frac{S_Q}{p S_Q}$.
\end{theorem}

\noindent
Topologia de Zariski.
$$
\Spec(S)=\{Q:Q \text{ primo}\}
$$
$$
\text{fechado }=V(I)= \{ Q: Q \supseteq I\}.
$$

$S$ is a domain, $\Spec(S)$ is an irreducible space,
$\implies $ open nonempty sets are dense.


\medskip\noindent
$k$ field. An algebraic set is  $V(I) = Z(I)$ for an ideal 
$I \subset k[x_1,…,x_n]$. $I$ is prime if and only if
$X:=V(I)$ is irreducible.
Recall that a polynomial map is given
by polynomial functions in the entries.
Any such map yields in the category of algebras a map
\begin{align*}
k[Y]=A[Y]=\mathcal{O}_Y=\frac{k[y_1,…,y_m]}{J}
&\xrightarrow{\varphi^*}
\frac{k[x_1,…,x_n]}{I}\\
y_i&\mapsto \overline{f_i(x)}
\end{align*}

\noindent
where $Y:=V(J)$.
$\varphi^*(g)=0$ $\forall  g \in J$,
$g \in J \iff g(f_1,…,f_m)\in I$
where $f_i$ are the coordinate polynomial functions of $\varphi$.

$\varphi$ is an isomorphism if it has a right and left inverse morphism.
$\varphi:X \to X$ is an automorphism
if it is an isomorphism.

\medskip\noindent
Question: determine the automorphism group $\text{Aut}(X)$.

If $X=\mathbb{A}^1$, a morphism
\begin{align*}
\varphi: k[x] &\longrightarrow k[x] \\
x &\longmapsto f(x)
\end{align*}

\noindent
has an inverse if and only if there exists $f(x)$ such that $f(g(x))=x$.
[This forces $f$ to have degree 1 since
composition $f \circ g$ must have degree
$\text{deg} f \cdot \text{deg} g = \text{deg} x=1$.]
So $f(x)=ax+b$, $a\neq 0$.
Thus, we can parametrize the automorphism by two numbers
$a,b$ with $a \neq 0$. This is a quasi-affine variety.

For $X=\mathbb{A}^2$ 
the situation is a little more involved.
de Jonequeres, Triangular group.
Hierzch Jung 1942, chre.
Wout Van der Kulk, 1953, chr arbitrary.
$$
\text{Aut}(\mathbb{A}^2)
=\left<\text{linear, triangular}\right>
$$

For $\text{Aut}(\mathbb{A}^3)$,
the question if whether $\text{Aut}(\mathbb{A}^3)=
\left<\text{linear, triangular}\right>$.
This is false in general

\begin{definition}
\label{definition-tame}
An auotomorphism $\sigma \in \text{Aut}(\mathbb{A}^3)$ 
is called {\it tame} if $\sigma \in \left<\text{linear, triangular}\right>$ 
and {\it wild} otherwise.
\end{definition}

\noindent
Nagata's conjecture is that
a $\sigma:k[x,y,z] \to k[x,y,z]$,
$(x,y,z) \mapsto (x+(x^2-yz)z,y+2(x^2yz)x+(x^2-yz)^2z,z)$ 
is wild.

Shiguero Kuroda 2014. Introduce a monomial order
$\leq $ in $k[x,y,z]$.
[The following is not a criterion;
this just says "if this happens, then the morphism is wild",
but not an iff.]
Let 
\begin{align*}
\sigma: k[x,y,z] &\longrightarrow k[x,y,z] \\
\leavevmode &\longmapsto (f_1,f_2,f_3)
\end{align*}

\noindent
$\sigma$ is wild if
\begin{enumerate}
\item $\text{int}(f_1),\text{int}(f_2), \text{int}(f_3)$
are linearly dependent over $\mathbb{Z}$,
pairwise linear over $\mathbb{Z}$.
\item $\text{int}(f_{i_1}) \neq  p \text{int}(f_{i_2})+q(\text{int}(f_{i_3})$ 
for $p,q \in \mathbb{Z}_{\geq 0}$.
\end{enumerate}

\medskip\noindent
Jacobian conjecture:
a polynomial map $\varphi:\mathbb{C}^n \to \mathbb{C}^n$
[with coordinate functions] $(f_1(\mathbf{x}),…,f_n(\mathbf{x}))$
is an automorphism if and only if
$\det \text{Jacob}(\varphi)=\text{constant}\neq 0$.

The forward implication is clear: 
if  $\varphi \psi=\text{id}$
then $\text{Jac}(\varphi)(\psi)\text{Jac}(\psi)=\text{Jac}(\text{id})=1$.

[Bass-Conell-Wright, 1982]
Let $\varphi:\mathbb{C}^n \to \mathbb{C}^n$ 
a polynomial map with
$\det \text{Jac}\varphi=$ constant. Then
$\varphi$ is invertible (its inverse is a polynomial map)
[note that by the holomorphic inverse function theorem
we already know that there exists a local inverse
that is a series, but here we see it's actually polynomial]
if and only if $\varphi$ is injective
if and only if 
$\varphi$ is proper ($\varphi^{-1}(\text{compact})=\text{compact}$).

\begin{theorem}
\label{theorem-jacobian-conjecture}
If the Jacobian conjecture is valid for all  $n$ and
for any polynomial of degree $\leq 3$,
then it's valid.
\end{theorem}

\begin{theorem}
\label{theorem-jacobian-conjecture2}
If the Jacobian conjecture is valid over $\mathbb{C}$
then it is valid over any domain.
\end{theorem}








\end{document}

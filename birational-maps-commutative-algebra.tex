\input{preamble}

\begin{document}

\title{Birational maps from the commutative algebra viewpoint}
\maketitle

\noindent
Minicourse by Hamid Hassanzadeh,
OPEGA 2026, UFPE.

\hfill Notes at
\href{http://github.com/danimalabares/stack}{
github.com/danimalabares/stack}


\medskip\noindent
O estudo da equivalência de variedades algébricas a partir da biracionalidade é um dos temas centrais da Geometria Algébrica. Nessa direção, a classificação de variedades pode ser vista como parte da chamada geometria birracional. De modo análogo, podemos concentrar-nos na estrutura dos mapas racionais que definem biracionalidades entre variedades — este será o ponto de partida do minicurso. Apresentarei aplicações de métodos de Álgebra Comutativa, como o uso de syzygies no estudo dos ideais de base de mapas racionais, e discutirei critérios algébricos que fornecem ferramentas computacionais para verificar a biracionalidade de tais mapas. Introduziremos também alguns métodos no Macaulay2 para testar biracionalidade. Além disso, veremos como diferentes teorias de multiplicidades — como mixed multiplicity e j-multiplicity — desempenham um papel importante na abordagem algébrica dos mapas racionais.

\tableofcontents

\section{Plan}
\label{section-plan}

\begin{itemize}
\item A. Simis, A. Doria, - 2012 (Adv. Math) Criterio de Biracionalidade
\item Simis, - 2012: Plane Cremona.
\item Simis, -, 2017. Bounds for degrees of Birational maps with CM graph.
\item Simis, Chardin 2021:
degree of rational map versus syzygy.
\item Mostafazadeh, 2024: loci de $\text{Bir}(X)$.
\item \texttt{RationalMaps.m2}
\end{itemize}

\section{Introdução}
\label{section-intro}

\noindent
Let $R$ be a ring.
We are interested in $Mod_R$ and $R$-algebras categories.
The first one is abelian, while the second one isn't.
There are other diferences.

We are interested in discussing flat $R$-algebras
and flat $R$-modules.

\begin{theorem}[Grothendieck Generic freeness theorem]
\label{theorem-grothendieck}
Let $R$ be noetherian domain, $\varphi:R \to S$,
$S$ a finitely generated $R$-algebra.
Then there exists $f \in R$
such that $\varphi:R_f \to S_{\varphi(f)}$,
$S_{\varphi(f)}$ is a free $R_f$-module.
$S_{\varphi(f)}$ is a flat $R_f$-module.
\end{theorem}

\begin{theorem}[Dimension of fibers theorem (Matsumura 15.1)]
\label{theorem-dimension-of-fibers}
$\varphi:R \to S$, $R$ Noetherian,
$S$ Noetherian, $Q \in \Spec(S)$,
$p=\varphi^{-1}(Q) \in \Spec(R)$,
\begin{enumerate}
\item $ht Q \leq  ht p + \dim \frac{S_Q}{pS_Q}$ 
[Shafarevich section 3].
\item If $\varphi$ is flat then
$\text{ht}Q=\text{ht}p+\dim \frac{S_Q}{pS_Q}$.
\end{enumerate}

\noindent
If $Q \not\ni \varphi(f) \implies $
$S_Q$ is a free $R_p$-module (by Grothendieck's theorem).
$\dim S_Q= \dim R_p+\dim \frac{S_Q}{p S_Q}$.
\end{theorem}

\noindent
Topologia de Zariski.
$$
\Spec(S)=\{Q:Q \text{ primo}\}
$$
$$
\text{fechado }=V(I)= \{ Q: Q \supseteq I\}.
$$

$S$ is a domain, $\Spec(S)$ is an irreducible space,
$\implies $ open nonempty sets are dense.


\medskip\noindent
$k$ field. An algebraic set is  $V(I) = Z(I)$ for an ideal 
$I \subset k[x_1,…,x_n]$. $I$ is prime if and only if
$X:=V(I)$ is irreducible.
Recall that a polynomial map is given
by polynomial functions in the entries.
Any such map yields in the category of algebras a map
\begin{align*}
k[Y]=A[Y]=\mathcal{O}_Y=\frac{k[y_1,…,y_m]}{J}
&\xrightarrow{\varphi^*}
\frac{k[x_1,…,x_n]}{I}\\
y_i&\mapsto \overline{f_i(x)}
\end{align*}

\noindent
where $Y:=V(J)$.
$\varphi^*(g)=0$ $\forall  g \in J$,
$g \in J \iff g(f_1,…,f_m)\in I$
where $f_i$ are the coordinate polynomial functions of $\varphi$.

$\varphi$ is an isomorphism if it has a right and left inverse morphism.
$\varphi:X \to X$ is an automorphism
if it is an isomorphism.

\medskip\noindent
Question: determine the automorphism group $\text{Aut}(X)$.

If $X=\mathbb{A}^1$, a morphism
\begin{align*}
\varphi: k[x] &\longrightarrow k[x] \\
x &\longmapsto f(x)
\end{align*}

\noindent
has an inverse if and only if there exists $f(x)$ such that $f(g(x))=x$.
[This forces $f$ to have degree 1 since
composition $f \circ g$ must have degree
$\text{deg} f \cdot \text{deg} g = \text{deg} x=1$.]
So $f(x)=ax+b$, $a\neq 0$.
Thus, we can parametrize the automorphism by two numbers
$a,b$ with $a \neq 0$. This is a quasi-affine variety.

For $X=\mathbb{A}^2$ 
the situation is a little more involved.
de Jonequeres, Triangular group.
Hierzch Jung 1942, chre.
Wout Van der Kulk, 1953, chr arbitrary.
$$
\text{Aut}(\mathbb{A}^2)
=\left<\text{linear, triangular}\right>
$$

For $\text{Aut}(\mathbb{A}^3)$,
the question if whether $\text{Aut}(\mathbb{A}^3)=
\left<\text{linear, triangular}\right>$.
This is false in general

\begin{definition}
\label{definition-tame}
An auotomorphism $\sigma \in \text{Aut}(\mathbb{A}^3)$ 
is called {\it tame} if $\sigma \in \left<\text{linear, triangular}\right>$ 
and {\it wild} otherwise.
\end{definition}

\noindent
Nagata's conjecture is that
a $\sigma:k[x,y,z] \to k[x,y,z]$,
$(x,y,z) \mapsto (x+(x^2-yz)z,y+2(x^2yz)x+(x^2-yz)^2z,z)$ 
is wild.

Shiguero Kuroda 2014. Introduce a monomial order
$\leq $ in $k[x,y,z]$.
[The following is not a criterion;
this just says "if this happens, then the morphism is wild",
but not an iff.]
Let 
\begin{align*}
\sigma: k[x,y,z] &\longrightarrow k[x,y,z] \\
\leavevmode &\longmapsto (f_1,f_2,f_3)
\end{align*}

\noindent
$\sigma$ is wild if
\begin{enumerate}
\item $\text{int}(f_1),\text{int}(f_2), \text{int}(f_3)$
are linearly dependent over $\mathbb{Z}$,
pairwise linear over $\mathbb{Z}$.
\item $\text{int}(f_{i_1}) \neq  p \text{int}(f_{i_2})+q(\text{int}(f_{i_3})$ 
for $p,q \in \mathbb{Z}_{\geq 0}$.
\end{enumerate}

\medskip\noindent
Jacobian conjecture:
a polynomial map $\varphi:\mathbb{C}^n \to \mathbb{C}^n$
[with coordinate functions] $(f_1(\mathbf{x}),…,f_n(\mathbf{x}))$
is an automorphism if and only if
$\det \text{Jacob}(\varphi)=\text{constant}\neq 0$.

The forward implication is clear: 
if  $\varphi \psi=\text{id}$
then $\text{Jac}(\varphi)(\psi)\text{Jac}(\psi)=\text{Jac}(\text{id})=1$.

[Bass-Conell-Wright, 1982]
Let $\varphi:\mathbb{C}^n \to \mathbb{C}^n$ 
a polynomial map with
$\det \text{Jac}\varphi=$ constant. Then
$\varphi$ is invertible (its inverse is a polynomial map)
[note that by the holomorphic inverse function theorem
we already know that there exists a local inverse
that is a series, but here we see it's actually polynomial]
if and only if $\varphi$ is injective
if and only if 
$\varphi$ is proper ($\varphi^{-1}(\text{compact})=\text{compact}$).

\begin{theorem}
\label{theorem-jacobian-conjecture}
If the Jacobian conjecture is valid for all  $n$ and
for any polynomial of degree $\leq 3$,
then it's valid.
\end{theorem}

\begin{theorem}
\label{theorem-jacobian-conjecture2}
If the Jacobian conjecture is valid over $\mathbb{C}$
then it is valid over any domain.
\end{theorem}


\section{Rational maps of projective varieties}
\label{section-rational-maps-projective-varieties}

\noindent
$X=V(I) \subseteq \mathbb{P}^n$,
$I \subseteq k[x_0,…,x_n]$ homogeneous.

\begin{definition}
\label{definition-presentation-of-map}
A rational map is $\varphi:X \dashrightarrow Y$ 
for $X\subseteq \mathbb{P}^n$, $Y\subseteq \mathbb{P}^m$ 
given by $(f_0:…:f_m)$
is {\it presented} by
$f_i \in R=k[\mathbf{x}]/I_X$,
$\text{deg}f_i=\text{deg}f_j$,
$f_i$ homogeneous.
\end{definition}

\noindent
A map can have another presentation:
\begin{align*}
\varphi:X &\dashrightarrow Y \\
\leavevmode &(g_0:…:g_m)
\end{align*}

\noindent
For all $p \in \text{Dom}\varphi$,
 $$
(f_0(p):…:f_m(p))=(g_0(p):…:g_m(p)).
$$

If
$$
I_2
\begin{pmatrix}f_0(\mathbf{x})&f_m(\mathbf{x})\\ 
g_0(\mathbf{x})&g_m(\mathbf{x})\end{pmatrix}
\in I_X
$$
[then the presentations are equivalent].

\begin{exercise}
\label{exercise-standard-quadratic}
$X=V(y^3-x^2z) \subseteq \mathbb{P}^2$,
$\sigma:X \dashrightarrow X$, $(yz:xz:xy)$.
[Show that] $(xz:y^2:x^2)$ 
[is an equivalent presentation, that is,]
$$
I_2
\begin{pmatrix} 
yz & xz & xy\\
xz & y^2 & x^2 \end{pmatrix} 
\in (y^3-x^2z).
$$

\end{exercise}

\noindent
What are the possible presentations of a given rational map?
$$
f_0g_j=f_jg_0\qquad \text{ in }I_X.
$$
If $f_0 \neq 0 \implies  g_j=\left(\frac{g_0}{f_0}\right) f_j$.

\medskip\noindent
[We see that there is a 1-1 correspondence between]
$$
\text{Presentations of } \varphi
\xymatrix{\ar@{<~>}[r]&}
\Hom_R(I,R).
$$
From
$$
\xymatrix{
0\ar[r]
&I\ar[r]
&R\ar[r]
&R/I\ar[r]
&0
}
$$ 
we obtain
$$
\xymatrix{
0\ar[r]
&\underbrace{\Hom(R/I,R)}_{=0}\ar[r]
&\Hom(R,R)\ar[r]
&\Hom(I,R)\ar[r]
&\text{Ext}^1(R/I,R)\ar[r]
&\underbrace{\text{Ext}^1(R,R)
}_{=0}}
$$
And $\Hom(R/I,R)=\ast \iff \text{grade}(I) \geq 1$.

If $\text{Ext}^1(R/I,R)=0 \implies R \simeq\Hom(I,R)$.
Note that $\text{Ext}^1(R/I,R) \iff \text{grade}(I,R)\geq 2$.

[This would help us understand $\text{Bir}(X)_d$.
$$
\text{Domin}\varphi=\bigcup \text{Domin}(\text{presentation})=X.
$$


\section{Image and graph of a rational map}
\label{section-image-graph}

[Given a rational map]
$\varphi:X \dashrightarrow Y$, $(f_0:…:f_m)$,
the ideal of definition of the image of $\varphi$ 
is
$$
\mathcal{K}
=\{h(y) \in k[y_0,…,y_m]
:h(f_0(p),…,f_m(p)), \forall p \in \text{Dom}\varphi\}.
$$
[The coordinate ring of the image has a special name:]
$$
\frac{k[y_0:…:y_m]}{\mathcal{K}}
\simeq k[f_0,…,f_m]
\subseteq\frac{k[x_0,…,x_n]}{I_X}
$$
is called {\it special fiber $\mathcal{I}_I$ of the ideal $I\subseteq R$}.

The {\it analytic spread} is
$$
\dim(\text{Im}\varphi)
=\dim \mathcal{I}_I-1
=\ell_R(I)-1.
$$

Para um anel em geral
$f_0,…,f_m \in R$,
$\mathcal{I}_I=k[f_0t,…,f_mt] \subseteq R[t]$.
$$
\text{ht}(I) \leq  \ell(I) \leq  \dim R.
$$

One can also prove via Dedekind-Mertens lemma,
which says that 
$c(f)^nc(fg)=c(g)c(f)^{n+1}$,
where $c$ is the content of a polynomial
as in Gauss lemma,
that
$$
\ell(I_j) \leq  \ell(I) + \ell(J)-1.
$$

\medskip\noindent
The {\it graph} of $\varphi$ is
$$
\Gamma_\varphi
= \overline{\{(p_0:…:p_n)\times
(f_0(p):…:f_m(p))
\subseteq X \times \mathbb{P}^m\}:
p \in \text{Dom}(\varphi)\}}
$$
$$
\{
h(\mathbf{x},\mathbf{y})
\in k[x_0,…,x_n,y_0,…,y_m]
:h(\mathbf{p},\mathbf{f}(\mathbf{x}))=0
\forall p \in \text{Dom}(\varphi)\}.
$$

The  {\it ideal of definition of the graph} 
is given by homogeneous polynomial equations
of a presentation of $\varphi$.

The {\it Rees algebra} is
$$
\frac{R[y_0,…,y_m]}{J}
\simeq R[f_0t,…,f_\mapsto ] \subseteq R[t]
$$
where $J = $ polynomial equations of
$f_0,…,f_m$.


\section{Finite maps}
\label{section-finite-maps}

\noindent
[Another concept before we get to birational maps.]

\begin{definition}
\label{definition-finite-map}
A rational map $\varphi:X \dashrightarrow \mathbb{P}^m$
presented by $(f_0:…:f_m)$
is called a {\it finite map} if and only if
$\ell(I) = \dim R$.
$I$ maximal analytic spread.
\end{definition}

\noindent
[Keep in mind that dimension of image is analytic spread minus 1,
and dimension of domain is $\dim R-1$.]

\medskip\noindent
[The following is another version
of Hilbert's Nullstellensatz in commutative algebra:]

\begin{theorem}[Matsumura 5.6]
\label{theorem-matsumura-5.6}
$R$ Noetherian domain containing a field $k$,
$$
\dim (R) = \text{trdeg}_k Q(R).
$$
\end{theorem}

$$
[\text{Equations involving fraction fields of coordinate rings}]
$$
We obtain that
$\varphi:X \dashrightarrow Y$ 
is finite if and only if
$\dim(\varphi^{-1}(q))=0$ for generic $q \in Y$.
$\text{deg}(\varphi)=\# \{p : \varphi(p)=q\}$.

\medskip\noindent
How can we compute (algebraically) $\text{deg}(\varphi)$?

\begin{align*}
\text{deg}Z&=\sum_{\substack{z\text{ closed} \\ \text{
in $Z$}}}
\lambda(\mathcal{O}_{Z,z}\text{deg}(Z)\\
&=
\sum_{ p \not \supseteq I}
\lambda \left(\left(
\frac{R}{I_q}\right)_p\right) 
e\left(\frac{R}{p}\right)
\end{align*}

\noindent
since
$$
\{p:(f_0(p):…:f_m(p))
=(q_0:…:q)m): p \in \text{Domin}(\varphi)\},
$$
$$
I_q=I_2
\begin{pmatrix}f_0(\mathbf{x})&f_m(\mathbf{x})\\ 
q_0&q_m\end{pmatrix},\qquad  V(I_q).
$$
$$
I_q=Q_1\cap…\cap Q_t\cap\underbrace{…\cap Q_s}_{
\sqrt{Q_i}\supseteq I}.
$$

\begin{exercise}
\label{exercise-saturation}
$(I_q:I^\infty)=Q_1\cap…\cap Q_t$.
\end{exercise}

$$
\text{deg}(\varphi)
=e\left(\frac{R}{I_q:I^\infty}\right).
$$


\medskip\noindent
Integral closure (1,82).

$$
\xymatrix{
&  V \ar[dr]^{\pi_L}\\
X\ar@{-->}[ur]^{(g_0,…,g_N)}\ar@{-->}[rr]^{\varphi=(f_0,…,f_m)}
& & Y
}
$$
where $\pi_L$ is a linear map.
$$
\underbrace{(f_0,…,f_m)}_{J}
\subseteq \underbrace{(g_0,…,g_N)}_{I}.
$$


The center of the projection $\pi_L$ 
does not intersect $V$ if and only if $J$ 
is reduction of $I$, that is,
$I \subseteq \overline{J}$.

\medskip\noindent
Another way of stating this result is:
given $J \subseteq I$,
$$
J \subseteq \mathcal{F}_I
=k[g]
=\frac{k[y_0,…,y_m]}{k},
\ell_1,…,\ell_m.
$$
$J \subseteq I$ reduction if and only if
$(\ell_0,…,\ell_m)$ is $m$-primary.



\section{Summary}
\label{section-summary}

\noindent
Let $x \subset \mathbb{P}^n$ and $y \subset \mathbb{P}^m$.
$\varphi:x \dashrightarrow y$, $(f_0:…:f_m)$,
$r=k[x]=k[x_0,…,x_n]/I_x$,
$I=(f_0,…,f_n) \subseteq R$.

We defined $\varphi$ to be {\it finite} if $\ell(I) = \dim r$.
$[q(x),q(y)]<\infty$.
$\#\{p : p \in x: \varphi(p) = q,q\text{ general in }y\}$.

$$
i_q=i_2
\begin{pmatrix} f_0(\mathbf{x}&  \cdots &  f_n(\mathbf{x})\\
q_0& \cdots & q_m \end{pmatrix} 
\subseteq r
$$
$$
e\left(\frac{R}{I_q:I^\infty}\right) 
=\text{deg}(\varphi)
$$
[is the topological degree of $\varphi$.]

Eisenbud-Ulrich 2007: raw ideal.
$I_q:I^\infty=$ morphism fiber ideal.
Correspondence fiber ideal
$$
\xymatrix{
&\underbrace{\Gamma}_{\substack{\text{
graph} \\ \text{of }X}} \subseteq X \times\mathbb{P}^m
\ar[dr]_{\pi_2}\ar[dl]^{\substack{\pi_1 \\ \text{(birational)}}}\\
X \ar@{-->}[rr]& & Y \ni q
}
$$

\begin{align*}
\pi_2^* :\frac{k[Y]}{I_Y}&\to R_I=\frac{y_0,…,y_m]}{J}
y_i&\longmapsto \mathbf{y_i}=f_it \in R[\mathbf{f}t].
\end{align*}

$$
C_q=(I_qtR_I :_{R_I}I^{\infty}t) \cap R, \qquad  R \subseteq R_I.
$$
\begin{exercise}
\label{exercise-raw-ideal}
$C_q=(I_qtR_I:R_i^{\infty}t)
=\bigcup_{n=1}^\infty (I_q I^{n-1}:_R I^n) 
\subseteq(I_q:I^\infty)$.
\end{exercise}

$$
v(q)=\{p:(I_q)_p:\text{ is not a reduction of $I_p$}\}.
$$
$$
I_q(q:\psi)=(\underbrace{I_q:I}_{\text{raw ideal}}
\subseteq C_q \subseteq (I_q:I^\infty)
$$
Question: when to $(I_q:I)$ and $C_q$ have the
same saturation?

If $I$ has a [crossed: linear generalized] raw
 and $I$ is locally of linear type
$\implies (I_q:I)^{\text{sat}}=C_q^{\text{sat}}$.




\medskip\noindent
[Simis Chardin, 2021.]
$Q(X)=R_i{\underline{0}}=\left\{\frac{a}{b}:b\neq 0\right\}$.
$R_{(\underline{0})}=\left\{\frac{a}{b}:b\neq 0,b\text{
homogeneous}\right\}$.

 $$
k[x]_x=k[x,x^{-1}], R_{((0))}=k,
$$
$$
k(X)=R_{((0))}.
$$

\begin{lemma}
\label{lemma-degree}
Let $A = \bigoplus_{i=0}^\infty A_i$ 
be a graded domain over a field $A_0=F$
with $\dim A=1$.
Then $e(A)=[K(A):F]$.
\end{lemma}

\begin{proof}
There exists $d$ such that $d_F A_i=e$ 
for all $i \geq d$, $A_d=F\left<a_1,…,a_e\right>$.
Show that $K(A)= F \left<\frac{a_1}{a_1},…,\frac{a_e}{a_1}\right>$
(exercise).
\end{proof}

\noindent
[Then] 
\begin{align*}
e(S/J \otimes_B Q(Y))
&=[K(S/J \otimes Q(Y)):Q(Y)]\\
&=[K(S/J)(K_0):K(Y)(K_0)]\\
&=[K(\Gamma):K(Y)]\\
&=\text{deg} \varphi.
\end{align*}


[The ring $S/J \otimes_B Q(Y)$ is always Cohen-Macaulay of rank 1.]

\begin{theorem}
\label{theorem-minimal}
Suppose that $J$ is prime minimal over an ideal $I$ 
generated by elements of degrees $d_1 \geq  d_2 \geq …$.
$$
\text{deg}\varphi\leq  d_1…d_{t}e(R)
$$
where $t=\dim X$.

\end{theorem}

\noindent
We have the following corollary:
\begin{lemma}
\label{lemma-minimal}
$\text{deg}(\varphi)\leq  d_1… d_{d_\text{max}}e(R)$.
\end{lemma}

\section{Eisenbud-Goto conjecture}
\label{section-eisenbud-goto-conjecture}

\noindent
1982.
Let $X \subseteq \mathbb{P}^n$ irreducible
and ``non-degenerated''.

\begin{align*}
reg(X) &\leq  \text{deg}(X) - \text{codim}(X)\\
reg(X)&=
\text{min}\{r:H^1(X,\mathcal{O}_X(r-i))=0\forall  i >0\}\\
&=\text{min} \{r: H^2_{\underline{m}}(R)_{r-i}=0 \forall  i>0\}.
\end{align*}

[Regularity controls properties of the ideal
such as the degree of the generators in a Gröbner base.]

$$
reg(R/I_X)
=\text{max}
\{t_i+i:t_i=\text{max}\{d_{ji}\}\}
$$
$$
reg(R/f)=\text{deg}f-(n+1)+n=\text{deg}f-1.
$$

[Notice the term ``non-degenerate''. It means that
the ideal $I_X$ in  $R=\frac{k[x_0,…,x_n]}{I_X}$
does not contain any linear terms.]

\begin{align*}
e\left(\frac{S}{J}\otimes Q(Y)\right) &\geq 
n+1- \dim_{Q(Y)}(\tilde{J}_{\underbrace{1}_{\text{deg} X}}
+reg(\tilde{J})-2.\\
\text{deg}\varphi\geq n+q-\dim_{Q(Y)}(\tilde{J}_1)
=\text{ Jacobian dual rank}.\\
reg(\tilde{J}) \geq  \text{deg}(\text{gens})\geq 2\\
\text{deg}\varphi\geq n+1-r \text{jdrank}(\varphi).
\end{align*}

\noindent
[The conjecture is false in general,
but in some cases holds, such as in the Cohen-Macaulay case.]


\section{Summary}
\label{section-summary}

Recall our notations:
\begin{itemize}
\item $\varphi:X \dashrightarrow Y$, $(f_1,…,f_n)$.

\item $I=(f_0,…,f_n)$.

\item $R=k[x_0,…,x_n]/I_X$.

\item $\mathcal{R}_I(R)=R[y_0,…,y_n]/J=S/J$.

\item $B=k[y_0,…,y_m]$.

\item $k[Y]=k[y_0,…,y_m]/J\cap B=B/q$/

\item $S/J \otimes_B Q(Y)$, $Q(Y)=Q(B/q)$. CM dim 1, domain.

 \item $S/J \otimes )B Q(Y)
=\frac{Q(Y)[x_0,…,x_n]}{\underbrace{\tilde{J}}_{=\text{Im}\theta}}$,
$$
\xymatrix{
0\ar[r]
&J\ar[r]
&S\ar[r]
&S/J\ar[r]
&0
}
$$
$$
\xymatrix{
J \otimes Q(Y)\ar[r]^{\theta}
&S \otimes Q(Y)\ar[r]
&S/J \otimes_B Q(Y)\ar[r]
&0
}
$$
$S=k[\mathbf{x},\mathbf{y}]/I_X$.

\item $e\left(\frac{Q(Y)[\mathbf{x}]}{\tilde{J}}\right)=\text{deg}\varphi$.

\item $\text{djrank}=\dim_{Q(Y)}(\tilde{J})_1$ 

\item $\text{deg}(\varphi)\geq  (n-\text{jdrank}(\varphi))
+\text{reg}(\tilde{J})-1)$.
\end{itemize}



\begin{theorem}
\label{theorem-jacobian-rank-and-birationality}
$n=\text{jdrank}(\varphi)$ if and only if
$\varphi$ is birational.
\end{theorem}

\begin{proof}
$\text{deg}(\varphi)=e(Q(Y)[x_0,…,x_n]/\tilde{J})$.

($\implies$) If $\text{jdrank}(\varphi)=n \implies 
\tilde{J}_1=(\ell_1,…,\ell_n)
\implies e(A)=e(Q(Y)[x]/\tilde{J})$
where $A:=S/J\otimes_BQ(Y)$.
$A$ domain,  $\dim=1 \implies  \tilde{J}'=0
\implies e(A)=1=\text{deg}(\varphi)$.

($\impliedby$) $e(A)=1 \implies $ACM
$\implies $ quasi-unmixed
$\implies$ $\hat{A}$ is equidimensional
and $A$ is analytically unmixed
$\implies $ $A$ is a regular ring.
$A=\frac{Q(Y)[\mathbf{x}]}{\tilde{J}}$ (See [Bruns-Herzog §2]).
[Thus $\tilde{J}$ is linear,]
$\tilde{J}=(\ell_1,…,\ell_n)$ 
since $\text{codim}\tilde{J}=n$.
\end{proof}

\begin{proposition}
\label{proposition-regular-local-ring}
Let $(A,\mathfrak{m})$ a regular local ring and $I\subseteq \mathfrak{m}$.
If  $A/I$ is regular then $I$ is generated by a part
of a system of parameters.
\end{proposition}

\section{How to compute the Jacobian dual rank}
\label{section-computing-jacobian-dual-rank}

$$
R=\frac{k[x]}{I_X}\supseteq I=(f_0,…,f_m),
$$
$$
\mathcal{I}_I(R)=\frac{R[\mathbf{y}]}{J}.
$$
$$
\left<J_{(1,\ast)}\right>
=\left<\underbrace{g_1}_{\in R[\mathbf{y}]},
…,\underbrace{g_t}_{R[\mathbf{y}]}\right>
$$
lifts $g_1(\overline{Q}_i)$,
$Q_i \in k[\mathbf{x},\mathbf{y}]_{i,\ast)}$.
$$
k[\mathbf{y}]\ni
\begin{pmatrix} 
\frac{\partial Q_1}{\partial x_0}& \cdots
&\frac{\partial Q_t}{\partial x_0}\\
\vdots &  &  \vdots \\
\frac{\partial Q_1}{\partial x_n}&  \cdots
&  \frac{\partial Q_t}{\partial x_n}
 \end{pmatrix} 
=M
$$
Jacobian dual matrix.
$$
-:k[\mathbf{y}]\to \frac{k[t]}{I_Y}/
$$
$$
\psi=
\begin{pmatrix} 
\overline{\frac{\partial Q_1}{\partial x_0}}
&  \cdots & 
\overline{\frac{\partial Q_1}{\partial x_0}}\\
\vdots &  &  \vdots \\
\overline{\frac{\partial Q_1}{\partial x_n}}
&  \cdots & 
\overline{\frac{\partial Q_t}{\partial x_n}}
 \end{pmatrix} 
\in M_{(n+1) \times t}(k[Y]).
$$
$$
\tilde{\psi}
=\begin{pmatrix} 
\frac{\partial Q_1}{\partial x_0}    \\
&  \ddots & \\
&  &  \frac{\partial Q_t}{\partial x_n}
\end{pmatrix}
\Big|_{(f_0(\mathbf{x}),…,f_m(\mathbf{x})}
\in M_{(n+1)t}(R).
$$

\begin{proposition}
\label{proposition-rank-and-birationality}
$\text{rk}\psi=\text{rk}\tilde{\psi}=\text{jrank}(\varphi)
=\dim_{Q(Y)}\tilde{J}_1$,
$\text{rk}\psi=n$ if and only if
$\varphi$ is birational.
\end{proposition}

\begin{proof}
[Compute minors, add a row that is missing…
this way we obtain a candidate for the inverse map…]
\end{proof}

\section{Extra}
\label{section-extra}

\noindent
Consider
$$
\text{Bir}(X)_d
= \{ \varphi:X \dashrightarrow X,(f_0,…,f_n):\text{deg} f_i=d\}.
$$
$$
f_0=d_0x^d+a_1x_0^{d-1}x_1
+a_2x_0^{d-1}x_1x_2…
$$
Let $H_d \subseteq \mathbb{A}^{(n+1)\times\binom{n+d}{d}}$ 
be the points whose corresponding map is birational.
We have a map $H_d \to \text{Bir}(X)_d$.
[We have proved that] $H_d$ is a quasi-projective variety
and $\text{Bir}(X)_d$ is a constructible set.

[If the Cremona map has degree $d$,
what's the degree of the inverse map?]
$\varphi^{-1}\leq d^{n-1}$.
A proof of this is due to Gabber.

\section{Extra 2}
\label{section-extra2}

$$
D=\underbrace{\sum a_i P_i}_{E}
- \sum\underbrace{b_j}_{\in \mathbb{Z}_{\geq 0}}
\underbrace{q_j}_{\text{proj}},
\qquad \text{proj}(R)=X.
$$
$$
\mathcal{O}_X(D)=\tilde{M}
$$
\begin{align*}
\mathcal{O}_X(D)&=\tilde{M}\\
\text{prod}E&=\prod p_i^{[a_i]}=(f^{a_i}:…: f_k^{a_i})\\
\text{prod}F&=\prod q_j^{[b_j]}\\
\text{prod}E^{\ast \ast}&=
\Hom(\Hom(\text{prod}F,R',R))\\
\text{dual}&=\text{prod}E^{\ast \ast}\otimes
\Hom(\text{prod}(D^{\ast \ast},R)\\
M&=\Hom(\text{dual},R).
\end{align*}

\noindent
D. OO(D). divisorToModule(D)

\medskip\noindent
Divisor (Karl Schewede,…) [is the name of this Macaulay2 package].
$M$ rk 1, torsion-free. embedAsIdeal(M).

$$
\xymatrix{
\bigoplus R(-e_j)
\ar[r]^{
\begin{pmatrix} a_1\\
\vdots \\
a_n \end{pmatrix} }_\psi
&  R^n(-d_i)
\ar[r]
& M\ar[d]^{m_i \mapsto  f_i}\ar[r]& 0\\
& & R
}
$$
$(\underline{f}) \in \Ker \psi^t$.

$I=$.
$$
\psi^t:R^n(td_i) \to R(fe_j)
$$
$I=(\underline{f}) \in R^n(td_i)_a$.

mapToProjectiveSpace(D),
Map defined by base of $I_a$.









\end{document}

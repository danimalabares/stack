\IfFileExists{stacks-project.cls}{%
\documentclass{stacks-project}
}{%
\documentclass{amsart}
}

% For dealing with references we use the comment environment
\usepackage{verbatim}
\newenvironment{reference}{\comment}{\endcomment}
%\newenvironment{reference}{}{}
\newenvironment{slogan}{\comment}{\endcomment}
\newenvironment{history}{\comment}{\endcomment}

% For commutative diagrams we use Xy-pic
\usepackage[all]{xy}

% We use 2cell for 2-commutative diagrams.
\xyoption{2cell}
\UseAllTwocells

% We use multicol for the list of chapters between chapters
\usepackage{multicol}

% This is generally recommended for better output
\usepackage{lmodern}
\usepackage[T1]{fontenc}

% For cross-file-references
\usepackage{xr-hyper}

% Package for hypertext links:
\usepackage{hyperref}

% For any local file, say "hello.tex" you want to link to please
% use \externaldocument[hello-]{hello}
\externaldocument[introduction-]{introduction}
\externaldocument[conventions-]{conventions}
\externaldocument[sets-]{sets}
\externaldocument[categories-]{categories}
\externaldocument[topology-]{topology}
\externaldocument[sheaves-]{sheaves}
\externaldocument[sites-]{sites}
\externaldocument[stacks-]{stacks}
\externaldocument[fields-]{fields}
\externaldocument[algebra-]{algebra}
\externaldocument[brauer-]{brauer}
\externaldocument[homology-]{homology}
\externaldocument[derived-]{derived}
\externaldocument[simplicial-]{simplicial}
\externaldocument[more-algebra-]{more-algebra}
\externaldocument[smoothing-]{smoothing}
\externaldocument[modules-]{modules}
\externaldocument[sites-modules-]{sites-modules}
\externaldocument[injectives-]{injectives}
\externaldocument[cohomology-]{cohomology}
\externaldocument[sites-cohomology-]{sites-cohomology}
\externaldocument[dga-]{dga}
\externaldocument[dpa-]{dpa}
\externaldocument[sdga-]{sdga}
\externaldocument[hypercovering-]{hypercovering}
\externaldocument[schemes-]{schemes}
\externaldocument[constructions-]{constructions}
\externaldocument[properties-]{properties}
\externaldocument[morphisms-]{morphisms}
\externaldocument[coherent-]{coherent}
\externaldocument[divisors-]{divisors}
\externaldocument[limits-]{limits}
\externaldocument[varieties-]{varieties}
\externaldocument[topologies-]{topologies}
\externaldocument[descent-]{descent}
\externaldocument[perfect-]{perfect}
\externaldocument[more-morphisms-]{more-morphisms}
\externaldocument[flat-]{flat}
\externaldocument[groupoids-]{groupoids}
\externaldocument[more-groupoids-]{more-groupoids}
\externaldocument[etale-]{etale}
\externaldocument[chow-]{chow}
\externaldocument[intersection-]{intersection}
\externaldocument[pic-]{pic}
\externaldocument[weil-]{weil}
\externaldocument[adequate-]{adequate}
\externaldocument[dualizing-]{dualizing}
\externaldocument[duality-]{duality}
\externaldocument[discriminant-]{discriminant}
\externaldocument[derham-]{derham}
\externaldocument[local-cohomology-]{local-cohomology}
\externaldocument[algebraization-]{algebraization}
\externaldocument[curves-]{curves}
\externaldocument[resolve-]{resolve}
\externaldocument[models-]{models}
\externaldocument[functors-]{functors}
\externaldocument[equiv-]{equiv}
\externaldocument[pione-]{pione}
\externaldocument[etale-cohomology-]{etale-cohomology}
\externaldocument[proetale-]{proetale}
\externaldocument[relative-cycles-]{relative-cycles}
\externaldocument[more-etale-]{more-etale}
\externaldocument[trace-]{trace}
\externaldocument[crystalline-]{crystalline}
\externaldocument[spaces-]{spaces}
\externaldocument[spaces-properties-]{spaces-properties}
\externaldocument[spaces-morphisms-]{spaces-morphisms}
\externaldocument[decent-spaces-]{decent-spaces}
\externaldocument[spaces-cohomology-]{spaces-cohomology}
\externaldocument[spaces-limits-]{spaces-limits}
\externaldocument[spaces-divisors-]{spaces-divisors}
\externaldocument[spaces-over-fields-]{spaces-over-fields}
\externaldocument[spaces-topologies-]{spaces-topologies}
\externaldocument[spaces-descent-]{spaces-descent}
\externaldocument[spaces-perfect-]{spaces-perfect}
\externaldocument[spaces-more-morphisms-]{spaces-more-morphisms}
\externaldocument[spaces-flat-]{spaces-flat}
\externaldocument[spaces-groupoids-]{spaces-groupoids}
\externaldocument[spaces-more-groupoids-]{spaces-more-groupoids}
\externaldocument[bootstrap-]{bootstrap}
\externaldocument[spaces-pushouts-]{spaces-pushouts}
\externaldocument[spaces-chow-]{spaces-chow}
\externaldocument[groupoids-quotients-]{groupoids-quotients}
\externaldocument[spaces-more-cohomology-]{spaces-more-cohomology}
\externaldocument[spaces-simplicial-]{spaces-simplicial}
\externaldocument[spaces-duality-]{spaces-duality}
\externaldocument[formal-spaces-]{formal-spaces}
\externaldocument[restricted-]{restricted}
\externaldocument[spaces-resolve-]{spaces-resolve}
\externaldocument[formal-defos-]{formal-defos}
\externaldocument[defos-]{defos}
\externaldocument[cotangent-]{cotangent}
\externaldocument[examples-defos-]{examples-defos}
\externaldocument[algebraic-]{algebraic}
\externaldocument[examples-stacks-]{examples-stacks}
\externaldocument[stacks-sheaves-]{stacks-sheaves}
\externaldocument[criteria-]{criteria}
\externaldocument[artin-]{artin}
\externaldocument[quot-]{quot}
\externaldocument[stacks-properties-]{stacks-properties}
\externaldocument[stacks-morphisms-]{stacks-morphisms}
\externaldocument[stacks-limits-]{stacks-limits}
\externaldocument[stacks-cohomology-]{stacks-cohomology}
\externaldocument[stacks-perfect-]{stacks-perfect}
\externaldocument[stacks-introduction-]{stacks-introduction}
\externaldocument[stacks-more-morphisms-]{stacks-more-morphisms}
\externaldocument[stacks-geometry-]{stacks-geometry}
\externaldocument[moduli-]{moduli}
\externaldocument[moduli-curves-]{moduli-curves}
\externaldocument[examples-]{examples}
\externaldocument[exercises-]{exercises}
\externaldocument[guide-]{guide}
\externaldocument[desirables-]{desirables}
\externaldocument[coding-]{coding}
\externaldocument[obsolete-]{obsolete}
\externaldocument[fdl-]{fdl}
\externaldocument[index-]{index}

% Theorem environments.
%
\theoremstyle{plain}
\newtheorem{theorem}[subsection]{Theorem}
\newtheorem{proposition}[subsection]{Proposition}
\newtheorem{lemma}[subsection]{Lemma}

\theoremstyle{definition}
\newtheorem{definition}[subsection]{Definition}
\newtheorem{example}[subsection]{Example}
\newtheorem{exercise}[subsection]{Exercise}
\newtheorem{situation}[subsection]{Situation}

\theoremstyle{remark}
\newtheorem{remark}[subsection]{Remark}
\newtheorem{remarks}[subsection]{Remarks}

\numberwithin{equation}{subsection}

% Macros
%
\def\lim{\mathop{\mathrm{lim}}\nolimits}
\def\colim{\mathop{\mathrm{colim}}\nolimits}
\def\Spec{\mathop{\mathrm{Spec}}}
\def\Hom{\mathop{\mathrm{Hom}}\nolimits}
\def\Ext{\mathop{\mathrm{Ext}}\nolimits}
\def\SheafHom{\mathop{\mathcal{H}\!\mathit{om}}\nolimits}
\def\SheafExt{\mathop{\mathcal{E}\!\mathit{xt}}\nolimits}
\def\Sch{\mathit{Sch}}
\def\Mor{\mathop{\mathrm{Mor}}\nolimits}
\def\Ob{\mathop{\mathrm{Ob}}\nolimits}
\def\Sh{\mathop{\mathit{Sh}}\nolimits}
\def\NL{\mathop{N\!L}\nolimits}
\def\CH{\mathop{\mathrm{CH}}\nolimits}
\def\proetale{{pro\text{-}\acute{e}tale}}
\def\etale{{\acute{e}tale}}
\def\QCoh{\mathit{QCoh}}
\def\Ker{\mathop{\mathrm{Ker}}}
\def\Im{\mathop{\mathrm{Im}}}
\def\Coker{\mathop{\mathrm{Coker}}}
\def\Coim{\mathop{\mathrm{Coim}}}

% Boxtimes
%
\DeclareMathSymbol{\boxtimes}{\mathbin}{AMSa}{"02}

%
% Macros for moduli stacks/spaces
%
\def\QCohstack{\mathcal{QC}\!\mathit{oh}}
\def\Cohstack{\mathcal{C}\!\mathit{oh}}
\def\Spacesstack{\mathcal{S}\!\mathit{paces}}
\def\Quotfunctor{\mathrm{Quot}}
\def\Hilbfunctor{\mathrm{Hilb}}
\def\Curvesstack{\mathcal{C}\!\mathit{urves}}
\def\Polarizedstack{\mathcal{P}\!\mathit{olarized}}
\def\Complexesstack{\mathcal{C}\!\mathit{omplexes}}
% \Pic is the operator that assigns to X its picard group, usage \Pic(X)
% \Picardstack_{X/B} denotes the Picard stack of X over B
% \Picardfunctor_{X/B} denotes the Picard functor of X over B
\def\Pic{\mathop{\mathrm{Pic}}\nolimits}
\def\Picardstack{\mathcal{P}\!\mathit{ic}}
\def\Picardfunctor{\mathrm{Pic}}
\def\Deformationcategory{\mathcal{D}\!\mathit{ef}}

%Dani's additions
\usepackage{graphicx} %figures


\begin{document}

\title{Lie algebras}
\maketitle

\phantomsection
\label{section-phantom}
\hfill
\href{http://github.com/danimalabares/stack}{github.com/danimalabares/stack}

\tableofcontents

\section{Basic definitions and examples}
\label{section-basic-definitions-and-examples}

\begin{definition}
\label{definition-algebra}
An {\it algebra} is a vector space over a field $\mathbb{F}$ endowed with a
binary operation that is bilinear
\begin{align*}
a(\lambda b+\mu c)&=\lambda ab+\mu ac\\
(\lambda b + \mu c) a &= \lambda b a + \mu c a
\end{align*}
\end{definition}

\begin{definition}
\label{definition-Lie-algebra}
A {\it Lie algebra} is an algebra $\mathfrak{g}$ 
with a product $[\cdot,\cdot]$ we call 
{\it bracket} that satisfies
\begin{enumerate}
\item $[x,x]=0\qquad \forall x\in \mathfrak{g}$,
\item (Jacobi identity.)
\end{enumerate}
\end{definition}

\begin{definition}
\label{definition-simple-Lie-algebra}
A {\it simple Lie algebra} is a Lie algebra that has no nontrivial proper
ideals.
\end{definition}

As a rule of thumb I keep in mind the following silly computation:
$$
\text{log}\det A=\text{log} \prod_{i}\lambda_i=\sum \text{log}\lambda_i
=``\text{tr}\text{log}A''
$$
And recall that exponent map goes from $T_eG= \mathfrak{g} \to G$, so that
logarithm would go from $G\to \mathfrak{g}$. This is why I remember that the
condition on a classical Lie group of 
{\it having determinant 1} goes to {\it having vanishing trace} in the 
Lie algebra. (Because $\text{log}1=0$.)

\begin{example}
\label{example-Lie-algebras}
\begin{enumerate}
\item The {\it special linear Lie algebra} 
$$
\mathfrak{sl}_n=\{\text{Mat}_n|Tr(A)=0\}
$$
which is just obvious from the slogan above.
\item The {\it special orthogonal Lie algebra} 
$$
\mathfrak{so}_n=\{A\in \text{Mat}_n|A+A^{\mathbf{T}}=0\}
$$
which is obvious from: $\text{SO}(n)=$isometries, so
$\left<v,v\right>=\left<Av,Av\right>=\left<v,A^{\mathbf{T}}Av\right>$, so
$\text{SO}(n)=\{A \in \text{Mat}_n:A^{-1}=A^{\mathbf{T}}\}$, and then
$0=
\text{log}1=\text{log}(A A^{\mathbf{T}})=\text{log}A+\text{log}A^{\mathbf{T}}$.

\item The {\it symplectic Lie algebra}
$$
\mathfrak{sp}_{2n}=\{A \in \text{Mat}_{2n}:\Omega A+A^{\mathbf{T}}\Omega=0\}
$$
where $\Omega=\begin{pmatrix}
0&\text{Id}_n\\ 
-\text{Id}_n & 0
\end{pmatrix}$.

{\bf Pending.} Why this makes sense?
\end{enumerate}
\end{example}

\begin{example}
\label{example-associative-algebra-gives-Lie-algebra}
If $A$ is an associative Lie algebra, then $A$ with the bracket  $[a,b]=ab-ba$
is a Lie algebra, denoted by  $A_-$. (It is an exercise to verify Jacobi's
identity.) This gives  $\mathfrak{gl}_V:=\text{End}(V)_-$.
\end{example}

\begin{definition}
\label{definition-Lie-subalgebra}
A {\it Lie subalgebra} is a vector subspace $\mathfrak{h} \subset \mathfrak{g}$
such that $[\mathfrak{h},\mathfrak{h}] \subset \mathfrak{h}$.
\end{definition}

\begin{definition}
\label{definition-ideal}
A subspace $\mathfrak{h}\subset \mathfrak{g}$ is called an {\it ideal} if
$[\mathfrak{h},\mathfrak{g}]\subset\mathfrak{h}$.
\end{definition}

\medskip\noindent
Recall that a bilinear form
$(\cdot,\cdot):\mathfrak{g}\times\mathfrak{g}\to \mathbb{C}$ is {\it invariant}
if $([a,b],c])=(a,[b,c])$ for all  $a,b,c \in \mathfrak{g}$.

\begin{definition}
\label{definition-ad}
For any finite dimensional Lie algebra  $\mathfrak{g}$ and $x \in \mathfrak{g}$
we write
$$
\text{ad}_a=[a,-]
$$
\end{definition}

\begin{exercise}
\label{exercise-adjoint-representation}
Show that the map $\text{ad}:\mathfrak{g}\to \text{End}(\mathfrak{g})$,
$a \mapsto \text{ad}_a$ is a representation, 
i.e. a morphism of Lie algebras.
\end{exercise}

\begin{definition}
\label{definition-Killing-form}
The {\it Killing form} is
$$
\kappa(x,y)=\text{Tr}_{\mathfrak{g}}\text{ad}x\text{ad}y
$$
\end{definition}

\begin{exercise}
\label{exercise-invariant-bilinear-form-on-simple-Lie-algebra-is-symmetric}
Prove that an invariant bilinear form on a simple Lie algebra must in fact be
symmetric.
\end{exercise}

\begin{proof}
(David.) It's enough to show that $\mathfrak{g}$ is
{\it perfect}, i.e. that $[\mathfrak{g},\mathfrak{g}]=\mathfrak{g}$. In this
case, let $a,b \in \mathfrak{g}$ and suppose that $b=[x,y]$. Then
\begin{align*}
(a,b)&=(a,[x,y])=(a,-[y,x])=(-[a,y],x)=([y,a],x)\\
&=(y,[a,x])=(y,-[x,a])=(-[y,x],a)=([x,y],a)=(b,a)
\end{align*}
To confirm that $\mathfrak{g}$ is perfect just observe that
$[\mathfrak{g},\mathfrak{g}]$ is a nontrivial ideal of $\mathfrak{g}$.
\end{proof}

\begin{definition}
\label{definition-semisimple-Lie-algebra}
A {\it semisimple Lie algebra} is a direct sum of simple ones
$$
\mathfrak{g}=\mathfrak{g}_1\oplus\ldots\oplus\mathfrak{g}_s
$$
\end{definition}

\begin{theorem}[Cartan]
\label{theorem-finite-dimensional-Lie-algebra-is
-semisimple-iff-Killing-form-is-nondegenerate}
The finite dimensional Lie algebra $\mathfrak{g}$ is semisimple if and only if
its Killing form is nondegenerate.
\end{theorem}

\section{Nilpotent and solvable Lie algebras}
\label{section-nilpotent-and-solvable-Lie-algebras}

Let $\mathfrak{g}$ be a Lie algebra. Define
$$
\mathfrak{g}^1:=[\mathfrak{g},\mathfrak{g}],\quad 
\mathfrak{g}^2:=[\mathfrak{g},\mathfrak{g}^1],\quad 
\ldots,
\mathfrak{g}^n:=[\mathfrak{g},\mathfrak{g}^{n-1}]
$$
and
$$
\mathfrak{g}^{(1)}:=[\mathfrak{g},\mathfrak{g}],\quad 
\mathfrak{g}^{(2)}:=[\mathfrak{g}^{(1)},\mathfrak{g}^{(1)}],
\ldots,
\mathfrak{g}^{(n)}:=[\mathfrak{g}^{(n-1)},\mathfrak{g}^{(n-1)}].
$$

\begin{definition}
\label{definition-nilpotent-and-solvable-Lie-algebras}
A Lie algebra $\mathfrak{g}$ is
\begin{itemize}
\item {\it nilpotent} if there exists $n$ such that $\mathfrak{g}^n=0$,
\item {\it solvable} is there exists $n$ such that $\mathfrak{g}^{(n)}=0$.
\end{itemize}
\end{definition}

\section{Lie's theorem}
\label{section-Lie-theorem}

Lie's theorem says that solvable Lie algebras over closed fields of
characterstic not zero have weights.

\begin{definition}
\label{definition-weight-space}
Let $\mathfrak{h}$ be a Lie algebra, 
$\pi:\mathfrak{h}\to\mathfrak{gl}_V$ a representation of $\mathfrak{h}$ 
and $\lambda \in \mathfrak{h}^*$. The {\it weight space} of $\mathfrak{h}$
attached to $\lambda$ is
$$
V_\lambda^{\mathfrak{h}}:=\{v \in V|\pi(h)v=\lambda(h)v,\; 
\forall h \in \mathfrak{h}\}.
$$
If $V_\lambda^{\mathfrak{h}}\neq 0$, we say that $\lambda$ is a {\it weight} for
$\pi$.
\end{definition}

\begin{theorem}[Lie's theorem]
\label{theorem-Lie}
Let $\mathfrak{g}$ be a solvable Lie algebra and $\pi$ a representation of
$\mathfrak{g}$ on a finite dimensional vector space $V \neq 0$, over an
algebraically closed field $\mathbb{F}$ of characterstic 0. 
Then there exists a weight $\lambda \in \mathfrak{g}^*$ for $\pi$ 
(that is, $V_\lambda^\mathfrak{g} \neq \{0\}$).
\end{theorem}

\begin{exercise}
\label{exercise-corollaries-of-Lie-theorem}
Show the following two corollaries of Lie's theorem:
\begin{itemize}
\item for all representations $\pi$ of a solvable Lie algebra $\mathfrak{g}$ on
a finite dimensional vector space $\mathcal{V}$ over an algebraically closed
field $\mathbb{F}$, $\text{char}\mathbb{F}=0$, 
there exists a basis for $V$ for which the matrices
of $\pi(\mathfrak{g})$ are upper triangular;
\item a solvable subalgebra $\mathfrak{g} \subset \mathfrak{gl}_V$,
(where $V$ is finite-dimensional over an algebraically closed field
$\mathbb{F}$, $\text{char}\mathbb{F}=0$),
is contained in the subalgebra of upper triangular matrices
over $\mathbb{F}$ for some basis of $V$.
\end{itemize}
\end{exercise}

\section{Generalized Eigenspaces and Generalized Weight spaces}
\label{section-generalized-eigenspaces-and-generalized-weight-spaces}

\begin{definition}
\label{definition-generalized-eigenspace}
A {\it generalized eigenspace of $A \in \text{End}(V)$ 
with eigenvalue $\lambda$} is
$$
V_\lambda=\{v \in V:(A-\lambda \text{Id})^N=0
\text{ for some positive integer }N\}.
$$
\end{definition}

It turns out that any linear operator on an algebraically closed field gives a
decomposition into generalized eigenspaces via Jordan canonical form:

\begin{proposition}
\label{proposition-generalized-eigenspace-decomposition}
Let $A$ be a linear operator on a finite-dimensional
vector space $V$ over an algebraically closed
field $\mathbb{F}$, and let $\lambda_1,\ldots,\lambda_s$ 
be all eigenvalues of $A$, and $n_1,\ldots,n_s$ their 
multiplicities.
Then one has the generalized eigenspace decomposition:
$$
V=\bigoplus_{i=1}^s V_{\lambda_i},\qquad \dim V_{\lambda_i}=n_i.
$$
\end{proposition}

In particular, for a Lie algebra $\mathfrak{g}$ with a
representation 
(i.e. Lie algebra morphism)
$\pi:\mathfrak{g} \to \text{End}(V)$ we have
$$
V=\bigoplus_{\lambda \in \mathbb{F}}V^a_\lambda,\qquad 
V^a_\lambda=\{v \in V:(\pi(a)v-\lambda \text{Id})^Nv=0\text{ for some }
N \in \mathbb{N}\}
$$
And even more particularly, for the adjoint representation we have
$$
V=\bigoplus_{\lambda \in \mathbb{F}}\mathfrak{g}^a_\lambda,\qquad 
\mathfrak{g}_\alpha^a=\{b \in \mathfrak{g}:([a,\cdot]-\alpha \text{Id})^Nb=0,
\text{ for some }N \in \mathbb{N}\}
$$
And a little more generally we have
\begin{definition}
\label{definition-generalized-weight-space}
Let $\mathfrak{h}$ be a Lie algebra with a representation $\pi$ on a vector
space $V$, and $\lambda \in \mathfrak{h}^*$.
A {\it generalized weight space of $\mathfrak{h}$ in $V$ attached to $\lambda$} 
is
$$
V_\lambda^\mathfrak{h}=
\left\{v \in V: (\pi(a)-\lambda(a)\text{Id})^Nv=0,
\substack{ \text{ for some }N \in \mathbb{N}\\
\text{ depending on }a \in \mathfrak{h}\\
\text{for all }a \in \mathfrak{h}}\right\}
$$
\end{definition}

Under the right conditions, a nilpotent subalgebra 
$\mathfrak{h} \subset \mathfrak{g}$ permits
decomposing $V$ as a direct sum of generalized weight
spaces of $\mathfrak{h}$. Namely,

\begin{theorem}
\label{theorem-decomposition-into-generalized-weight-spaces}
Let $\mathfrak{g}$ be a finite-dimensional Lie algebra
and $\pi$ its representation on a finite-dimensional vector
space $V$, over an algebraically closed field
$\mathbb{F}$ of characteristic 0.
Let $\mathfrak{h}$ be a nilpotent subalgebra of $\mathfrak{g}$.
Then the following equalities hold:
$$
V=\bigoplus_{\lambda \in \mathfrak{h}^*}V_\lambda^\mathfrak{h}
$$
\begin{equation}
\label{equation-inclusion-for-general-representation}
\pi\left(\mathfrak{g}^\mathfrak{h}_\alpha\right)V_\lambda^\mathfrak{h}
\subseteq V_{\lambda+\alpha}^\mathfrak{h}
\end{equation}
\end{theorem}

Which in the case of the adjoint representation, looks like:

\begin{definition}
\label{definition-generalized-root-space-decomposition}
The {\it generalized root space decomposition} of a Lie algebra 
$\mathfrak{g}$ over an algebraically closed field $\mathbb{F}$ 
of characteristic zero with respect to a nilpotent subalgebra $\mathfrak{h}$
is the generalized weight space decomposition with respect to the
adjoint representation. That is,
$$
\mathfrak{g}=\bigoplus_{\alpha \in \mathfrak{h}^*}
\mathfrak{g}_\alpha^\mathfrak{h}
$$
and it has the property that
\begin{equation}
\label{equation-inclusion-for-adjoint-representation}
[\mathfrak{g}_\alpha^\mathfrak{h},\mathfrak{g}_\beta^\mathfrak{h}]
\subseteq\mathfrak{g}_{\alpha+\beta}^\mathfrak{h}
\end{equation}
\end{definition}

It is important to make the distinction between the generalized weight
space decomposition and the generalized root space decomposition;
``we will see its convenience in later lectures, as 
we try to better understand the functionals $\alpha$ appearing
in the decomposition''.

\begin{exercise}
\label{exercise-generalized-weight-space-decomposition-for-gl}
Take $\mathfrak{g}=\mathfrak{gl}_n(\mathbb{F})$ and 
$\mathfrak{h}=\{\text{diagonal matrices}\}$.
Find the generalized weight space decomposition in both the 
tautological and the adjoint representations,
and check the inclusions \ref{equation-inclusion-for-general-representation} and
\ref{equation-inclusion-for-adjoint-representation}. 
\end{exercise}

\section{Zarisky Topology and Regular Elements}
\label{section-Zarisky-topology-and-regular-elements}

Let $\mathfrak{g}$ be a finite-dimensional Lie algebra
of dimension $d$. For every element  $a \in \mathfrak{g}$
the characeristic polynomial
of $\text{ad}_a$ must be of the form
$$
\det_{\mathfrak{g}}(\text{ad}_a-\lambda\text{Id})
=(-\lambda)^d+c_{d-1}(-\lambda)^{d-1}+\ldots
+\det(\text{ad}_a).
$$
[to be honest I don't see why the constant
term must be de determinant of $\text{ad}_a$, but OK…]
since $[a,a]=0$, $\det(\text{ad}_a)=0$, i.e., the
constant term vanishes.

\begin{exercise}
\label{exercise-characteristic-polynomial-of-ada}
Show that $c_j$ is a homogeneous polynomial on $\mathfrak{g}$ 
of degree $d-j$.
\end{exercise}

\begin{proof}
The term ``polynomial on $\mathfrak{g}$'' is not clear.
But we follow \cite{KLAL} to interpret this is only as
being a polynomial on the $a^i$ where $a=a^iX_i$
for some basis $X_i$ of $\mathfrak{g}$.

Then we just notice that denoting $[X_i,X_j]:=L_{ij}$ we have
$$
[a,X_j]=[a^iX_i,X_j]=a^i[X_i,X_j]=a^iL_{ij}
$$
which is a linear combination of the $a^i$. Then the matrix
representation of $\text{ad}_a$ in terms of the basis $X_i$ is
$$
[\text{ad}_a]=\begin{pmatrix}
a^iL_{i1}^1&\cdots &a^iL_{in}^1\\
\vdots &&\vdots \\
a^iL_{i 1}^n&  \cdots &  a^iL_{in}^n
\end{pmatrix}
$$
substracting $\lambda \text{Id}$ and taking determinant
we obtain that a term with $\lambda^k$ would have 
for coefficient a product of $k$ of the linear
combinations $a^iL_{i\ell}^j$ for varying $j$ and $\ell$.
Such a product is understood to be a homogeneous polynomial
of degree $k$ in $\mathfrak{g}$.
\end{proof}

\begin{definition}
\label{definition-rank}
I think this goes for any Lie algebra $\mathfrak{g}$:
\begin{itemize}
\item The {\it rank} of $\mathfrak{g}$ is
the smallest integer $r$ such that $c_r(a)$ is not the 
zero polynomial on $\mathfrak{g}$.
\item An element $a \in \mathfrak{g}$ is called {\it regular} 
if $c_r(a)\neq 0$.
\item The {\it discriminant} of $\mathfrak{g}$ is
the nonzero polynomial $c_r(a)$ of degree $d-r$, what? 
\end{itemize}
\end{definition}

Explanation: we compute the polynomial characteristic w.r.t.
$\text{ad}_a$ for every $a \in \mathfrak{g}$.
Express this polynomial as a polynomial with coefficients in $\mathfrak{g}[t]$,
(here's the question: this polynomial seems to me to be a polynomial
in $\mathbb{F}$.)
This polynomial does not have a constant term.
But what is the next smallest-degree monomial? 1? 2?
That's the rank of the Lie algebra.

\section{Cartan subalgebra}
\label{section-Cartan-subalgebra}

Recall that $\mathfrak{h}\subset\mathfrak{g}$ is an ideal
if $[a,\mathfrak{h}]\subset\mathfrak{h}$ for all $a \in \mathfrak{g}$.
Maybe $\mathfrak{h}$ is not an ideal, but we 
can consider the largest subalgebra of $\mathfrak{g}$ 
where $\mathfrak{h}$ is an ideal.
This is called the normalizer.

\begin{definition}
\label{definition-normalizer}
Let $\mathfrak{h}$ be a subalgebra of a Lie algebra $\mathfrak{g}$.
The {\it normalizer} of $\mathfrak{h}$
is 
$N_\mathfrak{g}(\mathfrak{h}):=\{a \in \mathfrak{g}
|[a,\mathfrak{h}]\subset\mathfrak{h}\}$
\end{definition}

I think behind the Cartan subalgebra is that, roughly,
taking product with
anything not in $\mathfrak{h}$ leaves $\mathfrak{h}$.

\begin{definition}
\label{definition-Cartan-subalgebra}
A {\it Cartan subalgebra} of a Lie algebra $\mathfrak{g}$ is a subalgebra
$\mathfrak{h}$, satisfying the following two conditions:
\begin{enumerate}
\item $\mathfrak{h}$ is a nilpotent Lie algebra,
\item $N_\mathfrak{g}(\mathfrak{h})=\mathfrak{h}$.
\end{enumerate}
\end{definition}

\begin{proposition}
\label{proposition-diagonal-matrices-are-Cartan-subalgebra}
Let $\mathfrak{g}\subset \mathfrak{gl}_n(\mathbb{F})$ be a subalgebra
containing a diagonal matrix $a=\text{diag}(a_1,\ldots,a_n)$ 
with distinct $a_i$, and let $\mathfrak{h}$ be the subspace
of all diagonal matrices in $\mathfrak{g}$. Then $\mathfrak{h}$ 
is a Cartan subalgebra (of $\mathfrak{g}$ I guess).
\end{proposition}

\begin{proof}[Idea of proof]
Consider the following illustrative example:
\begin{align*}
&\begin{pmatrix}
a&b\\ 
c&d
\end{pmatrix}\begin{pmatrix}
\lambda_1&0\\ 
0&\lambda_2
\end{pmatrix}-\begin{pmatrix}
\lambda_1&0\\ 
0&\lambda_2
\end{pmatrix}\begin{pmatrix}
a&b\\ 
c&d
\end{pmatrix}\\
&=\begin{pmatrix}
a \lambda_1&b \lambda_2\\ 
c \lambda_3&d \lambda_2
\end{pmatrix}-
\begin{pmatrix}
\lambda_1a&\lambda_1b\\ 
\lambda_2c&\lambda_2d
\end{pmatrix}\\
&=\begin{pmatrix}
0&b\lambda_2-\lambda_1b\\ 
\lambda_3a-\lambda_2c&0
\end{pmatrix}\qquad 
\end{align*}
which says that any non-diagonal matrix would escape $\mathfrak{h}$.
\end{proof}

\begin{theorem}[Cartan]
\label{theorem-finite-dimensional-Lie-algebra-over-closed-field-has
-Cartan-subalgebra}
Let $\mathfrak{g}$ be a finite-dimensional Lie algebra over an 
algebraically closed field $\mathbb{F}$. Let $a \in \mathfrak{g}$ 
be a regular element (which exists since $\mathbb{F}$ is infinite),
and let $\mathfrak{g}=\bigoplus_{\lambda \in \mathbb{F}}\mathfrak{g}^a_\lambda$
be the generalized eigenspace decomposition of $\mathfrak{g}$ 
with respect to $\text{ad}_a$. Then $\mathfrak{g}_0^a$ 
is a Cartan subalgebra.
\end{theorem}

\begin{proof}[Idea of proof]
First recall what is $\mathfrak{h}=\mathfrak{g}_0^a$, the set of elements
$b\in\mathfrak{g}$ such that $\text{ad}_a^N(b)=0$.
\end{proof}

\begin{remark}
\label{remark-rank-is-dimension-of-that-subalgebra}
The rank of a Lie algebra is the dimension of $\mathfrak{g}_0^a=\mathfrak{h}$.
\end{remark}

\begin{proposition}
\label{proposition-g0-equals-h-for-any-Cartan-subalgebra-h}
Let $\mathfrak{g}$ be a finite-dimensional Lie algebra over
an algebraically closed field $\mathbb{F}$ of characteristic zero
and let $\mathfrak{h}\subset\mathfrak{g}$ be a Cartan subalgebra.
Then $\mathfrak{g}_0=\mathfrak{h}$ in the generalized 
weight space decomposition.
\end{proposition}

\begin{proof}[Idea of proof]
Engel's theorem…
\end{proof}



\section{Semisimple Lie algebras}
\label{section-semisimple-Lie-algebras}

\begin{definition}
\label{definition-radical}
A {\it radical} $R(\mathfrak{g})$ of a finite-dimensional Lie algebra
$\mathfrak{g}$ is a solvable ideal of $\mathfrak{g}$ of maximal possible
dimension.
\end{definition}

\begin{proposition}
\label{proposition-radical-ideal-contains-any-solvable-ideal-and-is-unique}
The radical ideal of $\mathfrak{g}$ contains any solvable ideal of
$\mathfrak{g}$ and is unique.
\end{proposition}

If $\mathfrak{g}$ is a finite dimensional solvable Lie algebra,
then $R(\mathfrak{g})=\mathfrak{g}$. 
The opposite case if when $R(\mathfrak{g})=0$.

\begin{definition}
\label{definition-semisimple-Lie-algebra}
A finite-dimensional Lie algebra $\mathfrak{g}$ is called {\it semisimple} if
$R(\mathfrak{g})=0$.
\end{definition}

In \cite[Lecture 11]{KLAL} we have the tools needed for a characterization
of semisimple Lie algebras in terms of the Killing form:

\begin{theorem}[Cartan]
\label{theorem-Cartan}
Let $\mathfrak{g}$ be a finite-dimensional Lie algebra
over a field of characteristic 0.
Then the Killing form on $\mathfrak{g}$ is non-degenerate is and only if
$\mathfrak{g}$ is semisimple.
Moreover, if $\mathfrak{g}$ is semisimple and $\mathfrak{a}\subset\mathfrak{g}$
is an ideal,
then the restriction of the Killing form to $\mathfrak{a}$,
$K|_{\mathfrak{a}\times\mathfrak{a}}$,
is also nondegenerate and coincides with the Killing form of $\mathfrak{a}$.
\end{theorem}

I think we need semisimplicity, i.e. nondegeneracy of the
Killing form for the following definition:

\begin{definition}
\label{definition-coroot}
Given a set of simple roots $\Pi=\{\alpha_1,\ldots,\alpha_\ell\}$,
we define the {\it coroot} 
\begin{equation}
\label{equation-coroot}
\alpha_i^\vee=\frac{2}{(\alpha_i,\alpha_i)}\nu^{-1}(\alpha_i)
\end{equation}
where $\nu:\mathfrak{h}\to\mathfrak{h}^*$ is the pairing
induced by the Killing form.
\end{definition}

\section{$\mathfrak{g}$-modules}
\label{section-g-modules}

\begin{definition}
\label{definition-g-module}
Let $\mathfrak{g}$ be a Lie algebra. A {\it $\mathfrak{g}$-module}
(also called a {\it representation of $\mathfrak{g}$}
is a vector space $V$ and a homomorphism
$$
\rho:\mathfrak{g}\to \text{End}(V)
$$
of Lie algebras, i.e.,
$$
\rho([x,y])=\rho(x)\rho(y)-\rho(y)\rho(x).
$$
A vector space $W \subset V$ such that $vW\subset W$ for all
$x \in \mathfrak{g}$ is called a
 {\it $\mathfrak{g}$-submodule}.
\end{definition}

\begin{definition}
\label{definition-irreducible-g-module}
An {\it irreducible $\mathfrak{g}$-module} is one that has
no $\mathfrak{g}$-submodules other than the
trivial ones (namely, 0 and $V$ itself).
\end{definition}

It is possible to classify and describe the 
irreducible finite-dimensional
$\mathfrak{g}$-modules. This is in contrast
with all the modules of a Lie algebra $\mathfrak{g}$,
which is ``impossible in general''.

\section{Verma module}
\label{section-Verma-module}

Let $M$ be a $\mathfrak{g}$-module. Recall 
Definition \ref{definition-weight-space}: 
let $\mathfrak{h}$ be a Lie algebra, 
$\pi:\mathfrak{h}\to\mathfrak{gl}_V$ a representation of $\mathfrak{h}$ 
and $\lambda \in \mathfrak{h}^*$. The {\it weight space} of $\mathfrak{h}$
attached to $\lambda$ is
$$
V_\lambda^{\mathfrak{h}}:=\{v \in V|\pi(h)v=\lambda(h)v,\; 
\forall h \in \mathfrak{h}\}.
$$
We have
that a weight space in $M$ is a nonzero subspace
$M_\lambda \subset M$ such that
$$
hv=\lambda(h)v \qquad \forall h \in \mathfrak{h}\subset\mathfrak{g}
\text{ and }v \in M,
$$
and in this case we call $\lambda$ a weight.
If $M$ is a direct sum of weight spaces, we
call it a {\it weight module}.

The {\it support} of a weight module $M$ is
\begin{equation}
\label{equation-support-of-weight-module}
\text{supp}(M):=\{\lambda \in \mathfrak{h}^*:M_\lambda\neq 0\}
\end{equation}
We call elements of $M_\lambda$ {\it vectors
of weight $\lambda$}.

\begin{definition}
\label{definition-highest-root}
$\Lambda \in \mathfrak{h}^*$ is a 
{\it highest weight} of $M$ 
if $\Lambda \in \text{supp}(M)$
and for all positive roots $\alpha \in \Delta_+$
we have that $\Lambda+\alpha \not\in\text{supp}(M)$.
\end{definition}

\begin{exercise}
\label{exercise-highest-weight-of-an-irreducible-g-module-is-unique}
The highest weight of an irreducible $\mathfrak{g}$-module
is unique.
\end{exercise}

\begin{proof}

\end{proof}

\begin{definition}
\label{definition-Verma-module}
\begin{reference}
\cite[Definition 4.1]{jethro-rt}
\end{reference}
Let $\Lambda \in \mathfrak{h}^*$
(a highest weight, I suppose).
The {\it Verma module} $M(\Lambda)$
of highest weight $\Lambda$ is
$$
M(\Lambda)=U(\mathfrak{g})\otimes_{U(\mathfrak{h}\oplus \mathfrak{n}_+}
\mathbb{C}_\Lambda,
$$
where $\mathbb{C}_\Lambda=\mathbb{C}v_\Lambda$
is the $1$-dimensional $(\mathfrak{h} \oplus \mathfrak{n}_+)$-module
defined by $\mathfrak{n}_+v_\Lambda=0$
and $hv_\Lambda=\left<\Lambda,h\right>v_\Lambda$
for all $h \in \mathfrak{h}$.
\end{definition}

Long story short, by the PBW Theorem we
have an isomorphism $U(\mathfrak{n}_-) \to M(\Lambda)$
of $U(\mathfrak{n}_-)$ modules.

\section{Weyl character formula}
\label{section-Weyl-character-formula}

There are at least two Weyl character formulas.
This one is \cite[Theorem 10.14]{Hall}:

\begin{theorem}[Weyl Character Formula]
\label{theorem-Weyl-character-formula-Hall}
If $(\pi,V_\mu)$ is an irreducible representation of
$\mathfrak{g}$ with highest weight $\mu$, then
\begin{equation}
\label{equation-Weyl-character-formula-Hall}
\chi_\pi(H)=\frac{\sum_{w \in W}\det(w)e^{\left<w\cdot(\mu+\delta),H\right>}}
{\sum_{w \in W}\det(w)e^{\left<w\cdot \delta,H\right>}}
\end{equation}
for all $H \in \mathfrak{h}$ for which the denominator is nonzero.
\end{theorem}

And this one is \cite[Theorem 25.3]{KLAL}:

\begin{theorem}[Weyl Character Formula]
\label{theorem-Weyl-character-formula-KLAL}
Let $R=\prod_{\alpha \in \Delta_+}(1-e^{-\alpha}$.
If  $\Lambda \in P_+$, then
\begin{equation}
\label{equation-Weyl-character-formula-KLAL}
e^{\rho}R \text{ch}L(\Lambda)=\sum_{w \in W}\det w e^{w(\Lambda+\rho)}
\end{equation}
\end{theorem}

\begin{exercise}
\label{exercise-Weyl-character-formula-in-sl3}
Let $\mathfrak{g}=\mathfrak{sl}_3$. Compute $\text{ch}_{L(\lambda)}$
for some $\lambda$,
\begin{enumerate}
\item $\lambda=0$,
\item $\lambda=\omega_1$,
\item $\lambda=\omega_1+\omega_2$,
\end{enumerate}
etc.
\end{exercise}

\begin{proof}
No matter what formula I use, I must compute
the determinants of the Weyl reflections
and the $\rho$ (which I think is $\delta$ for Hall),
which is just the sum of all the fundamental weights,
which in turn are the dual of the coroot vectors.
Finally applying the reflections to $\rho$ and
to $\lambda+\rho$ would yield the result.
\begin{enumerate}
\item (Find roots of $\mathfrak{sl}_3$.) We look for $\alpha \in \mathfrak{h}^*
= \{\text{diagonal matrices}\}$ such that
$$
[H,E_{ij}]=\alpha(H)E_{ij}\qquad \forall H \in \mathfrak{h}
$$
where $E_{ij}$ is the matrix that has zero in every entry but 
in the $(i,j)$-th where it has a 1. One obtains that
$[H,E_{ij}]=(h_i-h_j)E_{ij}$, so that the roots are
$\alpha_{ij}(h)=h_i-h_j$. 

\item (Compute the Killing form.) Recall that by definition
$\kappa(H,H')=\text{Tr}(\text{ad}_H\text{ad}_{H'})$.
But in this case, we have $\kappa(H,H')=\text{Tr}(HH')$ 
because $\text{ad}_H\text{ad}_{H'}=[H,[H',A]]$

The first thing Jethro did was to consider a basis
$$
H_1=\begin{pmatrix}
1&0&0\\ 
0&-1&0\\
0&0&0
\end{pmatrix},
\qquad
H_2=\begin{pmatrix}
0&0&0\\ 
0&1&0\\
0&0&-1
\end{pmatrix}
$$
and then compute the Killing form via
\begin{align*}
H_1H_1&=2,\qquad H_1H_2=-1,\qquad H_2H_1=-1,\qquad H_2H_2=2.
\end{align*}

\item (Find the dual basis vectors.) Now we use the musical
isomorphism $\nu:\mathfrak{h}\to \mathfrak{h}^*$, 
$H\mapsto (-,H)$ to find
$$
\nu(H_1)=
$$
\item (Find fundamental weights.) It looks like $\alpha_1=2 \omega_1-\omega_2$
and $\alpha_2=2\omega_2-\omega_1$. This says 
$\omega_1=2\omega_2-\alpha_2=2(2\omega_1-\alpha_1)-\alpha_2$ so
$3\omega_1=2\alpha_1+\alpha_2$. And then 
$\omega_2=2(\frac{2}{3}\alpha_1+\frac{1}{3}\alpha_2)-\alpha_1
=\frac{1}{3}(\alpha_1+2\alpha_2)$.

\item (Find $\rho$.) Then $\rho=\omega_1+\omega_2=\alpha_1+\alpha_2$.

\item (Find Weyl reflections.) To find Weyl reflections I need to compute
first the numbers $(\alpha_i, \alpha_j)$. I think this is still not
very clear --- have to use Killing form? But OK, I know we must have
$(\alpha_i,\alpha_i)=2$, and the both of the crossed ones give $-1$.
This should give by bilinearity the values on $\omega_i$ also.

\item (Find determinants of Weyl reflections.) I guess for this
I have to evaluate the reflections in each of the w

\item (Evaluate $\rho$ and $\lambda+\rho$ on Weyl reflections.)

\item (Find the Weyl character.)
\end{enumerate}

What's funny in this whole story is that
in the end we can just find the Weyl character geometrically.
For every $\lambda=m \omega_1+n \omega_2$
I want to compute its orbit under the Weyl group $W$.
This means 


I would start by finding the determinant 
of the Weyl reflections…
\end{proof}

\bibliography{my}
\bibliographystyle{amsalpha}

\end{document}

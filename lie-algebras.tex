\IfFileExists{stacks-project.cls}{%
\documentclass{stacks-project}
}{%
\documentclass{amsart}
}

% For dealing with references we use the comment environment
\usepackage{verbatim}
\newenvironment{reference}{\comment}{\endcomment}
%\newenvironment{reference}{}{}
\newenvironment{slogan}{\comment}{\endcomment}
\newenvironment{history}{\comment}{\endcomment}

% For commutative diagrams we use Xy-pic
\usepackage[all]{xy}

% We use 2cell for 2-commutative diagrams.
\xyoption{2cell}
\UseAllTwocells

% We use multicol for the list of chapters between chapters
\usepackage{multicol}

% This is generally recommended for better output
\usepackage{lmodern}
\usepackage[T1]{fontenc}

% For cross-file-references
\usepackage{xr-hyper}

% Package for hypertext links:
\usepackage{hyperref}

% For any local file, say "hello.tex" you want to link to please
% use \externaldocument[hello-]{hello}
\externaldocument[introduction-]{introduction}
\externaldocument[conventions-]{conventions}
\externaldocument[sets-]{sets}
\externaldocument[categories-]{categories}
\externaldocument[topology-]{topology}
\externaldocument[sheaves-]{sheaves}
\externaldocument[sites-]{sites}
\externaldocument[stacks-]{stacks}
\externaldocument[fields-]{fields}
\externaldocument[algebra-]{algebra}
\externaldocument[brauer-]{brauer}
\externaldocument[homology-]{homology}
\externaldocument[derived-]{derived}
\externaldocument[simplicial-]{simplicial}
\externaldocument[more-algebra-]{more-algebra}
\externaldocument[smoothing-]{smoothing}
\externaldocument[modules-]{modules}
\externaldocument[sites-modules-]{sites-modules}
\externaldocument[injectives-]{injectives}
\externaldocument[cohomology-]{cohomology}
\externaldocument[sites-cohomology-]{sites-cohomology}
\externaldocument[dga-]{dga}
\externaldocument[dpa-]{dpa}
\externaldocument[sdga-]{sdga}
\externaldocument[hypercovering-]{hypercovering}
\externaldocument[schemes-]{schemes}
\externaldocument[constructions-]{constructions}
\externaldocument[properties-]{properties}
\externaldocument[morphisms-]{morphisms}
\externaldocument[coherent-]{coherent}
\externaldocument[divisors-]{divisors}
\externaldocument[limits-]{limits}
\externaldocument[varieties-]{varieties}
\externaldocument[topologies-]{topologies}
\externaldocument[descent-]{descent}
\externaldocument[perfect-]{perfect}
\externaldocument[more-morphisms-]{more-morphisms}
\externaldocument[flat-]{flat}
\externaldocument[groupoids-]{groupoids}
\externaldocument[more-groupoids-]{more-groupoids}
\externaldocument[etale-]{etale}
\externaldocument[chow-]{chow}
\externaldocument[intersection-]{intersection}
\externaldocument[pic-]{pic}
\externaldocument[weil-]{weil}
\externaldocument[adequate-]{adequate}
\externaldocument[dualizing-]{dualizing}
\externaldocument[duality-]{duality}
\externaldocument[discriminant-]{discriminant}
\externaldocument[derham-]{derham}
\externaldocument[local-cohomology-]{local-cohomology}
\externaldocument[algebraization-]{algebraization}
\externaldocument[curves-]{curves}
\externaldocument[resolve-]{resolve}
\externaldocument[models-]{models}
\externaldocument[functors-]{functors}
\externaldocument[equiv-]{equiv}
\externaldocument[pione-]{pione}
\externaldocument[etale-cohomology-]{etale-cohomology}
\externaldocument[proetale-]{proetale}
\externaldocument[relative-cycles-]{relative-cycles}
\externaldocument[more-etale-]{more-etale}
\externaldocument[trace-]{trace}
\externaldocument[crystalline-]{crystalline}
\externaldocument[spaces-]{spaces}
\externaldocument[spaces-properties-]{spaces-properties}
\externaldocument[spaces-morphisms-]{spaces-morphisms}
\externaldocument[decent-spaces-]{decent-spaces}
\externaldocument[spaces-cohomology-]{spaces-cohomology}
\externaldocument[spaces-limits-]{spaces-limits}
\externaldocument[spaces-divisors-]{spaces-divisors}
\externaldocument[spaces-over-fields-]{spaces-over-fields}
\externaldocument[spaces-topologies-]{spaces-topologies}
\externaldocument[spaces-descent-]{spaces-descent}
\externaldocument[spaces-perfect-]{spaces-perfect}
\externaldocument[spaces-more-morphisms-]{spaces-more-morphisms}
\externaldocument[spaces-flat-]{spaces-flat}
\externaldocument[spaces-groupoids-]{spaces-groupoids}
\externaldocument[spaces-more-groupoids-]{spaces-more-groupoids}
\externaldocument[bootstrap-]{bootstrap}
\externaldocument[spaces-pushouts-]{spaces-pushouts}
\externaldocument[spaces-chow-]{spaces-chow}
\externaldocument[groupoids-quotients-]{groupoids-quotients}
\externaldocument[spaces-more-cohomology-]{spaces-more-cohomology}
\externaldocument[spaces-simplicial-]{spaces-simplicial}
\externaldocument[spaces-duality-]{spaces-duality}
\externaldocument[formal-spaces-]{formal-spaces}
\externaldocument[restricted-]{restricted}
\externaldocument[spaces-resolve-]{spaces-resolve}
\externaldocument[formal-defos-]{formal-defos}
\externaldocument[defos-]{defos}
\externaldocument[cotangent-]{cotangent}
\externaldocument[examples-defos-]{examples-defos}
\externaldocument[algebraic-]{algebraic}
\externaldocument[examples-stacks-]{examples-stacks}
\externaldocument[stacks-sheaves-]{stacks-sheaves}
\externaldocument[criteria-]{criteria}
\externaldocument[artin-]{artin}
\externaldocument[quot-]{quot}
\externaldocument[stacks-properties-]{stacks-properties}
\externaldocument[stacks-morphisms-]{stacks-morphisms}
\externaldocument[stacks-limits-]{stacks-limits}
\externaldocument[stacks-cohomology-]{stacks-cohomology}
\externaldocument[stacks-perfect-]{stacks-perfect}
\externaldocument[stacks-introduction-]{stacks-introduction}
\externaldocument[stacks-more-morphisms-]{stacks-more-morphisms}
\externaldocument[stacks-geometry-]{stacks-geometry}
\externaldocument[moduli-]{moduli}
\externaldocument[moduli-curves-]{moduli-curves}
\externaldocument[examples-]{examples}
\externaldocument[exercises-]{exercises}
\externaldocument[guide-]{guide}
\externaldocument[desirables-]{desirables}
\externaldocument[coding-]{coding}
\externaldocument[obsolete-]{obsolete}
\externaldocument[fdl-]{fdl}
\externaldocument[index-]{index}

% Theorem environments.
%
\theoremstyle{plain}
\newtheorem{theorem}[subsection]{Theorem}
\newtheorem{proposition}[subsection]{Proposition}
\newtheorem{lemma}[subsection]{Lemma}

\theoremstyle{definition}
\newtheorem{definition}[subsection]{Definition}
\newtheorem{example}[subsection]{Example}
\newtheorem{exercise}[subsection]{Exercise}
\newtheorem{situation}[subsection]{Situation}

\theoremstyle{remark}
\newtheorem{remark}[subsection]{Remark}
\newtheorem{remarks}[subsection]{Remarks}

\numberwithin{equation}{subsection}

% Macros
%
\def\lim{\mathop{\mathrm{lim}}\nolimits}
\def\colim{\mathop{\mathrm{colim}}\nolimits}
\def\Spec{\mathop{\mathrm{Spec}}}
\def\Hom{\mathop{\mathrm{Hom}}\nolimits}
\def\Ext{\mathop{\mathrm{Ext}}\nolimits}
\def\SheafHom{\mathop{\mathcal{H}\!\mathit{om}}\nolimits}
\def\SheafExt{\mathop{\mathcal{E}\!\mathit{xt}}\nolimits}
\def\Sch{\mathit{Sch}}
\def\Mor{\mathop{\mathrm{Mor}}\nolimits}
\def\Ob{\mathop{\mathrm{Ob}}\nolimits}
\def\Sh{\mathop{\mathit{Sh}}\nolimits}
\def\NL{\mathop{N\!L}\nolimits}
\def\CH{\mathop{\mathrm{CH}}\nolimits}
\def\proetale{{pro\text{-}\acute{e}tale}}
\def\etale{{\acute{e}tale}}
\def\QCoh{\mathit{QCoh}}
\def\Ker{\mathop{\mathrm{Ker}}}
\def\Im{\mathop{\mathrm{Im}}}
\def\Coker{\mathop{\mathrm{Coker}}}
\def\Coim{\mathop{\mathrm{Coim}}}

% Boxtimes
%
\DeclareMathSymbol{\boxtimes}{\mathbin}{AMSa}{"02}

%
% Macros for moduli stacks/spaces
%
\def\QCohstack{\mathcal{QC}\!\mathit{oh}}
\def\Cohstack{\mathcal{C}\!\mathit{oh}}
\def\Spacesstack{\mathcal{S}\!\mathit{paces}}
\def\Quotfunctor{\mathrm{Quot}}
\def\Hilbfunctor{\mathrm{Hilb}}
\def\Curvesstack{\mathcal{C}\!\mathit{urves}}
\def\Polarizedstack{\mathcal{P}\!\mathit{olarized}}
\def\Complexesstack{\mathcal{C}\!\mathit{omplexes}}
% \Pic is the operator that assigns to X its picard group, usage \Pic(X)
% \Picardstack_{X/B} denotes the Picard stack of X over B
% \Picardfunctor_{X/B} denotes the Picard functor of X over B
\def\Pic{\mathop{\mathrm{Pic}}\nolimits}
\def\Picardstack{\mathcal{P}\!\mathit{ic}}
\def\Picardfunctor{\mathrm{Pic}}
\def\Deformationcategory{\mathcal{D}\!\mathit{ef}}

%Dani's additions
\usepackage{graphicx} %figures


\begin{document}

\title{Lie algebras}
\maketitle

\phantomsection
\label{section-phantom}
\hfill
\href{http://github.com/danimalabares/stack}{github.com/danimalabares/stack}

\tableofcontents

\section{Basic definitions and examples}
\label{section-basic-definitions-and-examples}

\begin{definition}
\label{definition-algebra}
An {\it algebra} is a vector space over a field $\mathbb{F}$ endowed with a
binary operation that is bilinear
\begin{align*}
a(\lambda b+\mu c)&=\lambda ab+\mu ac\\
(\lambda b + \mu c) a &= \lambda b a + \mu c a
\end{align*}
\end{definition}

\begin{definition}
\label{definition-Lie-algebra}
A {\it Lie algebra} is an algebra $\mathfrak{g}$ 
with a product $[\cdot,\cdot]$ we call 
{\it bracket} that satisfies
\begin{enumerate}
\item $[x,x]=0\qquad \forall x\in \mathfrak{g}$,
\item (Jacobi identity.)
\end{enumerate}
\end{definition}

\begin{definition}
\label{definition-simple-Lie-algebra}
A {\it simple Lie algebra} is a Lie algebra that has no nontrivial proper
ideals.
\end{definition}

As a rule of thumb I keep in mind the following silly computation:
$$
\text{log}\det A=\text{log} \prod_{i}\lambda_i=\sum \text{log}\lambda_i
=``\text{tr}\text{log}A''
$$
And recall that exponent map goes from $T_eG= \mathfrak{g} \to G$, so that
logarithm would go from $G\to \mathfrak{g}$. This is why I remember that the
condition on a classical Lie group of 
{\it having determinant 1} goes to {\it having vanishing trace} in the 
Lie algebra. (Because $\text{log}1=0$.)

\begin{example}[Classical Lie algebras]
\label{example-classical-Lie-algebras}
\begin{enumerate}
\item The {\it special linear Lie algebra} 
$$
\mathfrak{sl}_n=\{\text{Mat}_n|Tr(A)=0\}
$$
which is just obvious from the slogan above.
\item The {\it special orthogonal Lie algebra} 
$$
\mathfrak{so}_n=\{A\in \text{Mat}_n|A+A^{\mathbf{T}}=0\}
$$
which is obvious from: $\text{SO}(n)=$isometries, so
$\left<v,v\right>=\left<Av,Av\right>=\left<v,A^{\mathbf{T}}Av\right>$, so
$\text{SO}(n)=\{A \in \text{Mat}_n:A^{-1}=A^{\mathbf{T}}\}$, and then
$0=
\text{log}1=\text{log}(A A^{\mathbf{T}})=\text{log}A+\text{log}A^{\mathbf{T}}$.

\item The {\it symplectic Lie algebra}
$$
\mathfrak{sp}_{2n}=\{A \in \text{Mat}_{2n}:\Omega A+A^{\mathbf{T}}\Omega=0\}
$$
where $\Omega=\begin{pmatrix}
0&\text{Id}_n\\ 
-\text{Id}_n & 0
\end{pmatrix}$.

{\bf Pending.} Why this makes sense?
\end{enumerate}
\end{example}

\begin{example}
\label{example-associative-algebra-gives-Lie-algebra}
If $A$ is an associative Lie algebra, then $A$ with the bracket  $[a,b]=ab-ba$
is a Lie algebra, denoted by  $A_-$. (It is an exercise to verify Jacobi's
identity.) This gives  $\mathfrak{gl}_V:=\text{End}(V)_-$.
\end{example}

\begin{definition}
\label{definition-Lie-subalgebra}
A {\it Lie subalgebra} is a vector subspace $\mathfrak{h} \subset \mathfrak{g}$
such that $[\mathfrak{h},\mathfrak{h}] \subset \mathfrak{h}$.
\end{definition}

\begin{definition}
\label{definition-ideal}
A subspace $\mathfrak{h}\subset \mathfrak{g}$ is called an {\it ideal} if
$[\mathfrak{h},\mathfrak{g}]\subset\mathfrak{h}$.
\end{definition}

\section{Killing form}
\label{section-Killing-form}

\begin{definition}
\label{definition-ad}
For any finite dimensional Lie algebra  $\mathfrak{g}$ and $x \in \mathfrak{g}$
we write
$$
\text{ad}(x)=[x,-]
$$
\end{definition}

\begin{definition}
\label{definition-Killing-form}
The {\it Killing form} is
$$
\kappa(x,y)=\text{Tr}_{\mathfrak{g}}\text{ad}x\text{ad}y
$$
\end{definition}

\begin{exercise}
\label{exercise-invariant-bilinear-form-on-simple-Lie-algebra-is-symmetric}
Prove that an invariant bilinear form on a simple Lie algebra must in fact be
symmetric.
\end{exercise}

\begin{definition}
\label{definition-semisimple-Lie-algebra}
A {\it semisimple Lie algebra} is a direct sum of simple ones
$$
\mathfrak{g}=\mathfrak{g}_1\oplus\ldots\oplus\mathfrak{g}_s
$$
\end{definition}

\begin{theorem}[Cartan]
\label{theorem-Cartan}
The finite dimensional Lie algebra $\mathfrak{g}$ is semisimple if and only if
its Killing form is nondegenerate.
\end{theorem}

\section{Engel's Theorem}
\label{section-Engel-theorem}

\section{Nilpotent and solvable Lie algebras}
\label{section-nilpotent-and-solvable-Lie-algebras}

\section{Lie's theorem}
\label{section-Lie-theorem}

Lie's theorem says that solvable Lie algebras over closed fields of
characterstic not zero have weights.

\begin{definition}
\label{definition-weight-space}
Let $\mathfrak{h}$ be a Lie algebra, 
$\pi:\mathfrak{h}\to\mathfrak{gl}_V$ a representation of $\mathfrak{h}$ 
and $\lambda \in \mathfrak{h}^*$. The {\it weight space} of $\mathfrak{h}$
attached to $\lambda$ is
$$
V_\lambda^{\mathfrak{h}}:=\{v \in V|\pi(h)v=\lambda(h)v,\; 
\forall h \in \mathfrak{h}\}.
$$
If $V_\lambda^{\mathfrak{h}}\neq 0$, we say that $\lambda$ is a {\it weight} for
$\pi$.
\end{definition}

My first impression on this is concept is: a weight of a representation is a
functional for which there is at least one vector (like an eigenvector) with the
property that 
multiplying this vector times any other vector coming from the Lie algebra 
is a multiple of the vector (so the weight gives me kind of an eigenvalue).
\begin{lemma}[Lie's Lemma]
\label{lemma-Lie}
Some condition on invariance used for…
\end{lemma}

\begin{theorem}[Lie's theorem]
\label{theorem-Lie}
Let $\mathfrak{g}$ be a solvable Lie algebra and $\pi$ a representation of
$\mathfrak{g}$ on a finite dimensional vector space $V \neq 0$, over an
algebraically closed field $\mathbb{F}$ of characterstic 0. 
Then there exists a weight $\lambda \in \mathfrak{g}^*$ for $\pi$ 
(that is, $V_\lambda^\mathfrak{g} \neq \{0\}$.
\end{theorem}

\begin{exercise}
\label{exercise-corollaries-of-Lie-theorem}
Show the following two corollaries of Lie's theorem:
\end{exercise}


\section{Weights}
\label{section-weights}




\bibliography{my}
\bibliographystyle{amsalpha}

\end{document}


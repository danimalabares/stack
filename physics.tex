\input{preamble}

\begin{document}

\title{Physics}
\maketitle

\phantomsection
\label{section-phantom}
\hfill
\href{http://github.com/danimalabares/stack}{github.com/danimalabares/stack}

\tableofcontents

\section{History}
\label{section-history}

The first force discovered was gravity, by Newton. Then there was
electromagnetism by Coulomb. The formulas are very similar.

Faraday introduced notion of field as a ``convenient" way to study
electromagnetism. Then Maxwell's equations. These were about 1870's.

Special relativity in 1905. General relativity in 1915: Einstein equation, Hilbert
ation.

QM. Schroedinger equation 1920. Heisenberg equation 1929. Bohr Rule 1925.

QFT: particles created/annihilated. Dirac, 1929, spinors. Feynman, Tomonaga,
Dyson. Wilson in 1970s with ``Renormalization".

In 1953 Yang-Mills introduced non-abelian gauge theory.

Higgs interaction field, 1964.

\section{Tensions in QM and GR}
\label{section-tensions}

Gravity and QM don't agree at a very basic observation: vacuum energy.

Quantization of a very simple variation problem, for a scalar field, says that
energy of the vacum is infinite. But there's an experiment that says that the
cosmological constant (which is energy of the vacuum) is
 $\rho_{\operatorname{cc}}\approx 5 \times 10^{-10}\frac{J}{m^3}$.

One there was a variation problem for the functional
$$
S=\int dt \int d^3x \frac{1}{2}\left\{ \frac{1}{2}\dot\varphi^2-(\nabla \varphi)^2-m^2\varphi^2\right\} 
$$
whose solution, upon quantization, gives an infinite integral. But it should be
the cosmological constant! A tiny number.

We want to experimentally find a system that is governed both by quantum
mechanics and relativity. An experiment called ``Micro g" by Aspa-Meyer shows
that gravitational laws in a 1-millimetre, i.e. $10^{-6}$ kg particles. The size
of the proton is $10^{-27}$ kg, where we see quantum mechanics. Experimentalists
at PUC-Rio have work with sizes of the orders $10^{-18}$ kg.

\section{General Relativity}
\label{section-general-relativity}
\begin{slogan}
The mass determines the metric.
\end{slogan}
Consider the variational problem:

\section{Neutrinos}
\label{section-neutrinos}

Protons and neutrons have very similar mass of $m_p\approx940$ MeV, while
electrons have mass of $m_e \approx 0.5$ MeV. MeV is $10^{6}$ electronvolts,
where one eV is approximately $1.6\times 10^{-19}$ J. This is standard in high
energy physics, they use electronvolts instead of Joules. Recall that
2J=1N$\times$1m.

So most of the things we see are protons since they are so much larger than
electrons. But protons (and neutrons) are not elementary particles.

Here's the standard model:
\begin{tabular}{c c c c}
Quarks & $\begin{bmatrix} u\\d \end{bmatrix}_L $ & $\begin{bmatrix} c\\s
\end{bmatrix}_L$ & $\begin{bmatrix} t\\b \end{bmatrix}_L$\\
Leptons & $\begin{bmatrix} \nu_e\\e^- \end{bmatrix}_L $ & $\begin{bmatrix}
\nu_\mu\\ \mu^-
\end{bmatrix}_L$ & $\begin{bmatrix} \nu_\tau\\ \tau^- \end{bmatrix}_L$\\
Generation & 1st & 2nd & 3rd\\
Bosons & $g$, $\gamma$ , $\omega^\pm$, $z$ \\
Higgs Bosson & $H$
\end{tabular}
It is very particular that nature repeats itself three times. The $L$ in those
matrix actually means left-handed, and accounts for chirality. Only left-handed
fermions have weak interaction. Right-handed have electromagnetic interaction,
gravitational interaction, but not weak interaction.

And then there's neutrinos. They have negative helicity (chirality). Being
left-handed, mathematically, means to have helicity $-1$. I think this means
that the spin is left-handed. But chirality and helicity are not the same:
helicity is observer-dependent, and chirality is not. Almost all neutrinos we
can see (\% 99.99999…) have negative helicity, but not all of them.

Consider the following:
$$
n+\nu_e\leftrightarrow p+e^-
$$
But it's not completely correct: we'd better put $d$ instead of $n$, and $u$
instead of $p$: the $d$ and $u$ quarks, instead of the neutrons and protons.

Now consider the following reaction: a neutron decays into a proton, an electron
and an antineutrino:
$$
n\to p+e^+\overline{\nu}_e
$$
Protons is very stable, that's why we are here. But neutron decays in only 15
minutes.

By experimental data, we can conclude that neutrinos' mass is consistent with
zero. But if they have mass, it should be much smaller than the electron's $m_v
\leq 0.5$ eV. And the electron is already the lightest fermion!

If the mass of the neutrino was zero, i.e. $m_\nu=0$, then $v_\nu=c$ in vacuum,
which would imply that 
$$
\xymatrix{
\nu_e\ar[r]^{L}&\nu_e\ar[r]\nu_e\\
0:00&0:00&0:00
}
$$
meaning: time doesn't pass! And this means the state of the particle cannot
change.

\bibliography{my}
\bibliographystyle{amsalpha}

\end{document}

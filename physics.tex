\input{preamble}

\begin{document}

\title{Physics}
\maketitle

\phantomsection
\label{section-phantom}

\tableofcontents

\section{History}
\label{section-history}

The first force discovered was gravity, by Newton. Then there was
electromagnetism by Coulomb. The formulas are very similar.

Faraday introduced notion of field as a ``convenient" way to study
electromagnetism. Then Maxwell's equations. These were about 1870's.

Special relativity in 1905. General relativity in 1915: Einstein equation, Hilbert
ation.

QM. Schroedinger equation 1920. Heisenberg equation 1929. Bohr Rule 1925.

QFT: particles created/annihilated. Dirac, 1929, spinors. Feynman, Tomonaga,
Dyson. Wilson in 1970s with ``Renormalization".

In 1953 Yang-Mills introduced non-abelian gauge theory.

Higgs interaction field, 1964.

\section{Tensions in QM and GR}
\label{section-tensions}

Gravity and QM don't agree at a very basic observation: vacuum energy.

Quantization of a very simple variation problem, for a scalar field, says that
energy of the vacum is infinite. But there's an experiment that says that the
cosmological constant (which is energy of the vacuum) is
 $\rho_{\operatorname{cc}}\approx 5 \times 10^{-10}\frac{J}{m^3}$.

One there was a variation problem for the functional
$$
S=\int dt \int d^3x \frac{1}{2}\left\{ \frac{1}{2}\dot\varphi^2-(\nabla \varphi)^2-m^2\varphi^2\right\} 
$$
whose solution, upon quantization, gives an infinite integral. But it should be
the cosmological constant! A tiny number.

We want to experimentally find a system that is governed both by quantum
mechanics and relativity. An experiment called ``Micro g" by Aspa-Meyer shows
that gravitational laws in a 1-millimetre, i.e. $10^{-6}$ kg particles. The size
of the proton is $10^{-27}$ kg, where we see quantum mechanics. Experimentalists
at PUC-Rio have work with sizes of the orders $10^{-18}$ kg.

\section{General Relativity}
\label{section-general-relativity}
\begin{slogan}
The mass determines the metric.
\end{slogan}
Consider the variational problem:


\bibliography{my}
\bibliographystyle{amsalpha}

\end{document}


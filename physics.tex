\IfFileExists{stacks-project.cls}{%
\documentclass{stacks-project}
}{%
\documentclass{amsart}
}

% For dealing with references we use the comment environment
\usepackage{verbatim}
\newenvironment{reference}{\comment}{\endcomment}
%\newenvironment{reference}{}{}
\newenvironment{slogan}{\comment}{\endcomment}
\newenvironment{history}{\comment}{\endcomment}

% For commutative diagrams we use Xy-pic
\usepackage[all]{xy}

% We use 2cell for 2-commutative diagrams.
\xyoption{2cell}
\UseAllTwocells

% We use multicol for the list of chapters between chapters
\usepackage{multicol}

% This is generally recommended for better output
\usepackage{lmodern}
\usepackage[T1]{fontenc}

% For cross-file-references
\usepackage{xr-hyper}

% Package for hypertext links:
\usepackage{hyperref}

% For any local file, say "hello.tex" you want to link to please
% use \externaldocument[hello-]{hello}
\externaldocument[introduction-]{introduction}
\externaldocument[conventions-]{conventions}
\externaldocument[sets-]{sets}
\externaldocument[categories-]{categories}
\externaldocument[topology-]{topology}
\externaldocument[sheaves-]{sheaves}
\externaldocument[sites-]{sites}
\externaldocument[stacks-]{stacks}
\externaldocument[fields-]{fields}
\externaldocument[algebra-]{algebra}
\externaldocument[brauer-]{brauer}
\externaldocument[homology-]{homology}
\externaldocument[derived-]{derived}
\externaldocument[simplicial-]{simplicial}
\externaldocument[more-algebra-]{more-algebra}
\externaldocument[smoothing-]{smoothing}
\externaldocument[modules-]{modules}
\externaldocument[sites-modules-]{sites-modules}
\externaldocument[injectives-]{injectives}
\externaldocument[cohomology-]{cohomology}
\externaldocument[sites-cohomology-]{sites-cohomology}
\externaldocument[dga-]{dga}
\externaldocument[dpa-]{dpa}
\externaldocument[sdga-]{sdga}
\externaldocument[hypercovering-]{hypercovering}
\externaldocument[schemes-]{schemes}
\externaldocument[constructions-]{constructions}
\externaldocument[properties-]{properties}
\externaldocument[morphisms-]{morphisms}
\externaldocument[coherent-]{coherent}
\externaldocument[divisors-]{divisors}
\externaldocument[limits-]{limits}
\externaldocument[varieties-]{varieties}
\externaldocument[topologies-]{topologies}
\externaldocument[descent-]{descent}
\externaldocument[perfect-]{perfect}
\externaldocument[more-morphisms-]{more-morphisms}
\externaldocument[flat-]{flat}
\externaldocument[groupoids-]{groupoids}
\externaldocument[more-groupoids-]{more-groupoids}
\externaldocument[etale-]{etale}
\externaldocument[chow-]{chow}
\externaldocument[intersection-]{intersection}
\externaldocument[pic-]{pic}
\externaldocument[weil-]{weil}
\externaldocument[adequate-]{adequate}
\externaldocument[dualizing-]{dualizing}
\externaldocument[duality-]{duality}
\externaldocument[discriminant-]{discriminant}
\externaldocument[derham-]{derham}
\externaldocument[local-cohomology-]{local-cohomology}
\externaldocument[algebraization-]{algebraization}
\externaldocument[curves-]{curves}
\externaldocument[resolve-]{resolve}
\externaldocument[models-]{models}
\externaldocument[functors-]{functors}
\externaldocument[equiv-]{equiv}
\externaldocument[pione-]{pione}
\externaldocument[etale-cohomology-]{etale-cohomology}
\externaldocument[proetale-]{proetale}
\externaldocument[relative-cycles-]{relative-cycles}
\externaldocument[more-etale-]{more-etale}
\externaldocument[trace-]{trace}
\externaldocument[crystalline-]{crystalline}
\externaldocument[spaces-]{spaces}
\externaldocument[spaces-properties-]{spaces-properties}
\externaldocument[spaces-morphisms-]{spaces-morphisms}
\externaldocument[decent-spaces-]{decent-spaces}
\externaldocument[spaces-cohomology-]{spaces-cohomology}
\externaldocument[spaces-limits-]{spaces-limits}
\externaldocument[spaces-divisors-]{spaces-divisors}
\externaldocument[spaces-over-fields-]{spaces-over-fields}
\externaldocument[spaces-topologies-]{spaces-topologies}
\externaldocument[spaces-descent-]{spaces-descent}
\externaldocument[spaces-perfect-]{spaces-perfect}
\externaldocument[spaces-more-morphisms-]{spaces-more-morphisms}
\externaldocument[spaces-flat-]{spaces-flat}
\externaldocument[spaces-groupoids-]{spaces-groupoids}
\externaldocument[spaces-more-groupoids-]{spaces-more-groupoids}
\externaldocument[bootstrap-]{bootstrap}
\externaldocument[spaces-pushouts-]{spaces-pushouts}
\externaldocument[spaces-chow-]{spaces-chow}
\externaldocument[groupoids-quotients-]{groupoids-quotients}
\externaldocument[spaces-more-cohomology-]{spaces-more-cohomology}
\externaldocument[spaces-simplicial-]{spaces-simplicial}
\externaldocument[spaces-duality-]{spaces-duality}
\externaldocument[formal-spaces-]{formal-spaces}
\externaldocument[restricted-]{restricted}
\externaldocument[spaces-resolve-]{spaces-resolve}
\externaldocument[formal-defos-]{formal-defos}
\externaldocument[defos-]{defos}
\externaldocument[cotangent-]{cotangent}
\externaldocument[examples-defos-]{examples-defos}
\externaldocument[algebraic-]{algebraic}
\externaldocument[examples-stacks-]{examples-stacks}
\externaldocument[stacks-sheaves-]{stacks-sheaves}
\externaldocument[criteria-]{criteria}
\externaldocument[artin-]{artin}
\externaldocument[quot-]{quot}
\externaldocument[stacks-properties-]{stacks-properties}
\externaldocument[stacks-morphisms-]{stacks-morphisms}
\externaldocument[stacks-limits-]{stacks-limits}
\externaldocument[stacks-cohomology-]{stacks-cohomology}
\externaldocument[stacks-perfect-]{stacks-perfect}
\externaldocument[stacks-introduction-]{stacks-introduction}
\externaldocument[stacks-more-morphisms-]{stacks-more-morphisms}
\externaldocument[stacks-geometry-]{stacks-geometry}
\externaldocument[moduli-]{moduli}
\externaldocument[moduli-curves-]{moduli-curves}
\externaldocument[examples-]{examples}
\externaldocument[exercises-]{exercises}
\externaldocument[guide-]{guide}
\externaldocument[desirables-]{desirables}
\externaldocument[coding-]{coding}
\externaldocument[obsolete-]{obsolete}
\externaldocument[fdl-]{fdl}
\externaldocument[index-]{index}

% Theorem environments.
%
\theoremstyle{plain}
\newtheorem{theorem}[subsection]{Theorem}
\newtheorem{proposition}[subsection]{Proposition}
\newtheorem{lemma}[subsection]{Lemma}

\theoremstyle{definition}
\newtheorem{definition}[subsection]{Definition}
\newtheorem{example}[subsection]{Example}
\newtheorem{exercise}[subsection]{Exercise}
\newtheorem{situation}[subsection]{Situation}

\theoremstyle{remark}
\newtheorem{remark}[subsection]{Remark}
\newtheorem{remarks}[subsection]{Remarks}

\numberwithin{equation}{subsection}

% Macros
%
\def\lim{\mathop{\mathrm{lim}}\nolimits}
\def\colim{\mathop{\mathrm{colim}}\nolimits}
\def\Spec{\mathop{\mathrm{Spec}}}
\def\Hom{\mathop{\mathrm{Hom}}\nolimits}
\def\Ext{\mathop{\mathrm{Ext}}\nolimits}
\def\SheafHom{\mathop{\mathcal{H}\!\mathit{om}}\nolimits}
\def\SheafExt{\mathop{\mathcal{E}\!\mathit{xt}}\nolimits}
\def\Sch{\mathit{Sch}}
\def\Mor{\mathop{\mathrm{Mor}}\nolimits}
\def\Ob{\mathop{\mathrm{Ob}}\nolimits}
\def\Sh{\mathop{\mathit{Sh}}\nolimits}
\def\NL{\mathop{N\!L}\nolimits}
\def\CH{\mathop{\mathrm{CH}}\nolimits}
\def\proetale{{pro\text{-}\acute{e}tale}}
\def\etale{{\acute{e}tale}}
\def\QCoh{\mathit{QCoh}}
\def\Ker{\mathop{\mathrm{Ker}}}
\def\Im{\mathop{\mathrm{Im}}}
\def\Coker{\mathop{\mathrm{Coker}}}
\def\Coim{\mathop{\mathrm{Coim}}}

% Boxtimes
%
\DeclareMathSymbol{\boxtimes}{\mathbin}{AMSa}{"02}

%
% Macros for moduli stacks/spaces
%
\def\QCohstack{\mathcal{QC}\!\mathit{oh}}
\def\Cohstack{\mathcal{C}\!\mathit{oh}}
\def\Spacesstack{\mathcal{S}\!\mathit{paces}}
\def\Quotfunctor{\mathrm{Quot}}
\def\Hilbfunctor{\mathrm{Hilb}}
\def\Curvesstack{\mathcal{C}\!\mathit{urves}}
\def\Polarizedstack{\mathcal{P}\!\mathit{olarized}}
\def\Complexesstack{\mathcal{C}\!\mathit{omplexes}}
% \Pic is the operator that assigns to X its picard group, usage \Pic(X)
% \Picardstack_{X/B} denotes the Picard stack of X over B
% \Picardfunctor_{X/B} denotes the Picard functor of X over B
\def\Pic{\mathop{\mathrm{Pic}}\nolimits}
\def\Picardstack{\mathcal{P}\!\mathit{ic}}
\def\Picardfunctor{\mathrm{Pic}}
\def\Deformationcategory{\mathcal{D}\!\mathit{ef}}

%Dani's additions
\usepackage{graphicx} %figures


\begin{document}

\title{Physics}
\maketitle

\phantomsection
\label{section-phantom}
\hfill
\href{http://github.com/danimalabares/stack}{github.com/danimalabares/stack}

\tableofcontents

\section{History}
\label{section-history}

The first force discovered was gravity, by Newton. Then there was
electromagnetism by Coulomb. The formulas are very similar.

Faraday introduced notion of field as a ``convenient" way to study
electromagnetism. Then Maxwell's equations. These were about 1870's.

Special relativity in 1905. General relativity in 1915: Einstein equation, Hilbert
ation.

QM. Schroedinger equation 1920. Heisenberg equation 1929. Bohr Rule 1925.

QFT: particles created/annihilated. Dirac, 1929, spinors. Feynman, Tomonaga,
Dyson. Wilson in 1970s with ``Renormalization".

In 1953 Yang-Mills introduced non-abelian gauge theory.

Higgs interaction field, 1964.

\section{Electromagnetism}
\label{section-electromagnetism}

Probably the one Gauge theory you are all familiar with is electromagnetism.
It's was the first \textit{field theory} in which there were dynamical objects
which are functions $\phi(\vec{x},t)$ which could be measured, etc. It is based
in what we call an abelian Gauge theory (for historical reasons) with group
$\mathsf{U}(1)$. Its \textit{dynamical field} is a connection $A$, which is
(locally?) a 1-form on Minkowsky space $\mathbb{R}^{3,1}$, over a
$\mathsf{U}(1)$-bundle. And the \textit{physical observables} are the
\textit{field strengths} (i.e. curvature) $F=dA=dt\wedge \vec{x}
\vec{E}\ldots$, whose components are the familiar electric and
magnetics fields.

Maxwell unified electricity and magnetism into these theory, whose equations are
simple second order PDEs:
\[dF=0,\qquad d\wedge F=0\]
which follow from an action functional
\[\delta=\frac{1}{4e^2}\int_{M}F\wedge * F.\]
These equations were the first \textit{field theory}. Having dynamics described
by a function $\phi(\vec{x},t)$ of space and time. The action is invariant under
Lorentz transformations i.e. in the group $\mathsf{SO}(3,1)$. The solutions of
Maxwell equations are light $\sim$ waves of ripples in this field strength.

\section{Yang-Mills theory}
\label{section-Yang-Mills-theory}

A generalization of gauge theory to non 

YM theory has a mass gap, which produces a scale that ED in the vaccum does not
have. We will be discussing non-abelian electromagnetism just like
electromagnetism has no scale as a dimensionless coupling. And we will produce a
scale, a mass gap, from quantum effects.

What does it mean to define or prove the existence of a QFT?

We need a Hilbert space $\mathcal{H}$, states in a Hilbert space $|\psi\rangle$,
positive norm states (we identify probability with the norm of states, and
probability should be positive). We want {\it relativistic invariant},
symmetries, Poincaré invariants: translation in space and time $x \mapsto x+a$ +
Lorentz transformations $\mathsf{SO}(3,1)$. In quantum mechanics, such
symmetries of the theories are represented by unitary operators under which
states, the fields, transform covariantly. And we want a {\it vacuum
state} $|\Omega\rangle$, a ground state of the system, that is invariant under
translations and Lorentz transformations. We're looking for all this.

We also want that the generators of these symmetry transofrmations, the momenta
$P^n$, have definite spectral properties. In particular, the energy $P^0$ should
have spectrum bounded from below $P^0\geq 0$. We call
$\mathcal{P}^2=E^2-P^2=m^2$ the {\it mass}.

In the case of ENM, the spectrum of the energy  $P^0$ is continuous.

The spectrum of the mass operator has a base state $|\Omega\rangle$, and then
there is a discrete state $m_G$, $G$ standing for \textit{gap}. An the pairs of
these particles of twice the mass, $2m_G$, and some extra kinetic energy, are
continuum of these multiparticle sates. So, in a QFT where we have massive
particles and no massless particles like lightrays, \textit{there is a gap
between the vacuum, the ground state, and the first state that can be created
with a minimal energy equal to the mass of that particle}, and then
multiparticle states, and perhaps some isolated states.

\section{Tensions in QM and GR}
\label{section-tensions}

Gravity and QM don't agree at a very basic observation: vacuum energy.

Quantization of a very simple variation problem, for a scalar field, says that
energy of the vacum is infinite. But there's an experiment that says that the
cosmological constant (which is energy of the vacuum) is
 $\rho_{\operatorname{cc}}\approx 5 \times 10^{-10}\frac{J}{m^3}$.

One there was a variation problem for the functional
$$
S=\int dt \int d^3x \frac{1}{2}\left\{ \frac{1}{2}\dot\varphi^2-(\nabla \varphi)^2-m^2\varphi^2\right\} 
$$
whose solution, upon quantization, gives an infinite integral. But it should be
the cosmological constant! A tiny number.

We want to experimentally find a system that is governed both by quantum
mechanics and relativity. An experiment called ``Micro g" by Aspa-Meyer shows
that gravitational laws in a 1-millimetre, i.e. $10^{-6}$ kg particles. The size
of the proton is $10^{-27}$ kg, where we see quantum mechanics. Experimentalists
at PUC-Rio have work with sizes of the orders $10^{-18}$ kg.

\section{General Relativity}
\label{section-general-relativity}
\begin{slogan}
The mass determines the metric.
\end{slogan}
Consider the variational problem:

\section{Standard Model}
\label{section-standard-model}

\begin{slogan}
Standard model = symmetries + representations
\end{slogan}

Recall: in electromagnetism we have
$$
\psi\rightsquigarrow e^{i\alpha(x)}\psi,\qquad 
A_\mu\rightsquigarrow A_\mu+\partial_\mu\alpha(x)
$$
so that the gauge group is $\text{U}(1)$.

In the standard model we have gauge group
$\text{SU}(3)\times\text{SU}(2)\times\text{U}(1)$. So for a gauge transformation
we do
$$
\psi\rightsquigarrow U(x)\psi,\qquad 
A_\mu\rightsquigarrow U(x)(A_\mu+i\partial_\mu)U^\dag(x)
$$
We have the following table as a result of several theoretic and experimental
resereach:
\begin{table}[H]
\centering
\caption{Standard Model (incomplete)}
\label{tab:standard-model}
\begin{tabular}{c c c c c}
Field &	$\text{SU}(3)$ & $\text{SU}(2)$ & $\text{U}(1)$ &
$\text{SL}(2,\mathbb{C})$ \\
gluons, $G_\mu$ & 8 & 1 & 0 (charge) & \\
$W_\mu$ & & & & &\\
$B_\mu$ &&&&\\
\hline
$q^i_L$, $i=1,2,3$ &&&&\\
$\ell^i_L$ &&&&\\
$u^i_R$ &&&&\\
$d^i_R$ &&&&\\
$e^i_R$ &&&&\\
\hline
$H$ Higgs &&&&
\end{tabular}
\end{table}

Confinement: as two quarks spread apart, the potential energy between them grows
linearly, i.e. $V(r)=\sigma r$ where $r$ is the distance. (Due to strong
coupling.)

\section{Neutrinos}
\label{section-neutrinos}


Protons and neutrons have very similar mass of $m_p\approx940$ MeV, while
electrons have mass of $m_e \approx 0.5$ MeV. MeV is $10^{6}$ electronvolts,
where one eV is approximately $1.6\times 10^{-19}$ J. This is standard in high
energy physics, they use electronvolts instead of Joules. Recall that
2J=1N$\times$1m.

So most of the things we see are protons since they are so much larger than
electrons. But protons (and neutrons) are not elementary particles.

Here's the standard model:
\begin{tabular}{c c c c}
Quarks & $\begin{bmatrix} u\\d \end{bmatrix}_L $ & $\begin{bmatrix} c\\s
\end{bmatrix}_L$ & $\begin{bmatrix} t\\b \end{bmatrix}_L$\\
Leptons & $\begin{bmatrix} \nu_e\\e^- \end{bmatrix}_L $ & $\begin{bmatrix}
\nu_\mu\\ \mu^-
\end{bmatrix}_L$ & $\begin{bmatrix} \nu_\tau\\ \tau^- \end{bmatrix}_L$\\
Generation & 1st & 2nd & 3rd\\
Bosons & $g$, $\gamma$ , $\omega^\pm$, $z$ \\
Higgs Bosson & $H$
\end{tabular}
It is very particular that nature repeats itself three times. The $L$ in those
matrix actually means left-handed, and accounts for chirality. Only left-handed
fermions have weak interaction. Right-handed have electromagnetic interaction,
gravitational interaction, but not weak interaction.

\section{Neutrinos}
\label{section-neutrinos}

And then there's neutrinos. They have negative helicity (chirality). Being
left-handed, mathematically, means to have helicity $-1$. I think this means
that the spin is left-handed. But chirality and helicity are not the same:
helicity is observer-dependent, and chirality is not. Almost all neutrinos we
can see (\% 99.99999…) have negative helicity, but not all of them.

Consider the following:
$$
n+\nu_e\leftrightarrow p+e^-
$$
But it's not completely correct: we'd better put $d$ instead of $n$, and $u$
instead of $p$: the $d$ and $u$ quarks, instead of the neutrons and protons.

Now consider the following reaction: a neutron decays into a proton, an electron
and an antineutrino:
$$
n\to p+e^+\overline{\nu}_e
$$
Protons is very stable, that's why we are here. But neutron decays in only 15
minutes.

By experimental data, we can conclude that neutrinos' mass is consistent with
zero. But if they have mass, it should be much smaller than the electron's $m_v
\leq 0.5$ eV. And the electron is already the lightest fermion!

If the mass of the neutrino was zero, i.e. $m_\nu=0$, then $v_\nu=c$ in vacuum,
which would imply that 
$$
\xymatrix{
\nu_e\ar[r]^{L}&\nu_e\ar[r]\nu_e\\
0:00&0:00&0:00
}
$$
meaning: time doesn't pass! And this means the state of the particle cannot
change.

\bibliography{my}
\bibliographystyle{amsalpha}

\end{document}

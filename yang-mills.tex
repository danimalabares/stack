% This file is part of the Stack Project
% https://github.com/danimalabares/stack

\section{Yang--Mills theory}
\bibliographystyle{alpha}
\begin{minipage}{\textwidth}
	\begin{minipage}{1\textwidth}
		Personal notes on YM \hfill Daniel González Casanova Azuela
		
  {\small
  \hfill\href{https://github.com/sergunchik/daniel}{github.com/sergunchik/daniel}}
	\end{minipage}
\end{minipage}\vspace{.2cm}\hrule

\vspace{10pt}
{\huge Personal notes on Yang-Mills theory}

Disclaimer: these are very sketchy notes just to keep track of what I read and
remember key ideas.

\section{December 14: day 1}

\subsection{Reading \cite{reportdouglas}}

\begin{enumerate}
\item YM theory is a generalization of Maxwell's theory of electromagnetism.
\item Has classical and quantum version.
\item Classical version is some PDEs.
\item Quantum version has no satisfactory mathematical definition.
\item YM on a lattice (there are other very different ways to think about YM
theory)
	\begin{enumerate}
 \item Start with a lattice $\Gamma$ like $\mathbb{Z}^4 \subset \mathbb{R}^4$.
 Put some graph structure on $\Gamma$. Define a $G$-connection to be a function
 from edges to $G$. Define the curvature. YM action is the sum of squares of the
 curvatures.
 \item Quantize: define a quantum lattice on a subgraph $\gamma \subset \Gamma$
 with a functional integral $Z$ that is called \textit{\textbf{partition
 function}}. It is of physical interest like the expectation values of quantum
 observables.
 \item (\textbf{Objective}.) To devine a functional integral on all of $\Gamma$
 by taking the limit of $Z$ over larger and larger $\gamma$.
 \item (\textbf{Existence}.) There are some properties this construction should
 have.
 \item (\textbf{Mass gap}.) There are some other properties this construction
 should have.
	\end{enumerate}
 This could be explained without $\Gamma$, or partition functions, or taking
 limits.
\item There are some QFTs similar to YMT--- ``There are two classes of quantum
field theories which are generally believed to bear a close similarity to
four-dimensional Yang-Mills theory. The first is the two-dimensional nonlinear
sigma model with target space a group manifold \( G\)…"

 ``The orher broad class of models with great simlarity to YM" are 4dssYMTs. In
 those we consider also fermions, and the quantum Hamiltonian has a square root.
 These theories help with the problem of renormalization. Actually there's a
 reason why solving the problem here would solve the problem in the original
 problem.

 This property has been discovered by physicists from experiment and confirmed
 by computer simulations, but it still has not been understood from a
 theoretical point of view. Progress in establishing the existence of the
 Yang-Mills theory and a mass gap will require the introduction of fundamental
 new ideas both in physics and in mathematics.
\item There are some recent results (2000's).
\item In 1994 Seiberg-Witten had a clever ansatz to solve a problem.
Computations were hard, but they were solved by N. Roughly, this made SW some
sort of mirror-symmetry for 4d topology.
\item Integrability of $N=4$ ssYM is harder to explain in mathematical terms but
it is important. It has to do with AdS/CFT.
\end{enumerate}

From \href{https://www.claymath.org/millennium/yang-mills-the-maths-gap/}{here}:
{\color{3}\begin{quotation}
 The successful use of Yang-Mills theory to describe the strong interactions of
 elementary particles depends on a subtle quantum mechanical property called the
 “mass gap”: the quantum particles have positive masses, even though the
 classical waves travel at the speed of light.
\end{quotation}}

\subsection{Video of a lecture by David Gross on YM}

\href{https://www.youtube.com/watch?v=vMiY7zlBOFI}{Here's the video}.

\subsubsection{Motivation}

\begin{thing4}{Millenium problem}\leavevmode
Existence of YM theory in 4 dimensions with gauge group $G$, and finite mass
gap.
\end{thing4}
Gauge theories are examples of QFTs. In 4 dimensions they have never been proven
to exist. But the existence of these QFTs is not that interesting.

Why is there a prize on this?
\begin{enumerate}
\item These theories (YM theories) are the standrd theory of elementary
particles. It's the best most complete ambitious physical theories we've had.
\item Physicists have ways to discover truth that isn't guided solely by
mathematics. First part of this problem: it's hard to know how it would fail
since we don't know what it would mean to prove the non-existence of QFT. It
would be possible to define a YM theory which \textit{doesn't} have a positive
mass gap. That would shock us all since we know it's false.

 Like walking on a rope over a precipice without looking down: we have nature to
 guide us. We know YMT is correct by expermient, analogues, etc… David Gross
 takes strong bets against the unexistence of positive mass gap.
\end{enumerate}
So: there are no mathematics to explain this.

\subsubsection{Gauge theory}

Probably the one Gauge theory you are all familiar with is electromagnetism.
It's was the first \textit{field theory} in which there were dynamical objects
which are functions $\phi(\vec{x},t)$ which could be measured, etc. It is based
in what we call an abelian Gauge theory (for historical reasons) with group
$\mathsf{U}(1)$. Its \textit{dynamical field} is a connection $A$, which is
{\color{4}locally?} a 1-form on Minkowsky space $\mathbb{R}^{3,1}$, over a
$\mathsf{U}(1)$-bundle. And the \textit{physical observables} are the
\textit{field strengths} (i.e. curvature) $F=dA=dt\wedge \vec{x}
\vec{E}{\color{3}\ldots}$, whose components are the familiar electric and
magnetics fields.

Maxwell unified electricity and magnetism into these theory, whose equations are
simple second order PDEs:
\[dF=0,\qquad d\wedge F=0\]
which follow from an action functional
\[\delta=\frac{1}{4e^2}\int_{M}F\wedge * F.\]
These equations were the first \textit{field theory}. Having dynamics described
by a function $\phi(\vec{x},t)$ of space and time. The action is invariant under
Lorentz transformations i.e. in the group $\mathsf{SO}(3,1)$. The solutions of
Maxwell equations are light $\sim$ waves of ripples in this field strength.

{\color{3}\begin{quotation}
	All the theories we discuss in YM are generalizations of that.
\end{quotation}}

The constant in the action functional is called the \textit{electric charge}.

{\color{3}\bfseries Units.}\hspace{.5em}In physics we need to specify the units
of the physical things we are measuring. Like $L,T$ energy, mass, etc. Also
there appear some universal constants like velocity of light $c$, and Plank's
constant $\hslash=\frac{mL^2}{T}$. We can somehow put them to 1. Having units
means that \textit{everything is expressible by a length}. So the $e$ in the
action functional above is $\frac{e^2}{\hslash c}=\frac{1}{137.035\ldots}$

That's electrodynamics in the vaccum--- it's very trivial and it's even trivial
to quantize it (you can easily construct the Hilbert space).

Actually, nobody has ever proven the existence of a QED (quantum
electrodynamics?) as opposed to electromagnetism coupled to electrons.

\begin{thing14}{QED at very short distances}\leavevmode
 QFT assumes the existence of fields-- functions of spacetime. This means we
 believe that we live in a continuum, that we can go down to arbitrarily small
 distances. If we put a cutoff, destroy the continuum at very short distances,
 say, at $10^{-3}$ cm. Then the non existence of QED (this simple abelian gauge
 theory) is well-defined. We can argue that the assymtpotic expansion (this is
 related to the theoretical approximation of some constant that can also be
 appoximated experimentally) would approximate with erros of small order. So you
 can use asymptotic expans. But the theory may not work at other levels of
 energy. Actually we believe that QED theory to arbitrarily short distances does
 not exist. And for the very same reasons that we believe that, we believe that
 non-abelian gauge theories, even just vaccum non-abelian gauge theories, which
 are highly non-trivial, \textit{do exist}. And you can really go to arbitrarily
 short distances; it's a complete theory.
\end{thing14}

\subsection{YM}

YM theory has a mass gap, which produces a scale that ED in the vaccum does not
have. We will be discussing non-abelian electromagnetism just like
electromagnetism has no scale as a dimensionless coupling. And we will produce a
scale, a mass gap, from quantum effects.

\begin{question}\leavevmode
	What does it mean to define or prove the existence of a QFT?
\end{question}

We need a Hilbert space $\mathcal{H}$, states in a Hilbert space $|\psi\rangle$,
positive norm states (we identify probability with the norm of states, and
probability should be positive). We want \textit{\textbf{relativistic
invariant}}, symmetries, Poincaré invariants: translation in space and time $x
\mapsto x+a$ + Lorentz transformations $\mathsf{SO}(3,1)$. In quantum mechanics,
such symmetries of the theories are represented by unitary operators under which
states, the fields, transform covariantly. And we want a \textit{\textbf{vacuum
state}} $|\Omega\rangle$, a ground state of the system, that is invariant under
translations and Lorentz transformations. We're looking for all this.

We also want that the generators of these symmetry transofrmations, the momenta
$P^n$, have definite spectral properties. In particular, the energy $P^0$ should
have spectrum bounded from below $P^0\geq 0$. We call
$\mathcal{P}^2=E^2-P^2=m^2$ the \textit{\textbf{mass}}.

In the case of ENM, the spectrum of the energy  $P^0$ is continuous.

The spectrum of the mass operator has a base state $|\Omega\rangle$, and then
there is a discrete state $m_G$, $G$ standing for \textit{gap}. An the pairs of
these particles of twice the mass, $2m_G$, and some extra kinetic energy, are
continuum of these multiparticle sates. So, in a QFT where we have massive
particles and no massless particles like lightrays, \textit{there is a gap
between the vacuum, the ground state, and the first state that can be created
with a minimal energy equal to the mass of that particle}, and then
multiparticle states, and perhaps some isolated states.

\begin{thing5}{Really interesting dynamical issue}\leavevmode
 how to produce YM theory with roughly the same structure as classical
 electromagnetism a mass gap.
\end{thing5}

\section{Reading \cite{yangmills}}

Like in the video by David Gross, in this paper there is a quick review of
electromagnetic theory as the most basic gauge theory: the gauge group is
\(\mathsf{U}(1)\), which has associated a connection $A$, which is locally a
1-form ({\color{6}like in Dani's symplectic geometry presentation}), whic
differential is the two form \(F=dA\) ({\color{6}which is the curvature and the
force}), so that Maxwell's equations read \(0=dF=d * F\), {\color{6}where I'm
still not sure how Hodge star acts here}. And the upshot (or one of the upshots)
is that this system describes both ``large scale electric and magnetic fields"
\textit{and}, as Maxwell discovered, the propagation of light waves at the speed
of light.

And then what is YM theory?
\begin{quotation}
 ``[…] and the discovery of Yang-Mills theory, also known as \textit{non-abelian
 gauge theory}."
\end{quotation}
So in YM we replace the gauge group \(\mathsf{U}(1)\) with a compact
({\color{6}non-abelian I suppose}) gauge group \(G\). The definition of
curvature changes to \(F=dA+A \wedge A\) and Maxwell's equations are replaced by
YM equations \(0=d_AF=d_A * F\) where \(d_A\) is the gauge-covariant extension
of the exterior derivative ({\color{6}probably just the most natural extension
of the exterior derivative to something that behaves well along with the new
\(G\) }).

Despite interesting advantages, some difficulties arise in this theory: ``the
masless nature of classical YM was a seious obstacle to applying YM theory to
the other forces […]" And then ``In the 1960s and 1970s, physicists overcame
these obstacles to the physiscal interpretation of non-abelian gauge theory.",
basicall introding an additional ``Higgs field".

``The solution to the problem of massless YM fields for the strong interaction
has a completely different nature. […] by discovering a remarkable property of
YM theory itself […] called `àsymptotic freedom" […] that at short distances the
field displays quantum behaviour very similar to its classical behavior; yet at
long distances the classical theory is no longer a good guide to the quantum
behaviour of the field.

In \textbf{Section 3. Quantum Fields} some basic concepts related to QFTs are
explained, so that in \textbf{Section 4. The problem} we have:

\begin{thing7}{Yang-Mills Existence and Mass Gap}\leavevmode
 Prove that for any compact simple gauge group \(G\), a non-trivial quantum
 Yang-Mills theory exists on \(\mathbb{R}^4\) and has a mass gap \(\Delta>0\).
 Existence includes establishing axiomatic properties at least as strong as
 those cited in [35,45]. Existence includes establishing axiomatic properties at
 least as strong as those cited in [35,45].
\end{thing7}

\section{Lucas' talk on January 17}

Here's a few notes on what I picked up from today's talk. Probably Lucas can
share his own notes.

\subsection{Plan}

\begin{enumerate}
\item QFT (today).
\item AQFT.
 \item  QYM
\begin{enumerate}
\item mass gap.
\item Perturbative \(\times\) lattice
\item Example \(d=2\).
\item Difficulties.
\end{enumerate}
\end{enumerate}


\subsection{Standard model}

Fermions correspond to mass. They are divided in Quarks ($u$, $c$, $t$, $d$,
$s$, $b$ ) and Leptons (\(e^-\) (electron), \(\mu^-, \tau^-, \nu_e, \nu_\mu,
\nu_\tau\). Quarks are much more massive than leptons. For a long time leptons
were just asumed to have zero mass.

Bosons correspond to ``mass carriers". We will discuss a lot about these today.
Gauge bosons: foton: $g$ and \(Y\); *something* decay (radiation): \(Z\) and
\(W\). Scalar (Higg) boson \(H\).

\begin{question}\leavevmode
What's a proton: it's a combination of three fundamental quarks: up up down,
$uud$. And neutrons are up down down. (So protons and neutrons are not
fundamental.)
\end{question}

So the Boson part is very related to light, and the Fermion part to gravity.
Well light may be studied by Electromagnetism (and its quantum version QED), for
which the particle is a foton \(Y\) and the gauge group is \(\mathsf{U}(1)\).
Then there's Electroweak force (or field?) which includes Higgs boson in its
particle and its gauge group is \(\mathsf{SU}(2)\times \mathsf{U}(1)\). And
finally also Quantum Chromo Dynamics (QCD) with partigle \(g\) and gauge group
\(\mathsf{SU}(3)\).

\begin{remark}\leavevmode
\begin{enumerate}
\item Pretty much everything has an antimatter, which has the same mass but
(opposite charge?).
\end{enumerate}
\end{remark}

\subsection{Classical \(\rightsquigarrow \) Quantum}

\textbf{Classical:} ``deterministic".
\begin{itemize}
\item Space of fields \(A\).
\item Action functional \(S(A)= \int_A \mathcal{L}(A)\).
\item Equation of motion (EOM) = Euler-Lagrange equations for \(S\) (\(\delta
S=0\), first vol.).
\end{itemize}

\textbf{Quantum}: ``probabilistic". A particle becomes a wave [drawing
double-slit-experiment-like]
\begin{itemize}
\item \(Z_{\operatorname{euc}}=\int e^{S(A)/\hslash}dA\).
\item \(\left<O(A)\right>=\int O(A)e^{S(A)/\hslash}dA\).
\end{itemize}

\begin{remark}\leavevmode
The measure \(dA\) is usually ill-defined, but there are ways to solve it (some
cases).

\(\hslash \to 0\): the exp in \(Z\) gets localized at the minimum of \(S
\rightsquigarrow \)EL, EOM.
\end{remark}

\begin{example}\leavevmode
\(M^3\), \(G=\) compact gauge group. \(S_{\operatorname{cs}}(A)=\frac{k}{4\pi}
\int_M \operatorname{tr}(A \wedge dA+\frac{2}{3}A \wedge A \wedge A)\), \(Z=
\int e^{i S(A)/\hslash}dA\).
\end{example}

\begin{example}\leavevmode
\begin{enumerate}[label=(\alph*)]
\item Fields=space of Riemannian metrics. This is \textit{\textbf{Quantum
gravity}}.
\item Fix auxiliary manifold $N$: fields are functions \(M \to N\). This gives a
\textit{\textbf{sigma-model}}. Dimension 1 \(\sigma\) model is actually Quantum
Mechanics (QM).
\item Fix principal \(G\)-bundle \(P \to M\). Fields: \(G\)-connections (``gauge
fields"). Sections of associated vector bundles (``mother fields"). This is
\textit{\textbf{Quantum Gauge Theory}}.
\end{enumerate}
\end{example}

\section{January 24}

\subsection{Sergey S.}

A surface \(\Sigma\), choose a set of \(N\) points \(P_i \in \Sigma\) which give
a triangulation. Then we take piecewise linear compactly supported functions
\(\phi\), which generate the finite-dimensional hilbert space \(\mathcal{H}_N\).
Consider \(\mathcal{H}_N \to \mathcal{H})\).

\subsection{Bruno}

\(\Sigma\) compact, \(L\) holomorphic pre-quantum line bundle.
\(V_n=H^{0}(\Sigma,L^{\otimes n})\). Since \(\Sigma\), the sections of line
bundles are \textbf{finite dimensional}. The BT operators are \(Tx_j \in
\operatorname{End}(V_n)\).

Looks like we are tryng to understand what a \textbf{quantum surface} is. This
was defined by Koontsevich and collaborators in \cite{kon}.

Here's Bruno saying what Altan told him about what Kontsevich said in that
video:

\begin{quotation}
 \(\Sigma \xrightarrow{\text{minimal surface}} \mathbb{R}^n\).
 \(L=K_\Sigma^{1/2}\) hermitian, metric. \(\mathcal{H}_n=H^{0}(\Sigma,L^{\otimes
 n})\cap L^2\). \(T_{x_i}\in B(\mathcal{H}_n)\) by BT.
\end{quotation}

Here's Bruno saying some other ideas:
\begin{quotation}
 Start with \((M, \omega)\) Kähler \(\mathcal{L}\) holomorphic line bundle, \(f
 \in C^\infty (M)\) \( T_f=\pi_\mathcal{H} \circ M_f\). \(\mathcal{H}:
 H^{0}(M,\mathcal{L})\hookrightarrow
 L^2(M,\mathcal{L})=\mathcal{H}^{\operatorname{pre}}\).
\end{quotation}


Here are Altan's ideas:
\begin{quotation}
\(\Sigma\) a compact surface. \(\Sigma \hookrightarrow \mathbb{R}^n\) a minimal
surface. Then \(x_1,\ldots, x_N : \Sigma \to \mathbb{R}\). BT quantization:
\(X_1,\ldots,X_n \in \operatorname{Mat}_{n \times n}\). Quantum surface.
\end{quotation}

\subsection{Lucas: basic TQFT}

\begin{upshot}\leavevmode
One the axioms in 1-extended TQFT \textbf{is Atiyah-Floer conjecture} 
\end{upshot}

\textbf{Notation:} \(M^{(d)}\) is a closed oriented $d$-manifold.
\(\underline{M^{(d)}}\) is compact \(d\)-manifold \textit{with} boundary.

TQFT: 2functors: classical TQFT is given by two functors. For any closed
manifold you associate a vector space,
\[\mathcal{H}:Y^{(n-1)}\mapsto \mathcal{H}(Y) \in \operatorname{Vect}\]
and also
\[Z: \underline{X}^{(n) }\mapsto Z(x) \in \mathcal{H}(Y), \qquad \partial X=Y.\]
Satisfying some axioms:

\begin{enumerate}
\item \(\mathcal{H}(Y \sqcup Y'=\mathcal{H} (Y) \otimes \mathcal{H}(Y')\).
\item  \(\mathcal{H}(\overline{Y}) = \mathcal{H}(Y)^*\).
\item \(\mathcal{H}(Y)^* \otimes \mathcal{H}(Y)\otimes \mathcal{H}(Y')\to
\mathcal{H}(Y')\), \(Z(\underline{X}\mapsto Z(\underline{X}^\sharp\).
\end{enumerate}

And then there is the ``1-extended TQFT", where now to a closed surface we
associate a category
\[\Sigma \mapsto  \mathcal{C}(\Sigma)\]
and
\[\underline{Y} \mapsto \]

\subsection{Altan: Atiyah-Floer Homology}

\begin{thing4}{Abstract}
I will give an overview of Floer theory and share my perception of the
Atiyah-Floer conjecture. Your help and remarks are very welcome, especially from
Lucas who has studied it all before.
\end{thing4}

Recall that last week we also had Altan

\begin{thing4}{Abstract from last week}
Tomorrow I am going to present the history of the Atiyah-Floer conjecture,
introduce Z/8Z-graded instanton Floer homology and recall some
infinite-dimensional Morse theory developed by Witten. I hope we'll have some
examples, too. Sorry if you've already heard it somewhere else.

I'm postponing the general discussion of Lagrangian Floer homology and further
intuition/physical interpretation of the Atiyah-Floer conjecture until our next
meeting (Wednesday?).
\end{thing4}

\subsubsection{Morse homology}

\begin{remark}\leavevmode
 (Pretty much everything I know about Morse homology was learnt in
 \texttt{gabriel-anotado.pdf}.) Today we went over the basic construction of
 Morse homology, which is section 3.3 of Gabriel's project. We have defined the
 differential on the chain of complexes generated by the critical points of
 index \(k\) of a Morse function as \(\partial[x]=\sum_{y \in
 \operatorname{Crit}f}\# \mathcal{M}_{f,y}(x,y)[y]\). So as I recall that's the
 number of trayectories from \(x\) to \(y\), and the index decreases one-by-one
 along trajectories so its image is the chain group of degree one less. Then
 Altan started the proof of \(\partial^2=0\). In contrast with Gabriel's work
 (the more usual approach I think), today we have used \(\mathbb{Z}\)
 coefficients (unfortunately we were unable to prove \(\partial^2=0\) with
 \(\mathbb{Z}\)-coefficients).
\end{remark}

\((M,f)\), \(M\) manifold and \(f:M\to \mathbb{R}\). We ask that $f$ be a Morse
function, i.e. in the discrete set \(\operatorname{Crit}f=\{x:df_x=0\}\) the
Hessian is nondegenerate. So locally
\(f(x)=c+x_1^2+\ldots+x_k^2-x^2_{k+1}+\ldots+x_n^2\) by Morse lemma.

The \textit{\textbf{index}} of \(f\) is the negative part of the signature of
the Hessian at a given critical point.

\begin{remark}[Lucas]\leavevmode
The examples of doughnut, heart, with height function (I think) are
\textit{generic}, meaning most cases are like this and it's $\mathsf{OK}$ to
think of these examples when doing these things. (In particular Morse functions
are dense.)
\end{remark}

I have interpreted that a proof that Morse homology is isomorphic to singular
may be done using the stable/unstable manifolds to give \(M\) a CW structure.

\section{January 31}

\subsection{Altan}

\subsubsection{Recall}

\(Y\) a 3-dimensional manifold (a homology sphere). A \(\begin{tikzcd}
E\arrow[d]\\
Y
\end{tikzcd}\)-principal \(\mathsf{SU}(2)\)-bundle. \(\mathcal{A}\) all
connections on \(E\). \(\mathcal{G}\) the gauge group. The
\textit{\textbf{Chern-Simmons functional}} \(\operatorname{ C S}:
\mathcal{A}/\mathcal{G}\to \mathbb{R}/4\pi^2\mathbb{Z}\)




%\bibliography{~/github/config/bibliography.bib}
\bibliography{b.bib}

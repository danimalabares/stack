\section{Exercises}
\label{section-exercises}

\begin{exercise}[Geodésica sem pontos conjugados é localmente minimizante]
\label{exercise-geodesica-sem-pontos-conjugados-e-localmente-minimizante}
$\gamma$ geodésica sem pontos conjugados em $[0,a]$. Então $\gamma$ é localmente
minimizante.
\end{exercise}

\begin{proof}
Temos dois casos:
\begin{enumerate}
\item Se $E''(0)<0$ usamos o teorema do índice. Isto é, não é possível que
 $E''(0)<0$ porque isso implica que o índice da forma do índice é diferente
 de zero, ou seja, existem pontos conjugados.
\item Se $E''(0)=0$,
\begin{enumerate}
\item[(0)] Como $\gamma$ não tem pontos conjugados, então $I$ não tem
nulidade em todo  $\mathcal{V}$.
\item $I_{\mathcal{V}^-}$ não tem índice nem nulidade. Segue do ponto anterior + 
algum passo na prova.
\item $I_{\mathcal{V}^-}$ não tem índice.
\item 2+3 dão que $I_{\mathcal{V}^-}>0$.
\item Logo $I>0$.
\end{enumerate}
\end{enumerate}
\end{proof}

\begin{exercise}[Warped product]
\label{exercise-wraped-product}
Para $(M,g_M)$ e $(N,g_N)$ e $f:M \to \mathbb{R}_+$, definimos o {\it warped
product} como sendo o produto cartesiano $M\times N$ com a métrica 
 $g_M+f^2g_N$.

Mostre que se os fatores são completos, o warped product é completo.
\end{exercise}

\subsection{Highlight exercises}
\label{subsection-highlight-exercises}
\begin{exercise}
\label{exercise-minimal-intersection}
$(M^n,g)$ completa $\operatorname{Ric}_M>0$. $A$, $B$ hipersuperfícies mínimas
completas. Então  $A \cap B \neq \emptyset$.
\end{exercise}

\begin{proof}
Suponha que existe uma $\gamma$ mínima com comprimento positivo ligando as 
duas variedades. Tem dois caminhos:
\begin{enumerate}
\item Deixa o campo exponencial ao longo da $\gamma$ e ajusta nos extremos
para fazer ele tangente às variedades. Nesse caso já tem que o temo
$\left<f_{ss},\gamma'\right>|_{0}^a$ se anula. Então so mostrar que $I(V,V)$  é
zero. {\bf Onde se usa que são mínimas? Intentar construir esta?}
\item Começa com uma geodésica tangente e transporta paralelamente um vetor
tangente $e_i$ a  $M_1$. Ele chega tangente a  $M_2$ porque a geodésica é
minimizante (cf exer passado).
\end{enumerate}
\end{proof}

\begin{exercise}[\cite{doc} Chapter IX, Exercise 5]
\label{exercise-two-submanifolds}
Sejam $N_1$ e $N_2$ duas subvariedades fechadas e disjuntas de uma variedade
Riemanniana compacta.
\begin{enumerate}
\item Mostre que a distância entre $N_1$ e $N_2$ é realizada por uma geodésica
	$\gamma$ perpendicular a ambas $N_1$ e $N_2$.
\item Mostre que, para qualquer variação ortogonal $h(t,s)$ de $\gamma$, com
$h(0,s) \in N_1$ e $h(\ell,s) \in N_2$, tem-se para a fórmula da segunda
variação a seguinte expressão
$$
\frac{1}{2}E''(0)=I_{\ell}(V,V)+
\left<V(\ell),S_{\gamma'(\ell)}^{(2)}V(\ell)\right>
-\left<V(0),S^{(2)}_{\gamma'(0)}(V(0))\right>
$$
\end{enumerate}
\end{exercise}


\begin{exercise}
%\label{exercise}
{\bf (Qualificação)}Prove que $M^{2n},K_M>0$ compact, então ela é simplesmente conexa.
\end{exercise}

\begin{exercise}
%\label{exercise-}
$M^{2n+1}$, $K_M>0$ compacta \(\implies\) orientável.
\end{exercise}

O exercício anterior usa Exercício 6 da lista 6. Ou, equivalentemente,

\begin{exercise}
%\label{exercise-}
Se $M^{2n+1}$ tem $K_M >0$ e $\gamma:S^1 \to M$ inverta orientação, então é
possível encurtar $\gamma$ na sua classe de homotopia.
\end{exercise}

\section{Lista 3}
\label{section-lista-3}

\begin{exercise}[Curvas minimizantes]
\label{exercise-curvas-minimizantes}
\begin{enumerate}
\item Seja $\gamma$ uma curva suave por partes parametrizada por comprimento de
 arco, conectando $p$ a $q$, pontos de uma variedade Riemanniana $(M,g)$.
 Mostre que se $d(p,q)=\ell(\gamma)$, então $\gamma$ é uma geodésica.
 Dizemos que $\gamma$ realiza a distância entre $p$ e $q$.
\item Suponha que $\gamma,\sigma:[0,2]\to M$ são geodésicas distintas
 e satisfazem: $\gamma(0)=\sigma(0):=p$, $\gamma(1)=\sigma(1):=q$, 
$\gamma$ e $\sigma$ realizam a distância entre $p$ e $q$.
 Mostre que $\gamma$ não realiza a distância entre $p$ e $\gamma(1+s)$ 
para nenhum $s>0$.
\end{enumerate}
\end{exercise}

\begin{proof}
\begin{enumerate}
\item Primeiro suponha que $\gamma$ é suave. Podemos ver que trata-se de uma 
geodésica usando seu campo de velocidades como campo variacional na primeira
fórmula da variação:
$$
E'(0)=\int_0^a |\gamma'|=0 \implies \gamma'\equiv 0
$$
Uma geodésica quebrada não pode ser minimizante porque em uma vizinhança 
totalmente normal (cf. \ref{proposition-totally-normal-neighbourhoods}) 
do ponto singular podemos achar uma geodésica radial (suave)
 que acurta a distância entre qualquer ponto antes do ponto singular
 e qualquer ponto depois do ponto singular.
\item Se as duas geodésicas se intersectam não tangencialmente, podemos 
construir um caminho mais curto entre $p$ e $\gamma(1+s)$ ao longo de $\sigma$,
a menos de suavizar a quina perto de $q$. Se as curvas se intersectam
 tangencialmente devem ser a mesma.
\end{enumerate}
\end{proof}

\section{Lista 6}
\label{section-lista-6}

\begin{exercise}
\label{exercise-l6-6}
Seja $M^{2n}$ uma variedade Riemanniana de dimensão para, completa, orientável e
com curvatura seccional $K>0$. Seja $\gamma$ uma geodésica fechada em $M$ de
comprimento $\ell(\gamma)$. Mostre que existem curvas livremente homotópicas a
$\gamma$ em $M$, arbitrariamente próximas de $\gamma$, que possuem comprimento
menor que $\ell(\gamma)$.
\end{exercise}


\section{Lista 7}
\label{section-lista-7}

\begin{exercise}
\label{exercise-minimizing-implies-no-conjugate-points}
Seja $\gamma:[0,a]\to M$ uma geodésica em uma variedade Riemanniana $M$.

 Prove que se $\gamma$ é minimizante, então $\gamma$ não possui pontos
 conjugados em  $(0,a)$. Encontre um exemplo de geodésica $\gamma:[0,a] \to M$ 
sem pontos conjugados que não é minimizante.
\end{exercise}

\begin{proof}
Por contrapositiva, suponha que existem dois pontos conjugados
 ao longo de $\gamma$ e vamos provar que ela não pode ser minimizante. 
Por que não posso simplesmente pegar a variação associada ao campo de 
Jacobi? Essa variação me da duas geodésicas que convergem num ponto,
 e portanto $\gamma$ não pode ser minimizante depois desse ponto. 
Resposta: porque nada me assegura que essas geodésicas 
\end{proof}

\section{Lista 8}
\label{section-lista-8}

\begin{exercise}
\label{exercise-l8-1}
Prop. 2.12 do capítulo XIII, \cite{doc}. Seja $p \in M$. 
Suponha exista um ponto $q \in C_m(p)$ que realiza a distância de $p$ a 
$C_m(p)$. Então:
\begin{enumerate}
\item ou existe uma geodésica 
minimizante $\gamma$ de $p$ a $q$ ao longo da qual $q$ é 
conjugado a $p$,
\item ou existem exatamente duas geodésicas 
minimizantes $\gamma$ e $\sigma$ de $p$ a $q$; além disto,
$\gamma'(\ell)=-\sigma'(\ell)$, 
$\ell=d(p,q)$. 
\end{enumerate}
\end{exercise}

\begin{proof}
No final da secção 1 do Capítulo XIII está especificado que as variedades que se
consideram no capítulo são completas. Portanto podemos supor que existe
uma geodésica minimizante $\gamma$ ligando $p$ e $q$.

Então podemos aplicar a Proposição \ref{proposition-cut-point-characterization}:
 um ponto $q$ é o cut point de $p$ ao longo de uma 
geodésica minimizante $\gamma$ se e somente se
alguma das seguintes condições é verdadeira: 
(a) $q$ é o primeiro ponto conjugado a $p$ ao longo de $\gamma$, ou 
(b) existem duas geodésicas minimizantes ligando $p$ e $q$.

Se (a) é verdadeira, terminamos. Se (b) é verdadeira, temos que existe outra
geodésica minimizante $\sigma$ ligando $p$ e $q$.

{\bf Intento pessoal (não funcionou):}  considere a variação por geodésicas 
$$
f(s,t):=\operatorname{exp}_{\gamma(t)}
(s\operatorname{exp}_{\gamma(t)}^{-1}\sigma(t))
$$
cujo campo de Jacobi 
$$
J(t)=d_{t\text{exp}_{\gamma(t)}^{-1}\sigma(t)}
\text{exp}_p(\text{exp}_{\gamma(t)}^{-1}\sigma(t))
$$
se anula em $t=0,1$. Pensei que esse campo de Jacobi estava bem definido pelo 
Corolário 2.8, Cap. XIII (Lema 
\ref{lemma-outside-cut-locus-exists-minimizing-geodesic}), que assegura
 que $\text{exp}_p$ é injetiva fora do cut locus de $p$. Porém, precisaria
 que a exponencial ao longo de $\gamma$,
 $\text{exp}_{\gamma(t)}$, for injetiva fora do cut locus de $p$. 
Essa condição seria garantida se $d(p,q)=i(M)$, ou seja, se a exponencial
 $\text{exp}_{p'}$ for injetiva na bola de raio $d(p,q)$ para qualquer 
$p' \in M$. Em conclusão: minha variação não está bem definida. 
(Se estivesse, com
isso terminaria o exercício, pois obtemos que o único caso em que $p$ não é
conjugado a $q$ é se $\gamma'(\ell)=-\sigma'(\ell)$, i.e. se o campo de Jacobi 
associado à variação é nulo.) 

\medskip

A variação certa é produzida no \cite{doc} supondo que 
$\gamma'(\ell)\neq -\sigma'(\ell)$; isso implica que existe um vetor
 $V \in T_q M$ tal que 
$\left<\gamma'(\ell),V\right>< 0$ e $\left<\sigma'(\ell),V\right>< 0$.

Uma maneria simples de conferir a existência desse vetor $V$ é olhando para o 
plano gerado por $\gamma'(\ell)$ e $\sigma'(\ell)$ (usando que são linearmente
independentes). O conjunto de vetores $V$ nesse plano satisfazendo 
$\left<\gamma'(\ell),V\right><0$ é um semiespaço, e o mesmo acontece com 
 os vetores satisfazendo $\left<\sigma'(\ell),V\right><0$. Esses dois semiespaços
devem ter interseção não vazia precisamente porque 
$\gamma'(\ell)\neq -\sigma'(\ell)$.

Agora usamos o fato de que $p$ e $q$ não são conjugados para 
achar uma vizinhança do vetor $\ell \gamma'(0)$ onde $\text{exp}_p$ é um
difeomorfismo e assim levantar alguma
curva $r$ que realize o vetor $V$ ao espaço tangente,
 digamos $v:(-\varepsilon,\varepsilon) \to T_q M$.

A variação então está dada por
$$
f(s,t):=\text{exp}_p\left(\frac{t}{\ell}v(s)\right)
$$
Podemos aplicar a primeira fórmula da variação, na qual: o termo com a
integral é zero porque $\gamma$ é uma geodésica; o termo da sumatoria é zero
porque trata-se de uma variação suave; e o primeiro termo dos extremos é zero
porque a variação fixa o ponto de partida. Obtemos que 
$$
E'(0)=\left<V,\gamma'(0)\right><0
$$
Note que Manfredo escreve a equação anterior usando o funcional de distância. 
Isso também é válido: de fato a primeira fórmula da variação pode ser 
deduzida de maneira análoga (feito em sala para variações próprias) 
 para o funcional de distância no caso de geodésicas parametrizadas por 
comprimento de arco (cf. Teo. 6.3 \cite{ler}).

O fato da derivada do funcional de comprimento ser negativa nos diz que para
valores perto de $s=0$ podemos achar curvas com menor comprimento. Mais
precisamente, existe um $s>0$ tal que $\ell(f(s,\cdot))<\ell(\gamma)$.

É claro que o mesmo procedimento genera uma variação $\tilde{f}$ de $\sigma$, e 
podemos supor que a mesma $s>0$ faz $\ell(\tilde{f}(s,\cdot))<\ell(\sigma)$.

Concluímos seguindo o argumento do Professor Manfredo: se 
$\ell(\gamma_s)=\ell(\sigma_s)$, então temos duas geodésicas com o mesmo
comprimento ligando $p$ e $\gamma_s(\ell)=\sigma_s(\ell)$; isso significa 
que existe um ponto $\tilde{t} \in (0,\ell]$ tal que $\gamma_s(\tilde{t})$ é
 o cut point de $p$. Porém, isso contradiz o fato de que $q$ realiza a 
distância entre $p$ e o seu cut locus.

Finalmente, se $\ell(\gamma_s)<\ell(\sigma_s)$, segue que o cut point de $p$
 ao longo de $\sigma_s$ está a distância menor do que $q$, que de novo contradiz
a nossa hipótese.
\end{proof}

\begin{exercise}
\label{exercise-l8-2}
Proposição 2.13, Cap. XIII \cite{doc}. Se a curvatura seccional $K$ de uma
variedade Riemanniana completa $M$ satisfaz
$$
0< K_{\operatorname{min}}\leq K\leq K_{\operatorname{max}},
$$
Então
\begin{enumerate}
\item $i(M) \geq \pi/\sqrt{K_{\operatorname{max}}}$, ou
\item existe uma geodésica fechada $\gamma$ em $M$, cujo comprimento é menor do 
que o de qualquer outra geodésica fechada em $M$, tal que
$$
i(M)=\frac{1}{2}\ell(\gamma)
$$
\end{enumerate}
\end{exercise}

\begin{proof}
Suponha que (1) não é verdadeiro. 

Note que pelo Teorema de Bonnet-Myers 
\ref{theorem-Bonnet-Myers}, $M$ é compacta e portanto $C_m(p)$ é compacto para
todo $p$. Isso nos permite achar dois pontos $p$ e $q$ cuja distância é o raio
de injetividade  $i(M)$.

Então podemos usar o exercício anterior para $p$ e $q$.  Primeiro vamos ver o
que acontece no caso da segunda possibilidade daquele exercício, i.e. que
existam exatamente duas geodésicas ligando $p$ e $q$, digamos $\gamma$ é
$\sigma$, tais que $\gamma'(\ell)=-\sigma'(\ell)$ onde $\ell=d(p,q)$.  Considere
$\gamma * \overline{\sigma}$, onde $*$ é a concatenação de curvas e 
$\overline{\sigma}$ é a curva $\sigma$ percorrida em sentido contrário. É claro
que $\gamma * \overline{\sigma}$ é uma geodésica fechada de comprimento $2
d(p,q)$. Note que por definição  $d(p,q)=\ell=i(M)$, como queríamos.

Essa geodésica fechada tem a distância mínima entre as geodésicas fechadas, já
que se tivéssemos alguma outra com distância menor, podemos pegar um ponto
qualquer nela e o seu cut point ficaria a distância exatamente a metade do
comprimento do laço (pois existem duas geodésicas que chegam nele: cada
metade do laço partindo em direções opostas do ponto inicial escolhido),
contradizendo o fato de que $i(M)=d(p,q)$.

Portanto, basta descartar o primeiro caso do exercício anterior. Para chegar a
 uma contradição, suponha que $p $ é conjugado a $q$, i.e. que existe um campo
 de Jacobi $J$ que se anula em $p$ e em $q$. Podemos usar o teorema de Rauch
\ref{theorem-Rauch} para comparar esse campo com $\tilde{J}$, 
um campo em $S^n_{K_{\text{max}}}$, a esfera de curvatura constante
 $K_{\text{max}}$. Como $K \leq  K_{\text{max}}$,
concluímos que $\tilde{J}$ deve se anular em $\ell=d(p,q)=i(M)$. Absurdo, pois
estamos supondo que (1) não é verdadeiro, i.e. que
 $i(M) < \pi/\sqrt{K_{\text{max}}}$.
\end{proof}

\begin{exercise}
\label{exercise-proposição-3.4}
\cite{doc}, Capítulo XIII, Proposição 3.4. Se a curvatura seccional $K$ de uma
variedade Riemanniana $M^n$, compacta, orientável e de dimensão par, satisfaz 
$0<K \leq 1$, então $i(M) \geq \pi$.
\end{exercise}

\begin{proof}
Para obter uma contradição, suponha que existe um ponto $p \in M$ tal que
 $d(p, C_m(p))<\pi$. Como $M$ é compacta, sabemos que $C_m(p)$ é compacto e
portanto existe uma geodésica $\gamma$ ligando $p$ com $q \in C_m(p)$ e
realizando a distância $d(p,C_m(p))$.

Pelo teorema de Rauch, sabemos que qualquer geodésica não pode ter pontos
conjugados antes de alcançar comprimento $\pi$. Explicitamente, se $J$ é um
campo de Jacobi ao longo de alguma geodésica $\gamma$ tal que $J(0)=0$ e 
$J(\ell(\gamma))=0$, comparando com um campo $\tilde{J}$ em $S^n$ tal que
 $\tilde{J}(0)=0$, $|\tilde{J}'(0)|=|J'(0)|$ e 
$\left<J,\gamma'\right>=\left<\tilde{J},\tilde{\gamma}\right>$, concluímos que
 $|\tilde{J}|\leq |J|$ ao longo de $\gamma$. Como as geodésicas de $S^n$ não
 tem pontos conjugados antes de atingir comprimento $\pi$, concluímos que 
$J$ não pode se anular antes desse ponto.

Isso mostra que, como estamos supondo que $d(p,C_m(p))<\pi$, devemos estar no
caso (2) do Exercício \ref{exercise-l8-1}, i.e. existem exatamente duas
geodésicas $\gamma$ e $\sigma$ ligando $p$ e $q$, e
$\gamma'(\ell)=-\sigma'(\ell)$. Repetindo o procedimento do Exercício anterior
 \ref{exercise-l8-3}, podemos usar essas duas geodésicas para achar uma 
geodésica fechada de comprimento $2i(M)$, cujo comprimento e menor do que o
comprimento de qualquer outra geodésica fechada.

{\bf Observação.} Na prova do teorema de Synge \ref{theorem-Synge}, o Professor
Manfredo afirma que ``Como $M$ é compacta e tem curvatura positiva, $K\geq
\delta>0$"; que não me parece imediato, pois a curvatura seccional não é uma
função definida em $M$. Porém, não preciso mostrar isso para garantir a
existência do loop geodésico; apenas é necessário que $M$ seja compacta para
garantir a existência do ponto $q$.

Minha primeira ideia foi usar o teorema de Synge \ref{theorem-Synge}, 
que nos diz que como $M$ é de dimensão par, orientável, compacta e de 
curvatura positiva, o grupo fundamental dela deve ser trivial. 
Isso significa que o laço geodésico que achamos é homotópico a uma constante,
mas isso não conduz a contradição nenhuma.

O caminho certo é usar o exercício 6 da lista 6, (cf.
\ref{exercise-l6-6}), onde mostrei que, nestas condições, i.e. $M$ de dimensão
par, completa, orientável e com curvatura seccional positiva, existem curvas
livremente homotópicas a qualquer geodésica fechada $\gamma$, que possuem
comprimento menor que $\ell(\gamma)$.

A condição dessas curvas serem ``arbitrariamente próximas" a $\gamma$ significa
que são as curvas de uma variação com campo variacional arbitrariamente pequeno.

\end{proof}

\begin{exercise}
\label{exercise-l8-11}
Seja $M$ uma variedade Riemanniana completa, de curvatura seccional não
negativa. Sejam $\gamma,\sigma:[0,\infty) \to M$ geodésicas tais que
$\gamma(0)=\sigma(0)$. Se $\gamma$ é um raio e
$\angle(\gamma'(0),\sigma'(0))<\frac{1}{2}\pi$, então $\lim_{t \to \infty}
\rho(\sigma(0),\sigma(t))=\infty$ (i.e. $\sigma$ vai para o infinito).
\end{exercise}


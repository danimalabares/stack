\section{Exercises}
\label{section-exercises}

\begin{exercise}
\label{exercise-pts}
$\gamma$ geodésica sem pontos conjugados em $[0,a]$. Então $\gamma$ é localmente
minimizante.
\end{exercise}

\begin{proof}
Temos dois casos:
\begin{enumerate}
\item Se $E''(0)<0$ usamos o teorema do índice.
\item Se $E''(0)=0$,
\begin{enumerate}
\item[(0)] Como $\gamma$ não tem pontos conjugados, então $I$ não tem
		nulidade em todo  $\mathcal{V}$.
\item $I_{\mathcal{V}^-}$ não tem índice nem nulidade. Segue do ponto anterior +
	algum passo na prova.
\item $I_{\mathcal{V}^-}$ não tem índice.
\item 2+3 dão que $I_{\mathcal{V}^-}>0$.
\item Logo $I>0$.
\end{enumerate}
\end{enumerate}
\end{proof}

\begin{exercise}
\label{exercise-wraped-product}
Para $(M,g_M)$ e $(N,g_N)$ e $f:M \to \mathbb{R}_+$, definimos o {\it warped
product} como sendo o produto cartesiano $M\times N$ com a métrica 
 $g_M+f^2g_N$.

Mostre que se os fatores são completos, o warped product é completo.
\end{exercise}

\subsection{Highlight exercises}
\label{subsection-highlight-exercises}
\begin{exercise}
\label{exercise-minimal-intersection}
$(M^n,g)$ completa $\operatorname{Ric}_M>0$. $A$, $B$ hipersuperfícies mínimas
completas. Então  $A \cap B \neq \emptyset$.
\end{exercise}

\begin{proof}
Suponha que existe uma $\gamma$ mínima com comprimento positivo ligando as 
duas variedades. Tem dois caminhos:
\begin{enumerate}
\item Deixa o campo exponencial ao longo da $\gamma$ e ajusta nos extremos
para fazer ele tangente às variedades. Nesse caso já tem que o temo
$\left<f_{ss},\gamma'\right>|_{0}^a$ se anula. Então so mostrar que $I(V,V)$  é
zero. {\bf Onde se usa que são mínimas? Intentar construir esta?}
\item Começa com uma geodésica tangente e transporta paralelamente um vetor
tangente $e_i$ a  $M_1$. Ele chega tangente a  $M_2$ porque a geodésica é
minimizante (cf exer passado).
\end{enumerate}
\end{proof}

\begin{exercise}[\cite{doc} Chapter IX, Exercise 5]
\label{exercise-two-submanifolds}
Sejam $N_1$ e $N_2$ duas subvariedades fechadas e disjuntas de uma variedade
Riemanniana compacta.
\begin{enumerate}
\item Mostre que a distância entre $N_1$ e $N_2$ é realizada por uma geodésica
	$\gamma$ perpendicular a ambas $N_1$ e $N_2$.
\item Mostre que, para qualquer variação ortogonal $h(t,s)$ de $\gamma$, com
$h(0,s) \in N_1$ e $h(\ell,s) \in N_2$, tem-se para a fórmula da segunda
variação a seguinte expressão
$$
\frac{1}{2}E''(0)=I_{\ell}(V,V)+
\left<V(\ell),S_{\gamma'(\ell)}^{(2)}V(\ell)\right>
-\left<V(0),S^{(2)}_{\gamma'(0)}(V(0))\right>
$$
\end{enumerate}
\end{exercise}


\begin{exercise}
%\label{exercise}
{\bf (Qualificação)}Prove que $M^{2n},K_M>0$ compact, então ela é simplesmente conexa.
\end{exercise}

\begin{exercise}
%\label{exercise-}
$M^{2n+1}$, $K_M>0$ compacta \(\implies\) orientável.
\end{exercise}

O exercício anterior usa Exercício 6 da lista 6. Ou, equivalentemente,

\begin{exercise}
%\label{exercise-}
Se $M^{2n+1}$ tem $K_M >0$ e $\gamma:S^1 \to M$ inverta orientação, então é
possível encurtar $\gamma$ na sua classe de homotopia.
\end{exercise}

\begin{exercise}

\end{exercise}

\section{Lista 8}
\label{section-l8}

\begin{exercise}
Prop. 2.12 do capítulo XIII, \cite{doc}. Seja $p \in M$. 
Suponha exista um ponto $q \in C_m(p)$ que realiza a distância de $p$ a 
$C_m(p)$. Então:
\begin{enumerate}
\item ou existe uma geodésica 
minimizante $\gamma$ de $p$ a $q$ ao longo da qual $q$ é 
conjugado a $p$,
\item ou existem exatamente duas geodésicas 
minimizantes $\gamma$ e $\sigma$ de $p$ a $q$; além disto,
$\gamma'(\ell)=-\sigma'(\ell)$, 
$\ell=d(p,q)$. 
\end{enumerate}
\end{exercise}

\begin{proof}[Solução en processo]
Suponha por enquanto que existe
uma geodésica minimizante $\gamma$ ligando $p$ e $q$.

Então podemos aplicar a Proposição \ref{proposition-cut-point-characterization}:
 um ponto $q$ é o cut point de $p$ ao longo de uma 
geodésica minimizante $\gamma$ se e somente se
alguma das seguintes condições é verdadeira: 
(a) $q$ é o primeiro ponto conjugado a $p$ ao longo de $\gamma$, ou 
(b) existem duas geodésicas minimizantes ligando $p$ e $q$.

Se (a) é verdadeira, terminamos. 
Se (b) é verdadeira, considere a variação por geodésicas 
$$
f(s,t):=\operatorname{exp}_{\gamma(t)}
(s\operatorname{exp}_{\gamma(t)}^{-1}\sigma(t))
$$
cujo campo de Jacobi 
$$
J(t)=d_{t\text{exp}_{\gamma(t)}^{-1}\sigma(t)}
\text{exp}_p(\text{exp}_{\gamma(t)}^{-1}\sigma(t))
$$
Se anula em $t=0,1$. Note que se $\gamma'(\ell)=-\sigma'(\ell)$ 
o campo de Jacobi é nulo, e não necessariamente $p$ é conjugado a $q$ ao longo
de $\gamma$. Em qualquer outro caso voltamos no caso (a).

Agora vamos mostrar 

\end{proof}

\begin{exercise}
Proposição 2.13, Cap. XIII \cite{doc}. Se a curvatura seccional $K$ de uma
variedade Riemanniana completa $M$ satisfaz
$$
0\leq K_{\operatorname{min}}\leq K\leq K_{\operatorname{max}},
$$
Então
\begin{enumerate}
\item $i(M) \geq \pi/\sqrt{K_{\operatorname{max}}}$, ou
\item existe uma geodésica fechada $\gamma$ em $M$, cujo comprimento é menor do 
que o de qualquer outra geodésica fechada em $M$, tal que
$$
i(M)=\frac{1}{2}\ell(\gamma)
$$

\end{enumerate}
\end{exercise}



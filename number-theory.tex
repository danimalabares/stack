\input{preamble}

\begin{document}

\title{Number theory}
\maketitle

\phantomsection
\label{section-phantom}
\hfill
\href{http://github.com/danimalabares/stack}{github.com/danimalabares/stack}

\tableofcontents

\bibliography{my}
\bibliographystyle{amsalpha}

\section{Unique factorization domains}
\label{section-UFD}

\begin{lemma}
\label{lemma-Gauss}
If $R$ is a UFD then $R[x]$ is a (U)FD.
\end{lemma}

\begin{proof}
By induction on the degree of $f \in R[x]$. Write $f=cf'$ for
$c=\text{gcd}(\text{coefficients of $f$})$. Then I claim that $f=cf'$ is a
factorization of $f$ into irreducible elements. $f'$ is irreducible because it
is linear and primitive; that is, writing $f'=c'f''$ is impossible because
there is no way to factor a non-unit number from the coefficients of $f'$;
indeed, in such case, $cc'$ would be a {\it larger} common factor of $f$. Here I
define larger as follows using that $R$ is UFD: if $a=a_1\ldots a_k$ and
$b=b_1\ldots b_\ell$ then the gdc is the product of all common factors of $a$
and $b$.

To conclude our induction now suppose that degree-$n$ elements have
factorization and let $f$ be of degree $n+1$. If $f$ is irreducible we are done.
Otherwise $f$ can be expressed as the product of two positive degree elements,
each of which is expressed in a product of irreducibles.

For now I won't prove uniqueness.
\end{proof}

\begin{lemma}
\label{lemma-in-UFD-irreducible-elements-generate-prime-ideals}
In a UFD, irreducible elements generate prime ideals.
\end{lemma}

\begin{proof}
Let $f$ be irreducible. Suppose that $pq\in(f)$. Then $pq=fg$ for some  $g\in
R$. Then $p_1\ldots p_kq_1\ldots q_\ell=fg_1\ldots g_m$. But since $f$ is
irreducible and factorization is unique we conclude that $f$ must be one of the
 $p_i$ or one of the $q_i$. Then $f$ divides $p$ or $q$, i.e. either $p\in(f)$
or $q\in(f)$.
\end{proof}

\end{document}


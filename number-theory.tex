\IfFileExists{stacks-project.cls}{%
\documentclass{stacks-project}
}{%
\documentclass{amsart}
}

% For dealing with references we use the comment environment
\usepackage{verbatim}
\newenvironment{reference}{\comment}{\endcomment}
%\newenvironment{reference}{}{}
\newenvironment{slogan}{\comment}{\endcomment}
\newenvironment{history}{\comment}{\endcomment}

% For commutative diagrams we use Xy-pic
\usepackage[all]{xy}

% We use 2cell for 2-commutative diagrams.
\xyoption{2cell}
\UseAllTwocells

% We use multicol for the list of chapters between chapters
\usepackage{multicol}

% This is generally recommended for better output
\usepackage{lmodern}
\usepackage[T1]{fontenc}

% For cross-file-references
\usepackage{xr-hyper}

% Package for hypertext links:
\usepackage{hyperref}

% For any local file, say "hello.tex" you want to link to please
% use \externaldocument[hello-]{hello}
\externaldocument[introduction-]{introduction}
\externaldocument[conventions-]{conventions}
\externaldocument[sets-]{sets}
\externaldocument[categories-]{categories}
\externaldocument[topology-]{topology}
\externaldocument[sheaves-]{sheaves}
\externaldocument[sites-]{sites}
\externaldocument[stacks-]{stacks}
\externaldocument[fields-]{fields}
\externaldocument[algebra-]{algebra}
\externaldocument[brauer-]{brauer}
\externaldocument[homology-]{homology}
\externaldocument[derived-]{derived}
\externaldocument[simplicial-]{simplicial}
\externaldocument[more-algebra-]{more-algebra}
\externaldocument[smoothing-]{smoothing}
\externaldocument[modules-]{modules}
\externaldocument[sites-modules-]{sites-modules}
\externaldocument[injectives-]{injectives}
\externaldocument[cohomology-]{cohomology}
\externaldocument[sites-cohomology-]{sites-cohomology}
\externaldocument[dga-]{dga}
\externaldocument[dpa-]{dpa}
\externaldocument[sdga-]{sdga}
\externaldocument[hypercovering-]{hypercovering}
\externaldocument[schemes-]{schemes}
\externaldocument[constructions-]{constructions}
\externaldocument[properties-]{properties}
\externaldocument[morphisms-]{morphisms}
\externaldocument[coherent-]{coherent}
\externaldocument[divisors-]{divisors}
\externaldocument[limits-]{limits}
\externaldocument[varieties-]{varieties}
\externaldocument[topologies-]{topologies}
\externaldocument[descent-]{descent}
\externaldocument[perfect-]{perfect}
\externaldocument[more-morphisms-]{more-morphisms}
\externaldocument[flat-]{flat}
\externaldocument[groupoids-]{groupoids}
\externaldocument[more-groupoids-]{more-groupoids}
\externaldocument[etale-]{etale}
\externaldocument[chow-]{chow}
\externaldocument[intersection-]{intersection}
\externaldocument[pic-]{pic}
\externaldocument[weil-]{weil}
\externaldocument[adequate-]{adequate}
\externaldocument[dualizing-]{dualizing}
\externaldocument[duality-]{duality}
\externaldocument[discriminant-]{discriminant}
\externaldocument[derham-]{derham}
\externaldocument[local-cohomology-]{local-cohomology}
\externaldocument[algebraization-]{algebraization}
\externaldocument[curves-]{curves}
\externaldocument[resolve-]{resolve}
\externaldocument[models-]{models}
\externaldocument[functors-]{functors}
\externaldocument[equiv-]{equiv}
\externaldocument[pione-]{pione}
\externaldocument[etale-cohomology-]{etale-cohomology}
\externaldocument[proetale-]{proetale}
\externaldocument[relative-cycles-]{relative-cycles}
\externaldocument[more-etale-]{more-etale}
\externaldocument[trace-]{trace}
\externaldocument[crystalline-]{crystalline}
\externaldocument[spaces-]{spaces}
\externaldocument[spaces-properties-]{spaces-properties}
\externaldocument[spaces-morphisms-]{spaces-morphisms}
\externaldocument[decent-spaces-]{decent-spaces}
\externaldocument[spaces-cohomology-]{spaces-cohomology}
\externaldocument[spaces-limits-]{spaces-limits}
\externaldocument[spaces-divisors-]{spaces-divisors}
\externaldocument[spaces-over-fields-]{spaces-over-fields}
\externaldocument[spaces-topologies-]{spaces-topologies}
\externaldocument[spaces-descent-]{spaces-descent}
\externaldocument[spaces-perfect-]{spaces-perfect}
\externaldocument[spaces-more-morphisms-]{spaces-more-morphisms}
\externaldocument[spaces-flat-]{spaces-flat}
\externaldocument[spaces-groupoids-]{spaces-groupoids}
\externaldocument[spaces-more-groupoids-]{spaces-more-groupoids}
\externaldocument[bootstrap-]{bootstrap}
\externaldocument[spaces-pushouts-]{spaces-pushouts}
\externaldocument[spaces-chow-]{spaces-chow}
\externaldocument[groupoids-quotients-]{groupoids-quotients}
\externaldocument[spaces-more-cohomology-]{spaces-more-cohomology}
\externaldocument[spaces-simplicial-]{spaces-simplicial}
\externaldocument[spaces-duality-]{spaces-duality}
\externaldocument[formal-spaces-]{formal-spaces}
\externaldocument[restricted-]{restricted}
\externaldocument[spaces-resolve-]{spaces-resolve}
\externaldocument[formal-defos-]{formal-defos}
\externaldocument[defos-]{defos}
\externaldocument[cotangent-]{cotangent}
\externaldocument[examples-defos-]{examples-defos}
\externaldocument[algebraic-]{algebraic}
\externaldocument[examples-stacks-]{examples-stacks}
\externaldocument[stacks-sheaves-]{stacks-sheaves}
\externaldocument[criteria-]{criteria}
\externaldocument[artin-]{artin}
\externaldocument[quot-]{quot}
\externaldocument[stacks-properties-]{stacks-properties}
\externaldocument[stacks-morphisms-]{stacks-morphisms}
\externaldocument[stacks-limits-]{stacks-limits}
\externaldocument[stacks-cohomology-]{stacks-cohomology}
\externaldocument[stacks-perfect-]{stacks-perfect}
\externaldocument[stacks-introduction-]{stacks-introduction}
\externaldocument[stacks-more-morphisms-]{stacks-more-morphisms}
\externaldocument[stacks-geometry-]{stacks-geometry}
\externaldocument[moduli-]{moduli}
\externaldocument[moduli-curves-]{moduli-curves}
\externaldocument[examples-]{examples}
\externaldocument[exercises-]{exercises}
\externaldocument[guide-]{guide}
\externaldocument[desirables-]{desirables}
\externaldocument[coding-]{coding}
\externaldocument[obsolete-]{obsolete}
\externaldocument[fdl-]{fdl}
\externaldocument[index-]{index}

% Theorem environments.
%
\theoremstyle{plain}
\newtheorem{theorem}[subsection]{Theorem}
\newtheorem{proposition}[subsection]{Proposition}
\newtheorem{lemma}[subsection]{Lemma}

\theoremstyle{definition}
\newtheorem{definition}[subsection]{Definition}
\newtheorem{example}[subsection]{Example}
\newtheorem{exercise}[subsection]{Exercise}
\newtheorem{situation}[subsection]{Situation}

\theoremstyle{remark}
\newtheorem{remark}[subsection]{Remark}
\newtheorem{remarks}[subsection]{Remarks}

\numberwithin{equation}{subsection}

% Macros
%
\def\lim{\mathop{\mathrm{lim}}\nolimits}
\def\colim{\mathop{\mathrm{colim}}\nolimits}
\def\Spec{\mathop{\mathrm{Spec}}}
\def\Hom{\mathop{\mathrm{Hom}}\nolimits}
\def\Ext{\mathop{\mathrm{Ext}}\nolimits}
\def\SheafHom{\mathop{\mathcal{H}\!\mathit{om}}\nolimits}
\def\SheafExt{\mathop{\mathcal{E}\!\mathit{xt}}\nolimits}
\def\Sch{\mathit{Sch}}
\def\Mor{\mathop{\mathrm{Mor}}\nolimits}
\def\Ob{\mathop{\mathrm{Ob}}\nolimits}
\def\Sh{\mathop{\mathit{Sh}}\nolimits}
\def\NL{\mathop{N\!L}\nolimits}
\def\CH{\mathop{\mathrm{CH}}\nolimits}
\def\proetale{{pro\text{-}\acute{e}tale}}
\def\etale{{\acute{e}tale}}
\def\QCoh{\mathit{QCoh}}
\def\Ker{\mathop{\mathrm{Ker}}}
\def\Im{\mathop{\mathrm{Im}}}
\def\Coker{\mathop{\mathrm{Coker}}}
\def\Coim{\mathop{\mathrm{Coim}}}

% Boxtimes
%
\DeclareMathSymbol{\boxtimes}{\mathbin}{AMSa}{"02}

%
% Macros for moduli stacks/spaces
%
\def\QCohstack{\mathcal{QC}\!\mathit{oh}}
\def\Cohstack{\mathcal{C}\!\mathit{oh}}
\def\Spacesstack{\mathcal{S}\!\mathit{paces}}
\def\Quotfunctor{\mathrm{Quot}}
\def\Hilbfunctor{\mathrm{Hilb}}
\def\Curvesstack{\mathcal{C}\!\mathit{urves}}
\def\Polarizedstack{\mathcal{P}\!\mathit{olarized}}
\def\Complexesstack{\mathcal{C}\!\mathit{omplexes}}
% \Pic is the operator that assigns to X its picard group, usage \Pic(X)
% \Picardstack_{X/B} denotes the Picard stack of X over B
% \Picardfunctor_{X/B} denotes the Picard functor of X over B
\def\Pic{\mathop{\mathrm{Pic}}\nolimits}
\def\Picardstack{\mathcal{P}\!\mathit{ic}}
\def\Picardfunctor{\mathrm{Pic}}
\def\Deformationcategory{\mathcal{D}\!\mathit{ef}}

%Dani's additions
\usepackage{graphicx} %figures


\begin{document}

\title{Number theory}
\maketitle

\phantomsection
\label{section-phantom}
\hfill
\href{http://github.com/danimalabares/stack}{github.com/danimalabares/stack}

\tableofcontents

\bibliography{my}
\bibliographystyle{amsalpha}

\section{Greatest common divisor and least common multiple}
\label{section-gcd-and-lcm}

Following is the universal definition of lcm and gdc:

\begin{definition}
\label{definition-lcm-and-gdc-universal}
A {\it least common multiple} of $a,b \in R$ is an element $d$ such that if for
every $c\in R$, if $a,b|c$ then $d|c$.

A {\it greatest common divisor} of $a,b \in R$ is an element $d$ such that
for every  $c \in R$,  if $c|a,b$ then  $c|d$.
\end{definition}

But it looks like it shall be easier to define them
as follows:

\begin{definition}
\label{definition-lcm-and-gdc-polynomial}
A {\it least common multiple} of $f,g \in R[x]$ is the monic polynomial $d$ of
smallest degree such that $f,g|d$.

A {\it greatest common divisor} of $f,g\in R[x]$ is the monic polynomial $d$ of
greatest degree such that $d|f,g$.
\end{definition}

\begin{lemma}
\label{lemma-existence-of-lcm-and-gcd}
If $R$ is a UFD, in $R[x]$ there exist lcm and mcd, and are unique up to 
multiplication by units.
\end{lemma}

\begin{proof}
We can order by degree, and those of the same degree order by the leading
term… but this requires an order in $R$.
\end{proof}

\section{Unique factorization domains and the Gauss lemma}
\label{section-UFD}

The point of the Gauss lemma is that a polynomial $f$ over the field of fractions
$F$ of a UFD $R$ can be expressed as a rational number called the content of $f$
times a polynomial over $R$ called the primitive part. The point of the Gauss
lemma is that it allows us to make sure that the content will in fact be an
element of $R$.

\medskip\noindent

\begin{lemma}
\label{lemma-irreducible-is-prime}
If $R$ is a UFD then irreducible elements are prime.
\end{lemma}

\begin{proof}
If $f$ is irreducible and $f|ab$ then $f$ must be an element of the unique
factorization of $a$ or $b$.
\end{proof}

\begin{lemma}
\label{lemma-gcd-linear-combination}
There exist elements  $p,q \in R[x]$ such that $fp+gq=\text{gcd}(f,g)$.
\end{lemma}

\begin{proof}
By defining $d$ as the element of least degree of the form $fp+gq$, we obtain
that $d$ must be $\text{gcd}(f,g)$. Indeed, we first check by highschool
division that $d$ must divide any element of the form $fx+gy$. In particular
$d$ divides both $f$ and $g$, so $d \leq \text{gcd}(f,g)$. Also, since
$\text{gcd}(f,g)$ divides both $f$ and $g$, it divides $d=xf+yg$. (Indeed, if
$c:=\text{gcd}(f,g)$ is such that $ca=f,cb=g$ then $c(ax+by)=cax+cby=fx+gy=d$.
\end{proof}

\begin{lemma}
\label{lemma-relatively-prime-and-divides-implies-divides}
If $a|bc$ and $(a,b)=1$, then $a|c$.
\end{lemma}

\begin{proof}
By Lemma \ref{lemma-gcd-linear-combination}, there exist elements $p,q$ such
that $ap+bq=1\implies apc+bcq=c$, so that $a$ divides both summands and thus
divides $c$.
\end{proof}

\begin{lemma}
\label{lemma-polynomials-over-fields-are-UFD}
If $F$ is a field then $F[x]$ is a UFD.
\end{lemma}

\begin{proof}
The existence of the factorization is an easy induction on the degree of $f\in
R[x]$. If $f$ has degree zero, then we are done. If $f$ has positive degree and
is irreducible, we are done. If $f$ has positive degree and it is reducible,
i.e. $f=gh$, then both $g$ and $h$ have strictly smaller degree and we may apply
induction hypothesis to obtain a decomposition of each factor into irreducible
elements.

To prove uniqueness suppose that $f$ has two factorizations in irreducible
elements  $f_1\ldots f_k=f_1'\ldots f_\ell'$. Then $f_1|f_1'\ldots f_\ell'$, so
that it must divide some $f_i'$, which implies that 
$\text{deg}f_1 \leq\text{deg}f_i$. Similarly, $f_1'$ must divide some $f_i$,
meaning that $\text{deg}f_1'\leq \text{deg}f_i$. Supposing that we arrange the
$f_i$ and $f_i'$ in ascending order of degree, we obtain that $f_i=f_i'$ for all
$i$.
\end{proof}

The true definition of content is the following

\begin{definition}
\label{definition-order-of-fraction}
Let $a \in F$. If $p \in R$ is irreducible, define the {\it order} of $a$ at $p$
to be the number $r$ such that $a=p^rb$ for some fraction $b$ that does not have
$p$ as factor in the numerator nor in the denominator.
\end{definition}

We can pick the fraction $a$ so that its denominator and numerator have no prime
factor in common, making the order unique by the uniqueness of the factorization
in $R$.

\begin{definition}
\label{definition-content}
\begin{reference}
\cite[p. 181]{Lang-algebra}
\end{reference}
The {\it  order of $f\in R[x]$ at $p$} is the minimium order of
$p$ in any of the coefficients of $f$. The {\it content} of $f$ is the product
of every irreducible factor $p$ of $f$ raised to its order. A polynomial is
called {\it primitive} if it has content 1.
\end{definition}

As explained in the beginning of this section, the point of the Gauss
lemma is that it will allow us to make sure that the content of $f$ is an
element of $R$.

\begin{lemma}[True Gauss lemma]
\label{lemma-true-Gauss}
\begin{reference}
\cite{Chapter 4, Theorem 2.1}{Lang-algebra}
\end{reference}
Let $R$ be a UFD and $F$ its fraction field. Let $f,g\in F[x]$. Then the
$\text{cont}(fg)=\text{cont}(f)\text{cont}(g)$.
\end{lemma}

This clearly implies that the product of two primitive polynomials being
primitive, and the converse I think is true as well.

\begin{proof}
Since $\text{cont}(bf)=b\text{cont}(f)$ it's enough to suppose that  $f$ and $g$
are primitive. Indeed, taking content of $fg$ we obtain on one hand
$\text{cont}(fg)$ and on the other hand
$\text{cont}(\text{cont}(f)\text{pp}(f)\text{cont}(g)\text{pp}(g))
=\text{cont}(f)\text{cont}(g)\text{cont}(\text{pp}(f)\text{pp}(g))$. 
Thus we would finish if we show that the product of two primitive polynomials is
primitive.

Put another way, it's enough to show will show that if $f,g$ are primitive (as
are the primitive parts of $f$ and $g$ in the preceding paragraph) then $fg$ is
primitive. This means that I will show that any $p$ is not a factor of all the
coefficients in $fg$. A coefficient of $fg$ looks as
$c_j=\sum_{i=0}^ja_ib_{j-i}$.  Since $f,g$ are primitive then $p$ cannot divide
all $a_i$ and all $b_i$.  Thus we can pick the maximum $a_s$ and the maximum
$b_t$ that $p$ does not divide. Then the $c_j$ containing the product $a_sb_t$
will not be divided by $p$, for $p$ cannot divide $a_sb_t$ because it is prime,
but will divide the remaining terms in the sum $c_j$; thus if $p$ divided $c_j$
it would also divide $a_sb_t$lmail.
\end{proof}

\begin{lemma}
\label{lemma-R-UFD-implies-R[x]-UFD}
\begin{reference}
\cite[Chapter 4, Theorem 2.3]{Lang-algebra}
\end{reference}
If $R$ is a UFD then $R[x]$ is a UFD.
\end{lemma}

\begin{proof}
Let $f\in R[x]$ be distinct from 0. Since $F[x]$ is UFD we can write
$f=f_1\ldots f_k$. Taking primitive part on both sides and using Gauss lemma we
get $c\text{pp}(f)=c \text{pp}(f_1)\ldots\text{pp}(f_k)$. The importance of
Gauss lemma is that $c \in R$, so that we may split it as a product of
irreducibles of $R$, which are irreducibles of $R[x]$ as well. The primitive
parts of the factors are also irreducible in $R[x]$, for if $c_i$ is the content
of $f_i$ we observe that $c_i\text{pp}(f_i)=f_i$ is irreducible in $F[x]$ and
thus we couldn't factor $\text{pp}(f_i)$ into two nonunit elements of $R[x]$.

The uniqueness of the factorization follows from the uniqueness of the
factorization in $F[x]$, for another factorization in $R[x]$ would be a
factorization in $F[x]$, making the number of factors the same and equality up
to a rational number. After taking primitive part we see that the rational
number must be $1$.
\end{proof}

\begin{lemma}
\label{lemma-in-UFD-irreducible-elements-generate-prime-ideals}
In a UFD, irreducible elements generate prime ideals.
\end{lemma}

\begin{proof}
Let $f$ be irreducible. Suppose that $pq\in(f)$. Then $pq=fg$ for some  $g\in
R$. Then $p_1\ldots p_kq_1\ldots q_\ell=fg_1\ldots g_m$. But since $f$ is
irreducible and factorization is unique we conclude that $f$ must be one of the
 $p_i$ or one of the $q_i$. Then $f$ divides $p$ or $q$, i.e. either $p\in(f)$
or $q\in(f)$.
\end{proof}

\section{Unique factorization domains (alternative approach)}
\label{section-UFD-alternative}


\begin{lemma}
\label{lemma-primitive-part-unique-up-to-unit}
$\text{pp}(f)$ is unique up to multiplication by unit.
\end{lemma}

\begin{proof}
This follows from uniqueness up to units of gdc and lcm.
\end{proof}

\begin{definition}
\label{definition-primitive-part}
If $f\in F[x]$, define its {\it primitive part} to be a primitive polynomial
$\text{pp}(f)$ such that $c\text{pp}(f)=df$ where
\begin{itemize}
\item $d\in R$ is the minimum common
factor of the denominators of the coefficients of $f$, which may be defined as
the product of the least elements among all the factors (in their unique
factorization) of the denominators such that $df \in R[x]$.
\item $c\in R$ is the greatest common divisor of the coefficients of $df$, which
may be defined as the product of the greatest number of common elements in the
factorizations of each of the coefficients in $df$.
\end{itemize}
\end{definition}

\begin{lemma}[Gauss]
\label{lemma-Gauss}
Let $R$ be a ring and $F$ its field of fractions. A primitive polynomial $f\in
R[x]$ is irreducible in $F[x]$ if and only if it is irreducible in $R[x]$.
\end{lemma}

\begin{proof}
The direct implication is easy: suppose $f$ is irreducible in $F[x]$ and let
$f=gh$ for $g,h\in R[x]$. Then $g,h$ are also in $F[x]$ so that either of them
must be a unit of $F$, i.e. a fraction. But since they are polynomials in
$R[x]$, then it must be an element of $R$. But since $f$ is primitive we obtain
a contradiction unless the number is a unit.

For the converse, suppose that $f$ is irreducible in $R[x]$. To obtain a 
contradiction suppose that $f=gh$ in $F[x]$ with $g,h$ not units in $F[x]$.
Then $f=c\text{pp}(f)=c\text{pp}(g)\text{pp}(h)$ by the True Gauss lemma. 
Notice that we cannot take primitive parts of $g$ and $h$ separately since this
would give perhaps rational contents.
\end{proof}

\begin{lemma}
\label{lemma-irreducible-in-Fx-implies-primitive-part-irreducible}
If $f\in F[x]$ is irreducible (in $F[x]$), then its primitive part
$\text{pp}(f)$ is irreducible in $R[x]$.
\end{lemma}

\begin{proof}
Suppose that $\text{pp}(f)=gh$ for two non-unit polynomials $g,h\in F[x]$.
Multiplying by $c$ as in Definition \ref{definition-primitive-part} we obtain
$c\text{pp}(f)=cgh \implies df=cgh \implies f=\frac{c}{d}gh$, a contradiction
since $f$ is irreducible in $F[x]$. This shows that
$\text{pp}(f)$ is irreducible in $F[x]$. By Gauss lemma, since $\text{pp}(f)$ is
irreducible in $F[x]$ and primitive, it must be irreducible in $R[x]$.
\end{proof}

\begin{lemma}
\label{lemma-Gauss-lemma-corollary}
If $R$ then $R[x]$ is a UFD.
\end{lemma}

\begin{proof}
Let $f\in R[x]$. Let $c$ be the greatest common divisor of the coefficients of
$f$, so that $f=cf'$. Factor $f'$ in $F[x]$ as $f'=f_1'\ldots f_k'$. Take
primitive parts to write $f=c'\text{pp}(f_1')\ldots \text{pp}(f_k')$. By Lemma
\ref{lemma-irreducible-in-Fx-implies-primitive-part-irreducible} each of the
$\text{pp}(f_i')$ is primitive in $R[x]$. By the True Gauss lemma we get that
$c'=c\in R$. Since $R$ is a UFD we also obtain that
$c'$ may be factored into irreducible elements of $R$, which are also
irreducible in $R[x]$.

Uniqueness of the factorization follows from uniqueness of the factorization in
$F[x]$ and Gauss lemma: suppose that $f$ may also be factored into irreducible
elements of $R[x]$ as $g_1\ldots g_\ell$. We wish to show that this is a
factorization in irreducible elements of $F[x]$, which we know to be unique up
to multiplication by units of $F$. To show this we use the converse implication
of Gauss lemma \ref{lemma-Gauss}: after taking primitive parts to obtain
$\text{pp}(g_i)$, which are irreducible and primitive elements of $R[x]$, which
must be irreducible in $F[x]$. Then $i=\ell$, the constants coincide and
$\text{pp}(g_i)=\text{pp}(f_i)$. Since taking primitive part is unique up to
multiplication by unit, we are done.

By induction on the degree of $f \in R[x]$. Write $f=cf'$ for
$c=\text{gcd}(\text{coefficients of $f$})$. Then I claim that $f=cf'$ is a
factorization of $f$ into irreducible elements. $f'$ is irreducible because it
is linear and primitive; that is, writing $f'=c'f''$ is impossible because
there is no way to factor a non-unit number from the coefficients of $f'$;
indeed, in such case, $cc'$ would be a {\it larger} common factor of $f$. Here I
define larger as follows using that $R$ is UFD: if $a=a_1\ldots a_k$ and
$b=b_1\ldots b_\ell$ then the gdc is the product of all common factors of $a$
and $b$.

To conclude our induction now suppose that degree-$n$ elements have
factorization and let $f$ be of degree $n+1$. If $f$ is irreducible we are done.
Otherwise $f$ can be expressed as the product of two positive degree elements,
each of which is expressed in a product of irreducibles.
\end{proof}

\end{document}


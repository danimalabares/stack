\input{preamble}

\begin{document}

\title{Factoring Cremona transformations}
\maketitle

\noindent
Minicourse by Carolina Araujo and Sokratis Zikas.
OPEGA 2026, UFPE.

\hfill Notes at
\href{http://github.com/danimalabares/stack}{
github.com/danimalabares/stack}


\medskip\noindent
Abstract.
O grupo de Cremona em dimensão n é o grupo das transformações birracionais do espaço projetivo de dimensão n. O celebrado teorema de Noether-Castelnuovo (1871-1901) afirma que o grupo de Cremona em dimensão 2 é gerado pelos automorfismos lineares e por uma única transformação quadrática. Em dimensões superiores, não há uma descrição simples do grupo de Cremona em termos de geradores, e a situação é bem mais complicada. Por outro lado, técnicas de geometria birracional, em particular o MMP (Minimal Model Program), fornecem uma maneira de fatorar transformações de Cremona como composições de elos elementares. Essa teoria, conhecida como "Programa de Sarkisov", tem se mostrado extremamente útil no estudo do grupo de Cremona em dimensão superior. Neste minicurso, faremos uma introdução ao MMP e ao Programa de Sarkisov.
\phantomsection

\tableofcontents

\section{Cremona group}
\label{section-cremona-group}

\noindent
Let $\text{Cr}_n(k):=\text{Aut}_k k(x_1,…,x_n)$.

\begin{definition}
\label{definition-rational-map}
$f:X \dashrightarrow Y$ is {\it rational} 
if it is defined in an open dense set $U_X \subseteq X$.
It is {\it birational} if it admits an inverse;
equivalently, if there is $U_Y \subseteq Y$ open dense
such that $f:U_X \to U_Y$ is an isomorphism.
\end{definition}

\noindent
Fix $n =2$ and consider the map
$(x_1,x_2) \mapsto  \left(\frac{1}{x_1},\frac{1}{x_2}\right)$.

\begin{example}
\label{example-standard-quadratic-transformation}
\begin{align*}
\sigma: \mathbb{P}^2 &\dashrightarrow \mathbb{P}^2\\
(x_0,x_1,x_2) &\longmapsto (x_1x_2,x_0x_2,x_0x_1).
\end{align*}

\noindent
Then $\sigma=\sigma^{-1}$, with
\begin{align*}
\{x_i=0\}\mapsto p_i,\qquad \qquad \{x_0=0\}\mapsto (1:0:0).
\end{align*}

\noindent
$\sigma$: standard quadratic transformation.
\end{example}

\medskip\noindent
Question (Enrique, 1895):
Is  $\text{Cr}_n(k)$ simple?
[i.e. does it have any nontrivial normal subgroups?]
Do there exists any homomorphisms $\text{Cr}_n(k) \to H?$ 

$\text{PGL}_{n-1}(k) \cong \text{Aut}(\mathbb{P}^n)
\subseteq \text{Cr}_n(k)$.

\medskip\noindent
How to find homomorphism to a group $H$,
$\text{Cr}_n(k) \to H$.

\begin{enumerate}
\item Find a set of generators $G$ by $\text{Cr}_n(k)$.

\item Get a set $R$ of relators.
$\text{Cr}_n(k)=\left<G|R\right>$.

\item Map the generators to arbitrary elements of $H$
and check that the relators are mapped to $1_H$.
\end{enumerate}

\begin{theorem}[Noether-Castelnuovo, 1872]
\label{theorem-noether-castelnuovo}
$\text{Cr}_2(\mathbb{C})$ is generated by two
automorphisms of $\mathbb{P}^2$ and $\sigma$.
\end{theorem}

\begin{theorem}[Gizatulin, 1983]
\label{theorem-gizatulin}
Description of the relators with respect to two generators
$\{\text{PGL}_3(k),\sigma\}$.
\end{theorem}

\begin{remark}
\label{remark-dim-2}
Contat-Lamy, 2010. $\text{Cr}_2(k)$ is not simple for $k=\bar{k}$.
Lonjoi, 2017: any field.
[This is the proof that uses an action on an
infinite-dimensional hyperbolic space.
The same technique does not work for higher dimensions.]
\end{remark}

\begin{theorem}[Hudson, 1927 \& Pan, 1999]
\label{theorem-hudson-pan}
Any set of [nonlinear] generators for $\text{Cr}_n(k)$, $n\geq 3$,
is uncountable.
\end{theorem}

\noindent
… and we don't know any relators!


\section{MMP and Sarkisov theory}
\label{section-mmp-and-sarkisov}

\noindent
[MMP is an algorithm.]

$$
\xymatrix{
X:\substack{\text{smooth} \\ \text{projective}}
\ar[r]
&  \text{MMP}
\ar[r]
&  X_{\text{min}}: \substack{\text{mild singular} \\ \text{projective}}
}
$$
so that
\begin{itemize}
\item  $X \sim_{bir} X_{\text{min}}$
\item $X_{\text{min}}$ is the ``simpler'' than $X$
\begin{itemize}
\item $K_{X_{\text{min}}}$ is ``more positive'',
\item $\rho(X_{\text{min}}) \leq  \rho(X)$.
\end{itemize}
\item The process is realized in ``elementary''steps.
\end{itemize}

\noindent
The outputs of MMP
are of two types:
\begin{enumerate}
\item minimal models, or
\item Mori fiber spaces (Mfs)
\end{enumerate}

\medskip\noindent
Sarkisov program.
An algorithm for decomposing birational maps
among Mfs into ``simpler'' maps.

[The idea of these maps is to copy the action of 
the standard quadratic transformation $\sigma$
from Example \ref{example-standard-quadratic-transformation}:
we blow up three points (vertices of a triangle, the $p_i$),
then we get a ``hexagonal'' arangement,
and contract three lines (the other three)
to get back at a triangle).]

$$
\xymatrix{
\text{triangle}\ar@{~>}[r]
&\text{hexagon}\ar@{~>}[r]
&\text{triangle}
}
$$
$$
\xymatrix{
\underbrace{\mathbb{P}^2}_{\text{triangle}}
\ar@{~>}[r]
&\underbrace{\mathbb{F}_1}_{\text{Hirzebruch}}
\ar@{~>}[r]
&\ast
\ar@{~>}[r]
&\mathbb{P}^q \times \mathbb{P}^1
\ar@{~>}[r]
&\ast
\ar@{~>}[r]
&\ast
\ar@{~>}[r]
&\underbrace{\mathbb{P}^2}_{\text{triangle}}
}
$$

\begin{theorem}[Corb, 1995 and Hacon-McKeman, 2011]
\label{theorem-corcb-hacon-mckeman}
Any birational map among Mfs can be factores as
a composition of Sarkisov links
if and only if
the Sarkisov links are generators of the groupoid
$$
\underbrace{\text{BirMori}
}_{\supseteq \text{Cr}_n(k)}
=\left\{f:X\dashrightarrow Y
\substack{f\text{ birational} \\ X,Y\text{ Mfs,}\\
X,Y \sim_{\text{bir}}\mathbb{P}^n}\right\}
$$
\end{theorem}

\begin{theorem}[Blanc-Lamy-Zimmerman]
\label{theorem-blanc-lamy-zimmerman}
Description of the relators among the Sarkisov links.
\end{theorem}

\begin{theorem}[BLZ, 2021]
\label{theorem-blz-2021}
$\text{Cr}_n(k)$ is not simple for $n \geq 3$,
$\varphi_i$ only appearing in relations of the form
$\varphi_i \circ X \circ \varphi_i^{-1} \circ X^{-1}=\text{id}$,
$\varphi_i^2=\text{id}$.
\begin{align*}
\text{BirMor}(\mathbb{P}^n) &\longrightarrow \mathbb{Z}/2\mathbb{Z} \\
\varphi &\longmapsto 1\\
\text{links}\neq \varphi&\longmapsto 0
\end{align*}

\end{theorem}






\end{document}

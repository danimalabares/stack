\input{preamble}

\begin{document}

\title{Factoring Cremona transformations}
\maketitle

\noindent
Minicourse by Carolina Araujo and Sokratis Zikas.
OPEGA 2026, UFPE.

\hfill Notes at
\href{http://github.com/danimalabares/stack}{
github.com/danimalabares/stack}


\medskip\noindent
Abstract.
O grupo de Cremona em dimensão n é o grupo das transformações birracionais do espaço projetivo de dimensão n. O celebrado teorema de Noether-Castelnuovo (1871-1901) afirma que o grupo de Cremona em dimensão 2 é gerado pelos automorfismos lineares e por uma única transformação quadrática. Em dimensões superiores, não há uma descrição simples do grupo de Cremona em termos de geradores, e a situação é bem mais complicada. Por outro lado, técnicas de geometria birracional, em particular o MMP (Minimal Model Program), fornecem uma maneira de fatorar transformações de Cremona como composições de elos elementares. Essa teoria, conhecida como "Programa de Sarkisov", tem se mostrado extremamente útil no estudo do grupo de Cremona em dimensão superior. Neste minicurso, faremos uma introdução ao MMP e ao Programa de Sarkisov.
\phantomsection

\tableofcontents

\section{Cremona group}
\label{section-cremona-group}

\noindent
Let $\text{Cr}_n(k):=\text{Aut}_k k(x_1,…,x_n)$.

\begin{definition}
\label{definition-rational-map}
$f:X \dashrightarrow Y$ is {\it rational} 
if it is defined in an open dense set $U_X \subseteq X$.
It is {\it birational} if it admits an inverse;
equivalently, if there is $U_Y \subseteq Y$ open dense
such that $f:U_X \to U_Y$ is an isomorphism.
\end{definition}

\noindent
Fix $n =2$ and consider the map
$(x_1,x_2) \mapsto  \left(\frac{1}{x_1},\frac{1}{x_2}\right)$.

\begin{example}
\label{example-standard-quadratic-transformation}
\begin{align*}
\sigma: \mathbb{P}^2 &\dashrightarrow \mathbb{P}^2\\
(x_0,x_1,x_2) &\longmapsto (x_1x_2,x_0x_2,x_0x_1).
\end{align*}

\noindent
Then $\sigma=\sigma^{-1}$, with
\begin{align*}
\{x_i=0\}\mapsto p_i,\qquad \qquad \{x_0=0\}\mapsto (1:0:0).
\end{align*}

\noindent
$\sigma$: standard quadratic transformation.
\end{example}

\medskip\noindent
Question (Enrique, 1895):
Is  $\text{Cr}_n(k)$ simple?
[i.e. does it have any nontrivial normal subgroups?]
Do there exists any homomorphisms $\text{Cr}_n(k) \to H?$ 

$\text{PGL}_{n-1}(k) \cong \text{Aut}(\mathbb{P}^n)
\subseteq \text{Cr}_n(k)$.

\medskip\noindent
How to find homomorphism to a group $H$,
$\text{Cr}_n(k) \to H$.

\begin{enumerate}
\item Find a set of generators $G$ by $\text{Cr}_n(k)$.

\item Get a set $R$ of relators.
$\text{Cr}_n(k)=\left<G|R\right>$.

\item Map the generators to arbitrary elements of $H$
and check that the relators are mapped to $1_H$.
\end{enumerate}

\begin{theorem}[Noether-Castelnuovo, 1872]
\label{theorem-noether-castelnuovo}
$\text{Cr}_2(\mathbb{C})$ is generated by two
automorphisms of $\mathbb{P}^2$ and $\sigma$.
\end{theorem}

\begin{theorem}[Gizatulin, 1983]
\label{theorem-gizatulin}
Description of the relators with respect to two generators
$\{\text{PGL}_3(k),\sigma\}$.
\end{theorem}

\begin{remark}
\label{remark-dim-2}
Contat-Lamy, 2010. $\text{Cr}_2(k)$ is not simple for $k=\bar{k}$.
Lonjoi, 2017: any field.
[This is the proof that uses an action on an
infinite-dimensional hyperbolic space.
The same technique does not work for higher dimensions.]
\end{remark}

\begin{theorem}[Hudson, 1927 \& Pan, 1999]
\label{theorem-hudson-pan}
Any set of [nonlinear] generators for $\text{Cr}_n(k)$, $n\geq 3$,
is uncountable.
\end{theorem}

\noindent
… and we don't know any relators!


\section{MMP and Sarkisov theory}
\label{section-mmp-and-sarkisov}

\noindent
[MMP is an algorithm.]

$$
\xymatrix{
X:\substack{\text{smooth} \\ \text{projective}}
\ar[r]
&  \text{MMP}
\ar[r]
&  X_{\text{min}}: \substack{\text{mildly singular} \\ \text{projective}}
}
$$
so that
\begin{itemize}
\item  $X \sim_{bir} X_{\text{min}}$
\item $X_{\text{min}}$ is the ``simpler'' than $X$
\begin{itemize}
\item $K_{X_{\text{min}}}$ is ``more positive'',
\item $\rho(X_{\text{min}}) \leq  \rho(X)$.
\end{itemize}
\item The process is realized in ``elementary''steps.
\end{itemize}

\noindent
The outputs of MMP
are of two types:
\begin{enumerate}
\item minimal models, or
\item Mori fiber spaces (Mfs)
\end{enumerate}

\medskip\noindent
Sarkisov program.
An algorithm for decomposing birational maps
among Mfs into ``simpler'' maps.

[The idea of these maps is to copy the action of 
the standard quadratic transformation $\sigma$
from Example \ref{example-standard-quadratic-transformation}:
we blow up three points (vertices of a triangle, the $p_i$),
then we get a ``hexagonal'' arangement,
and contract three lines (the other three)
to get back at a triangle).]

$$
\xymatrix{
\text{triangle}\ar@{~>}[r]
&\text{hexagon}\ar@{~>}[r]
&\text{triangle}
}
$$
$$
\xymatrix{
\underbrace{\mathbb{P}^2}_{\text{triangle}}
\ar@{~>}[r]
&\underbrace{\mathbb{F}_1}_{\text{Hirzebruch}}
\ar@{~>}[r]
&\ast
\ar@{~>}[r]
&\mathbb{P}^1 \times \mathbb{P}^1
\ar@{~>}[r]
&\ast
\ar@{~>}[r]
&\ast
\ar@{~>}[r]
&\underbrace{\mathbb{P}^2}_{\text{triangle}}
}
$$

\begin{theorem}[Corb, 1995 and Hacon-McKeman, 2011]
\label{theorem-corcb-hacon-mckeman}
Any birational map among Mfs can be factored as
a composition of Sarkisov links
if and only if
the Sarkisov links are generators of the groupoid
$$
\underbrace{\text{BirMori}
}_{\supseteq \text{Cr}_n(k)}
=\left\{f:X\dashrightarrow Y
\substack{f\text{ birational} \\ X,Y\text{ Mfs,}\\
X,Y \sim_{\text{bir}}\mathbb{P}^n}\right\}
$$
\end{theorem}

\begin{theorem}[Blanc-Lamy-Zimmerman]
\label{theorem-blanc-lamy-zimmerman}
Description of the relators among the Sarkisov links.
\end{theorem}

\begin{theorem}[BLZ, 2021]
\label{theorem-blz-2021}
$\text{Cr}_n(k)$ is not simple for $n \geq 3$,
$\varphi_i$ only appearing in relations of the form
$\varphi_i \circ X \circ \varphi_i^{-1} \circ X^{-1}=\text{id}$,
$\varphi_i^2=\text{id}$.
\begin{align*}
\text{BirMor}(\mathbb{P}^n) &\longrightarrow \mathbb{Z}/2\mathbb{Z} \\
\varphi &\longmapsto 1\\
\text{links}\neq \varphi&\longmapsto 0
\end{align*}

\end{theorem}

\section{Review of surfaces}
\label{section-surfaces}

\begin{definition}
\label{definition-surface}
A {\it surface} is a smooth projective variety
of dimension 2.
\end{definition}

\begin{example}
\label{example-surfaces}
\begin{enumerate}
\item $S=\mathbb{P}^2$ 

\item
 $$
\xymatrix{
S=\mathbb{P}^1\times \mathbb{P}^1
\ar@{^{(}->}[r]
&z(xy-zw)\subset \mathbb{P}^3_{(x:y:z:w)}
}
$$

\item Blowup de $\mathbb{P}^2$ em $p$.
[Picture showing an arbitrary line $\mathbb{P}^1$ contained
in the projective plane $\mathbb{P}^2$.
Given $P$ not in the line,
we can project any point $Q$ not in the line
to the line.
This gives a rational map
$\pi_P:\mathbb{P}^2\setminus \{P\}\dashrightarrow \mathbb{P}^1$.
Consider the graph $\Gamma$ of $\pi_P$,
it is contained in $\mathbb{P}^2\setminus\{ P\} \times \mathbb{P}^1$.
Its closure, $\overline{\Gamma}$ 
is contained in $\mathbb{P}^2\times \mathbb{P}^1$.
We have projections $p:\mathbb{P}^2\times \mathbb{P}^1 \to \mathbb{P}^2$
and $q:\mathbb{P}^2\times P^1\to \mathbb{P}^1$.
Note that $E=p^{-1}(P) \cong \mathbb{P}^1$
and $S\setminus E \xrightarrow{\cong}\mathbb{P}^1\setminus\{P\}$.
More generally,
$$
\tilde{S}=Bl_PS \xrightarrow{\pi}S
$$
$$
E=\pi^{-1}(P)\cong \mathbb{P}^1 \to P
$$
for $P \in S$.
\end{enumerate}
\end{example}

\begin{definition}
\label{definition-picard}
$S$ surface,
$\text{Pic}(S) = \text{Div}(S)/\sim \cong \{\text{invertible sheaves}\}/\sim$.

$$
\text{Div}(S)
=\left\{\sum_{\text{finite}}n_iD_i:
n_i \in \mathbb{Z},
D_i \subset S\text{ irreducible curve}\right\}.
$$
$$
\text{Div}(S) \iff D-D'=\text{div}(f)=(f)_0 - (f_\infty),\qquad 
f \in K(S)\setminus \{0\}.
$$
\end{definition}



\section{The role of the blowup in surface birational geometry}
\label{section-role-of-blowup}

\noindent
Let $S$ and $S'$ be two (projective nonsingular) surfaces.
\begin{itemize}
\item Any birational morphism $f:S \to S'$
is a composition of blowups.

\item Any birational map $f:S \dashrightarrow S'$ 
factors as
$$
\xymatrix{
&  W \ar[dl]^{p}\ar[dr]_q\\
S\ar@{-->}[rr]&& S'
}
$$
where $W$ is a surface and $p,q$ are compostions
of blowups.
\end{itemize}

\noindent
[We wish to find the simplest variety in the
same birational class.]

\begin{definition}
\label{definition-minimal-surface}
A surface $S$ is called {\it minimal} if 
[if whenever we have a birational map from $S$ to another surface then 
that morphism is an isomorphism]
if any birational morphism $S \to S'$ is an isomorphism.
\end{definition}

\section{Intersection theory for surfaces}
\label{section-intersection-surfaces}

\begin{theorem}
\label{theorem-intersection-form}
Let $S$ be a surface.
There exists a unique symmetric bilinear form
$\text{Pic}(S) \times \text{Pic}(S)\to \mathbb{Z}$
such that
if $C$ and  $C'$ are two curves intersecting transversally, 
then $[C]\cdot[C']=\#(C \cap C')$.
\end{theorem}

\begin{example}
\label{example-intersection}
\begin{enumerate}
\item $S=\mathbb{P}^1$, $\text{Pic}(S)=\mathbb{Z}[H]$ where
$H$ is a line (hyperplane).
$H^2=1$.

\item $S=\mathbb{P}^1 \times \mathbb{P}^1$,
$\text{Pic}(S)=\mathbb{Z}[H_1] \oplus \mathbb{Z}[H_2]$
there $H_i$ is the class determined by either projection.
We have
$$
\begin{cases}
H_1^2=0&=H^2_2 \\
H_1\cdot H_2 &= 1.
\end{cases}
$$
In this case $C \subset S$, $C=Z(F)\subset \mathbb{P}^1_{(x:y)}
\times \mathbb{P}^1_{(z:w)}$,
$F \in k[x,y,z,w]$ 
bi homogeneous of degree $(d_1,d_2)$.

\item $S=Bl_P\mathbb{P}^2 \xrightarrow{\pi}\mathbb{P}^2 $, 
$\text{Pic}(S)=\mathbb{Z}[\pi^*H] \oplus\mathbb{Z}[E]$.
$$
\begin{cases}
\pi^*H \cdot \pi^*H&=H\cdot H=1\\
\pi^*H\cdot E&=0\\
E^2&=?
\end{cases}
$$
$$
\pi^* \ell=\tilde{\ell}+E,\qquad  \ell \text{ line }\subset \mathbb{P}^2
$$
\begin{align*}
E^2&=E\cdot(\pi^* \ell - \tilde{\ell})
=\underbrace{E\cdot \pi^*\ell}_{0}
-\underbrace{E\cdot \tilde{\ell}}_{1}=-1.
\end{align*}
\end{enumerate}
\end{example}

\noindent
In general [for blowups],
let
\begin{align*}
\pi:\tilde{S}=Bl_PS&\longrightarrow S \\
\mathbb{P}^1 \cong E &\longmapsto P
\end{align*}

$$
\text{Pic}(\tilde{S})=\pi^*\text{Pic}(S) \oplus \mathbb{Z}E
$$
$$
\begin{cases}
\pi^*D\cdot\pi^*D'&=D\cdot D \\
\pi^*D\cdot E&=0\\
E^2&=-1.
\end{cases}
$$
\begin{definition}
\label{definition-minus1-curve}
A {\it $(-1)$ curve} in a surface $S$ is a curve $C \subset S$ 
such the $C \cong \mathbb{P}^1$ and $C^2=-1.$
\end{definition}

\begin{theorem}[Castelnuovo's contractibility theorem]
\label{theorem-castelnuovo}
Let $S$ be a surface and $C\subset S$ a $(-1)$-curve.
Then there exists a surface $S' \ni P$ such that
$S \cong Bl_PS'$ and via this isomorphism
$C$ is an exceptional curve.
\end{theorem}

\noindent
Notation. We say that $S\cong Bl_PS' \to S'$ 
is a contraction of $C$.

As a corollary,
\begin{lemma}
\label{lemma-corollary-castelnuovo}
A surface $S$ is minimal if $S$ does not contain any $(-1)$ curves.
\end{lemma}

\section{MMP for surfaces}
\label{section-mmp-for-surfaces}

Let $S$ be a (nonsingular projective) surface.

$$
\xymatrix{
&\text{Does $S$ contain any $(-1)$ curves?}
\ar[dl]\ar[dr]\\
\substack{\text{Não} \\ \text{Stop!}\\
\text{$S$ is a minimal}\\\text{surface}}
&  & 
\substack{\text{Sim} \\ \text{Pick a $(-1)$ curve}\\
\text{Apply Castelnuovo}}\ar@/^2.5pc/[ul]^{\substack{
\text{substitute} \\ S=S'\\\text{where }S=Bl_PS'\\
\text{and repeat}}}
}
$$

\begin{definition}
\label{definition-numerical-equivalence}
Let $S$ be a surface and $D,D' \in \text{Div}(S)$.
$D$ and $D'$ are {\it numerically equivalent} if for any curve $C \subset S$ 
we have $D\cdot C= D'\cdot C$.
$\text{Num}(S)=\text{Div}/ \equiv=\text{Pic}(S)/\equiv$.
\end{definition}

\begin{theorem}
\label{theorem-num}
$\text{Num}(S)$ is a finite-rank free abelian group.
Such a rank is called the {\it Picard number} 
$\rho(S):=\text{rk}(\text{Num}(S))$.
\end{theorem}

\begin{remark}
\label{remark-pic-number}
[The rank of the blowup is one larger.]
$\tilde{S}=Bl_PS$ Then
\begin{align*}
\text{Pic}(\tilde{S})&=\pi^*\text{Pic}(S) \oplus \mathbb{Z}\cdot E\\
\text{Num}(S)&=\pi^*\text{Num}(S) \oplus \mathbb{Z}[E].
\end{align*}
\end{remark}

\noindent
[The following examples shows how we may arrive
at different surface…]

\begin{example}
\label{example-arrive-at-different-surfaces}
Let $S= Bl_{P,Q}\mathbb{P}^2$ and denote
$E_P$ and $E_Q$ the exceptional divisors of $P$ and $Q$.
Let $\ell$ be the line containing $P$ and $Q$.
Consider a lift  $\tilde{\ell} \cong \mathbb{P}^1$.
Then $\pi^* \ell = \tilde{\ell}+E_P+E_Q$
and $(\tilde{\ell})^2=-1$.
[One may show that in fact these are the only $(-1)$ curves.]
Applying Castelnuovo we obtain a contraction of $\tilde{\ell}$,
$S \to \mathbb{P}^1\times \mathbb{P}^1$,
where $\tilde{\ell}$ is mapped to a point 
$R \in \mathbb{P}^1 \times\mathbb{P}^1$.
[It is AG jargon that the blowup of two points is $\mathbb{P}^1\times
\mathbb{P}^1$ 
with …]
Since $\mathbb{P}^1 \times \mathbb{P}^1 \cong Q \subset \mathbb{P}^3$,
[a doubly ruled surface?]
[We also have a map $Q \xrightarrow{\pi_R} \mathbb{P}^2$.
It looks like the two lines passing through $R$
(since it is a doubly ruled surface)
are mapped to two points under  $\pi_R$.]
\end{example}

$$
\xymatrix{
\substack{S \\ \text{rational}\\\text{surface}\\
\text{i.e., }S \sim_{bir}\mathbb{P}^2}
\ar@{~>}[r]^{\text{MMP}}
&\mathbb{P}^2,\mathbb{P}^1\times \mathbb{P}^1, \mathcal{F}_n,
\qquad  n\geq 2.
}
$$
\begin{definition}
\label{definition-scroll}
A {\it scroll} is a surface $S \xrightarrow{\pi}\mathbb{P}^1$ 
such that all fibers are isomorphic to $\mathbb{P}^1$.
$S= \mathbb{P}_{\mathbb{P}^1}(\mathcal{E})$
where $\mathcal{E}$ is a vector bundle of rank 2.

$$
\mathcal{E}=\mathcal{O}_{\mathbb{P}^1}\oplus \mathcal{O}_{\mathbb{P}^1}(n)
$$
$$
\mathbb{P}(\mathcal{E}) \cong 
\mathbb{P}(\mathcal{E} \otimes \mathcal{O}_{\mathbb{P}^1}(n)).
$$
\end{definition}
$$
\mathbb{F}_n=\mathbb{P}_{\mathbb{P}^1}(\mathcal{O} \oplus \mathcal{O}(n)).
$$
\begin{exercise}
\label{exercise-intersection-theory-hirzebruch}
$\text{Pic}(\mathbb{F}_n)=\mathbb{Z}\cdot[f] \oplus \mathbb{Z}\cdot[E]$
$$
\begin{cases}
f^2&=0\\
f\cdot E &= 1\\
E^2&=-n.
\end{cases}
$$

\end{exercise}

\section{Summary}
\label{section-summary}

$$
\xymatrix{
\substack{S\text{ rational} \\ \text{surface}}
\ar@{~>}[r]^{\text{MMP}}
&  \substack{S=\mathbb{P}^2\\\text{or} \\ S=\mathbb{F}_m, \quad
\substack{m=0 \\ \text{or }m \geq 2}}.
}
$$

\begin{exercise}
\label{exercise-rational}
Given $n_0>1$,
construct $S$ such that there exists MMP $\xymatrix{\ar@{<~>}[r]&}$ 
$\mathbb{F}_m$ for all $m=0$ or $2 \leq  m \leq  n_0$.
\end{exercise}

\medskip\noindent
Consider $\mathbb{F}_m$ and $\sigma^2=-m$.
If $m\geq 1 \exists !$ curve $C \subset \mathbb{F}_m$
such that $C^2<0$  $(C=\sigma)$.
If $m=0$, $\mathbb{F}_0=\mathbb{P}^1 \times \mathbb{P}^1$.


$$
\xymatrix{
&\substack{\text{variety containing }
\\\pi^* f,E,\widetilde{f_P}\cong\mathbb{P}^1\\
\tilde{\sigma}}
\ar[dl]^{Bl_P}_\pi
\ar[dr]^{\substack{\text{contraction} \\ \text{of $\widetilde{f_P}$}}}
\ar[d]\\
\substack{\mathbb{F}_m \\
\text{containing}\\
f,f_P,P}
\ar[d]\ar@{-->}[rr]^{\text{Sarkisov link}}
&\mathbb{P}
&\substack{\mathbb{F}_n \\ 
\text{containing}\\
\tilde{E},\tilde{\tilde{\sigma}},Q\\
n=?}
\ar[d]\\
\mathbb{P}^1&& \mathbb{P}^1.
}
$$
\begin{align*}
(\tilde{f}_P)^2&=\tilde{f}_P\cdot(\pi^*f-E)=-1\\
\pi^*f_P …
\end{align*}

\noindent
[Computations show that $n=m+1$.]

\section{Positivity of the canonical class }
\label{section-positivity-canonical}

\noindent
Now we want to reinterpret the MMP
in terms of ``positivity of the canonical class''.

\begin{definition}
\label{definition-canonical-class}
Let $S$ be a surface and $\Omega^1_S$ 
the Kähler differentials on $S$.
The canonic sheaf is $\omega_S=\Lambda^2\Omega_S^1$
[because the dimension of $S$ is 2].
Thus, $[\omega_S]=K_S \in \text{Pic}(S)=\text{Div}/\sim$
[is well-defined].
\end{definition}

\noindent
More concretely,
$\omega$ is a rational 2-form in $S$,
$\text{div}(\omega)=(\omega)_0-(\omega)_\infty$.

\begin{example}
\label{example-canonical-class-of-p2}
\begin{enumerate}
\item $S=\mathbb{P}^2$, $\text{Pic}(S)=\mathbb{Z}\cdot[H]$.
[We look for the zeroes and poles of the form.]
[Choose an affine open] $U=\mathbb{A}^2_{\left(
\underbrace{\frac{x}{z}}_{u},
\underbrace{\frac{y}{z}_{t}}\right) }\subset \mathbb{P}^2$
where $\omega=du \wedge dt$.

Let $V=\mathbb{A}^2_{\left(
\frac{y}{x},\frac{z}{x}\right)}$.
[We look for the coefficient function of]
$\omega= ? dv \wedge ds$.

In $U \cap V$, [computations…]
Thus, $\omega=-\frac{1}{s^3}dv \wedge ds$.
[Thus, this form has a pole of degree 3 at infinity.]

We conclude that $K_{\mathbb{P}^2}=-3H$.

\item $S=\mathbb{P}^1 \times \mathbb{P}^1$,
$\text{Pic}(S)=\mathbb{Z}\cdot H_1 \oplus \mathbb{Z}\cdot H_2$,
[Exercise:] $K_S=-2H_1 - 2H_2$.
\begin{lemma}[Adjunction formula]
\label{lemma-adjunction}
Let $X$ be a smooth projective variety
and $Y \subset X$ an irreducible hypersurface.
Then $K_Y=(K_X+Y)|_Y$.
\end{lemma}

\noindent
[Then]
$$
S=\mathbb{P}^1 \times \mathbb{P}^1 \cong S_2 \subset \mathbb{P}^3
$$
$$
K_S = (-4H+2H)|_S=-2H|_S
$$

\item $S= \mathbb{F}_m$, $\text{Pic}(\mathbb{F}_m)=\mathbb{Z}\cdot f
\oplus \mathbb{Z}\cdot\sigma$… [More computations].
\end{enumerate}
\end{example}

\medskip\noindent
\begin{definition}
\label{definition-positive}
A divisor $D \in \text{Div}(S)$
is {\it nef} if [it has nonegative intersection number with
any curve in $S$] $D\cdot C \geq 0$ for any curve $C \subset S$.
\end{definition}

\begin{example}
\label{example-positive-divisors}
\begin{enumerate}
\item (A very ample divisor.) Consider
$$
\xymatrix{
S \ar@{^{(}->}[r]^f
&\mathbb{P}^n
}
$$
where $\text{Pic}(\mathbb{P}^n)=\mathbb{Z}\cdot[H]$,
then $f^*H$ is very ample since
$f^*H\cdot C = f_*(C)\cdot H >0$.

\item (Ample divisor.) Recall that $D$ is ample if $mD$ 
is very ample for some $m>0$.
Let $f:S \to \mathbb{P}^m$ be a morphism,
$D= f^* H$, $C \subset S$.
Then
$$
D\cdot C
=H \cdot f_\ast(C)
=
\begin{cases}
>0\qquad &\text{if }f_*(C)\text{ is a curve} \\
=0\qquad &\text{if }f(C) = pt.
\end{cases}
$$
[Qualquer $D$ que seja o pullback da classe do hiperplano
sob alguma $f$ é dito de semi-amplo.]
$D$ is called {\it semi-ample}.
\end{enumerate}
\end{example}

\section{Kleimann amplness theorem}
\label{section-kleimann}

\noindent
``nef= limit of ample''.

\medskip\noindent
Recall that we defined
$$
\text{Num}(S) = \text{Div}(S)/\equiv = \mathbb{Z}^{\rho(S)}.
$$
Now let
$$
N'(S):= \text{Num}(S) \otimes_{\mathbb{Z}}\mathbb{R}=\mathbb{R}^{\rho(S)}.
$$
[Now we consider the following cones:]

$\overline{NE(S)}$,
the closed cone generated by all classes $[C]$ of curves $C \subset S$,
called the {\it Mori cone}. $\overline{NE(S)}\subset N'(S)$.

Let
$D \in \text{Div}(S)$.
$D$ is nef if and only if 
$D\cdot\alpha \geq 0\forall \alpha \in \overline{NE(S)}$.

$\text{Nef}(S)$ is the closed convex cone generated by
the nef divisors = $(\overline{NE_1(S)})^\vee \subset N^1(S)$.

\begin{theorem}[Kleimann's amplitude theorem]
\label{theorem-kleimann}
Let $D \in \text{Div}(S)$.
$D$ is ample if and only if $D\cdot \alpha>0$ for all
$\alpha \in \overline{NE_1(S)}\setminus\{0\}$
if and only if $D \in \text{Int}(\text{Nef}(S))$.
\end{theorem}

\begin{definition}
\label{definition-face-extremal-rays}
Let $C \subset \mathbb{R}^n$ a closed convex cone.
A {\it face} $F \subset N$ is a subcone such that
$u,v \in N$ and $u+v \in F$ imply
that $u,v \in F$.

An {\it extremal ray} of $N$ is a face of [maximal?] dimension.
\end{definition}

\begin{example}
\label{example-cone}
\begin{enumerate}
\item 
$S= \mathbb{F}_m$,
$N_1(\mathbb{R}^2)$.
$K_S \cdot f=-2$,
$K_S \cdot \sigma=n-2$.

\item (The extremal ray of the exceptional divisor
is an extremal ray.)

$$
\xymatrix{
\tilde{S}=Bl_PS\ar[r]^{\pi}
&S\\
\mathbb{P}^1\ar[r]\ar@{^{(}->}[u]
&pt\ar@{^{(}->}[u].
}
$$
Then $\mathbb{R}_{\geq 0}E$ is an extremal
ray of $\overline{NE_1(\tilde{S})}$.

$K_S\cdot E=-1$.
[ $(-1)$-curves always have $-1$ intersection number with the canonical
divisor.]
\end{enumerate}
\end{example}

\begin{exercise}
\label{exercise-adjunction-again}
$C \subset S$, $C$ a $(-1)$-curve if and only if
$C^2 <0$ and $C\cdot K_S<0$
if and only if $C^2=-1$ and  $K_S\cdot C=-1$.
\end{exercise}

\begin{remark}
\label{remark-minimal-surface}
If $K_S$ is nef, then  $S$ is a minimal surface.
$\mathbb{P}^2$ is minimal but $K_{\mathbb{P}^2}=-3H$.
\end{remark}

\begin{definition}
\label{definition-minimal-model}
$S$ is  {\it minimal model} if $K_S$ is nef.
\end{definition}

\begin{theorem}[Mori's cone theorem]
\label{theorem-mori-cone}
Let $S$ be a surface.
\begin{align*}
\overline{NE}_1(S)&=\overline{NE}_1(S)^{K_S \geq 0}
+\sum_i\mathbb{R}_{\geq 0}[C_i]\\
&=\overline{NE}_1(S)^{K_S+A \geq 0}
+ \sum_{\text{finite}}\mathbb{R}_{\geq 0}[C_i], \qquad C_i\cong
\mathbb{P}^1.
\end{align*}
\end{theorem}

$$
\left[ 
\substack{\text{
Figure of $N^1(S)$ split into two regions} \\ 
\text{by $K_S$. One region is positive}\\
\text{and the other negative. In the boundary}\\
\text{we see $K_S + \varepsilon A$, ample.}} \right] 
$$

\begin{theorem}[Classification of the $C_i$ 's]
\label{theorem-classification}
\begin{itemize}
\item $C_i^2<0 \implies C_i$ is a $(-1)$ curve.
\item $C_i^2 = 0 \implies $ $S \cong \mathbb{F}_m$ and $C_i=f$.
\item  $C_i^2 >0 \implies S \cong \mathbb{P}^2$ and $C_i=\lambda H$.
\end{itemize}
\end{theorem}



\section{Summary }
\label{section-summary}

\noindent
[Today: interperetation of the classical MMP
in modern language.]

\medskip\noindent
Recall:
\begin{itemize}
\item $D \in \text{Div}(S)$ is {\it nef} if
$D\cdot C \geq 0$ for all $C \subset S$.

\item [Take the blow up of a surface at a point:]
$\tilde{S}=Bl_PS \xrightarrow{\pi}S$ with exceptional divisor $E$,
then $K_S\cdot E=-1$ and $E^2=-1$.

\item $S= \mathbb{F}_m$, $K_S\cdot f=-2$.
$f^2=0$.

\item\leavevmode [The Mori cone is the convex cone generated
by the curves in $S$:]
$\overline{NE}_1(S)=\overline{\left<\text{curves in $S$}\right>}
\subset \underbrace{N^1(S)}_{=\mathbb{R}^e}
=\text{Num}(S) \otimes_\mathbb{Z} \mathbb{R}$.

\item Theorem. Let $S$ be a surface. Then
\begin{enumerate}
\item (Cone theorem.) [
Picture of the hyperplane $K_S$ limiting two regions,
a positive and a negative one. The negative side is locally
polyhedral - while on the positive
side it may be round. We also consider a small
perturbation $K_S+\varepsilon A$.] 

\item (Contraction theorem.)
Any face on the negative side of the Mori cone admits a contraction.
[See definitions below.]

\item (Classification of contractions
of $K_S$-negative extrmal rays.)
\begin{itemize}
\item If $C^2<0$, then  $\text{cont}_R=Bl_P$.
\item If $C^2=0$, then  $S= \mathbb{F}_m$
and $\text{cont}_R:\mathbb{F}_m \to \mathbb{P}^1$.
\item If $C^2 \geq 0$,
then $S \cong \mathbb{P}^1$.
\end{itemize}
\end{enumerate}

\item \leavevmode [A face is a hyperplane that encosta no cone.]
A {\it face} $F$ of a cone is such that  $v,u \in \overline{ NE}(S)$
and $v+u \in F$ then $u,v \in F$. 

\begin{definition}
\label{definition-contraction-of-face}
Let $F \subset \overline{NE}_1(S)$ be a face.
A {\it contraction} of $F$ is a morphism
$\text{cont}_F:S \to X$ 
with connected fibers,
$X$ normal such that
given a curve $C \subset S$,
$$
\text{cont}_F(C) = pt \iff
[C] \in F.
$$
\end{definition}

\begin{remark}
\label{remark-contractions}
$\text{cont}_F$ may or not exist, but if it exists it is unique.
\end{remark}
\end{itemize}

\section{Modern version of MMP}
\label{section-modern-mmp}

\noindent
[Instead of asking whether there are any $(-1)$ curves,
now we ask if the canonical divisor is positive:]

$$
\xymatrix{
&\text{Is $K_S$ nef?}
\ar[dl]\ar[dr]\\
\substack{\text{Yes} \\ \text{Stop!}\\
\text{$S$ is a minimal}\\\text{model}}
&  & 
\substack{\text{No} \\
\text{Pick a $K_S$ negative}\\
\text{extremal ray}\\
\text{and take}\\
f=\text{cont}_R:S \to X.}
\ar[dr]\ar[dl]\\
& 
\substack{\dim X=2 \\
\text{Then} \\
S=Bl_P\underbrace{S'}_{X}}
\ar[uu]^{\substack{S:=S' \\ 
\rho(S')\\
\rho(S)=1}}.
&& 
\substack{\dim X <2 \\ 
\text{Stop!}\\
\text{In this case}\\
f:S \to X\\
\text{is a Mori fiber space}\\
\text{second item in}\\
\text{classification part of thm}}.
}
$$


\begin{example}
\label{example-hirzebruch}
$S=\mathbb{F}_m$,
we computed that $K_S=-2\sigma -(2+m)f$.
$K_S\cdot f=-2$,
$K_S\cdot \sigma = 2m-2-m=m-2$.
\end{example}

\noindent
[Notice the diagram is more complicated now: there are more options.
Why is this formulation convenient?]

Now fix a divisor $D \in \text{Pic}(S)$.
We may consider now the $D$-MMP:
$$
\xymatrix{
&\text{Is $D$ nef?}
\ar[dl]\ar[dr]\\
\substack{\text{Yes} \\ \text{Stop!}\\
\text{$S$ is a $D$-minimal}\\\text{model}}
&  & 
\substack{\text{No} \\
\text{Pick a $D$ negative}\\
\underbrace{\text{extremal ray}}_{\text{may not exist!}}\\
\text{and take}\\
f=\text{cont}_R:S \to X.}
\ar[dr]^{\substack{\text{type} \\\text{A}}}
\ar[dl]_{\substack{\text{type} \\ \text{B}}}\\
& 
\substack{\dim X=2 \\
\text{Then} \\
f:S \to S'}
\ar[uu]^{\substack{
X\text{ may be singular}\\
D:=f_* D}}
&& 
\substack{\dim X <2 \\ 
\text{Stop!}\\
\text{In this case}\\
f:S \to X\\
\text{is a $D$-Mori fiber space}\\
\text{second item in}\\
\text{classification part of thm}}.
}
$$

Some remarks:
\begin{itemize}
\item The final result of MMP (or $D$-MMP)
may not be unique,
but its type (A or B)
depends only on the birational class of $S$
(or  $D=K_S$).

\item When MMP ends in B
(i.e.  $S$ is a [rational] ruled surface)
the result is in general not unique.
\end{itemize}

\begin{example}
\label{example-mmp-rational-surface}
When $S$ is a rational surface,
the possible results are wither
$\mathbb{P}^2 \to pt$ 
or
$\mathbb{F}_m \to \mathbb{P}^1$.
\end{example}


\medskip\noindent
When does a $D$-MMP exist?

\begin{itemize}
\item $S$ toric.
($D$-MMP exists for any  $D$.)

\item $S$ Mori dream space (MDS).
($D$-MMP exists for any $D$.)
[This is the reason for the name.]
If $\text{Pic}(S)=\mathbb{Z}^{\oplus m}$,
the {\it Cox ring} is
 $\text{Cox}(S)=\bigoplus_{D \in \text{Pic}(S)}H^0(S,D)$.
$S$ is a {\it Mori dream space} if $\text{Cox}(S)$ is finitely generated.

\item Let $S$ be any surface, $D=K+\varepsilon A$,
$A$ ample.
[With such a $D$ we always guarantee that the program exists.]

\begin{remark}
\label{remark-ample}
If $D \in \overline{NE}(S)$,
the $D$-MMP always ends in a $D$-minimal model.
\end{remark}
\end{itemize}

\noindent
[Recall, the classical version consists of
a sequence of blowdowns (or blowups)
starting from $S$ and finishing in $S'$.]
$$
\xymatrix{
S \ar[r]&\cdots
\ar[r]
&S' \ar[d]_g\\
& &\underbrace{X}_{\substack{\text{$D$-ample} \\ \text{model}}}
}
$$
$D'=g^*AX$, then 
$D'$ is nef (semi-ample).

\medskip\noindent
[Extra: A K3 is a MDS is it has a trivial automorphism group
(for $D$-MMP).]

\section{Geography of ample models}
\label{section-geography}

\begin{example}
\label{example-toric-surface}
[The following is a toric surface;
we worked on this two lectures ago.]
$S= Bl_{P,Q}\mathbb{P}^2$
[The cone has dimension 3, but we can make a transverse section.
Imagine that the rest of the cone is behind the screen:]
$$
\xymatrix{
  &           &  \tilde{\ell}
\ar@{-}[dl]\ar@{-}[dr]\\
  &  \tilde{\ell}_Q\ar@{-}[ddl]
\ar@{-}[dr]\ar@{-}[rr]&  &  \tilde{\ell}_{P} \ar@{-}[ddr]\ar@{-}[dl]\\
  &                          & H\\
E_P\ar@{-}[urr]\ar@{-}[uur]
\ar@{-}[rrrr]& & & & E_Q\ar@{-}[ull]\ar@{-}[uul]
}
$$

[$H$ should be in the center of the of the quadrilateral
with vertices $E_P$,  $E_Q$  $\tilde{\ell}_P$ and $\tilde{\ell}_Q$.
$Nef(S)$ is the triangle with vertices
 $\tilde{\ell}_Q, \tilde{\ell}_P$ and $H$.]
\end{example}

$$
\xymatrix@C=8em{
S\ar[r]^-{f}_{\text{blowups}}
&\underbrace{S'}_{\substack{
D\text{-minimal}\\\text{model}}}
\ar[r]^-{\varphi |mD|}
&\underbrace{X}_{\substack{
\text{$D$-ample} \\ \text{model}}}=X(S,D).
}
$$
\begin{definition}
\label{definition-mori-equivalence}
Let $D, D' \in \overline{NE}(S)$.
$D \sim_{\text{Mori}}D'$ if
the morphisms
$$
S \to X(S,D)\text{ and }
S \to X(S,D')
$$
are the same
(up to $X(S,D) \cong X(S,D')$).

\end{definition}

[Inside $Nef(S)$ the model is itself.
In the lower edge of the triangle,
the model is $\mathbb{P}^2 \to pt$.
In the edge from $\tilde{\ell}_p$ to $\tilde{\ell}$,
$\mathbb{P}^1 \times \mathbb{P}^1 \xrightarrow{p_1}\mathbb{P}^1$.
In the edge $\tilde{\ell}_p$ to $E_Q$,
$\mathbb{F}_1 \to \mathbb{P}^1$.]

[This figure shows us all the Mfs that may be obtained from $S$.
The results are codified in the faces of the cone decomposition.
The strategy of the proof of Sarkisov theorem
is to move around in the boundary of the cone;
that is, passing from one Mfs to another.]

\begin{theorem}[Sarkisov program]
\label{theorem-sarkisov}
Any birational map of Mfs is a composition of
{\it Sarkisov links}, which may be either of the follwing:
\begin{enumerate}
\item 
$$
\xymatrix{
&\mathbb{F}_1
\ar[dl]\ar[dr]\\
\ar[d]\mathbb{P}^2 & & \mathbb{P}^1\\
pt
}
$$

\item 
$$
\xymatrix{
&  \tilde{S}\ar[dl]^{Bl_P}\ar[dr]\\
\mathbb{F}_m \ar[d]\ar@{-->}[rr]&&  \mathbb{F}_{m \pm 1}\ar[d]\\
\mathbb{P}^1 & & \mathbb{P}^1
}
$$
\item 
$$
\xymatrix{
&\mathbb{F}_1\ar[dl]\ar[dr]\\
\mathbb{P}^1 &  &  \mathbb{P}^2\ar[d]\\
&& pt
}
$$

\item  $\mathbb{F}_0 = \mathbb{P}^1 \times \mathbb{P}^1$,
$$
\xymatrix{
\mathbb{F}_0\ar[d]_{p_1} \ar@{=}[r]&  \mathbb{F}_0\ar[d]^{p_2}\\
\mathbb{P}^1&\mathbb{P}^1
}
$$
\end{enumerate}
\end{theorem}

\noindent
[In fact, the Sarkisov links appear when
we pass from one chamber to another in the Mori cone.]




\section{Proof of Sarkisov theorem}
\label{section-proof-sarkisov}

\begin{itemize}
\item 
Let $D \in N^1(S)_d$ be of the form $K_S + A$, $A$ ample.

\item A birational map $f:S \to T$ is called a 
 {\it $D$-minimal model} if it is the result of a $D$-MMP.

\item A map $g:S \to B$ is called an
{\it ample model of $D$} if it factos from a $D$-minimal model
$f:S \to T$ followed by the contraction of $f_*D$.

\item (Geography of ample models.)
We pick two ample divisors $A_1,…,A_k, K_S \not\in \overline{NE}(S)$,
 $$
\mathcal{C}=
\{D=A_0K_S + \sum_{i=1}na_iA_i:
a_0,…,a_k \geq 0\}\cap \overline{NE}(S).
$$
$D_1,D_2$ are Mori equivalent if they have the same ample model.

\item A {\it Mori chamber} is an equivalence class
$(\mathcal{A}_i)_{i \in I}$ with ample model
$f:S \to T_i$. $\mathcal{A}_I$ is a {\it big chamber}
if $f_i$ is birational.
\end{itemize}

\begin{proposition}
\label{proposition-mori-chambers}
\begin{enumerate}
\item The chamber decomposition 
[of the Mori cone $\mathcal{C}$ in Mori chambers]
is finite.
(Cone theorem + $K_S \not\in \overline{ \text{NE}}(S)$.)

\item The chambers are convex subcones.
$(\mathcal{A}_i=
\left<f_i^*\text{Nef},f_i\text{-exeptional divisors}\right>$.

\item $\mathcal{A}_i$ big chamber if and only if
\label{item-3}
$\dim \mathcal{A}_i = \dim \mathcal{C}$.

\item If $\mathcal{A}_i$ is big,
$\overline{\mathcal{A}_i}=
\{D \in \mathcal{C}:f_i:S \to T_i\text{ is a minimal $D$-model}\}$.
(ample $\to$ nef).

\item If $\overline{\mathcal{A}_i}\cap\mathcal{A}_j \neq \emptyset$ 
there exists $\rho_{ij}:T_i \to T_j$ making the diagram
$$
\xymatrix{
A \ar[r]^{f_i}\ar[dr]_{f_j}&  T_i\\
& T_j
}
$$
commute and $\rho(T_i)-\rho(T_j)
=\dim(\mathcal{A}_i)-\dim(\overline{\mathcal{A}_i}
-\dim(\overline{\mathcal{A}_i} \cap \mathcal{A}_j)$.
\end{enumerate}
\end{proposition}


\begin{proof}[Proof of Sarkisov program]
[Start with a birational map of Mfs]
$$
\xymatrix{
& W\ar[dl]_{p_1}\ar[dr]^{p_2}\\
S_1\ar@{-->}[rr]^{\varphi}\ar[d]_{\eta_1}
&&S_2\ar[d]^{\eta_2}\\
B_1
&&B_2
}
$$
$\mathcal{C}_w$, there exist $D_1,D_2 \in \mathcal{C}$
such that $\eta_i \circ p_i$ is an ample model of $D_i$.
By (\ref{item-3}), $D_i \in \partial \overline{\text{NE}}(W)$.

[The idea is to show that there exists a way that joins
those two points which contained completely in the 
boundary.
After showing it exists,
we show that this path is in fact a Sarkisov link decomposition.]

The visible boundary is
$$
\partial^+ \mathcal{C}
=\left\{D \in \partial \mathcal{C}:
\substack{ D= \partial \mathcal{C} \cap [K_W,D^0 ]\\
\text{for some }D^0 \in \text{Int}(\mathcal{C}) }\right\}
\subseteq \partial\mathcal{C}.
$$
[Geometrically, we know that the canonical divisor
is not in the cone -- because it is not ample.]

\begin{lemma}
\label{lemma-boundary}
\begin{itemize}
\item The visible boundary is contained in $\partial \overline{
\text{NE}}(W)$.

\item $\dim \partial^+\mathcal{C}=\dim \mathcal{C}-1$.
\item $\partial^+\mathcal{C}$ is connected.
\end{itemize}
\end{lemma}

\begin{proof}
\begin{align*}
D \in \partial^+ \mathcal{C}_1 D&=
t K_W+(1-t) A_0,\qquad  t \in [0,1)\\
A_0 \in \text{Int}\mathcal{C} 
&  \implies A_0 = \sum a_i A_i,\qquad a_i>0, \forall i=1,…,k\\
D\not\in \partial \mathcal{D} &\implies 
D \in \partial \overline{\text{NE}}(W)
\end{align*}
where $\mathcal{D}:=\{a_0K_W+\sum a_iA_i:a_i \geq 0\}$ 
which appears in the definition of $\mathcal{C}$.
\end{proof}

\noindent
Let $V_2 \subseteq N^1(W)$ 
be a 2-dimensional affine space.
Take a slice of the cone:
$\mathcal{C} \cap V_2=\mathcal{P}$.
[This space has to be general in the sense that]
$\text{codim}_{V_2}(\mathcal{A}_i \cap V_2)=
\text{codim}_{\mathcal{C}}\mathcal{A}_i$.
[We obtain a polygon with a chamber decomposition.
Let's see what happens at a point $\Theta$ where two
(or more) chambers meet.
Zooming in, we see a some lines $\mathcal{O}_0,…,\mathcal{O}_k$
meeting at $\Theta$, which determine chambers
$\overline{\mathcal{A}}_1, \overline{\mathcal{A}}_2,…,
\overline{\mathcal{A}}_k$.]

[The $\mathcal{O}_i$ are the intersections of the sheaves.
The $\mathcal{O}_i$ are not chambers themselves:]

\begin{proposition}
\label{proposition-o-are-not-chambers}
 \begin{enumerate}
\item $0<i<k$, $\mathcal{O}_i \subseteq \mathcal{A}_i$ or 
$\mathcal{O}_i \subseteq \mathbb{A}_{i+1}$.

\item $D_0 \in \text{ rel int}\mathcal{O}_0$,
the ample model of $D_0$ is a Mfs.

\item $k<3$.

\item 
Let $\mathcal{A}$ be the chamber such that $\Theta \in \mathcal{A}$.
If there exists $i$ such that
$\dim(\overline{\mathcal{A}}_k \cap \mathcal{A})=0$ 
then there exists a Sarkisov link between the
Mfs of Proposition \ref{proposition-mori-chambers}.
\end{enumerate}
\end{proposition}

\begin{proof}
\begin{align*}
\text{Type I}
&\xymatrix{\ar@{<~>}[r]&}k=2, \mathcal{O}_q \subseteq \mathcal{A}_2\\
\text{Type II}&\xymatrix{\ar@{<~>}[r]&}k=3\\
\text{Type III}&\xymatrix{\ar@{<~>}[r]&}
k=2,\mathcal{O}_q \subseteq \mathcal{A}_1\\
\text{Type IV}&\xymatrix{\ar@{<~>}[r]&} k=1.
\end{align*}
\end{proof}
\end{proof}






\section{About simplicity of the Cremona group}
\label{section-simplicity-Cremona-group}

Recall that the {\it Cremona group} is defined by
$$
\text{Cr}_n(k)=
\{f:\mathbb{P}^n \dashrightarrow \mathbb{P}^n:
f\text{ is birational}\}.
$$
Question by Enriques (1865):
is $\text{Cr}_n(k)$ simple?

We know that $\text{Cr}_n(k)$ is not simple in the following cases:
\begin{enumerate}
\item $n=2$, $k$ any field.
Contat-Lamy '13, Lonjou '16.

\item $n=2$, $k$ perfect field
such that there exists a Galois orbit of size 8.
Lamy-Zimmermann '20.

\item $n \geq 3, k=\mathbb{C}$. Blanc-Lamy-Zimmermann '21.
\end{enumerate}

[The second and third item they use a different method
to define the automorphisms.]

\begin{theorem}
\label{theorem-cr3}
There is an isomorphism $\text{Cr}_3(\mathbb{C}) \simeq
G \ast (\ast_J \mathbb{Z}/2\mathbb{Z})$
where $J$ is uncountable.
\end{theorem}

With corollaries:

\begin{lemma}
\label{lemma-corollary}
$\text{Aut}(\text{Cr}_3(\mathbb{C}))$ 
is not generated by inner automorphisms
nor automorphisms of the field $\mathbb{C}$.
\end{lemma}

Déserti '06: $\text{Aut}(\text{Cr}_2(\mathbb{C}))$ 
is generated by such automorphisms.

\medskip\noindent
[We shall prove Theorem \ref{theorem-cr3}.]

Define a groupoid by 
$$
\underbrace{\text{BirMori}(\mathbb{P}^n)}_{\text{Cr}_n(\mathbb{C})}
=
\left\{f:X \dashrightarrow Y
:\substack{X,Y \text{ Mfs}, X,Y \sim_{\text{bir}}\mathbb{P}^n \\ 
f\text{ birational}}\right\}.
$$
\begin{definition}
\label{definition-rank-r-fibration}
Let $r \geq 1$.
A morphism  $\eta:X \to B$ is a {\it rank $r$ fibration}
if
\begin{enumerate}
\item $\dim X > \dim B, \rho(X)-\rho(B) = r$.
\item $X$ is a MDS over $B$
\item $-K_X$ is $\eta$-big.
\item The result of any
$D$-MMP over $B$ is
$\mathbb{Q}$-factorial is terminal.
\item $\exists \Delta_B \geq 0$ 
such that $(B,\Delta_B)$ is kld.
\end{enumerate}
\end{definition}

In the first case there exists a 2-ray game.
In case (2) the 2-ray game ends.
In the last 3 cases the 2-ray game
stays in the category of MMP.

\begin{itemize}
\item Rank 1 fibration = Mfs.
\item Rank 2 fibration = Sarkisov link.
\end{itemize}

\begin{theorem}[BLZ]
\label{theorem-blz2}
The relations among Sarkisov links are generated by
by the dominated relations of rank 3 fibrations.
\end{theorem}

$$
\text{BirMori}=
\left<
\substack{\text{Sarkisov} \\ \text{links}}
:\substack{\text{relations given by} \\ 
\text{rank-3 fibrations}}\right>.
$$

Let $(g,d) \in
\{(2,8),(6,9),(10,10),(11,14)\}$,
$C \in U_{g,d}\subseteq H_{g,d}^{\leq}$,
$J=\bigcup U_{g,d}$,
$X=Bl_C \mathbb{P}^3 \dashrightarrow \mathbb{P}^3$.

\begin{proposition}
\label{proposition-rank-2}
Let $X \to \mathbb{P}^3 \to \Spec(\mathbb{C})$
be a rank-2 fibration.
$$
\xymatrix{
X\ar@{-->}[rr]\ar[d]
&&X\ar[d]\\
\ar[dr]\mathbb{P}^3\ar@{-->}[rr]^{\chi_C}
&&\mathbb{P}^3 \ar[dl]\\
&\Spec(\mathbb{C}).
}
$$
with $\chi_C^2=\text{id}$.
\end{proposition}

$$
\xymatrix{
X\ar[dd]\ar[dr]&&& X\ar[dl]\ar[dd]\\
&Z \ar[r]^{t \mapsto -t}& Z\\
\mathbb{P}^3 \ar@{-->}[rrr]&&&\mathbb{P}^3.
}
$$
\begin{proposition}
\label{proposition-no-rank-3}
There are no rank 3 fibrations
dominated $X \to \Spec(\mathbb{C})$.
\end{proposition}








\end{document}

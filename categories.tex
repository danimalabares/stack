\input{preamble}

\begin{document}

\title{Categories}
\maketitle

\phantomsection
\label{section-phantom}
\hfill
\href{http://github.com/danimalabares/stack}{github.com/danimalabares/stack}

\tableofcontents

\section{Definitions}
\label{section-definition-categories}

\noindent
We recall the definitions, partly to fix notation.

\begin{definition}
\label{definition-category}
A {\it category} $\mathcal{C}$ consists of the following data:
\begin{enumerate}
\item A set of objects $\Ob(\mathcal{C})$.
\item For each pair $x, y \in \Ob(\mathcal{C})$ a set of morphisms
$\Mor_\mathcal{C}(x, y)$.
\item For each triple $x, y, z\in \Ob(\mathcal{C})$ a composition
map $ \Mor_\mathcal{C}(y, z) \times \Mor_\mathcal{C}(x, y)
\to \Mor_\mathcal{C}(x, z) $, denoted $(\phi, \psi) \mapsto
\phi \circ \psi$.
\end{enumerate}
These data are to satisfy the following rules:
\begin{enumerate}
\item For every element $x\in \Ob(\mathcal{C})$ there exists a
morphism $\text{id}_x\in \Mor_\mathcal{C}(x, x)$ such that
$\text{id}_x \circ \phi = \phi$ and $\psi \circ \text{id}_x = \psi $ whenever
these compositions make sense.
\item Composition is associative, i.e., $(\phi \circ \psi) \circ \chi =
\phi \circ ( \psi \circ \chi)$ whenever these compositions make sense.
\end{enumerate}
\end{definition}

\begin{definition}
\label{definition-functor}
A {\it functor} $F : \mathcal{A} \to \mathcal{B}$
between two categories $\mathcal{A}, \mathcal{B}$ is given by the
following data:
\begin{enumerate}
\item A map $F : \Ob(\mathcal{A}) \to \Ob(\mathcal{B})$.
\item For every $x, y \in \Ob(\mathcal{A})$ a map
$F : \Mor_\mathcal{A}(x, y) \to \Mor_\mathcal{B}(F(x), F(y))$,
denoted $\phi \mapsto F(\phi)$.
\end{enumerate}
These data should be compatible with composition and identity morphisms
in the following manner: $F(\phi \circ \psi) =
F(\phi) \circ F(\psi)$ for a composable pair $(\phi, \psi)$ of
morphisms of $\mathcal{A}$ and $F(\text{id}_x) = \text{id}_{F(x)}$.
\end{definition}

\section{Monomorphisms}
\label{section-monomorphisms-and-epimorphisms}

\begin{definition}
\label{definition-mono-epi}
Let $\mathcal{C}$ be a category and let $f : X \to Y$ be
a morphism of $\mathcal{C}$.
\begin{enumerate}
\item We say that $f$ is a {\it monomorphism} if for every object
$W$ and every pair of morphisms $a, b : W \to X$ such that
$f \circ a = f \circ b$ we have $a = b$.
\item We say that $f$ is an {\it epimorphism} if for every object
$W$ and every pair of morphisms $a, b : Y \to W$ such that
$a \circ f = b \circ f$ we have $a = b$.
\end{enumerate}
\end{definition}

\begin{definition}
\label{definition-presheaves-injective-surjective}
Let $\mathcal{C}$ be a category, and let $\varphi : \mathcal{F}
\to \mathcal{G}$ be a map of presheaves of sets.
\begin{enumerate}
\item We say that $\varphi$ is {\it injective} if for every object
$U$ of $\mathcal{C}$ the map $\varphi_U : \mathcal{F}(U)
\to \mathcal{G}(U)$ is injective.
\item We say that $\varphi$ is {\it surjective} if for every object
$U$ of $\mathcal{C}$ the map $\varphi_U : \mathcal{F}(U)
\to \mathcal{G}(U)$ is surjective.
\end{enumerate}
\end{definition}

\begin{lemma}
\label{lemma-mono-epi}
The injective (resp.\ surjective) maps defined above
are exactly the monomorphisms (resp.\ epimorphisms) of
$\textit{PSh}(\mathcal{C})$. A map is an isomorphism
if and only if it is both injective and surjective.
\end{lemma}

\section{Presheaves}
\label{section-presheaves}

\begin{definition}
\label{definition-presheaves-sets}
A {\it presheaf of sets} on $\mathcal{C}$ is a contravariant
functor from $\mathcal{C}$ to $\textit{Sets}$. {\it Morphisms
of presheaves} are transformations of functors. The category
of presheaves of sets is denoted $\textit{PSh}(\mathcal{C})$.
\end{definition}

\section{Internal Hom}
\label{section-internal-hom}

\noindent
{\bf Upshot.} Internal Hom is when the
Hom set of two objects in some category
is in also an object of the category.
Down-to-earth, that for two sheaves $\mathcal{F},\mathcal{G}$,
$U\mapsto \Hom(\mathcal{F}(U),\mathcal{G}(U)$ 
is also a sheaf, called $\SheafHom$.

\medskip\noindent
I start with Stacks Project approach.

Let $(X, \mathcal{O}_X)$ be a ringed space.
Let $\mathcal{F}$, $\mathcal{G}$ be $\mathcal{O}_X$-modules.
Consider the rule
$$
U \longmapsto \Hom_{\mathcal{O}_X|_U}(\mathcal{F}|_U, \mathcal{G}|_U).
$$
It follows from the discussion in Sheaves, Section
\ref{sheaves-section-glueing-sheaves} that this is a sheaf of
abelian groups. In addition, given an element
$\varphi \in \Hom_{\mathcal{O}_X|_U}(\mathcal{F}|_U, \mathcal{G}|_U)$
and a section $f \in \mathcal{O}_X(U)$ then we can define
$f\varphi \in \Hom_{\mathcal{O}_X|_U}(\mathcal{F}|_U, \mathcal{G}|_U)$
by either precomposing with multiplication by $f$ on $\mathcal{F}|_U$
or postcomposing with multiplication by $f$ on $\mathcal{G}|_U$ (it gives
the same result). Hence we in fact get a sheaf of $\mathcal{O}_X$-modules.
We will denote this sheaf
$\SheafHom_{\mathcal{O}_X}(\mathcal{F}, \mathcal{G})$.
There is a canonical ``evaluation'' morphism
$$
\mathcal{F}
\otimes_{\mathcal{O}_X}
\SheafHom_{\mathcal{O}_X}(\mathcal{F}, \mathcal{G})
\longrightarrow
\mathcal{G}.
$$
For every $x \in X$ there is also a canonical morphism
$$
\SheafHom_{\mathcal{O}_X}(\mathcal{F}, \mathcal{G})_x
\to
\Hom_{\mathcal{O}_{X, x}}(\mathcal{F}_x, \mathcal{G}_x)
$$
which is rarely an isomorphism.

\medskip\noindent
{\bf Cartesian closed category} In the category of sets there is a bijection
$\Hom(X\times Y, Z)\cong \Hom(X, \text{Hom}(Y, Z))$ that depends
naturally on $X$, $Y$ and $Z$. The notions related to this bijection are
“Cartesian closed category”, “currying” and “internal Hom”.

\begin{definition}
\label{definition-Cartesian-closed-category}
A category $\mathcal{C}$ is {\it Cartesian closed} if:
\begin{enumerate}
\item $\mathcal{C}$ has all finite products 
(Caveat: some require that $\mathcal{C}$ has all finite limits)
\item For any object $Y$ the functor $- \times Y$ has a right adjoint,
which we will denote by $\text{Map}(Y,-)$ or by $-^Y$ .
\end{enumerate}
\end{definition}

\begin{remark}
\label{remark-currying}
By section 3 \href{https://ncatlab.org/nlab/show/internal+hom }{here}, the
second property above implies that we get a functor 
$\text{Map}(-,-) :\mathcal{C}^{\text{op}} \times \mathcal{C} \to \mathcal{C}$,
and moreover we get natural isomorphisms 
$\Hom(X, \text{Map}(Y, Z)) \cong \Hom(X \times Y, Z)$
and $\text{Map}(X, \text{Map}(Y, Z))\cong \text{Map}(X \times Y, Z)$.
\end{remark}



\bibliography{my}
\bibliographystyle{amsalpha}



\end{document}


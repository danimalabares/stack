\IfFileExists{stacks-project.cls}{%
\documentclass{stacks-project}
}{%
\documentclass{amsart}
}

% For dealing with references we use the comment environment
\usepackage{verbatim}
\newenvironment{reference}{\comment}{\endcomment}
%\newenvironment{reference}{}{}
\newenvironment{slogan}{\comment}{\endcomment}
\newenvironment{history}{\comment}{\endcomment}

% For commutative diagrams we use Xy-pic
\usepackage[all]{xy}

% We use 2cell for 2-commutative diagrams.
\xyoption{2cell}
\UseAllTwocells

% We use multicol for the list of chapters between chapters
\usepackage{multicol}

% This is generally recommended for better output
\usepackage{lmodern}
\usepackage[T1]{fontenc}

% For cross-file-references
\usepackage{xr-hyper}

% Package for hypertext links:
\usepackage{hyperref}

% For any local file, say "hello.tex" you want to link to please
% use \externaldocument[hello-]{hello}
\externaldocument[introduction-]{introduction}
\externaldocument[conventions-]{conventions}
\externaldocument[sets-]{sets}
\externaldocument[categories-]{categories}
\externaldocument[topology-]{topology}
\externaldocument[sheaves-]{sheaves}
\externaldocument[sites-]{sites}
\externaldocument[stacks-]{stacks}
\externaldocument[fields-]{fields}
\externaldocument[algebra-]{algebra}
\externaldocument[brauer-]{brauer}
\externaldocument[homology-]{homology}
\externaldocument[derived-]{derived}
\externaldocument[simplicial-]{simplicial}
\externaldocument[more-algebra-]{more-algebra}
\externaldocument[smoothing-]{smoothing}
\externaldocument[modules-]{modules}
\externaldocument[sites-modules-]{sites-modules}
\externaldocument[injectives-]{injectives}
\externaldocument[cohomology-]{cohomology}
\externaldocument[sites-cohomology-]{sites-cohomology}
\externaldocument[dga-]{dga}
\externaldocument[dpa-]{dpa}
\externaldocument[sdga-]{sdga}
\externaldocument[hypercovering-]{hypercovering}
\externaldocument[schemes-]{schemes}
\externaldocument[constructions-]{constructions}
\externaldocument[properties-]{properties}
\externaldocument[morphisms-]{morphisms}
\externaldocument[coherent-]{coherent}
\externaldocument[divisors-]{divisors}
\externaldocument[limits-]{limits}
\externaldocument[varieties-]{varieties}
\externaldocument[topologies-]{topologies}
\externaldocument[descent-]{descent}
\externaldocument[perfect-]{perfect}
\externaldocument[more-morphisms-]{more-morphisms}
\externaldocument[flat-]{flat}
\externaldocument[groupoids-]{groupoids}
\externaldocument[more-groupoids-]{more-groupoids}
\externaldocument[etale-]{etale}
\externaldocument[chow-]{chow}
\externaldocument[intersection-]{intersection}
\externaldocument[pic-]{pic}
\externaldocument[weil-]{weil}
\externaldocument[adequate-]{adequate}
\externaldocument[dualizing-]{dualizing}
\externaldocument[duality-]{duality}
\externaldocument[discriminant-]{discriminant}
\externaldocument[derham-]{derham}
\externaldocument[local-cohomology-]{local-cohomology}
\externaldocument[algebraization-]{algebraization}
\externaldocument[curves-]{curves}
\externaldocument[resolve-]{resolve}
\externaldocument[models-]{models}
\externaldocument[functors-]{functors}
\externaldocument[equiv-]{equiv}
\externaldocument[pione-]{pione}
\externaldocument[etale-cohomology-]{etale-cohomology}
\externaldocument[proetale-]{proetale}
\externaldocument[relative-cycles-]{relative-cycles}
\externaldocument[more-etale-]{more-etale}
\externaldocument[trace-]{trace}
\externaldocument[crystalline-]{crystalline}
\externaldocument[spaces-]{spaces}
\externaldocument[spaces-properties-]{spaces-properties}
\externaldocument[spaces-morphisms-]{spaces-morphisms}
\externaldocument[decent-spaces-]{decent-spaces}
\externaldocument[spaces-cohomology-]{spaces-cohomology}
\externaldocument[spaces-limits-]{spaces-limits}
\externaldocument[spaces-divisors-]{spaces-divisors}
\externaldocument[spaces-over-fields-]{spaces-over-fields}
\externaldocument[spaces-topologies-]{spaces-topologies}
\externaldocument[spaces-descent-]{spaces-descent}
\externaldocument[spaces-perfect-]{spaces-perfect}
\externaldocument[spaces-more-morphisms-]{spaces-more-morphisms}
\externaldocument[spaces-flat-]{spaces-flat}
\externaldocument[spaces-groupoids-]{spaces-groupoids}
\externaldocument[spaces-more-groupoids-]{spaces-more-groupoids}
\externaldocument[bootstrap-]{bootstrap}
\externaldocument[spaces-pushouts-]{spaces-pushouts}
\externaldocument[spaces-chow-]{spaces-chow}
\externaldocument[groupoids-quotients-]{groupoids-quotients}
\externaldocument[spaces-more-cohomology-]{spaces-more-cohomology}
\externaldocument[spaces-simplicial-]{spaces-simplicial}
\externaldocument[spaces-duality-]{spaces-duality}
\externaldocument[formal-spaces-]{formal-spaces}
\externaldocument[restricted-]{restricted}
\externaldocument[spaces-resolve-]{spaces-resolve}
\externaldocument[formal-defos-]{formal-defos}
\externaldocument[defos-]{defos}
\externaldocument[cotangent-]{cotangent}
\externaldocument[examples-defos-]{examples-defos}
\externaldocument[algebraic-]{algebraic}
\externaldocument[examples-stacks-]{examples-stacks}
\externaldocument[stacks-sheaves-]{stacks-sheaves}
\externaldocument[criteria-]{criteria}
\externaldocument[artin-]{artin}
\externaldocument[quot-]{quot}
\externaldocument[stacks-properties-]{stacks-properties}
\externaldocument[stacks-morphisms-]{stacks-morphisms}
\externaldocument[stacks-limits-]{stacks-limits}
\externaldocument[stacks-cohomology-]{stacks-cohomology}
\externaldocument[stacks-perfect-]{stacks-perfect}
\externaldocument[stacks-introduction-]{stacks-introduction}
\externaldocument[stacks-more-morphisms-]{stacks-more-morphisms}
\externaldocument[stacks-geometry-]{stacks-geometry}
\externaldocument[moduli-]{moduli}
\externaldocument[moduli-curves-]{moduli-curves}
\externaldocument[examples-]{examples}
\externaldocument[exercises-]{exercises}
\externaldocument[guide-]{guide}
\externaldocument[desirables-]{desirables}
\externaldocument[coding-]{coding}
\externaldocument[obsolete-]{obsolete}
\externaldocument[fdl-]{fdl}
\externaldocument[index-]{index}

% Theorem environments.
%
\theoremstyle{plain}
\newtheorem{theorem}[subsection]{Theorem}
\newtheorem{proposition}[subsection]{Proposition}
\newtheorem{lemma}[subsection]{Lemma}

\theoremstyle{definition}
\newtheorem{definition}[subsection]{Definition}
\newtheorem{example}[subsection]{Example}
\newtheorem{exercise}[subsection]{Exercise}
\newtheorem{situation}[subsection]{Situation}

\theoremstyle{remark}
\newtheorem{remark}[subsection]{Remark}
\newtheorem{remarks}[subsection]{Remarks}

\numberwithin{equation}{subsection}

% Macros
%
\def\lim{\mathop{\mathrm{lim}}\nolimits}
\def\colim{\mathop{\mathrm{colim}}\nolimits}
\def\Spec{\mathop{\mathrm{Spec}}}
\def\Hom{\mathop{\mathrm{Hom}}\nolimits}
\def\Ext{\mathop{\mathrm{Ext}}\nolimits}
\def\SheafHom{\mathop{\mathcal{H}\!\mathit{om}}\nolimits}
\def\SheafExt{\mathop{\mathcal{E}\!\mathit{xt}}\nolimits}
\def\Sch{\mathit{Sch}}
\def\Mor{\mathop{\mathrm{Mor}}\nolimits}
\def\Ob{\mathop{\mathrm{Ob}}\nolimits}
\def\Sh{\mathop{\mathit{Sh}}\nolimits}
\def\NL{\mathop{N\!L}\nolimits}
\def\CH{\mathop{\mathrm{CH}}\nolimits}
\def\proetale{{pro\text{-}\acute{e}tale}}
\def\etale{{\acute{e}tale}}
\def\QCoh{\mathit{QCoh}}
\def\Ker{\mathop{\mathrm{Ker}}}
\def\Im{\mathop{\mathrm{Im}}}
\def\Coker{\mathop{\mathrm{Coker}}}
\def\Coim{\mathop{\mathrm{Coim}}}

% Boxtimes
%
\DeclareMathSymbol{\boxtimes}{\mathbin}{AMSa}{"02}

%
% Macros for moduli stacks/spaces
%
\def\QCohstack{\mathcal{QC}\!\mathit{oh}}
\def\Cohstack{\mathcal{C}\!\mathit{oh}}
\def\Spacesstack{\mathcal{S}\!\mathit{paces}}
\def\Quotfunctor{\mathrm{Quot}}
\def\Hilbfunctor{\mathrm{Hilb}}
\def\Curvesstack{\mathcal{C}\!\mathit{urves}}
\def\Polarizedstack{\mathcal{P}\!\mathit{olarized}}
\def\Complexesstack{\mathcal{C}\!\mathit{omplexes}}
% \Pic is the operator that assigns to X its picard group, usage \Pic(X)
% \Picardstack_{X/B} denotes the Picard stack of X over B
% \Picardfunctor_{X/B} denotes the Picard functor of X over B
\def\Pic{\mathop{\mathrm{Pic}}\nolimits}
\def\Picardstack{\mathcal{P}\!\mathit{ic}}
\def\Picardfunctor{\mathrm{Pic}}
\def\Deformationcategory{\mathcal{D}\!\mathit{ef}}

%Dani's additions
\usepackage{graphicx} %figures


\begin{document}

\title{Symplectic geometry}
\maketitle

\phantomsection
\label{section-phantom}
\hfill
\href{http://github.com/danimalabares/stack}{github.com/danimalabares/stack}

\tableofcontents

\section{Coadjoint orbits and KKS theorem}
\label{section-coadjoint-orbits}

\noindent
The coadjoint action induces orbits in $\mathfrak{g}^*$.
The tangent space at any point of a
coadjoint orbit is given by infinitesimal vector
fields, that is,
$$
T_{\xi}\mathcal{O}=\{u_{\mathfrak{g}^*}(\xi):\forall u \in \mathfrak{g}\}.
$$

\noindent
We may introduce a symlpectic form
$\omega \in \Lambda^{2}T^*_\xi\mathcal{O}$
on the tangent space 
at some point of a
coadjoint orbit by
$$
\omega_\xi(X,Y):=\xi([u,v])=-u_{\mathfrak{g}^*}(\xi)v=v_{\mathfrak{g}^*}(\xi)(u)
$$
which does not depend on $u,v \in \mathfrak{g}$.

\begin{theorem}[KKS]
\label{theorem-KKS}
$\xi \mapsto  \omega_\xi$ is symplectic on $\mathcal{O}$.
\end{theorem}

\begin{proof}
Non degeneracy is easy.
Then show that $\omega$ is $G$-invariant.
Closedness follows from Jacobi identity.
\end{proof}

\section{Hamiltonian actions and moment map}
\label{section-hamiltonian-actions-and-moment-map}

\noindent
To warm-up we define real symplectic and Hamiltonian actions.

\begin{definition}
\label{definition-real-symplectic-and-hamiltonian-actions}
\begin{itemize}
\item An $\mathbb{R}$-action is {\it symplectic} if $\psi_t^* \iff L_X \omega=0$,
that is, $X$ is a symplectic field.
\item An $\mathbb{R}$-action is {\it hamiltonian} if $X$ is Hamiltonian,
that is, if there exists a function $H \in C^\infty(M)$ such that
$X=X_H$.
\end{itemize}
\end{definition}

\noindent
To define moment map suppose there exists a map $\hat{\mu}$ 
such that
$$
\xymatrixcolsep{3em}
\xymatrixrowsep{.7em}
\xymatrix{
&C^\infty(M)\ar[dd]&f\ar@{|->}[dd]\\
\mathfrak{g}\ar[dr]\ar[ur]^{\hat{\mu}}\\
u \ar@{|->}[dr] & \mathfrak{X}(M) & X_f\\
& u_M \ar@{=}[ur]
}
$$
Note that if  $\hat{\mu}$ exists, it's not uniquely determined,
and that we may suppose that it is linear.

It's equivalente to define $\hat{\mu}$ as a map 
$\hat{\mu}:\mathfrak{g}\to C^\infty(M)$ or as a map
\begin{align*}
\mu: M &\longrightarrow \mathfrak{g}^* \\
\hat{\mu}(u)(x) &= \left<\mu(x),u\right>.
\end{align*}

\begin{definition}
\label{definition-symplectic-and-hamiltonian-actions}
\begin{itemize}
\item A $G$-action is symplectic if $\psi_g^*\omega=\omega$ for all $g \in G$,
that is, if
$$
\xymatrix{
\psi:G\ar[r]\ar[dr]&\text{Dif}(M)\\
&\text{Symp}(M,\omega)\ar@{^{(}->}[u].
}
$$

\item A symplectic action $G \overset{\psi}{\mathbb{y}}(M,\omega)$
is {\it weakly Hamiltonian} if there exists a map $\mu:M \to \mathfrak{g}^*$ 
(or equivalently, $\hat{\mu}:\mathfrak{g} \to C^\infty(M)$ linear)
such that
$$
i_{u_M}\omega=d\left<\mu,u\right>,\qquad (\text{resp. } u_M=X_{\hat{\mu}(u)}).
$$

\item A weakly Hamiltonian action is {\it Hamiltonian} if it is also
$\mu$-equivariant, that is,
$\mu: G\mathbb{y} M \to g^*\overset{\text{Ad}^*}{\mathbb{x}}G$,
$$
\mu \circ \psi_g=(\text{Ad}^*)_g(\mu).
$$
In this case we call $\mu$ a {\it moment map}.
\end{itemize}

\end{definition}

\section{Noether's principle}
\label{section-Noether-principle}

\begin{theorem}[Noether's principle]
\label{theorem-Noether-principle}
Let $(M,\omega,\mu)$ be a Hamiltonian $G$-space with Hamiltonian
$H\in C^\infty(M)$. $H$ is $G$-invariant if and only if $\mu$ is preserved by
the Hamiltonian flow.
\end{theorem}

\begin{proof}
one line
\end{proof}

\section{Symplectic quotient}
\label{section-symplectic-quotient}

\noindent
In symplectic geometry we cannot always take quotients.
We need to use level-sets of the moment map.

\begin{theorem}[Marsden-Weinstein, Megre]
\label{theorem-marsden-weinstein-megre}
Let $G \mathbb{y} (M,\omega)$ be a Hamiltonian action
with moment map $\mu:M \to \mathfrak{g}^*$.
Suppose $0 \in \mathfrak{g}^*$ is a regular value.
By the equivariance of $\mu$, we have an action $G \mathbb{y} \mu^{-1}(0)$
given as follows. For $x \in \mu^{-1}(0)$, 
$\mu(\psi_g(x))=\text{Ad}_g^*(\mu(x))=0$.
Suppose that this action is regular (this will be defined later),
that $\mu^{-1}(0)/G$ is smooth, and that the projection is a submersion.
Then there exists a unique symplectic form $\omega_{\text{red}}$ in $M_0$ 
such that $i^*\omega=\pi_0^*\omega_{\text{red}}$.
\end{theorem}

\noindent
That is, the pullback of the symplectic form to the 
0 level-set of $\mu$ passes to the quotient:
$$
\xymatrix{
G \mathbb{y} \mu^{-1}(0)\ar[d]^{\pi_0}\ar@{^{(}->}[r]& M\\
(M_0:=\mu^{-1}(0)/G,\omega_{\text{red}})
}
$$

\noindent
A slightly more general setting is the following.
Hamiltonian action $G \mathbb{y} (M,\omega)$,
and moment map
$\mu:M \to \mathfrak{g}^*$.
$\xi \in \mathfrak{g}^*$ regular, $G_\xi \subseteq G$
stabilizer of $\xi$.
Then we have
$$
\xymatrix{
G_\xi \mathbb{y} \mu^{-1}(\xi)\ar[d]^{\pi_\xi}\ar@{^{(}->}[r]^{i_\xi}& M\\
(M_\xi=\mu^{-1}(\xi)/G_\xi,\omega_{\text{red}}).
}
$$

\section{Toric varieties}
\label{section-toric-varieties}

\noindent
For any compact connected symplectic manifold with an action of a torus,
$\mathbb{T}^n \mathbb{y} (M^{2n},\omega)\xrightarrow{\mu}\mathbb{R}^n
=(\mathfrak{t}^n)^*$, the image of $\mu$
is a polytope. By [Guilleimin-Sternberg] and [Atiyah], this polytope is
convex. Further, if the dimension of the torus is exactly half
the dimension of the manifold, by [Delzant], the moment polytope
determines the manifold, i.e. there is a one-to-one
correspondence between
Delzant polytopes and symplectic toric varieties.
Moduli space of polygons is an example of this.

In fact, this situation of having a torus action of half the dimension
is what we may call a ``totally integrable system''.

\section{(Completely) integrable systems}
\label{section-completely-integrable-systems}

\noindent
Let $(M,\omega)$ be a symplectic manifold.
Recall that

\begin{itemize}
\item $\{f,g\}=\omega(X_g,X_f)=L_{X_f}g=-L_{X_g}f$.
\item $X_{\{f,g\}}=[X_f,X_g]$
\end{itemize}

\noindent
If $\{f,g\}=0$ then the Hamiltonian flows of $f$ and $g$ commute.

\begin{definition}
\label{definition-completely-integrable-system}
Let $(M^{2n},\omega)$ and $H \in C^\infty(M)$.
$(M,\omega,H)$ is a {\it completely integrable system} 
(in Liouville's sense)
if there exist functions $f_1,\ldots,f_n \in C^\infty(M)$ such that
\begin{enumerate}
\item $H=f_1,\ldots,f_n$ are linearly independent
($df_1,\ldots,df_n$ are linearly independent in every point).

\item $\{ f_i, f_j\}=0$ for all $i,j$.
\end{enumerate}
\end{definition}

\noindent
Note that $F=(f_1,\ldots,f_n):M \to \mathbb{R}^n$ is a submersion.
For all $c \in \mathbb{R}^n$, $F^{-1}(c)$ 
are Lagrangian submanifolds.
$F^{-1}(c)$ is invariant by the flow of
$X_{f_i}$ for $i=1,\ldots,n$.

Explanation: $F^{-1}(c)$ is a level-set submanifold of $M$,
whose tangent space is generated by the $df_i$.
These are all in the kernel of $dF$ by the integrability condition,
namely $dF=(df_1,\ldots,df_n)$, $df_i(X_{f_j})=\{f_i,f_j\}=0$.

The idea is that finding these ``conserved quantities''
we manage to describe the Hamiltonian dynamics completely.
Showing that a system is integrable is a victory.
The moduli space of polygons is an example,
and I think that so is $(\mathbb{C}^2)^{[n]}$.

There is another method of using ``bihamiltonian actions''
(two Hamiltonian actions yielding the same dynamics)
/Lax pairs (defining the system using eigenvalues of matrices).
Examples of this are Toda, Caolengo-Moser.

\medskip\noindent
The following theorem says that the connected
component of a symplectic manifold with an integrable system


\begin{theorem}[Arnold-Liouville]
\label{theorem-arnold-liouville}
Let $(M^{2n},\omega)$, $H=f_1,\ldots,f_n$ be completely intgrable system
and $F:M \to \mathbb{R}^n$, $F=(f_1,\ldots,f_n)$.
Let $M_c$ be the connected component of $F^{-1}(c)$.
Then
\begin{enumerate}
\item 
\label{item-toric1}
If $M_c$ is compact, then $M_c \simeq
\mathbb{T}^n=\mathbb{R}^n/2\pi\mathbb{Z}^n=\{(\theta_1,\ldots,\theta_n) \text{
mod }2\pi\}$.

\item 
\label{item-toric2}
In angular coordinates, the Hamiltonian flow is linear, that is,
$\frac{d}{dt}^{-1}(t)=\alpha \in \mathbb{R}^n$ 
implies that $\theta(t)=\theta_0+\alpha t$.
\end{enumerate}
\end{theorem}

The proof of item \ref{item-toric1} is given by
the following proposition
putting $N=M$ and $X_1=X_{f_1},\ldots,X_n=X_{f_n}$.

\begin{proposition}
\label{proposition-torus}
Let $N$ be complete, connected, of dimension $n$.
Let $X_1,\ldots,X_n$ be linearly independent vector fields on $N$ 
such that $[X_i,X_j]=0$ for all $j$.
Then $N \simeq \mathbb{T}^n$.
\end{proposition}

\begin{proof}
First we present the steps:
\begin{enumerate}
\item Flows of $X_1,\ldots,X_n$ define an action $\mathbb{R}^n \mathbb{y} N$.

\item This action is transitive. Then $N \simeq \mathbb{R}^n/\Gamma$
where $\Gamma$ is the stabilizer.

\item Since $\dim N=n$, then  $\Gamma$ is a lattice of $\mathbb{R}^n$.

\item Since $N$ is compact, then $\Gamma$ is of maximum rank, that is,
$\Gamma=2\pi \mathbb{Z}^n \subset \mathbb{R}^n$ .
\end{enumerate}


Let $\varphi^i_t$ be the flow of $X_i$ (which are complete since $N$ is
compact).
Since $[X_i,X_j]=0$, then $\varphi_t^i \varphi_s^j=\varphi_s^j \varphi^i_t$.
Thus we have an action $\mathbb{R}^n\mathbb{y} N$
given by moving along the flows of the vector fields, that is
$$
\underbrace{(t_1,\ldots,t_n)}_{\mathbf{t}}\cdot x
=\underbrace{\varphi_{t_1}^1 \circ \ldots \circ \varphi_{t_n}^n(x)}_{
\varphi_{\mathbf{t}}(x)}
$$
To prove transitivity 
we can show that the orbit maps of this action are surjective.
Indeed, the image of the orbit map is open since it is a local
diffeomorphism (it's derivative is given by the vector fields,
which are linearly independent!),
and it is closed by a taking a point in the boundary and 
using the flow at this point (and commutativity of the flows).

To check that the stabilizer $\Gamma$ of this action is trivial
first note that all stabilizers are equal
(not only conjugate) since $\mathbb{R}^n$ is abelian.
Further, since the orbit map is a local diffeomorphism,
$\Gamma$ is discrete. Thus, it must be a lattice of rank $k \leq n$.

To show that in fact $k=n$ consider a linear map  $T$ 
that maps the generators of $\Gamma$ to the first $k$ canonical 
vectors of $\mathbb{R}^n$. By the following diagram, the arrow on the bottom
is well defined as in fact a bijective local diffeomorphism,
that is, a diffeomorphism.
$$
\xymatrix{
\mathbb{R}^k \times\mathbb{R}^{n-k} \simeq \mathbb{R}^n\ar[r]^{T}\ar[d]_{p}
\ar[dr]^{\psi_x \circ T}&
\mathbb{R}^n\ar[d]^{\psi_x=\substack{\text{orbit} \\ \text{map}}}\\
\mathbb{T}^k \times \mathbb{R}^{n-k}\ar[r]_{\overline{T}}&
N
}
$$
Thus the factor $\mathbb{R}^{n-k}$ must be a point since $N$ is
compact by hypothesis.
\end{proof}

\noindent
Item \ref{item-toric2} from Theorem \ref{theorem-arnold-liouville}
is obtained by considering the flows
on the latter diagram and is left as an exercise.

\noindent
In the case that $N$ is not compact we just conclude that
it $M_c$ is a product of a torus with an Euclidean space.

Perhaps the simplest example of an integrable system
is given by taking an open ball about the origin $B \subseteq \mathbb{R}^n$
and $B \times \mathbb{T}^n=\{(I_1,\ldots,I_n,\phi_1,\ldots,\phi_n)\}$ 
with the symplectic form $\omega=\sum_{i=1}^n dI_1 \wedge d\phi_i$ 
with $\{I_1,I_j\}=0$.

The following theorem says that in fact the {\bf symplectic form} of every
level set of the kind $M_c$, which we just proved
that is a torus, may is described locally as such a product.

\begin{theorem}
\label{theorem-local}
There is a neighbourhood $U$ of $M_c$ and
a symplectomorphism
$$
\xymatrix{
\sum d \mathcal{I} \wedge d\phi_i&\ar@{~>}[l]
B \times \mathbb{T}^n\ar[r]^{r}_{\simeq}\ar[d]_{p}&U \subseteq M\ar[d]^{F}
\ar@{~>}[r]
& \omega\\
&B\ar[r]_{\simeq}&F(U)
}
$$
\end{theorem}

\noindent
The coordinates $(I_1,\ldots,I_n,\phi_1,\ldots,\phi_n)$
are called {\it action-angle} coordinates.

The case when $h(I)$ does not depend on the angle coordinates $\phi_i$
we can solve the Hamiltonian equations easily:
$$
\begin{cases}
\dot I_i=0 \\
\dot \phi_i=-\frac{\partial h}{\partial I}(I)
\end{cases}
$$
so that $\phi_i(t)=\phi_0+\alpha t$ for some constants $\alpha$.
This is why integrable systems are actually easy to solve!

\section{Vertex algebras}
\label{section-vertex-algebras}

Mônadas, álgebras sobre mônadas. Exemplo: ultrafiltros (Manes, 1976). Campo
quântico, como serie de Fourier. Exemplo: Heisenberg algebra.

Problema: o producto de campos quânticos não é bem definido.

\bibliography{my}
\bibliographystyle{amsalpha}

\end{document}


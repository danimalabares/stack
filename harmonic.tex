\input{preamble}

\begin{document}

\title{}
\maketitle

\phantomsection
\label{section-phantom}
\hfill
\href{http://github.com/danimalabares/stack}{github.com/danimalabares/stack}

\tableofcontents

{\bf Goal.} Understand harmonic maps $T^2 \to S^3$. Then gauge theory and
algebraic geometry (spectral data).

Let $M$ be a compact Riemann surface, $G$ be a compact Lie group with a
bi-invariant metric and $f:M\to G$ a smooth map.

\begin{example}[To keep in mind]
\label{example-torus}
$M=T^2$ and $G=\text{SU}(2)\cong S^3$.
\end{example}

\begin{definition}
\label{definition-connection}
A {\it connection} is a map
$$
\nabla^A:\Omega^{0}(M;G)\to \Omega^{1}(M;E)
$$
where $\Omega^{p}(M;E)=\Gamma(M,\Lambda^{p}T^*M\otimes E)$.
\end{definition}

Consider the pullback bundle of $(TG,\nabla^{LC})$ over $M$, which we denote by
$(E,\nabla^A)$.

Locally, we can express
$$
\nabla^A_{\text{loc}}=d+A
$$
where $A$ is the {\it connection matrix}, a mtrix of 1-forms.

Extend  $\nabla^A$ to an operator
$$
d_A:\Omega^{p}(M;E)\to \Omega^{p+1}(M;E)
$$
by Leibniz rule
$$
d_A(\alpha\wedge\beta=d\alpha\wedge\beta+(-1)^{|\alpha|}\alpha\wedge d_A\beta
$$
And we have a {\it curvature} given by
$$
F_A=d^2_A\in\Omega^{2}(M,;\text{End}(E))
$$

\begin{lemma}
\label{lemma-derivative-of-differential}
$d_A(df)=0$
\end{lemma}

\begin{proof}
If $\alpha\in\Omega^{1}(M;E)$ and $X,Y\in\mathfrak{X}(M)$,
$$
d_A(\alpha)(X,Y)=\nabla_X^A(\alpha(Y))-\nabla_Y^A(\alpha(X))-\alpha([X,Y])
$$
So for $\alpha=df$… (use that LC is torsion-free!)
\end{proof}

\begin{definition}
\label{definition-Hodge-star}
The {\it Hodge star operator} is
$$
d^*_A=-*d_A*
$$
where
\begin{align*}
*: \Omega^{p}(M;E) &\longrightarrow \Omega^{2-p}(M;E) \\
\alpha\otimes s &\longmapsto *\alpha\otimes s
\end{align*}
since our manifold is dimension 2.
\end{definition}

\begin{definition}
\label{definition-energy-functional}
$$
E(f)=\frac{1}{2}\int_M|df|^2d\text{Vol}
$$
\end{definition}

\begin{lemma}
\label{lemma-critical-points-of-energy-functional}
The critical points of $E$ satisfy
$$
\tau(f):=\text{tr}_g(\nabla df)=0
$$
where $\nabla df$ is the {\it induced connection} on $T^*M\otimes E$, i.e.
$$
\nabla(df)(X,Y)=\nabla_X^A(df(Y))-df(\nabla_X^gY)
$$
\end{lemma}

\begin{lemma}
\label{lemma-harmonic}
$f$ is harmonic if and only if $d_A^*(df)=0$.
\end{lemma}

\begin{proof}
Choose coordinates such that $\nabla^g_XY$ vanishes at the point. Then
$$
\tau(f)=\nabla^A_{e_1}(df(e_1))+\nabla^A_{e_1}(df(e_2))
$$
Let's compute $d_A^*(df)=-*d_A*(df)$. First we find that $*df=e^2s_1-e^1s_2$ for
a dual frame $e^i$ and now no longer writing the tensor product. Then we get
that 
\begin{align*}
d_A(*df)&=de^2\otimes s_1-e^2\nabla^A(s_1)-
\end{align*}

\end{proof}
\begin{proof}

\end{proof}


\bibliography{my}
\bibliographystyle{amsalpha}

\end{document}


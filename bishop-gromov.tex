\IfFileExists{stacks-project.cls}{%
\documentclass{stacks-project}
}{%
\documentclass{amsart}
}

% For dealing with references we use the comment environment
\usepackage{verbatim}
\newenvironment{reference}{\comment}{\endcomment}
%\newenvironment{reference}{}{}
\newenvironment{slogan}{\comment}{\endcomment}
\newenvironment{history}{\comment}{\endcomment}

% For commutative diagrams we use Xy-pic
\usepackage[all]{xy}

% We use 2cell for 2-commutative diagrams.
\xyoption{2cell}
\UseAllTwocells

% We use multicol for the list of chapters between chapters
\usepackage{multicol}

% This is generally recommended for better output
\usepackage{lmodern}
\usepackage[T1]{fontenc}

% For cross-file-references
\usepackage{xr-hyper}

% Package for hypertext links:
\usepackage{hyperref}

% For any local file, say "hello.tex" you want to link to please
% use \externaldocument[hello-]{hello}
\externaldocument[introduction-]{introduction}
\externaldocument[conventions-]{conventions}
\externaldocument[sets-]{sets}
\externaldocument[categories-]{categories}
\externaldocument[topology-]{topology}
\externaldocument[sheaves-]{sheaves}
\externaldocument[sites-]{sites}
\externaldocument[stacks-]{stacks}
\externaldocument[fields-]{fields}
\externaldocument[algebra-]{algebra}
\externaldocument[brauer-]{brauer}
\externaldocument[homology-]{homology}
\externaldocument[derived-]{derived}
\externaldocument[simplicial-]{simplicial}
\externaldocument[more-algebra-]{more-algebra}
\externaldocument[smoothing-]{smoothing}
\externaldocument[modules-]{modules}
\externaldocument[sites-modules-]{sites-modules}
\externaldocument[injectives-]{injectives}
\externaldocument[cohomology-]{cohomology}
\externaldocument[sites-cohomology-]{sites-cohomology}
\externaldocument[dga-]{dga}
\externaldocument[dpa-]{dpa}
\externaldocument[sdga-]{sdga}
\externaldocument[hypercovering-]{hypercovering}
\externaldocument[schemes-]{schemes}
\externaldocument[constructions-]{constructions}
\externaldocument[properties-]{properties}
\externaldocument[morphisms-]{morphisms}
\externaldocument[coherent-]{coherent}
\externaldocument[divisors-]{divisors}
\externaldocument[limits-]{limits}
\externaldocument[varieties-]{varieties}
\externaldocument[topologies-]{topologies}
\externaldocument[descent-]{descent}
\externaldocument[perfect-]{perfect}
\externaldocument[more-morphisms-]{more-morphisms}
\externaldocument[flat-]{flat}
\externaldocument[groupoids-]{groupoids}
\externaldocument[more-groupoids-]{more-groupoids}
\externaldocument[etale-]{etale}
\externaldocument[chow-]{chow}
\externaldocument[intersection-]{intersection}
\externaldocument[pic-]{pic}
\externaldocument[weil-]{weil}
\externaldocument[adequate-]{adequate}
\externaldocument[dualizing-]{dualizing}
\externaldocument[duality-]{duality}
\externaldocument[discriminant-]{discriminant}
\externaldocument[derham-]{derham}
\externaldocument[local-cohomology-]{local-cohomology}
\externaldocument[algebraization-]{algebraization}
\externaldocument[curves-]{curves}
\externaldocument[resolve-]{resolve}
\externaldocument[models-]{models}
\externaldocument[functors-]{functors}
\externaldocument[equiv-]{equiv}
\externaldocument[pione-]{pione}
\externaldocument[etale-cohomology-]{etale-cohomology}
\externaldocument[proetale-]{proetale}
\externaldocument[relative-cycles-]{relative-cycles}
\externaldocument[more-etale-]{more-etale}
\externaldocument[trace-]{trace}
\externaldocument[crystalline-]{crystalline}
\externaldocument[spaces-]{spaces}
\externaldocument[spaces-properties-]{spaces-properties}
\externaldocument[spaces-morphisms-]{spaces-morphisms}
\externaldocument[decent-spaces-]{decent-spaces}
\externaldocument[spaces-cohomology-]{spaces-cohomology}
\externaldocument[spaces-limits-]{spaces-limits}
\externaldocument[spaces-divisors-]{spaces-divisors}
\externaldocument[spaces-over-fields-]{spaces-over-fields}
\externaldocument[spaces-topologies-]{spaces-topologies}
\externaldocument[spaces-descent-]{spaces-descent}
\externaldocument[spaces-perfect-]{spaces-perfect}
\externaldocument[spaces-more-morphisms-]{spaces-more-morphisms}
\externaldocument[spaces-flat-]{spaces-flat}
\externaldocument[spaces-groupoids-]{spaces-groupoids}
\externaldocument[spaces-more-groupoids-]{spaces-more-groupoids}
\externaldocument[bootstrap-]{bootstrap}
\externaldocument[spaces-pushouts-]{spaces-pushouts}
\externaldocument[spaces-chow-]{spaces-chow}
\externaldocument[groupoids-quotients-]{groupoids-quotients}
\externaldocument[spaces-more-cohomology-]{spaces-more-cohomology}
\externaldocument[spaces-simplicial-]{spaces-simplicial}
\externaldocument[spaces-duality-]{spaces-duality}
\externaldocument[formal-spaces-]{formal-spaces}
\externaldocument[restricted-]{restricted}
\externaldocument[spaces-resolve-]{spaces-resolve}
\externaldocument[formal-defos-]{formal-defos}
\externaldocument[defos-]{defos}
\externaldocument[cotangent-]{cotangent}
\externaldocument[examples-defos-]{examples-defos}
\externaldocument[algebraic-]{algebraic}
\externaldocument[examples-stacks-]{examples-stacks}
\externaldocument[stacks-sheaves-]{stacks-sheaves}
\externaldocument[criteria-]{criteria}
\externaldocument[artin-]{artin}
\externaldocument[quot-]{quot}
\externaldocument[stacks-properties-]{stacks-properties}
\externaldocument[stacks-morphisms-]{stacks-morphisms}
\externaldocument[stacks-limits-]{stacks-limits}
\externaldocument[stacks-cohomology-]{stacks-cohomology}
\externaldocument[stacks-perfect-]{stacks-perfect}
\externaldocument[stacks-introduction-]{stacks-introduction}
\externaldocument[stacks-more-morphisms-]{stacks-more-morphisms}
\externaldocument[stacks-geometry-]{stacks-geometry}
\externaldocument[moduli-]{moduli}
\externaldocument[moduli-curves-]{moduli-curves}
\externaldocument[examples-]{examples}
\externaldocument[exercises-]{exercises}
\externaldocument[guide-]{guide}
\externaldocument[desirables-]{desirables}
\externaldocument[coding-]{coding}
\externaldocument[obsolete-]{obsolete}
\externaldocument[fdl-]{fdl}
\externaldocument[index-]{index}

% Theorem environments.
%
\theoremstyle{plain}
\newtheorem{theorem}[subsection]{Theorem}
\newtheorem{proposition}[subsection]{Proposition}
\newtheorem{lemma}[subsection]{Lemma}

\theoremstyle{definition}
\newtheorem{definition}[subsection]{Definition}
\newtheorem{example}[subsection]{Example}
\newtheorem{exercise}[subsection]{Exercise}
\newtheorem{situation}[subsection]{Situation}

\theoremstyle{remark}
\newtheorem{remark}[subsection]{Remark}
\newtheorem{remarks}[subsection]{Remarks}

\numberwithin{equation}{subsection}

% Macros
%
\def\lim{\mathop{\mathrm{lim}}\nolimits}
\def\colim{\mathop{\mathrm{colim}}\nolimits}
\def\Spec{\mathop{\mathrm{Spec}}}
\def\Hom{\mathop{\mathrm{Hom}}\nolimits}
\def\Ext{\mathop{\mathrm{Ext}}\nolimits}
\def\SheafHom{\mathop{\mathcal{H}\!\mathit{om}}\nolimits}
\def\SheafExt{\mathop{\mathcal{E}\!\mathit{xt}}\nolimits}
\def\Sch{\mathit{Sch}}
\def\Mor{\mathop{\mathrm{Mor}}\nolimits}
\def\Ob{\mathop{\mathrm{Ob}}\nolimits}
\def\Sh{\mathop{\mathit{Sh}}\nolimits}
\def\NL{\mathop{N\!L}\nolimits}
\def\CH{\mathop{\mathrm{CH}}\nolimits}
\def\proetale{{pro\text{-}\acute{e}tale}}
\def\etale{{\acute{e}tale}}
\def\QCoh{\mathit{QCoh}}
\def\Ker{\mathop{\mathrm{Ker}}}
\def\Im{\mathop{\mathrm{Im}}}
\def\Coker{\mathop{\mathrm{Coker}}}
\def\Coim{\mathop{\mathrm{Coim}}}

% Boxtimes
%
\DeclareMathSymbol{\boxtimes}{\mathbin}{AMSa}{"02}

%
% Macros for moduli stacks/spaces
%
\def\QCohstack{\mathcal{QC}\!\mathit{oh}}
\def\Cohstack{\mathcal{C}\!\mathit{oh}}
\def\Spacesstack{\mathcal{S}\!\mathit{paces}}
\def\Quotfunctor{\mathrm{Quot}}
\def\Hilbfunctor{\mathrm{Hilb}}
\def\Curvesstack{\mathcal{C}\!\mathit{urves}}
\def\Polarizedstack{\mathcal{P}\!\mathit{olarized}}
\def\Complexesstack{\mathcal{C}\!\mathit{omplexes}}
% \Pic is the operator that assigns to X its picard group, usage \Pic(X)
% \Picardstack_{X/B} denotes the Picard stack of X over B
% \Picardfunctor_{X/B} denotes the Picard functor of X over B
\def\Pic{\mathop{\mathrm{Pic}}\nolimits}
\def\Picardstack{\mathcal{P}\!\mathit{ic}}
\def\Picardfunctor{\mathrm{Pic}}
\def\Deformationcategory{\mathcal{D}\!\mathit{ef}}

%Dani's additions
\usepackage{graphicx} %figures


\begin{document}

\title{Bishop-Gromov Theorem}
\maketitle

\phantomsection
\label{section-phantom}

\begin{theorem}[Bishop-Gromov]
\label{theorem-bishop-gromov}
Let $p \in M$ and $0\leq t \leq i(p)=d(p,C_m(p)$. Denote $B_t(p)$ the ball of
raduis $t$ centered in $p$ and $B_{t,k}$ the ball of radius $t$ in a space of
constant curvature $k$.

If $\text{Ric}\geq k$, then $\text{Vol}(B_t(p))/\text{Vol}(B_{t,k})$ is a non
increasing function.

Moreover, if there are numbers  $0<s<r\leq i(p)$ such that
$$
\frac{\text{Vol}(B_s(p))}{\text{Vol}(B_{s,k}}=
\frac{\text{Vol}(B_r(p))}{\text{Vol}(B_{r,k})}
$$
then $B_r(p)=B_{r,k}$ isometrically.
\end{theorem}

\begin{proof}
Use the exponential map a parametrization of the geodesic sphere with radius
$r$ and centre in $p\in M$. (This parametrization works at least locally, but why
have we said in lecture that ``it's a global chart''?) Then
$$
\text{Vol}(S^n_r(p))=\int_{S^n_r(p)}\text{Vol}_{S^n_r(p)}=
\int_{S^n_r(0)}\text{exp}_p^*\text{Vol}_{S^n_r}=
\int_{S^n_r(0)}|\det d_{rv} \text{exp}_p|\text{Vol}_{S^n_r(0)}
$$
This determinant can be given by a basis of tangent vectors to $S_r(0)$, each of
which yields a Jacobi field via Proposition 
\ref{proposition-everyday-jacobi-field}. These Jacobi fields are the columns 
of the matrix $\mathbb{J}$, which is the Jacobian matrix of the exponential we 
are interested in computing.

Now we want to differentiate this with respect to $r$. By Exercise
\ref{exercise-derivative-of-determinant}, we know that the derivative of the
determinant of a one-parameter family of invertible matrices $\mathbb{J}(r)$ is
given by $\det \mathbb{J}(r)\text{tr}(\mathbb{J}^{-1}(r)\mathbb{J}'(r))$.

By the discussion in Section \ref{section-Riccati-equation}, we know that
$AJ=J'$ where $A$ is the shape operator with respect to the unit normal
$\gamma'(r)$ and $J$ is a Jacobi field defined by
\ref{proposition-everyday-jacobi-field}.  This gives $A=J'J^{-1}$. These are the
diagonal entries of the matrix $\mathbb{J}$, so that its trace, which depends
only on the diagonal by Exercise 
\ref{exercise-trace-is-independent-of-coordinates}, is precisely $H/(n-1)$ where
$H$ satisfying this equation is defined as the mean curvature.

This data translates to the following equation
\begin{equation}
\label{equation-Bishop-Gromov-proof1}
\det\mathbb{J}'=\det\mathbb{J}H
\end{equation}
\begin{remark}[Pregunta]
\label{remark-pregunta}
Then it is argued that $H$ satisfies the following Riccati equation:
$$
H'+H^2+\mathcal{R}=0
$$
where $\mathcal{R}:=\text{Ric}(\gamma')+|A_0|^2$. I don't understand what is 
$A_0$. Why does this hold? I think
it's the key to take Eq. \ref{equation-Bishop-Gromov-proof1} to an equation of
the kind 
\begin{equation}
\label{equation-Sturm-type1}
\det\mathbb{J}''+\mathcal{R}\det\mathbb{J}=0
\end{equation}
so that we can apply Sturm. 
\end{remark}
\bigskip
Notice that in the case of constant curvature $k$, Eq. 
\ref{equation-Bishop-Gromov-proof1} becomes
\begin{equation}
\label{equation-Bishop-Gromov-proof2}
\det\overline{\mathbb{J}}'=\det\overline{\mathbb{J}}k
\end{equation}
which in turn should yield an equation of Sturm type, as in Remark
\ref{remark-pregunta}. Namely,
\begin{equation}
\label{equation-Sturm-type2}
\det\overline{\mathbb{J}}''+k\det\overline{\mathbb{J}}=0
\end{equation}
Now we want to compare the volume of the sphere in $M$ with the volume of a
sphere of the same raduis in the space of constant curvature 
$\mathbb{Q}_k^{n+1}$:
\begin{align*}
\frac{\text{Vol}(S_r^{n}(p))}{\text{Vol}(S^{n}_{r,k})}&=
\frac{\int_{S^n(0)}\det \mathbb{J}}{\int_{S^n(0)}\det\overline{\mathbb{J}}}
=\frac{1}{\text{Vol}(S^n(0))}
\int_{S^n(0)}\frac{\det\mathbb{J}}{\det\overline{\mathbb{J}}}
\end{align*}
By Sturm Theorem \ref{theorem-Sturm}, which we may apply since both 
$\det\mathbb{J}$ and $\det\overline{\mathbb{J}}$ satisfy Eqs.
 \ref{equation-Sturm-type1} and \ref{equation-Sturm-type2}, and moreover 
they both vanish at $r=0$ and their derivatives are $1$ at $r=0$ by Exercise
\ref{exercise-Jacobi-tensor-derivative}, and of course, since we are supposing
that $\text{Ric}M\geq k$, we conclude that the integrand is a non {\bf
decreasing} function.
\begin{remark}[Pregunta]
\label{remark-pregunta2}
Según Sturm, el cociente de las funciones en cuestión, en este caso,  
$\frac{\det\mathbb{J}}{\det\overline{\mathbb{J}}}$ debería ser una función no
decreciente, pero el resultado que buscamos es que sea no {\bf creciente}.
\end{remark}
\bigskip
Now we turn to computing the ratios of the volumes of balls of radius $r$. 
\begin{align*}
\frac{\text{Vol}(B_r(p))}{\text{Vol}(B_{r,k})}&=
\frac{\int_{S^n(0)}\int_0^r\det\mathbb{J}(t)dt\text{Vol}S^n(0)}
{\text{Vol}(S^n(0))\int_0^r \det\overline{\mathbb{J}}(t)dt}\\
&=\frac{1}{\text{Vol}(S^n(0))} \frac{\int_0^r}{}
\end{align*}
\begin{remark}[Preguntas]
\label{remark-pregunta3}
I have two questions:
\begin{enumerate}
\item We have said in lecture that we may apply Fubini's theorem by Gauss'
lemma. Why? By Gauss' Lemma the normal vectors are orthogonal to the spheres,
the constructions related to Riccati equation are valid (cf. Section
\ref{section-Riccati-equation}). By why does this allow to use Fubini? 
\item We can pull out the volume in the denominator because the volumes of
spheres in the space of constant curvature somehow does not depend on the radius
 $r$. But how exactly? I think  $\det \overline{\mathbb{J}}(r)$ does depend 
on $r$ (in the space of constant curvature), see Eq. 
\ref{equation-Bishop-Gromov-proof2}.
\end{enumerate}
\end{remark}
Then the equation continues to
\begin{align*}
&=\frac{1}{\text{Vol}(S^n(0))}
\int_{S^n(0)}\frac{\int_0^r \det \mathbb{J}(t)dt \text{Vol}S^n(0)}
{\int_0^r\det\overline{J}dt}\\
&=\frac{1}{\text{Vol}(S^n(0))}
\int_{S^n(0)}\frac{\int_0^r \frac{\det \mathbb{J}(t)}
{\det\overline{\mathbb{J}}(t)}
\det\overline{\mathbb{J}}(t)dt \text{Vol}S^n(0)}
{\int_0^r\det\overline{\mathbb{J}}dt}\\
&=\frac{1}{\text{Vol}(S^n(0))}
\int_{S^n(0)}\frac{\int_0^r \frac{\det \mathbb{J}(t)}
{\det\overline{\mathbb{J}}(t)}d\mu}{\mu[0,r]}\text{Vol}S^n(0)
\end{align*}
where we have introduced a measure 
$\mu:=\det \overline{\mathbb{J}}dt$.

By our discussion before, the integrand is a non decreasing function, almost as
required.

Finally, notice that by the rigidity part on Sturm's Theorem
\ref{theorem-Sturm}, the condition
$$
\frac{\text{Vol}(B_s(p))}{\text{Vol}(B_{s,k}}=
\frac{\text{Vol}(B_r(p))}{\text{Vol}(B_{r,k})}
$$
implies that $\det\mathbb{J}=\det\overline{\mathbb{J}}$ and $k=\mathcal{R}$. 

\begin{remark}[Pregunta]
\label{remark-pregunta4}
I'm not sure exactly how this shows that $B_r(p)=B_{r,k}$ isometrically.
\end{remark}
\end{proof}

\end{document}


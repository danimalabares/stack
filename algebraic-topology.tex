\input{preamble}

\begin{document}

\title{Algebraic Topology}
\maketitle

\phantomsection
\label{section-phantom}

\tableofcontents

\section{Covering spaces}
\label{section-covering-spaces}

\begin{proposition}[Unique lifting property]
\label{proposition-unique-lifting-property}
\cite{hat}, Proposition 1.34. Given a covering space $p:\tilde{X} \to X$ and a
map $f:Y \to X$, if two lifts $\tilde{f}_1,\tilde{f}_2:Y \to \tilde{X}$ of $f$
agree at one point of $Y$ and $Y$ is connected, then $\tilde{f}_1$ and
$\tilde{f}_2$ agree on all of $Y$.
\end{proposition}

\begin{proof}
We aim to show that the set of points where $\tilde{f}_1$ and $\tilde{f}_2$
agree is both open and closed in $Y$. Let $y \in Y$ and consider an evenly
covered neighbourhood $U$ of $f(y)$, so that $p^{-1}(U)$ is a disjoint union of
sets, each of which is homeomorphic to $U$ via $p$. Let $\tilde{U}_1$ and
$\tilde{U}_2$ be the sheets containing $\tilde{f}_1(y)$ and $\tilde{f}_2(y)$,
respectively. By continuity of $\tilde{f}_1$ and $\tilde{f}_2$, the intersection
$\tilde{f}^{-1}_1(\tilde{U}_1) \cap \tilde{f}_2^{-1}(\tilde{U}_2):=N$ is 
 an open neighbourhood of $y$.

Since  $\tilde{U}_1$ and $\tilde{U}_2$ are mapped homeomorphically to the set
$U \subset X$, we can only have that either $\tilde{U}_1$ coincides
with $\tilde{U}_2$, or they are disjoint. If they are disjoint, then the 
neighbourhood $N$ shows that the set of points where $\tilde{f}_1$ and
$\tilde{f}_2$ agree is closed; that's because in this case they don't agree at a
point, so that we are on the complement of the set where they agree, and we have
shown it's an open set (every point has a neighbourhood contained in the set).

If, on the other hand, $\tilde{U}_1$ and $\tilde{U}_2$ are the same, which
happens when $\tilde{f}_1(y)=\tilde{f}_2(y)$, then $\tilde{f}_1$ and 
$\tilde{f}_2$ agree throughout $N$, so that we have shown that the set where
$\tilde{f}_1$ and $\tilde{f}_2$ coincide is open.
\end{proof}

\begin{definition}[Properly discontinuous action]
\label{definition-properly-discontinuous-action}
See \cite{hat} p. 72. The action of a group $G$ on a space $Y$ is called 
{\it properly discontinuous} or {\it covering space action} if each $y \in Y$
has a neighbourhood $U$ such that $g_1(U) \cap g_2(U)\neq \emptyset$ for any
$g_1,g_2 \in G$ implies $g_1=g_2$.
\end{definition}

\begin{proposition}
\label{proposition-deck-transformations-act-properly-discontinuous}
The action of deck transformations on a covering space is a properly
discontinuous action.
\end{proposition}

\section{Hurewicz homomorphism}
\label{section-Hurewicz-homomorphism}

\begin{definition}[Hurewicz homomorphism]
	\begin{align*}
		h: \pi_{n}(X) &\longrightarrow H_{n}(X) \\
		[f] &\longmapsto f_{*}(\alpha)
	.\end{align*}
\end{definition}
where $\alpha$ is a generator of $ H_{n}(S^{n})$.

\subsection{Absolute and relative versions of Hurewicz theorem}
\label{subsection-absolute-and-relative-Hurewicz-theorem}

\begin{theorem}[Hurewicz]
\label{theorem-Hurewicz-absolute}
If $X$ is $n$-connected for some $n \geq 2$, them $\tilde{H}_i(X)=0$ for $i<n$
and $\pi_{n}(X)\cong H_n(X)$.
\end{theorem}

\begin{lemma}
\label{lemma-Hurewicz-homomorphism-gives-homotopy-equivalence}
If $X$ and $Y$ are simply connected CW-spaces and  $f:X \to Y$ induces
isomorphisms $H_n(X) \cong H_n(Y)$ for all $n$, then it is an homotopy
equivalence.
\end{lemma}

\begin{theorem}[Hurewicz]
\label{theorem-Hurewicz-relative}
If $(X,A)$ is $(n-1)$-connected, i.e. $\pi_{n}(X,A)=0$ for $i \leq n-1$, for 
some $n\geq 2$ and $A$ is simply connected, then $H_i(X,A)=0$ for $i\leq n-1$ 
and $\pi_{n}(X,A)\cong H_n(X,A)$.
\end{theorem}
Perhaps $X$ should be $(n-1)$-connected for the relative version too---
they are stated together in Hatcher.

\bibliography{my}
\bibliographystyle{amsalpha}

\end{document}


\input{preamble}

\begin{document}

\title{Algebra}
\maketitle

\phantomsection
\label{section-phantom}
\hfill
\href{http://github.com/danimalabares/stack}{github.com/danimalabares/stack}

\tableofcontents

\section{Rings}
\label{section-rings}

\noindent
So elementary that this is not in Stacks Project.

\begin{theorem}
\label{theorem-ideals-quotient-correspondence}
Let $R$ be a ring and $I \subset R$ an ideal.
There is a correspondence between ideals of $R$ containing $I$ 
and ideals of $R/I$.
\end{theorem}

\begin{proof}
We show that there are two maps between the set of ideals of $R$ 
containing $I$ and the ideals of $R/I$ whose compositions
are the identities in the corresponding set. One map is the projection
to the quotient, and the other map is taking the elements
whose equivalence class lies in a given ideal.

Suppose that $\tilde{J}$ is an ideal of $R/I$.
Define $J=\{j \in R : j+I \in \tilde{J}\}$, 
the set of elements that project to $\tilde{J}$.
Then $I \subset J$, because $i + I = I$ is the zero of $R/I$ 
which is contained in any ideal of $R/I$.
Also notice that $J$ is an ideal of $R$ since for any $r \in R$,
$rj+I \in \tilde{J}$ by $\tilde{J}$ being an ideal.
Notice that projecting $J$ back to $R/I$ gives $J$
by definition.

Conversely, we project to the quotient.
Let $J$ be any ideal of $R$ containing $I$ 
Let $\tilde{J}$ be the projection to $R/I$.
Then $\tilde{J}$ is an ideal of $R/I$ by $J$ being an ideal of $R$.
Consider the set $J'$ of all elements whose projection lies in $\tilde{J}$.
Then $J'=J$ by definition.
\end{proof}

\noindent
This is the essential tool in proving that $R/I$ is a field
when $I$ is a maximal ideal

\begin{lemma}
\label{lemma-quotient-ring-maximal-ideal-is-field}
Let $R$ be a ring and $I$ a maximal ideal of $R$.
Then $R/I$ is a field.
\end{lemma}

\begin{proof}
It's easy to prove that a ring is a field if and only if its
only ideals are $\{0\}$ and $R$. By Theorem \ref{theorem-correspondence-ideals},
the ideals of $R/I$ are in correspondence with ideals of $R$ containing $I$,
but only $R$ is such.
\end{proof}

\section{Fields}
\label{section-fields}

\noindent
Just for the record, any simple field extension, i.e. the smallest
field in $F$ containing a single element, is isomorphic
to either que ``rational function field $k(t)/k$''
or to one of the field extensions $k[t]/(P)$ 
with $P \in k[t]$ irreducible.
(Stacks Project tag \href{https://stacks.math.columbia.edu/tag/09G1}{09G1}.)

\section{Modules}
\label{section-modules}

It looks like most books (at least
\cite{eis} and \cite{Samuel-Zariski-Vol1})
define a module to be an abelian group
along with a certain operation with
elements of the ring.

Which secretly just says that

\begin{definition}
\label{definition-module}
Let $R$ be a ring.
An {\it (left) $R$-module} $M$ is an abelian group
such that there exists a (left) ring morphism
$$
R \to \text{End}(M).
$$
\end{definition}

\noindent
Indeed, the three usual requirements
correspond to:

\begin{enumerate}
\item The endomorphism corresponding to $a \in R$ 
respect the group structure of the group $M$:
$$
a(x+y)=ax+ay,
$$

\item The representation map is a map of rings
$$
(a+b)x=ax+by,\qquad (ab)x=a(bx).
$$
\end{enumerate}

\noindent


\section{Algebras}
\label{section-algebras}

\noindent
There appears to be no definition of ``algebra''
in Stacks Project.
But there is one in \cite{Eisenbud}:

\begin{definition}
\label{definition-algebra}
If $R$ is a commutative ring,
then a {\it commutative algebra} over $R$ 
is a commutative ring $S$ together
with a ring morphism $R \to S$.
\end{definition}

\noindent
But actually I was expecting that 
$S$ would be defined as 
a ring that is also an $R$-module.
So let us note that a morphism of rings
$R \to S$ gives a representation
$R \to \text{End}(S)$ via left multiplication.
But the other way around,
given an endomorphism associated
to some element in $R$,
how do we assign an element of $S$
so as to produce a map $R \to S$?

\section{Finitely-generated and finitely presented algebras}
\label{section-finitely-presented-generated}

\noindent
See Stacks Project tag \href{https://stacks.math.columbia.edu/tag/00F2}{00F2}.

Upshot: a finitely-generated $R$-algebra $S$ is such that
$S \simeq R[x_1,\ldots,x_n]/I$ for some ideal $I \subset R$.
Finitely-presentedness is when $I = (s_1,\ldots,s_k)$.

Fun fact: The second item in the following definition
is the algebraic counterpart to an
affine algebraic set (variety);
i.e. the reason why we hear that
``affine varieties are in correspondence
with finitely presented $k$-algebras''
(wasn't there a notion of reduced algebra in that phrase…?)
And the difference with that and finite type,
I think,
is that the kernel is finitely generated.

\begin{definition}
\label{definition-finite-type}
Let $R \to S$ be a ring map.
\begin{enumerate}
\item We say $R \to S$ is of {\it finite type}, or that {\it $S$ is a finite
type $R$-algebra} if there exist an $n \in \mathbf{N}$ and an surjection
of $R$-algebras $R[x_1, \ldots, x_n] \to S$.
\item We say $R \to S$ is of {\it finite presentation} if there
exist integers $n, m \in \mathbf{N}$ and polynomials
$f_1, \ldots, f_m \in R[x_1, \ldots, x_n]$
and an isomorphism of $R$-algebras
$R[x_1, \ldots, x_n]/(f_1, \ldots, f_m) \cong S$.
\end{enumerate}
\end{definition}

Informally, $M$ is a finitely presented $R$-module if and only if
it is finitely generated and the module of relations among these
generators is finitely generated as well.
A choice of an exact sequence as in the definition is called a
{\it presentation} of $M$.

So, probably an ``algebra'' $A$ over a ring  $R$ is when $A$ contains 
(or, is isomorphic to?)  $R[X]$ for some possibly very arbitrary set
 $X\subset R$.


\section{Finite and integral ring extensions}
\label{section-finite-ring-extensions}

\noindent
The upshot is: let $\varphi:R \to S$ be a ring map.
An element $s \in S$ is integral over $R$ if there
is a monic polynomial with coefficients in $R$
that gives zero when you put $s$ instead of the variable.

But it might be convenient to think about
finite (or, for humans, finitely-generated) $R$-modules:
$x$ is integral over $R$ if and only if $R[x]$
is a finitely-generated  $R$-module (i.e.,
not only finitely generated as an algebra
but also as a module).
See \href{
https://youtu.be/5fCHj80BQHw?si=o7tMC4tRsbBnwmJi}{Zvi Rosen's video}.
The forward implication is: suppose $x$ is integral,
then $x^n+r_1x^{n-1}+\ldots+r_n=0$, so
$x^n=-r_1x^{n-1}+\ldots+r_n$,
which ``says'' $R[x] \cong \bigoplus_{k=0}^{n-1}Rx^k$.

So maybe the details of that are not entirely trivial (Zvi uses some theorem,
and Stacks Project does lots of things).
But the point is: I think that after applying Spec,
the fibers of an integral extension are {\bf finite}.
Which of course has all to do with the finite-moduleness
I was trying to explain.




\section{Gröbner theory}
\label{section-grobner}

\noindent
So far here's some rather unformatted notes from
Lin's talk at Commutative Algebra school in Recife 2025.

Would SAGBI bases be related to this?

square free monomial ideals preserve regularity upon Grobner deformation
i.e. initial ideal!

Constantini-Fouli-Goel-Lin-Lindo-Liske-Mostafazadehfard, 2025.

https://sites.psu.edu/kul20/

How is Lin using the Reese algebra here?
Is it ture that the initial ideal of the Gröbner deformation
coincides with the initial ideal ``of the Resse algebra''?



\bibliography{my}
\bibliographystyle{amsalpha}

\end{document}

\input{preamble}

\begin{document}

\title{Differential Geometry}
\maketitle

\phantomsection
\label{section-phantom}

\tableofcontents

\section{Submanifolds}
\label{section-submanifolds}

\begin{remark}
\label{remark-ricci-equation}
Ricci equation may help prove that the normal bundle of a high codimension
submanifold is trivial.
\end{remark}


\section{Variações da energia}
\label{section-variações-energia}

\begin{theorem}[Bonnet Myers]
\label{theorem-bonnet-myers}
If $M$ has  $\operatorname{Ric} \leq \frac{1}{r^2}$, then
$\operatorname{diam}M\leq \pi r$. In particular, $M$ is compact.
\end{theorem}

\section{Teorema de comparação de Rauch}
\label{section-rauch}

\subsection{Index lemma}
\label{subsection-index-lemma}

\subsection{Teorema de Rauch}
\label{subsection-rauch}

\begin{theorem}[Rauch]
\label{theorem-rauch}
\end{theorem}

\begin{proposition}
\label{proposition-contraction}
Sejam $p \in M$, $\tilde{p} \in \tilde{M}$ e $i:T_pM \to T_{\tilde{p}}\tilde{M}$
uma isometria. Então $f:=\operatorname{exp}^{-1}_p \circ i \circ
\operatorname{exp}_{\tilde{p}}$ é uma contração métrica, i.e. $|f_*w|\leq |w|$
para todo $w \in T_pM$.

Mas ainda, se $\operatorname{exp}_p^{-1}$ está definida numa bola totalmente
convexa, $f$ é uma contração métrica.
\end{proposition}

\begin{proof}
Write an abitrary $w\in T_p M$ as a Jacobi variational field of the variation 
$f(s,t)=\operatorname{exp}_p(t(v+sw)$. The resulting field on $\tilde{M}$ is a
Jacobi field with respect to $iw$ using that $i$ is an isometry. The conditions
of Rauch theorem are satisfied and the desired inequality is obtained.

For the metric result we just integrate along a curve realising the distance and
compute.
\end{proof}

\section{Morse index theorem}
\label{section-morse-index}

\begin{slogan}
The problem is that the vector spaces of sections of smooth manifolds are infinite-dimensional. we are interested in computing
\end{slogan}

\begin{definition}
\label{definition-index-index-form}
$$
i(I_a):=\operatorname{sup}\{\dim L \subset \mathfrak{X}_\gamma:I_a |_{L \times L}<0\}
$$

\end{definition}

\begin{theorem}[Morse Index]
\label{theorem-morse-index}
O índice $i(I_t)$ é finito e igual ao número de pontos conjugados em $[0,t)$ ao
longo de $\gamma$.
\end{theorem}

\begin{proof}
\end{proof}

\section{Cut locus}
\label{section-cut-locus}

\begin{definition}
\label{definition-injectivity-radius}
The {\it injectivity radius} of a manifold $M$ is
$$
i(M)=\operatorname{inf}\{d(p,C_m(M)):p \in M\}
$$
\end{definition}

\section{Bishop-Gromov Theorem}
\label{section-bishop-gromov}

\begin{theorem}[Bishop-Gromov]
\label{theorem-bishop-gromov}

\end{theorem}

\begin{proof}
Tome a função exponencial como uma carta parametrizando a esfera geodésica de
raio $r$ com centro em $p$. Então
$$
\operatorname{Vol}_{S_r}=\int
\operatorname{Vol}_{S_r}=
\int_{S_r(0)}\operatorname{exp}_p^*\operatorname{Vol}_{S_r}
$$
Portanto, estamos interessados em calcular 
$$
|\det d_{rv} \operatorname{exp}_p|
$$
Avaliando em uma base de vetores tangentes a $S_r(0)$, notamos que cada um deles
nos dá um campo de Jacobi $J$ ao longo de $\gamma$.

Como $\gamma'(r)\perp J$, derivando obtemos $A J=J'$ onde $A$ é o operador de
forma de $S_r$ respeito ao normal interior $\gamma'(r)$ (cf. conta feita antes).

Mmm… no sé: creo que lo que quiero calcular realmente es
$$
\det J
$$
entonces no sé dondé entran ni $A$ ni $J'$… O sea si derivas con respecto a $r$
pues va… pero si no…

\end{proof}

\section{Toponogov's theorem}
\label{section-toponogov}

First recall

\begin{theorem}[Toponogov, local version]
\label{theorem-toponogov,local}
Let $o,p_1,p_2 \in B$ where $B$ is a totally convex ball of $M$. Let $\gamma_1$
and $\gamma_2$ be the normalized geodesics joining $o$ to $p_1$ and to $p_2$.

Suppose $\tilde{o}, \tilde{p}_1,\tilde{p}_2 \in \tilde{B}$ is a triple on 
another manifold $\tilde{M}$ such that the distances from $\tilde{o}$ to
$\tilde{p}_1$ and from $\tilde{o}$ to $\tilde{p}_2$ coincide with those in $M$.

Then for any $t \in [0,\ell(\gamma_1)]$ and $s \in [0,\ell(\sigma)]$,
$$
\tilde{d}(\tilde{\gamma}_1(t),\tilde{\gamma}_s)\leq d(\gamma_1(t),\gamma_2(s))
$$
\end{theorem}

\begin{exercise}[Lista 7]
\label{exercise-toponogov-from-rauch}
Prove the local version of Toponogov's theorem as a corollary of Rauch's
Comparison Theorem \ref{theorem-rauch}.
\end{exercise}

\begin{proof}
This is an application of Proposition \label{proposition:contraction} for $i$ as
given by the condition of mapping $\gamma'_1(0)$ to $\gamma_2'(0)$ and the angle
between them to the corresponding vectors and angle on $\tilde{M}$.
\end{proof}

\begin{theorem}[Toponogov, hinge version] \label{theorem-toponogov-hinge}
Let $M$ be a complete manifold and $K \geq k$. Let $o,p_1,p_2 \in M$, with
$p_1\neq  p_2$. Let $\gamma_1$ and $\gamma_2$ be normalized geodesics joining 
$o$ to $p_1$ and to $p_2$. Suppose $\gamma_1$ is minimizing.
Let $\gamma_0$ be a nonconstant geodesic joining $p_1$ to $p_2$ and suppose
$$
\ell(\gamma_0) \leq \ell(\gamma_1) +\ell(\gamma_2)
$$
If $k>0$ suppose additionally that $\ell(\gamma_0) \leq \pi/\sqrt{k}$.

Then there exists a geodesic triangle $\{\tilde{\gamma}_i:i=1,2,0\}$ in
$\mathbb{Q}_k^2$ such that $\ell(\tilde{\gamma}_i) = \ell(\gamma_i)$ and
$$
d(\tilde{p}_1,\tilde{p}_2) \leq  d(o,p_1)+d(o,p_2)
$$

Put another way (next lecture…) Let $\tilde{\gamma}_1$ and $\tilde{\gamma}_2$ 
be a corresponding angle in $\mathbb{Q}_k^2$. Then
$$
\tilde{d}(\tilde{\gamma}_1(t),\tilde{\gamma}(s)\leq d(\gamma_1(t),\gamma_2(s))
$$

\end{theorem}

\begin{remark}
For me a more geometric statement is



Suppose $\tilde{o}, \tilde{p}_1,\tilde{p}_2 \in \tilde{B}$ is a triple on 
another manifold $\tilde{M}$ such that 
$$
d(\tilde{p}_1,\tilde{p}_2) \leq  d(o,p_1)+d(o,p_2)
$$
If the sectional curvature $K$ of $M$ is $K>0$, then suppose additionally that
$\ell(\gamma_1$

Then for any $t \in [0,\ell(\gamma_1)]$ and $s \in [0,\ell(\sigma)]$,
$$
\tilde{d}(\tilde{\gamma}_1(t),\tilde{\gamma}(s)\leq d(\gamma_1(t),\gamma_2(s))
$$
So it's the same right? You look form $p_1$ instead…
\end{remark}


\section{Exercises}
\label{section-exercises}

\begin{exercise}
Calcule o diâmetro de $S^2$, $\mathbb{T}^2$, $\mathbb{R}P^{2}$.
\end{exercise}

\begin{proof}[Solution]
O diâmetro de $S^2$ pode ser calculado via o teorema de Bonnet Myers
\ref{theorem-bonnet-myers}: nenhuma geodésica é minimizante depois de atingir
comprimento $\pi r$, e temos uma geodésica que atinge esse comprimento: qualquer
uma!

O diâmetro de $\mathbb{T}^2$ é $1$. Isso é por simples geometria
euclidiana: é o diâmetro do cubo! Por definição, a métrica de  $\mathbb{T}^2$ é
a induzida pela projeção quociente.

O diâmetro de $\mathbb{R}P^{2}$ é $\pi/2$. Qualquer geodésica que liga dois
pontos a distância $\pi/2$ é minimizante, pois estamos na métrica esférica e
podemos pegar cartas esféricas onde dois pontos a essa distância estão contidos. 
Por outro lado, se tivéssemos dois pontos a distância maior, a geodésica 
esférica que percorre o ponto antípoda ao inicial, chega no ponto final mais 
rapidamente que inicial; portanto geodésicas de comprimento maior que $\pi/2$ 
não são minimizantes.
\end{proof}

\begin{exercise}
\label{exercise-pts}
$\gamma$ geodésica sem pontos conjugados em $[0,a]$. Então $\gamma$ é localmente
minimizante.
\end{exercise}

\begin{proof}
Temos dois casos:
\begin{enumerate}
\item Se $E''(0)<0$ usamos o teorema do índice.
\item Se $E''(0)=0$,
\begin{enumerate}
\item[(0)] Como $\gamma$ não tem pontos conjugados, então $I$ não tem
		nulidade em todo  $\mathcal{V}$.
\item $I_{\mathcal{V}^-}$ não tem índice nem nulidade. Segue do ponto anterior +
	algum passo na prova.
\item $I_{\mathcal{V}^-}$ não tem índice.
\item 2+3 dão que $I_{\mathcal{V}^-}>0$.
\item Logo $I>0$.
\end{enumerate}
\end{enumerate}
\end{proof}

\begin{exercise}
\label{exercise-wraped-product}
Para $(M,g_M)$ e $(N,g_N)$ e $f:M \to \mathbb{R}_+$, definimos o {\it warped
product} como sendo o produto cartesiano $M\times N$ com a métrica 
 $g_M+f^2g_N$.

Mostre que se os fatores são completos, o warped product é completo.
\end{exercise}

\subsection{Highlight exercises}
\label{subsection-highlight-exercises}
\begin{exercise}
\label{exercise-minimal-intersection}
$(M^n,g)$ completa $\operatorname{Ric}_M>0$. $A$, $B$ hipersuperfícies mínimas
completas. Então  $A \cap B \neq \emptyset$.
\end{exercise}

\begin{proof}
Suponha que existe uma $\gamma$ mínima com comprimento positivo ligando as 
duas variedades. Tem dois caminhos:
\begin{enumerate}
\item Deixa o campo exponencial ao longo da $\gamma$ e ajusta nos extremos
para fazer ele tangente às variedades. Nesse caso já tem que o temo
$\left<f_{ss},\gamma'\right>|_{0}^a$ se anula. Então so mostrar que $I(V,V)$  é
zero. {\bf Onde se usa que são mínimas? Intentar construir esta?}
\item Começa com uma geodésica tangente e transporta paralelamente um vetor
tangente $e_i$ a  $M_1$. Ele chega tangente a  $M_2$ porque a geodésica é
minimizante (cf exer passado).
\end{enumerate}
\end{proof}

\begin{exercise}[\cite{doc} Chapter IX, Exercise 5]
\label{exercise-two-submanifolds}
Sejam $N_1$ e $N_2$ duas subvariedades fechadas e disjuntas de uma variedade
Riemanniana compacta.
\begin{enumerate}
\item Mostre que a distância entre $N_1$ e $N_2$ é realizada por uma geodésica
	$\gamma$ perpendicular a ambas $N_1$ e $N_2$.
\item Mostre que, para qualquer variação ortogonal $h(t,s)$ de $\gamma$, com
$h(0,s) \in N_1$ e $h(\ell,s) \in N_2$, tem-se para a fórmula da segunda
variação a seguinte expressão
$$
\frac{1}{2}E''(0)=I_{\ell}(V,V)+
\left<V(\ell),S_{\gamma'(\ell)}^{(2)}V(\ell)\right>
-\left<V(0),S^{(2)}_{\gamma'(0)}(V(0))\right>
$$
\end{enumerate}
\end{exercise}


\begin{exercise}
%\label{exercise}
{\bf (Qualificação)}Prove que $M^{2n},K_M>0$ compact, então ela é simplesmente conexa.
\end{exercise}

\begin{exercise}
%\label{exercise-}
$M^{2n+1}$, $K_M>0$ compacta \(\implies\) orientável.
\end{exercise}

O exercício anterior usa Exercício 6 da lista 6. Ou, equivalentemente,

\begin{exercise}
%\label{exercise-}
Se $M^{2n+1}$ tem $K_M >0$ e $\gamma:S^1 \to M$ inverta orientação, então é
possível encurtar $\gamma$ na sua classe de homotopia.
\end{exercise}

\begin{exercise}

\end{exercise}

\section{Lista 8}
\label{section-l8}

\begin{exercise}
Prop. 2.12 do capítulo XIII, \cite{doc}. Seja $p \in M$. 
Suponha exista um ponto $q \in C_m(p)$ que realiza a distância de $p$ a 
$C_m(p)$. Então:
\begin{enumerate}
\item ou existe uma geodésica 
minimizante $\gamma$ de $p$ a $q$ ao longo da qual $q$ é 
conjungado a p,
\item ou existem exatamente duas geodésicas 
minimizantes $\gamma$ e $\sigma$ de $p$ a $q$; além disto, $\gamma'(\ell)$, 
$\ell=d(p,q)$. 
\end{enumerate}
\end{exercise}

\begin{proof}[Prova sem consultar outras referencias]
Foi provado em sala que um ponto $q$ está no cut locus $C_m(p)$ se e somente se
alguma das seguintes condições é verdadeira: (a) $q$ é o primeiro ponto
conjugado a $p$, ou (b) existem duas geodésicas minimizantes ligando $p$ e $q$.

Sponha que (a) não é verdadeira. Considere a variação por geodésicas 
$$
f(s,t):=\operatorname{exp}_{\gamma(t)}
(s\operatorname{exp}_{\gamma(t)}^{-1}\sigma(t))
$$
Note que se $\gamma'(\ell)=-\sigma'(\ell)$ o campo de Jacobi é nulo. Em outro
caso obtemos que $p$ é conjugado a $q$, absurdo.
\end{proof}

\begin{exercise}
Proposição 2.13, Cap. XIII \cite{doc}. Se a curvatura seccional $K$ de uma
variedade Riemanniana completa $M$ satisfaz
$$
0\leq K_{\operatorname{min}}\leq K\leq K_{\operatorname{max}},
$$
Então
\begin{enumerate}
\item $i(M) \geq \pi/\sqrt{K_{\operatorname{max}}}$, ou
\item existe uma geodésica fechada $\gamma$ em $M$, cujo comprimento é menor do 
que o de qualquer outra geodésica fechada em $M$, tal que
$$
i(M)=\frac{1}{2}\ell(\gamma)
$$

\end{enumerate}
\end{exercise}

\bibliography{my}
\bibliographystyle{amsalpha}

\end{document}


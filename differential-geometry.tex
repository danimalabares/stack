\input{preamble}

\begin{document}

\title{Differential Geometry}
\maketitle

\phantomsection
\label{section-phantom}

\tableofcontents

\section{Connections}
\label{section-connections}

\begin{example}
\label{example-euclidean-connection}
The usual connection in $\mathbb{R}^n$ is given by 
$$
\nabla_XY=X(Y^i)\partial_i
$$
\end{example}

\section{Riemannian submersions and immersions}
\label{section-Riemannian-submersions-and-immersions}

For any smooth manifold submersion $\pi:\tilde{M}\to M$ there is a bundle
isomorphism I love: 
\begin{equation}
\label{equation-submersion-bundle-decomposition}
\pi^*TM \oplus \Ker \pi\cong
T\tilde{M} \end{equation}
 where $\Ker\pi$ is the {\it kernel bundle}, whose
fiber is the kernel of $\pi_*$, and is also called the {\it vertical bundle}.
Proving that $\Ker\pi$ is in fact a bundle and confirming the isomorphism
\ref{equation-submersion-bundle-decomposition} is a simple exercise.

We call $\pi$ a {\it riemannian submersion} only when $\pi^*TM$ is isometric to
$TM$, which makes sense only when $\tilde{M}$ and $M$ are Riemannian manifolds.

Following notation of \cite{Milnor-Characteristic-Classes}, for any subbundle
$\xi \subset\eta$ over a Riemannian manifold, the {\it orthogonal complement
bundle} $\xi^\perp$ can be defined fiberwise as the orthogonal complement of the
fiber and shown to be a bundle as in the case of the kernel bundle. Then Milnor
defines the {\it normal bundle} when $\xi$ is the tangent bundle of a
submanifold. Finally this construction is noted to work just as well for any
isometric immersion $f:M\to \tilde{M}$, so that
\begin{equation}
\label{equation-immersion-bundle-decomposition}
f^*T\tilde{M}\cong TM\oplus T^\perp M
\end{equation}
Further, in problem 3-B we are asked to define for $\xi \subset \eta$ the {\it
quotient bundle} and show it's a bundle. And if $\eta$ has a Riemannian metric,
then $\xi^\perp\cong\eta/\xi$. So that's why we sometimes say that we can define
the normal bundle using a Riemannian metric but we don't need to.

An interesting thought is: given a subbundle, is there a manifold whose tangent
bundle is that subbundle?

\section{Geodesics}
\label{section-geodesics}

\begin{proposition}[Totally Normal Neighbourhoods]
\label{proposition-totally-normal-neighbourhoods}
For all $p \in M$ there is a neighbourhood $W \ni p$ and a number $\delta>0$ 
 such that for every $q \in W$, $B_\delta(q)$ is normal and
 $W \subset B_\delta(q)$. 
\end{proposition}

\section{Curvature}
\label{section-curvature}

The {\it curvature tensor} is
\begin{equation}
\label{equation-curvature-tensor}
R(X,Y)Z=\nabla_X\nabla_YZ-\nabla_Y\nabla_XZ-\nabla_{[X,Y]}Z
\end{equation}

The {\it Ricci tensor} is
\begin{align*}
	\text{Ric}: TM\times TM &\longrightarrow \mathbb{R} \\
	 \text{Ric}(X,Y)&=\frac{1}{n-1}\text{tr}(Z\mapsto R(Z,X)Y)
\end{align*}
It is symmetric.

The {\it Ricci curvature} is
\begin{align*}
	\text{Ric}: T^1M &\longrightarrow \mathbb{R} \\
	\text{Ric}(X)&\text{Ric}(X,X)=\frac{1}{n-1}\sum_{i=1}^{n-1}K(X,E_i)
\end{align*}
where $\{X,E_1,\ldots,E_{n-1}\}$ is an orthonormal basis.

The {\it scalar curvature} is
\begin{align*}
	\text{scal}: M &\longrightarrow \mathbb{R} \\
	\text{scal}(p)&=\frac{1}{n}\text{tr}\text{Ric}
=\frac{1}{n}\sum_i\text{Ric}(e_i,e_i)
\end{align*}
If a manifold has constant curvature $c$,
\begin{equation}
\label{equation-constant-curvature}
\left<R(X,Y)Z,W\right>=
c(\left<X,Z\right>\left<Y,W\right>-\left<X,W\right>\left<Z,Y\right>)
\end{equation}
since the right-hand-side of Equation \ref{equation-constant-curvature} is a
so-called curvature tensor, i.e. a tensor satisfying the symmetries of $R$.

\section{Jacobi Fields}
\label{section-jacobi-fields}

\begin{proposition}[Everyday Jacobi field]
\label{proposition-everyday-jacobi-field}
\cite{doc} Chapter V, Proposition 2.4. Let $\gamma:[0,a]\to M$ be a normalized
geodesic and let $J$ be a Jacobi field through $\gamma$ with $J(0)=0$.
Let $p:=\gamma(0)$ $w:=J'(0)$ and  $v:=\gamma'(0)$. 
Consider the variation
$$
f(s,t):=\text{exp}_{p}(t(v+sw))
$$
Then the associated Jacobi field is $J$. So in particular the field
$$
J(t)=d_{tv}\text{exp}_p(tw)
$$
is a Jacobi field.

{\bf Remark:} the variation can also be taken with respect to
any a curve $\sigma$ on $T_vT_{p}M$ passing through $v$ 
at $s=0$ with velocity $w$.
\end{proposition}

\begin{example}[Jacobi Fields in constant curvature]
\label{example-jacobi-fields-in-constant-curvature}
If $M$ has constant curvature $k$, a Jacobi field is of the form
$$
J(t)=\begin{cases}
	\qquad & \\
	\qquad &
\end{cases}
$$
This follows from Equation \ref{equation-constant-curvature}, 
\end{example}

\begin{proposition}[Critical points of exponential]
\label{proposition-critical-points-exp}
\cite{doc} Chapter V, Proposition 3.5. A vector $v \in T_pM$ is a critical
point of $\text{exp}_p$ if and only if the corresponding 
$\gamma_v(1)$ is a conjugate point of $p$ along $\gamma_v$.
Moreover, in such points, the dimension of the kernel of 
$\text{exp}_p$ at this vector is the multiplicity
of the conjugate point $\gamma_v(1)$ 
(i.e. the dimension of the space of Jacobi fields vanishing 
at the endpoints).
\end{proposition}

\begin{proof}
($\implies$) Suppose that $v$ is a critical point of $\text{exp}_p$ and 
let $w \in \ker d_v \text{exp}_p$ then a Jacobi field along $\gamma$ 
vanishing at the endpoints is given by
$$
J(t):=d_{tv}\text{exp}_ptw
$$
($\impliedby$) If you have a critical point along $\gamma$, then the 
differential of $\text{exp}_p$ has kernel.

(Mulitplicity.) The proof every element of the kernel gives a Jacobi
 field vanishing at the points. Linearly dependent elements will give 
 linearly dependent Jacobi fields.
\end{proof}
\section{Submanifolds}
\label{section-submanifolds}

Let $f:M \to \tilde{M}$ be an isometric immersion of Riemannian manifolds. When
computing the covariant derivative along two vector fields 
$X,Y \in\mathfrak{X}(M)$ we get a vetor field along $X$ that may be split into
its normal and tangent part thanks to Equation 
\ref{equation-immersion-bundle-decomposition}. We find that the tangent part is
in fact the Levi-Civita connection of $M$ since it is a symmetric and torsion
free connection on the tangent bundle. We also find that the normal part is a
symmetric tensor we call {\it second fundamental form}:
\begin{equation}
\label{equation-second-fundamental-form}
\nabla^{f}_X f_*Y=f_*\nabla_XY+\alpha(X,Y)
\end{equation}
In a similar fashion we take a normal section $\xi\in T^\perp M$. 
We call the tangent part {\it shape operator}. The normal part is a connection 
we call the {\it normal connection}.
\begin{equation}
\label{equation-shape-operator}
\nabla^f_X\xi=-A_\xi X+\nabla^\perp_X\xi
\end{equation}
Notice the adjointness relation
\begin{equation}
\label{equation-second-fundamental-form-and-shape-operator}
\left<\xi,\alpha(X,Y)\right>=-\left<A_\xi X,Y\right>
\end{equation}
which follows from computing $0=X\left<Y,\xi\right>$.

\section{Gauss, Codazzi, and Ricci equations}
\label{section-Gauss-Codazzi-Ricci}

The {\it Gauss equation} is
\begin{equation}
\label{equation-Gauss}
\left<\tilde{R}(X,Y)Z,W\right>=\left<R(X,Y)Z,W\right>
-\left<\alpha(X,Z),\alpha(Y,W)\right>+\left<\alpha(X,W),\alpha(Y,Z)\right>
\end{equation}
which follows from substituting Eq. \ref{equation-second-fundamental-form} 
into Equation \ref{equation-curvature-tensor}.

For the case of a hypersurface, the normal vector $\alpha(X,Y)$ must be
proportional to the unit normal $\nu$, and by Eq.
\ref{equation-second-fundamental-form-and-shape-operator} the proportionally
factor is exactly $\left<A_\nu X,Y\right>$. Using this and evaluating
 at an orthonormal pair $X,Y$ we obtain
\begin{equation}
\label{equation-Gauss-hypersurface}
K(X,Y)=\tilde{K}(X,Y)+\left<AX,X\right>\left<AY,Y\right>-\left<AX,Y\right>^2
\end{equation}

\begin{exercise}
\label{exercise-Gauss-sectional}
Use Eq. \ref{equation-Gauss-hypersurface} to show that no 
hypersurface $M \hookrightarrow \mathbb{R}^{n+1}$ can have constant
 negative curvature.
\end{exercise}

\begin{proof}[Solution]
After substituting $\tilde{K}(X,Y)=0$, we see that Eq. 
\ref{equation-Gauss-hypersurface} cannot be negative because it contains the
determinant of the matrix
$$
\begin{pmatrix}
\left<AX,X\right>&\left<AX,Y\right>\\ 
\left<AX,Y\right>&\left<AY,Y\right>
\end{pmatrix}
$$
whose eigenvalues are…
\end{proof}

\begin{example}
\label{example-curvature-of-sphere}
Using Eq. \ref{equation-Gauss-hypersurface} we can prove that the sectional
curvature of $S^n$ is 1 once we make sure that the shape operator of the sphere
is the identity. And this follows from the fact that the connection on
$\mathbb{R}^n$ is just directional derivative, see Example 
\ref{example-euclidean-connection}.

More explicitly, the normal component of $\tilde{\nabla}_X\nu$ will vanish (see
discussion in Section \ref{section-Riccati-equation}, so that
$\tilde{\nabla}_X\nu=-A_\nu X$. But the connection in Euclidean space is just
differentiating the vector field, that is, $\tilde{\nabla}_X\nu=\nu X$. And
finally, who's $\nu$? It's the $-\text{id}$ if you chose the inner normal. So
 $\tilde{\nabla}_X\nu=-X$.
\end{example}

\begin{example}
\label{example-hyperbolic-space}
\cite{doc}, Chapter VIII, Exercise 3. The unit normal of hyperbolic space in the
hyperboloid model is also the identity. This allows for a similar computation,
but Eq. \ref{equation-Gauss-hypersurface} doesn't hold exactly because the
metric is not Riemannian. In fact, the normal vector has norm $-1$, giving that
hyperbolic space has constant curvature $-1$.
\end{example}


\begin{remark}
\label{remark-ricci-equation}
Ricci equation may help prove that the normal bundle of a high codimension
submanifold is trivial.
\end{remark}

\section{Riccati equation}
\label{section-Riccati-equation}

Take any point and any tangent vector at that point. Take the geodesic and take
the geodesic sphere of raduis say $r$ with center in the point. It's
a hypersurface. Normal vectors are all multiples of the unit normal vector. But
who's that, the unit normal? Ah, it's the speed of the geodesic.

So? The normal derivative of any vector field tangent to the sphere with respect
to the unit normal $\gamma'(r):=\nu$ is…
$$
0=X\left<\nu,\nu\right>=2\left<\nabla^\perp_X \nu,\nu\right>
$$
that is, normal derivative of $\nu$ with respect to $X$ is perpendicular to
$\nu$. But normal derivative is normal so it's actually a multiple of $\nu$, 
which makes $\nu$ have norm zero or else $\nabla^\perp_X \nu=0$.

Now consider everyday Jacobi variation \ref{proposition-everyday-jacobi-field}
 $f(s,t):=\text{exp}_p(t(v+sw))$ with Jacobi field
$J(t)=d_{tv}\text{exp}_p(tw)$. Differentiate with using the pullback connection
along $f$:
\begin{align*}
\nabla_{\frac{d}{dt}}^\gamma
J(t)&=\nabla_{\partial_t}^ff_s=\nabla_{\partial_t}^f \partial_sf\\
&=\nabla_{\partial_s}^f \partial_t f=\nabla_{J}^f \gamma' = -A_{\gamma'}f
\end{align*}
using that by our discussion above the normal derivative component vanishes. We
conclude that
$$
J'=AJ
$$
Finally we differentiate that to obtain
$$
J''=A'J+AJ'=A'J+A^2J
$$
that is,
$$
A'+A^2+R_{\gamma'}=0
$$
since we can let $J(r)$ be any vector whatsoever and make this construction
work.

\begin{exercise}
\label{exercise-Jacobi-tensor-derivative}
Show that $\overline{J}'(0)=\text{id}$, where $\overline{J}:T_pM \to
T_{\gamma(r)M}$ has the value of the Jacobi field depending on $w$ via
Proposition \ref{proposition-everyday-jacobi-field} at time $r$.
\end{exercise}

\begin{proof}[Solution]
By Rafael. First we present a naive computation. We will use that
$A(t)(J(t))=J'(t)$,  $J(0)=0$ and $J'(0)\neq 0$. Let $\varepsilon>0$ and let
$s>0$ be such that
$$
\frac{\|J(s)-J(0)-sJ'(0)\|}{s}<\frac{\varepsilon}{2}
$$
and
$$
\|J'(s)-J'(0)\|<\frac{\varepsilon}{2}.
$$
Then
\begin{align*}
\left\|A(s)(J(s))-\frac{J(s)}{s}\right\|&
=\left\|J'(s)-\left(\frac{J(s)-J(0)}{s}\right)\right\|\\
&\leq \|J'(s)-J'(0)\|+\frac{\|J(s)-J(0)-sJ'(0)\|}{s}\\
&\leq\varepsilon/2+\varepsilon/2=\varepsilon
\end{align*}
This shows that $A=\frac{1}{s}J$ for small $s$, but we have differentiated as if
we were in $\mathbb{R}^n$. A similar computation but
picking $s$ such that
$$
\frac{|\left<J(s),e_i(s)\right>-\left<J(0),e_i(0)\right>
-s\left<J'(0),e_i(0)\right>|}{s}<\varepsilon/2
$$
and
$$
|\left<J'(0),e_i(0)\right>-\left<J'(s),e_i(s)\right>|<\varepsilon/2
$$
for a parallel frame $e_i$ along $\gamma$ allows to conclude as desired.
\end{proof}

\section{Hopf-Rinow Theorem}
\label{section-Hopf-Rinow-theorem}

\begin{theorem}[Hopf-Rinow]
\label{theorem-Hopf-Rinow}
The following are equivalent for any $p \in M$:
\begin{enumerate}
\item The domain of the exponential map $\text{exp}_p$ is $T_pM$.
\item Closed and bounded subsets are compact.
\item $(M,d)$ is a complete metric space.
\item $M$ is geodeiscally complete (i.e. every geodesic is maximally defined in
all of $\mathbb{R}$).
\item Existence of exhaustions.
\end{enumerate}
Moreover, each of the former conditions imply that there exists a minimizing 
geodesic joining any two points.
\end{theorem}

\section{Hadamard's theorem}
\label{section-Hadamard}

\section{First Variation Formula}
\label{section-first-variation}

\begin{theorem}[First Variation Formula]
\label{theorem-first-variation-formula}
If  $f:(-\varepsilon,\varepsilon)\times(-\delta,\delta) \to M$ is a variation of
the geodesic $\gamma(-\delta,\delta)\to M$, then
$$
\frac{1}{2}E'(0)=-\int_a^b\left<V,\gamma''\right>+\left<V,\gamma'\right>|_{a}^b
+\sum_{i=1}^{k-1}\left<V,\gamma'^{-}(t_{i})-\gamma'^{+}(t_i)\right>
$$
\end{theorem}

\section{Second Variation Formula}
\label{section-second-variation}

The {\it second variation formula} reads
\begin{equation}
\label{equation-second-variation}
E''(0)=\int_0^a |V'|-\int_0^a R_{\gamma'}V+
\left<\gamma'(t),\nabla_{\partial_s}V(t)\right>|_{0}^a+
\sum_{i}\left<\gamma'(t),V^-(t_i)-V^+(t_i)\right>
\end{equation}

Agora vou fazer uma pausa para lembrar como se escreve a forma do índica em
geral. Por definição, a forma de índice é 
$$
I_a(V,W):=\int_0^a
\left<V'W'\right>-\int_0^a\left<R_{\gamma'}V,W\right>$$
 para quaisquer campos $V,W \in \mathfrak{X}_\gamma$.

Então reescrevemos isso usando que a conexão ao longo de \(\gamma\) é métrica:
\begin{align*}
\left<V,W'\right>'&=\left<V',W'\right>+\left<V,W''\right>\\
\implies  \left<V',W'\right>&=\left<V,W'\right>'- \left<V,W''\right>
\end{align*}
Substituindo obtemos que
\begin{equation}
\label{equation-index-form-alternative-formula}
\begin{aligned}
I_a(V,W)&=\int_0^a\left<V,W'\right>'-\int_0^a \left<V,W''\right>-\int_0^a \left<R_{\gamma'}V,W\right>\\
&=\left<V,W'\right>|_{0}^a-\int_0^a \left<V,W''\right>-\int_0^a\left<R_{\gamma'}W,V\right>\\
&=\left<V,W'\right>|_{0}^a-\int_0^a \left<V,W''+R_{\gamma'}W\right>
\end{aligned}
\end{equation}

\section{Bonnet-Myers Theorem}
\label{subsection-Bonnet-Myers-theorem}

\begin{theorem}[Bonnet-Myers]
\label{theorem-Bonnet-Myers}
If $M$ is a complete manifold with $\text{Ric} \geq \frac{1}{r^2}$, then
$\text{diam}M\leq \pi r$. In particular, $M$ is compact.
\end{theorem}

\begin{proof}[Sketch of proof]
Let $p,q \in M$ be points with $d(p,q)>\pi r$. We will construct a Jacobi
field vanishing at the endpoints, so that these points are conjugate and thus
any geodesic $\gamma$ joining them cannot be minimizing.

Let $w_i$ be vectors in $T_pM$ which along with $\gamma'(0)$ form an orthonormal
basis. Define $W_i$ to be the parallel transport of $w_i$ along $\gamma$ and
$J_i:=\sin(\pi t/\ell)W_i$.

Computing the second variation formula, which reduces to the index of $J_i$,
which we may write as in Eq. \ref{equation-index-form-alternative-formula} to
obtain
\begin{align*}
\frac{1}{2}E''(0)&=-\int_0^\ell\left<J,J''-R_{\gamma'}J\right>
\approx\int_0^\ell \sin(\pi t)(1+K(W_i,\gamma'(t))
\end{align*}
Summing over all indices $i$ we arrive at the Ricci curvature, which must be
greater or equal than $1/r^2$ by hypothesis. In comparing with the length of the
curve $\ell>\pi r$, we find that the integrand must be
negative, which forces one of the indices to be negative.
\end{proof}

\begin{lemma}
\label{lemma-Bonnet-Myers-implies-compact-universal-covering-and-finite-pi1}
\cite{doc}, Chapter IX, Corollary 3.2. Let $M$ be a complete Riemannian manifold
with $\text{Ric}\geq \delta>0$. Then the universal covering of $M$ is compact
and has finite fundamental group.
\end{lemma}

\begin{proof}
The universal cover with the pullback metric satisfies the same curvature
hypothesis as $M$, making it compact by Bonnet-Myers Theorem
\ref{theorem-Bonnet-Myers}. Then the leafs of the action of deck transformations
must be finite, so that $\pi_1(M)=\text{Deck}(\pi)$ is finite.
\end{proof}


\section{Weinstein's theorem}
\label{section-Weinstein-theorem}

\begin{theorem}[Weinstein]
\label{theorem-Weinstein}
Let $M^n$ be oriented, compact, $K>0$ and $f \in \text{Iso}(M)$. Suppose that 
if $n$ is even, $f$ preserves orientation, and if $n$ is odd, $f$ reverts
orientation. Then $f$ has a fixed point.
\end{theorem}

\begin{proof}
Suponha que $f$ não tem pontos mínimos. Defina uma função
\begin{align*}
	g: M &\longrightarrow \mathbb{R} \\
	g(q) &=d(q,f(q)) 
\end{align*}
Como $M$ é compacta e essa função é contínua, possui um ponto mínimo que
chamamos de $p$. 
\end{proof}

\section{Synge's theorem}
\label{section-Synge-theorem}

\begin{theorem}[Synge]
\label{theorem-Synge}
Let $M^n$ be compact and $K>0$. Then
\begin{enumerate}
\item If $n$ is even,
$$
\pi_1(M)=\begin{cases}
	1\qquad & \text{if $M$ is orientable}\\
	\mathbb{Z}_2\qquad & \text{if $M$ is not orientable}
\end{cases}
$$
\item If $n$ is odd, then $M^n$ is orientable.
\end{enumerate}
\end{theorem}

\begin{proof}
First suppose $n$ is even and $M$ is orientable. Since $M$ is compact we must
have that $K\geq \delta>0$. Now consider the universal cover $\tilde{M}$. Then
it also satisfies the curvature bound (with the pullback metric)
, making it compact as well by Bonnet-Myers \ref{theorem-Bonnet-Myers}. 
Now pick a deck transformation, which must be
orientation-preserving if we choose the orientation on $\tilde{M}$ (simply
connected is always orientable) making the projection orientation-preserving.
Then $f$ must have a fixed point by Weinstein's theorem \ref{theorem-Weinstein},
which applies since we have shown that $\tilde{M}$ is compact. 
However, deck transformations cannot have fixes points unless they are the
 identity: this is a consequence of the ``Unique lifting property", which a 
slightly more general statement than the uniqueness of curve lifts that we 
used for Klinenberg's lemma exercise \ref{exercise-l7-1}. 
More exactly (cf. \ref{proposition-unique-lifting-property}): 
given a covering space and to maps from any space into the base, these coincide
 if they agree at one point only.
\end{proof}

\section{Index lemma}
\label{subsection-index-lemma}

\section{Rauch's comparison theorem}
\label{section-Rauch}

We motivate Rauch's comparison theorem with a result from ordinary
differential equations.

\begin{theorem}[Sturm]
\label{theorem-Sturm}
Suppose $f,\tilde{f},K,\tilde{K}:[0,a]\to \mathbb{R}$ are smooth functions
satisfying the following conditions. $\tilde{f} > 0$ in $(0,a]$, $f(0)=\tilde{f}(0)=0$,
$f'(0)=\tilde{f}'(0)$ and
$$
f''+Kf=0,\qquad \tilde{f}''+\tilde{K}\tilde{f}=0.
$$
If $K\leq \tilde{K}$, then $f/\tilde{f}$ is non decreasing. Moreover, if there
is any point $t \in [0,a]$ where $f(t)=\tilde{f}(t)$, then $f \equiv \tilde{f}$
and $K \equiv \tilde{K}$.
\end{theorem}

\begin{proof}
Since we want to show $f/\tilde{f}$ is non decreasing we are interested in
computing the sign of the derivative of this quotient. We thus calculate 
$f'\tilde{f}-f \tilde{f}'$. We only have information about second derivatives,
so we differentiate to obtain
$$
(f'\tilde{f}-f \tilde{f}')'=(\tilde{K}-K)f\tilde{f}
$$
We may integrate to find that 
\begin{equation}
\label{equation-Sturm}
f'\tilde{f}-f \tilde{f}'=\int_0^t (f'\tilde{f}-f \tilde{f}')'
=\int_0^t (\tilde{K}-K)f\tilde{f}
\end{equation}
If we show that $f\geq 0$ we get our result. So suppose that $t_0$ is the
smallest number $t_0 \in (0,a]$ where $f$ is zero. Then $f'(t_0)<0$ since before
that $f$ is positive, which in turn is because $f'(0)=\tilde{f}'(0)$ and
$\tilde{f}$ is strictly positive. But this contradicts Equation
\ref{equation-Sturm} since the integrand was positive up to $t_0$.

Now suppose that $f(t_0)=\tilde{f}(t_0)$ for some $t_0 \in (0,a]$. Then Equation
\ref{equation-Sturm} vanishes at $t_0$, and we conclude that $K \equiv
\tilde{K}$ in $[0,t_0]$. Somehow this implies that $f$ also equals $\tilde{f}$…
\end{proof}

\begin{theorem}[Rauch]
\label{theorem-Rauch}
Let $\gamma$ be a geodesic of $M$ and $\tilde{\gamma}$ a geodesic of
$\tilde{M}$. Suppose that $J \in \mathfrak{X}^J_\gamma$ and $\tilde{J}\in
\mathfrak{X}^J_{\tilde{\gamma}}$ satisfy the following conditions. $J(0)=0$,
$\tilde{J}(0)=0$, $|J'(0)|=|\tilde{J}'(0)|$ and 
$\left<J,\gamma'\right>=\left<\tilde{J},\tilde{\gamma}'\right>$. Also,
$$
J''+K J=0,\qquad \tilde{J}''+\tilde{K}\tilde{J}=0
$$
If $\tilde{\gamma}$ has no conjugate points and $\tilde{K}\geq K$, then 
$|J|/|\tilde{J}|$ is nondecreasing. Moreover, if $|J|$ and $|\tilde{J}|$
coincide in one point, then $|J|\equiv |\tilde{J}|$ and $K \equiv \tilde{K}$.
\end{theorem}

\begin{proof}
Let $f:=|J|^2$ and $\tilde{f}:=|\tilde{J}|^2$. I think the main difference with
Sturm is that the condition on Jacobi fields is not a condition on the
derivatives of the functions $f$ and $\tilde{f}$ we just defined. Note that
$$
f'=2\left<J',J\right>,\qquad J''=2(\left<J',J'\right>+\left<J'',J\right>).
$$
While we may substitute $-KJ$ on the second bracket, we still have the first
one. Perhaps this is the reason to define $U(t):=J(t)/|J(r)|$ and
$\tilde{U}(t):=\tilde{J}(t)/|\tilde{J}(r)|$. But wait, what's that $r$? There's
a trick in this proof.

Notice that $U$ and $\tilde{U}$ are Jacobi fields and $|U(r)|=|\tilde{U}(r)|$.
\end{proof}

\begin{proposition}
\label{proposition-contraction}
Sejam $p \in M$, $\tilde{p} \in \tilde{M}$ e $i:T_pM \to T_{\tilde{p}}\tilde{M}$
uma isometria. Então $f:=\text{exp}^{-1}_p \circ i \circ
\text{exp}_{\tilde{p}}$ é uma contração métrica, i.e. $|f_*w|\leq |w|$
para todo $w \in T_pM$.

Mas ainda, se $\text{exp}_p^{-1}$ está definida numa bola totalmente
convexa, $f$ é uma contração métrica.
\end{proposition}

\begin{proof}
Write an abitrary $w\in T_p M$ as a Jacobi variational field of the variation 
$f(s,t)=\text{exp}_p(t(v+sw)$. The resulting field on $\tilde{M}$ is a
Jacobi field with respect to $iw$ using that $i$ is an isometry. The conditions
of Rauch theorem are satisfied and the desired inequality is obtained.

For the metric result we just integrate along a curve realising the distance and
compute.
\end{proof}

\section{Moore's theorem}
\label{section-Moore}

\begin{theorem}[Moore]
\label{theorem-Moore}
Let $M^n \subset \tilde{M}^{n+p}$ be a compact submanifold of a Hadamard
manifold $\tilde{M}$. If $K\leq \tilde{K}$, then $p\geq n$.
\end{theorem}

\section{Morse index theorem}
\label{section-morse-index}

\begin{definition}[Index of index form]
\label{definition-index-index-form}
$$
i(I_a):=\text{sup}\{\dim L \subset \mathfrak{X}_\gamma:I_a |_{L \times L}<0\}
$$
\end{definition}

For the proof of Morse Index Theorem \ref{theorem-Morse-index} we will use the
following objects
\begin{equation}
\label{equation-index-theorem-spaces}
\begin{aligned}
\mathcal{V}_a&=\{V \in \mathfrak{X}_\gamma:\text{d.p.p., }V(0)=0, V(a)=0\}\\
\mathcal{V}_t^+&:=\{V \in \mathcal{V}_t:V(t_i)=0\}\\
\mathcal{V}_t^-& :=\{V \in \mathcal{V}_t:V|_{[t_j,t_{j+1}]}\in
\mathfrak{X}_{\gamma|_{[t_j,t_{j+1}]}}^J\}
\end{aligned}
\end{equation}

\begin{proposition}
\label{proposition-kernel-of-index-form}
\cite{doc}, Chapter X, Proposition 2.3. The kernel of the index form i
$$
\Ker I_t:=\{V \in \mathcal{V}_t:I_t(V,W)=0\forall W\in\mathcal{V}_t\}=
V_t \cap \mathfrak{X}_\gamma^J 
$$
that is, it is composed of smooth Jacobi fields along $\gamma$.
\end{proposition}

\begin{proof}[Sketch of proof]
The point here is to show that if $V \in \Ker$ then it is (1) a Jacobi
 field in each interval where $V$ is smooth, and (2) smooth in the whole
 interval.

We use the fact that $V$ is in the kernel to show that (1) the norm of 
$V-R_{\gamma'}V$ is zero, and (2) the left and right derivatives of $V$
coincide. The first part involves multiplying by a positive function $f$ that
vanishes at the singular points; this will make the ``singular part" of the
index form, i.e. the sum, vanish, so that we get only with the integral part,
which is the norm of the Jacobi equation, which will vanish since we are in the
kernel. The second part I'm not sure---why do we need to show that $V$ is
differentiable, or $C^2$ (since it's the solution of an ODE).
\end{proof}

\begin{lemma}
\label{lemma-index-form-is-degenerate-iff-conjugate-points}
\cite{doc}, Chapter X, Corollary 2.4. The index form is degenerate if and only 
if the initial and final points of $\gamma$ are conjugate. In this case, the 
nullity, i.e. dimension of $\Ker I_a$ coincides with the nullity of $\gamma(a)$ 
as a conjugate point of $\gamma(p)$, that is, the dimension of the space of 
Jacobi fields vanishing at the endpoints.
\end{lemma}

\begin{proof}
Just by Proposition \ref{proposition-kernel-of-index-form} and by the fact that
$\mathcal{V}_a$ is defined as the smooth vector fields vanishing in the
endpoints.
\end{proof}

\begin{proposition}
\label{proposition-V-is-direct-sum}
\cite{doc}, Chapter XI, Proposition 2.5. 
$\mathcal{V}_a=\mathcal{V}^+ \oplus \mathcal{V}^-$.
\end{proposition}

\begin{proof}
Assume that the partition $0<t_1<\ldots<t_k=a$ of $[0,a]$ is such that every
segment $\gamma|_{[t_{i-1},t_i]}$ is contained in a totally geodesic
neighbourhood, so that no such segment can have conjugate points.

Let $V \in \mathcal{V}_a$ and pick a field $J \in \mathcal{V}^-$ defined as the
piecewise Jacobi field with pairwise boundary conditions $J(t_j)=V(t_j)$. The
assumtion that $\gamma|_{[t_{i-1},t_i]}$ has no conjugate points ensures the
uniqueness of such $J$. Then $V-J\in\mathcal{V}^+$ and we obtain the result.
\end{proof}

\begin{theorem}[Morse Index]
\label{theorem-morse-index}
O índice $i(I_t)$ é finito e igual ao número de pontos conjugados em $[0,t)$ ao
longo de $\gamma$.
\end{theorem}

\begin{proof}
\begin{enumerate}
\item Define
\begin{align*}
\mathcal{V}_a&=\{V \in \mathfrak{X}_\gamma:\text{d.p.p. }V(0)=0, V(a)=0\}\\
\mathcal{V}_t^+&:=\{V \in \mathcal{V}_t:V(t_i)=0\}\\
\mathcal{V}_t^-& :=\{V \in \mathcal{V}_t:V|_{[t_j,t_{j+1}]}\in
\mathfrak{X}_{\gamma|_{[t_j,t_{j+1}]}}^J\}
\end{align*}
\item Note que
$$
\mathcal{V}_t=\mathcal{V}_t^+ \oplus  \mathcal{V}_t^-
$$
\item Note que 
$$
\mathcal{V}_t^-=\bigoplus_{j=1}^{i-1}T_{\gamma(t_j)}M
$$
porque para qualquer escolha de vetores
 $v_1 \in T_{\gamma(t_1)}M,\ldots,v_{i-1}\in T_{\gamma(t_{i-1})}M$ existe
 um único campo $J \in \mathcal{V}^-$ tal que $J(t_j)=v_j$, por ser a solução
 da EDO com condições de contorno.
\item Usamos Proposition \ref{proposition-kernel-of-index-form}.
\item …
\end{enumerate}
\end{proof}



\begin{theorem}[Jacobi's theorem]
\label{theorem-Jacobi-theorem}
Se $\gamma$ é minimizante, então não pode ter pontos conjugados. More generally,
if $\gamma:I \to M$ is a geodesic and $ \gamma(a)$ is conjugate to $\gamma(0)$
through $\gamma$ then for every $\delta>0$, $I_{a+\delta}\not \geq 0$.
\end{theorem}

\begin{proof}
Se $\gamma$ é minimizante, a segunda fórmula da variação é positiva para todo
campo em $\mathcal{V}=\{V\in \mathfrak{X}_\gamma:V \text{ d.p.p. },
V(a)=0,V(b)=0\}$. Ou seja, a forma do índice é uma forma bilinear positiva, e
portanto tem signatura zero, e pelo Teorema do Índice de Morse 
\ref{theorem-Morse-index} o número de pontos conjugados é zero.
\end{proof}

It is possible to prove this theorem without using Morse Index Theorem
\ref{theorem-Morse-index} by constructing a variation where the index form is
negative, see Exercise \ref{exercise-l7-4}.

\section{Cut locus}
\label{section-cut-locus}

\begin{definition}
\label{definition-cut-point}
O {\it cut point} de  $p$ ao longo de $\gamma$ é $\gamma(\rho(\gamma))$ 
onde
$$
\rho(\gamma)=\text{sup}\{t:\gamma|_{[0,t]}\text{ é minimizante}\}
$$

\end{definition}

\begin{proposition}
\label{proposition-cut-point-characterization}
\cite{doc} Chapter XIII, Proposition 2.2. Let $\gamma$ be a minimizing geodesic joining $p$ and $q$. Then $q$ is the cut
point of $p$ if and only if either of the following conditions hold:
\begin{enumerate}
\item $q$ is the first conjugate point of $p$ along $\gamma$.
\item There is another, distinct, minimizing geodesic $\sigma$ joining $p$ and
 $q$.
\end{enumerate}
\end{proposition}

\begin{proof}[Sketch of proof]
($\implies$) Since $q:=\gamma(r)$ is the cut point of $p$, we know that for every point
$\gamma(r+\varepsilon)$ along $\gamma$ after $q$ there must be some other geodesic
$\gamma_\varepsilon$ which realizes the distance between $p$ and
$\gamma(r+\varepsilon)$

Taking the limit as $\varepsilon \to 0$ we obtain another geodesic
(because we are taking limit inside $S^n \subset T_pM$, so we get convergence;
and because ``things will behave well under the limit")
which puts us in the second condition of the theorem,
unless the limit geodesic is the same as  $\gamma$ setwise.

Now there's a critical observation:
the two geodesics $\gamma$ and $\gamma_\varepsilon$ meet at the point 
 $\gamma(r+\varepsilon)$, so you can write
 $\gamma_\varepsilon(r+\delta(\varepsilon))=\gamma(r+\varepsilon)$
 for some function $\delta(\varepsilon)$ (that could be negative, doesn't
 matter).
 But since $\gamma_\varepsilon$ is the minimizing
 geodesic from $\gamma(0)$ to this point, then
 $|\delta(\varepsilon)|<\varepsilon$.

Right, the point is that defining
$$
u_\varepsilon:=(r+\varepsilon)v,\qquad
u'_\varepsilon:=r+\delta(\varepsilon))v_\varepsilon
$$
we see that $\text{exp}_p$ is not injective, since these two vectors
hit that point.

This being true for every $\varepsilon>0$, there is no neighbourhood of $vr$ 
where $\text{exp}_p$ is injective---much less a diffeomorphism.
So it cannot be a local diffeomorphism, and we know that critical points of
$\text{exp}_p$ are conjugate points of $p$. (Remember: if
$\text{exp}_p$ has a critical point, its differential has kernel,
and the map $d_{tv}\text{exp}_pw$, for $w$ in the kernel, gives 
a Jacobi field. See Proposition \ref{proposition-critical-points-exp})
\end{proof}

\begin{exercise}
\label{exercise-diametros}
Calcule o diâmetro de $S^2$, $\mathbb{T}^2$, $\mathbb{R}P^{2}$.
\end{exercise}

\begin{proof}[Solution]
O diâmetro de $S^2$ pode ser calculado via o teorema de Bonnet Myers
\ref{theorem-bonnet-myers}: nenhuma geodésica é minimizante depois de atingir
comprimento $\pi r$, e temos uma geodésica que atinge esse comprimento: qualquer
uma!

O diâmetro de $\mathbb{T}^2$ é $1$. Isso é por simples geometria
euclidiana: é o diâmetro do cubo! Por definição, a métrica de  $\mathbb{T}^2$ é
a induzida pela projeção quociente.

O diâmetro de $\mathbb{R}P^{2}$ é $\pi/2$. Qualquer geodésica que liga dois
pontos a distância $\pi/2$ é minimizante, pois estamos na métrica esférica e
podemos pegar cartas esféricas onde dois pontos a essa distância estão contidos. 
Por outro lado, se tivéssemos dois pontos a distância maior, a geodésica 
esférica que percorre o ponto antípoda ao inicial, chega no ponto final mais 
rapidamente que inicial; portanto geodésicas de comprimento maior que $\pi/2$ 
não são minimizantes.
\end{proof}

\begin{lemma}
\label{lemma-cut-point-reflexive}
If $q$ is the cut point of $p$ through $\gamma$, then $p$ is the cut point of
$p$ through $\gamma$.
\end{lemma}

\begin{lemma}
\label{lemma-not-cut-locus-minimizing}

\end{lemma}

\begin{proposition}
The function
\begin{align*}
	\rho: T^1_p &\longrightarrow [0,+\infty] \\
	v &\longmapsto \text{cut point along }\gamma_v 
\end{align*}
is continuous.
\end{proposition}

\begin{lemma}
\label{lemma-cut-locus-closed}
Because it's the image of the unit sphere in $T_pM$, which is closed.
\end{lemma}

\begin{lemma}
\label{lemma-cut-locus-topology}

\end{lemma}

\begin{lemma}
\label{lemma-outside-cut-locus-exists-minimizing-geodesic}
\cite{doc} Chapter XIII, Corollary 2.8. If $q \in M\setminus C_m(p)$, there exists a
 unique minimizing geodesic joining
$p$ and $q$.
\end{lemma}

\begin{proof}
By Hopf-Rinow \ref{theorem-Hopf-Rinow}, we know that
there exists a minimizing geodesic joining any two points. If there was more
than one such geodesic, by 
Proposition \ref{proposition-cut-point-characterization} 
we get that $q$ is the cut point of  $q$ along either of these geodesics, 
a contradiction.
\end{proof}

The former Lemma \ref{lemma-outside-cut-locus-exists-minimizing-geodesic} 
shows that the exponential map $\text{exp}_p$
 is injective outside the cut locus of $p$. 
This motivates the following definition.

\begin{definition}
\label{definition-injectivity-radius}
The {\it injectivity radius} of a manifold $M$ is
$$
i(M)=\text{inf}\{d(p,C_m(p)):p \in M\}
$$
\end{definition}

\section{Bishop-Gromov Theorem}
\label{section-bishop-gromov}

\begin{theorem}[Bishop-Gromov]
\label{theorem-Bishop-Gromov}
Let $p \in M$ and $0\leq t \leq i(p)=d(p,C_m(p)$. Denote $B_t(p)$ the ball of
raduis $t$ centered in $p$ and $B_{t,k}$ the ball of radius $t$ in a space of
constant curvature $k$.

If $\text{Ric}\geq k$, then $\text{Vol}(B_t(p))/\text{Vol}(B_{t,k})$ is a non
increasing function.

Moreover, if there are numbers  $0<s<r\leq i(p)$ such that
$$
\frac{\text{Vol}(B_s(p))}{\text{Vol}(B_{s,k}}=
\frac{\text{Vol}(B_r(p))}{\text{Vol}(B_{r,k})}
$$
then $B_r(p)=B_{r,k}$ isometrically.
\end{theorem}

\begin{proof}
Use the exponential map a parametrization of the geodesic sphere with radius
$r$ and centre in $p\in M$. (This parametrization works at least locally, but why
have we said in lecture that ``it's a global chart''?) Then
$$
\text{Vol}(S^n_r(p))=\int_{S^n_r(p)}\text{Vol}_{S^n_r(p)}=
\int_{S^n_r(0)}\text{exp}_p^*\text{Vol}_{S^n_r}=
\int_{S^n_r(0)}|\det d_{rv} \text{exp}_p|\text{Vol}_{S^n_r(0)}
$$
This determinant can be given by a basis of tangent vectors to $S_r(0)$, each of
which yields a Jacobi field via Proposition 
\ref{proposition-everyday-jacobi-field}. These Jacobi fields are the columns 
of the matrix $\mathbb{J}$, which is the Jacobian matrix of the exponential we 
are interested in computing.

Now we want to differentiate this with respect to $r$. By Exercise
\ref{exercise-derivative-of-determinant}, we know that the derivative of the
determinant of a one-parameter family of invertible matrices $\mathbb{J}(r)$ is
given by $\det \mathbb{J}(r)\text{tr}(\mathbb{J}^{-1}(r)\mathbb{J}'(r))$.

By the discussion in Section \ref{section-Riccati-equation}, we know that
$AJ=J'$ where $A$ is the shape operator with respect to the unit normal
$\gamma'(r)$ and $J$ is a Jacobi field defined by
\ref{proposition-everyday-jacobi-field}.  This gives $A=J'J^{-1}$. These are the
diagonal entries of the matrix $\mathbb{J}$, so that its trace, which depends
only on the diagonal by Exercise 
\ref{exercise-trace-is-independent-of-coordinates}, is precisely $H/(n-1)$ where
$H$ satisfying this equation is defined as the mean curvature.

This data translates to the following equation
\begin{equation}
\label{equation-Bishop-Gromov-proof1}
\det\mathbb{J}'=\det\mathbb{J}H
\end{equation}
\begin{remark}[Pregunta]
\label{remark-pregunta}
Then it is argued that $H$ satisfies the following Riccati equation:
$$
H'+H^2+\mathcal{R}=0
$$
where $\mathcal{R}:=\text{Ric}(\gamma')+|A_0|^2$. I don't understand what is 
$A_0$. Why does this hold? I think
it's the key to take Eq. \ref{equation-Bishop-Gromov-proof1} to an equation of
the kind 
\begin{equation}
\label{equation-Sturm-type1}
\det\mathbb{J}''+\mathcal{R}\det\mathbb{J}=0
\end{equation}
so that we can apply Sturm. 
\end{remark}
\bigskip
Notice that in the case of constant curvature $k$, Eq. 
\ref{equation-Bishop-Gromov-proof1} becomes
\begin{equation}
\label{equation-Bishop-Gromov-proof2}
\det\overline{\mathbb{J}}'=\det\overline{\mathbb{J}}k
\end{equation}
which in turn should yield an equation of Sturm type, as in Remark
\ref{remark-pregunta}. Namely,
\begin{equation}
\label{equation-Sturm-type2}
\det\overline{\mathbb{J}}''+k\det\overline{\mathbb{J}}=0
\end{equation}
Now we want to compare the volume of the sphere in $M$ with the volume of a
sphere of the same raduis in the space of constant curvature 
$\mathbb{Q}_k^{n+1}$:
\begin{align*}
\frac{\text{Vol}(S_r^{n}(p))}{\text{Vol}(S^{n}_{r,k})}&=
\frac{\int_{S^n(0)}\det \mathbb{J}}{\int_{S^n(0)}\det\overline{\mathbb{J}}}
=\frac{1}{\text{Vol}(S^n(0))}
\int_{S^n(0)}\frac{\det\mathbb{J}}{\det\overline{\mathbb{J}}}
\end{align*}
By Sturm Theorem \ref{theorem-Sturm}, which we may apply since both 
$\det\mathbb{J}$ and $\det\overline{\mathbb{J}}$ satisfy Eqs.
 \ref{equation-Sturm-type1} and \ref{equation-Sturm-type2}, and moreover 
they both vanish at $r=0$ and their derivatives are $1$ at $r=0$ by Exercise
\ref{exercise-Jacobi-tensor-derivative}, and of course, since we are supposing
that $\text{Ric}M\geq k$, we conclude that the integrand is a non {\bf
decreasing} function.
\begin{remark}[Pregunta]
\label{remark-pregunta2}
Según Sturm, el cociente de las funciones en cuestión, en este caso,  
$\frac{\det\mathbb{J}}{\det\overline{\mathbb{J}}}$ debería ser una función no
decreciente, pero el resultado que buscamos es que sea no {\bf creciente}.
\end{remark}
\bigskip
Now we turn to computing the ratios of the volumes of balls of radius $r$. 
\begin{align*}
\frac{\text{Vol}(B_r(p))}{\text{Vol}(B_{r,k})}&=
\frac{\int_{S^n(0)}\int_0^r\det\mathbb{J}(t)dt\text{Vol}S^n(0)}
{\text{Vol}(S^n(0))\int_0^r \det\overline{\mathbb{J}}(t)dt}\\
&=\frac{1}{\text{Vol}(S^n(0))} \frac{\int_0^r}{}
\end{align*}
\begin{remark}[Preguntas]
\label{remark-pregunta3}
I have two questions:
\begin{enumerate}
\item We have said in lecture that we may apply Fubini's theorem by Gauss'
lemma. Why? By Gauss' Lemma the normal vectors are orthogonal to the spheres,
the constructions related to Riccati equation are valid (cf. Section
\ref{section-Riccati-equation}). By why does this allow to use Fubini? 
\item We can pull out the volume in the denominator because the volumes of
spheres in the space of constant curvature somehow does not depend on the radius
 $r$. But how exactly? I think  $\det \overline{\mathbb{J}}(r)$ does depend 
on $r$ (in the space of constant curvature), see Eq. 
\ref{equation-Bishop-Gromov-proof2}.
\end{enumerate}
\end{remark}
Then the equation continues to
\begin{align*}
&=\frac{1}{\text{Vol}(S^n(0))}
\int_{S^n(0)}\frac{\int_0^r \det \mathbb{J}(t)dt \text{Vol}S^n(0)}
{\int_0^r\det\overline{J}dt}\\
&=\frac{1}{\text{Vol}(S^n(0))}
\int_{S^n(0)}\frac{\int_0^r \frac{\det \mathbb{J}(t)}
{\det\overline{\mathbb{J}}(t)}
\det\overline{\mathbb{J}}(t)dt \text{Vol}S^n(0)}
{\int_0^r\det\overline{\mathbb{J}}dt}\\
&=\frac{1}{\text{Vol}(S^n(0))}
\int_{S^n(0)}\frac{\int_0^r \frac{\det \mathbb{J}(t)}
{\det\overline{\mathbb{J}}(t)}d\mu}{\mu[0,r]}\text{Vol}S^n(0)
\end{align*}
where we have introduced a measure 
$\mu:=\det \overline{\mathbb{J}}dt$.

By our discussion before, the integrand is a non decreasing function, almost as
required.

Finally, notice that by the rigidity part on Sturm's Theorem
\ref{theorem-Sturm}, the condition
$$
\frac{\text{Vol}(B_s(p))}{\text{Vol}(B_{s,k}}=
\frac{\text{Vol}(B_r(p))}{\text{Vol}(B_{r,k})}
$$
implies that $\det\mathbb{J}=\det\overline{\mathbb{J}}$ and $k=\mathcal{R}$. 

\begin{remark}[Pregunta]
\label{remark-pregunta4}
I'm not sure exactly how this shows that $B_r(p)=B_{r,k}$ isometrically.
\end{remark}
\end{proof}

\section{Cheng's Theorem}
\label{section-Cheng-theorem}

Cheng's theorem says what happens in case of equality in the conditions of 
Bonnet-Myers Theorem \ref{theorem-Bonnet-Myers}.
\begin{theorem}[Cheng]
\label{theorem-Cheng}
Let $M$ be a complete Riemannian manifold with $\text{Ric}\geq 1$. If 
$\text{diam}M=\pi$ then $M=S^n$ isometrically.
\end{theorem}

\begin{proof}
By Bonnet-Myers Theorem \ref{theorem-Bonnet-Myers}, $M$ is a compact manifold
with diameter $\pi$. This means that $M$ is the closure of any ball of radius
$\pi$. Then, by Bishop-Gromov Theorem \ref{theorem-Bishop-Gromov}, it's enough 
to show the rigidity condition holds for radius $\pi$. That is, we must show
that there is $s<\pi$ such that
$$
\frac{\text{Vol}(B_s(p))}{\text{Vol}(B_{s,1}}= 
\frac{\text{Vol}(B_\pi(p)}{\text{Vol}B_{\pi,1}}
$$
Consider two points $p_1$ and $p_2$ at distance $\pi$. Then the balls of radius
$\pi/2$ with center at these points cannot intersect. This says that
$$
\text{Vol}M\geq B_{\pi/2}(p_1)+B_{\pi/2}(p_2)
$$
But since the volume of $M$ is exactly the volume of $B_\pi(p)$ for any $p$, we
see that
$$
\frac{\text{Vol}(B_\pi(p))}{2}\leq B_{\pi/2}(p_i),\qquad i=1,2
$$
\end{proof}

\section{Toponogov's theorem}
\label{section-Toponogov}

First recall a result we already proved using Rauch's theorem
\ref{theorem-Rauch}:

\begin{theorem}[Toponogov, local version]
\label{theorem-Toponogov-local}
Let $o,p_1,p_2 \in B$ where $B$ is a totally convex ball of $M$. Let $\gamma_1$
and $\gamma_2$ be the normalized geodesics joining $o$ to $p_1$ and to $p_2$.

Suppose $\tilde{o}, \tilde{p}_1,\tilde{p}_2 \in \tilde{B}$ is a triple on 
another manifold $\tilde{M}$ such that the distances from $\tilde{o}$ to
$\tilde{p}_1$ and from $\tilde{o}$ to $\tilde{p}_2$ coincide with those in $M$.

Then for any $t \in [0,\ell(\gamma_1)]$ and $s \in [0,\ell(\sigma)]$,
$$
\tilde{d}(\tilde{\gamma}_1(t),\tilde{\gamma}_2(s))\leq d(\gamma_1(t),\gamma_2(s))
$$
\end{theorem}

Meaning that this theorem is actually an exercise:

\begin{exercise}
\label{exercise-Toponogov-from-Rauch}
Lista 7, Exercício 10. Prove the local version of Toponogov's theorem as a 
corollary of Rauch's Comparison Theorem \ref{theorem-Rauch}.
\end{exercise}

\begin{proof}
This is an application of Proposition \label{proposition-contraction} for $i$ as
given by the condition of mapping $\gamma'_1(0)$ to $\gamma_2'(0)$ and the angle
between them to the corresponding vectors and angle on $\tilde{M}$.
\end{proof}

The global version of that Theorem \ref{theorem-Toponogov-local} is the
following. Notice it coincides with our intuition from comparing plane triangles
with spherical ones; spherical are fatter.

\begin{theorem}[Toponogov, hinge version]
\label{theorem-Toponogov-hinge}
$M$ complete, $K \geq k$. $\gamma_1,\gamma_2$ normalized geodesics with
$\gamma_1(0)=\gamma_2(0)$. Suppose $\gamma_1$ is minimizing.

If $k>0$ suppose additionally that $\ell(\gamma_2) \leq \pi/\sqrt{k}$ (because
that's where conjugate points appear).

Suppose that $\tilde{\gamma}_1,\tilde{\gamma}_2$ is a corresponding hinge in
$\mathbb{Q}_k^2$, that is, $\ell(\gamma_i)=\ell(\tilde{\gamma}_i)$ and
$\angle(\gamma_1'(0),\gamma_2'(0)=
\angle\tilde{\gamma}_1'(0),\tilde{\gamma}_2'(0)$. Then
$$
d(\gamma_1(\ell_1),\gamma_2(\ell_2)) \leq 
\tilde{d}(\tilde{\gamma}_1(\ell_1),\tilde{\gamma}_2(\ell_2))
$$
\end{theorem}

\begin{theorem}[Toponogov, metric version]
\label{theorem-Toponogov-metric}
Let $M$ be a complete manifold and $K \geq k$. Let $o,p_1,p_2 \in M$, with
$p_1\neq  p_2$. Let $\gamma_1$ and $\gamma_2$ be normalized geodesics joining 
$o$ to $p_1$ and to $p_2$. Suppose $\gamma_1$ is minimizing.
Let $\gamma_0$ be a nonconstant geodesic joining $p_1$ to $p_2$ and suppose
$$
\ell(\gamma_0) \leq \ell(\gamma_1) +\ell(\gamma_2)
$$
If $k>0$ suppose additionally that $\ell(\gamma_0) \leq \pi/\sqrt{k}$.

Then there exists a geodesic triangle $\{\tilde{\gamma}_i:i=1,2,0\}$ in
$\mathbb{Q}_k^2$ such that $\ell(\tilde{\gamma}_i) = \ell(\gamma_i)$ and
$$
d(\tilde{o},\tilde{\gamma}_0(t)) \leq  d(o,\gamma_0(t))\qquad \forall t \in
[0,\ell(\gamma_0)]
$$
\end{theorem}

\begin{proof}[Outline of proof]
This is the one we proved in class. Define distance functions
$\rho:=d(\sigma,\cdot)$ and $\tilde{\rho}:=\tilde{d}(\tilde{\sigma},\cdot)$.
\end{proof}

\begin{theorem}[Toponogov, hinge version 2]
\label{theorem-Toponogov-hinge-2}
Put another way (next lecture…) Let $\tilde{\gamma}_1$ and $\tilde{\gamma}_2$ 
be a corresponding angle in $\mathbb{Q}_k^2$. Then
$$
\tilde{d}(\tilde{\gamma}_1(t),\tilde{\gamma}(s)\leq d(\gamma_1(t),\gamma_2(s))
$$
\end{theorem}

\begin{theorem}[Toponogov, angle version]
\label{theorem-Toponogov-angle}
\cite{Cheeger-Ebin} Theorem 2.2.(A). Let $M$ be a complete manifold with 
$K_M\geq H$.

Let $(\gamma_1,\gamma_2,\gamma_3)$ determine a geodesic triangle in $M$. Suppose
$\gamma_1,$ $\gamma_3$ are minimal and if $H>0$, suppose $\ell(\gamma_2)\leq
\frac{\pi}{\sqrt{H}}$. Then $M^H$, the simply connected 2-dimensional space of
constant curvature $H$, there exists a geodesic triangle
$(\tilde{\gamma}_1,\tilde{\gamma}_2,\tilde{\gamma}_3)$ such that
$\ell(\gamma_i)=\ell(\tilde{\gamma}_i)$ and $\tilde{\alpha}_1 \leq\alpha_1$,
$\tilde{\alpha}_3 \leq \tilde{\alpha}_3$.

Except in case $H>0$ and $\ell(\gamma_i)=\frac{\pi}{\sqrt{H}}$ for some $i$, the
triangle in $M^H$ is uniquely determined.
\end{theorem}

\section{Gromov's theorem}
\label{section-Gromov-theorem}

\begin{theorem}[Gromov]
\label{theorem-Gromov}
$M^n$ complete, $K \geq 0$, then $\pi$ admits a set of $3^n$ generators. In
particular, it is finitely generated.
\end{theorem}

\begin{proof}
Consider the universal cover $\tilde{M}$ of $M$ with the pullback metric. Fix a
point $x \in \tilde{M}$ and consider the set
$$
\{f \in \Gamma:d(x,f(x))<r\},\qquad r>0
$$
By \ref{proposition-deck-transformations-act-properly-discontinuous}
every point must have a disjoint neighbourhood … 
\end{proof}

\section{Cartan's Theorem}
\label{section-Cartan-theorem}

\begin{definition}
\label{definition-free-homotopy-set}
The {\it free homotopy set} is the set of homotopy classes of loops based on any
point of $M$.
\end{definition}

\begin{definition}
\label{definition-geodesic-loop}
A {\it geodesic loop} is a geodesic whose starting and ending points are the
same. It may not be differentiable at such point.
\end{definition}

\begin{definition}
\label{definition-closed-geodesic}
A {\it closed geodesic} a smooth geodesic loop.
\end{definition}

\begin{theorem}[Cartan]
\label{theorem-Cartan}
Let $M^n$ be compact, and $\omega \in \hat{\pi}_1M$ nontrivial. Then there is a
closed geodesic $\gamma$ such that $[\gamma]=\omega$.
\end{theorem}

\begin{proof}
Consider a sequence of loops whose lengths converge to the number
$$
\ell:=\text{inf}\{\ell(c):c \in \omega\}>0
$$
Consider a sequence $c_n$ of curves in $\omega$ such that their lengths converge
to $\ell$. Then apply Arzelà-Ascoli Theorem \ref{theorem-Arzela-Ascoli}
 to ensure they converge to some curve $c$ and show this is a geodesic 
in $\omega$.

First suppose that each of the $c_n$ is a piecewise geodesic parametrized by
arclength (why can we do this?). We may then pick the supremum $L$ of all the
lengths since we are assuming that their lengths decrease. Then
$$
d(c_n(t),c_n(s)) \leq \ell \left(c_n|_{[t,s]}\right) \leq  N |t-s|
$$
This says that $\{c_n\}$ is equicontinuous (the same $N$ works for all $n$) as a
family of maps from 
\end{proof}

\section{Preissman's Theorem}
\label{section-Preissman-theorem}

\begin{definition}
\label{definition-translation}
An isometry $f \in \text{Iso}(M)$ is called a {\it translation} if there exists
a geodesic of $M$ such that $f(\gamma)=\gamma$. That is, $f \circ \gamma$ is a
reparametrization of $\gamma$.
\end{definition}

\begin{theorem}[Preissman]
\label{theorem-Preissman}
Let $M^n$ be compact with $K<0$. If $1 \neq  H \subset \pi_1(M)$ is abelian,
then $H=\mathbb{Z}$.
\end{theorem}

\begin{theorem}[Preissman]
\label{theorem-M-compact-K-negative-then-pi1-not-abelian}
If $M$ is compact and $K<0$, then $\pi_{1}(M)$ is not abelian.
\end{theorem}

\section{Byers' Theorem}
\label{section-Byers-theorem}

\begin{lemma}
\label{lemma-deck-fixes-a-geodesic-then-not-compact}
Let $M^n$ be a complete manifold with $K<0$. Suppose that every element of
$\text{Deck}(\pi)$, where $\pi$ is the universal covering, fixes the same
 geodesic $\tilde{\gamma}$. Then $M$ is not compact.
\end{lemma}

\begin{theorem}[Byers]
\label{theorem-Byers}
Let $M$ be compact, $K<0$ and $H$ soluble subgroup of $\pi_1(M)$, $H\neq \{e\}$.
Then $H$ is cyclic infinite. Moreover, $\pi_1(M)$ does not have any cyclic
subgroup of finite index.
\end{theorem}

\section{Cheeger-Gromoll Siplitting Theorem}
\label{section-Chegger-Gromoll-splitting-theorem}

\begin{theorem}[The Strong Maximum Principle]
\label{theorem-strong-maximum-principle}
\cite{pet} Theorem 7.1.7. If $f:(M,g) \to \mathbb{R}$ is continuous and
subharmonic, then $f$ is constant in a neighbourhood of every local maximum. In
particular, if $f$ has a global maximum, then $f$ is constant.
\end{theorem}

\begin{proof}
If $\Delta f>0$ then we can find a support function. Notice that Prof. Florit
shows the theorem for functions satisfying $\Delta f \leq 0$.
\end{proof}

\begin{lemma}
\label{lemma-}

\end{lemma}

The following lemma follows from the lemma above and gives a bound on the
Laplacian of the distance function.

\begin{lemma}[Calabi]
\label{lemma-Calabi}
If $\text{Ric} \geq 0$, for $\rho:=d(p,\cdot)$, it holds that 
$\Delta\rho \leq (n-1)/\rho$ on $M\setminus C_m(p)\cup\{p\}$.
\end{lemma}


\begin{theorem}[Cheeger-Gromoll]
\label{theorem-splitting}
Let $M$ be complete with  $\text{Ric}\geq0$. If $M$ has a line, then $M$ is
isometric to $N\times\mathbb{R}$ for some Riemannian manifold $N$.
\end{theorem}

\section{Exercises}
\label{section-exercises}

\begin{exercise}[Geodésica sem pontos conjugados é localmente minimizante]
\label{exercise-geodesica-sem-pontos-conjugados-e-localmente-minimizante}
$\gamma$ geodésica sem pontos conjugados em $[0,a]$. Então $\gamma$ é localmente
minimizante.
\end{exercise}

\begin{proof}
Temos dois casos:
\begin{enumerate}
\item Se $E''(0)<0$ usamos o teorema do índice. Isto é, não é possível que
 $E''(0)<0$ porque isso implica que o índice da forma do índice é diferente
 de zero, ou seja, existem pontos conjugados.
\item Se $E''(0)=0$,
\begin{enumerate}
\item[(0)] Como $\gamma$ não tem pontos conjugados, então $I$ não tem
nulidade em todo  $\mathcal{V}$.
\item $I_{\mathcal{V}^-}$ não tem índice nem nulidade. Segue do ponto anterior + 
algum passo na prova.
\item $I_{\mathcal{V}^-}$ não tem índice.
\item 2+3 dão que $I_{\mathcal{V}^-}>0$.
\item Logo $I>0$.
\end{enumerate}
\end{enumerate}
\end{proof}

\begin{exercise}[Warped product]
\label{exercise-wraped-product}
Para $(M,g_M)$ e $(N,g_N)$ e $f:M \to \mathbb{R}_+$, definimos o {\it warped
product} como sendo o produto cartesiano $M\times N$ com a métrica 
 $g_M+f^2g_N$.

Mostre que se os fatores são completos, o warped product é completo.
\end{exercise}

\subsection{Highlight exercises}
\label{subsection-highlight-exercises}
\begin{exercise}[Intersecting minimal hypersurfaces]
\label{exercise-intersecting-minimal-hypersurfaces}
$(M^n,g)$ completa $\operatorname{Ric}_M>0$. $A$, $B$ hipersuperfícies mínimas
completas. Então  $A \cap B \neq \emptyset$.
\end{exercise}

\begin{proof}
Suponha que existe uma $\gamma$ mínima com comprimento positivo ligando as 
duas variedades. Tem dois caminhos:
\begin{enumerate}
\item Deixa o campo exponencial ao longo da $\gamma$ e ajusta nos extremos
para fazer ele tangente às variedades. Nesse caso já tem que o temo
$\left<f_{ss},\gamma'\right>|_{0}^a$ se anula. Então so mostrar que $I(V,V)$  é
zero. {\bf Onde se usa que são mínimas? Intentar construir esta?}
\item Começa com uma geodésica tangente e transporta paralelamente um vetor
tangente $e_i$ a  $M_1$. Ele chega tangente a  $M_2$ porque a geodésica é
minimizante (cf exer passado).
\end{enumerate}
\end{proof}

\begin{exercise}
\label{exercise-two-submanifolds}
\cite{doc} Chapter IX, Exercise 5. Sejam $N_1$ e $N_2$ duas subvariedades fechadas e disjuntas de uma variedade
Riemanniana compacta.
\begin{enumerate}
\item Mostre que a distância entre $N_1$ e $N_2$ é realizada por uma geodésica
	$\gamma$ perpendicular a ambas $N_1$ e $N_2$.
\item Mostre que, para qualquer variação ortogonal $h(t,s)$ de $\gamma$, com
$h(0,s) \in N_1$ e $h(\ell,s) \in N_2$, tem-se para a fórmula da segunda
variação a seguinte expressão
$$
\frac{1}{2}E''(0)=I_{\ell}(V,V)+
\left<V(\ell),S_{\gamma'(\ell)}^{(2)}V(\ell)\right>
-\left<V(0),S^{(2)}_{\gamma'(0)}(V(0))\right>
$$
\end{enumerate}
\end{exercise}


\begin{exercise}
\label{exercise-even-dimension-positive-K-compact-is-simply-connected}
Prove que $M^{2n},K_M>0$ compact, então ela é simplesmente conexa.
\end{exercise}

\begin{exercise}
\label{exercise-odd-dimension-positive-K-is-orientable}
$M^{2n+1}$, $K_M>0$ compacta \(\implies\) orientável.
\end{exercise}

\begin{proof}
I want to show that $M^{2n+1}$ is diffeomorphic to its double orientable cover
 $\tilde{M}$. Since $M$ is odd-dimensional, so is $\tilde{M}$. 
\end{proof}

O exercício anterior usa Exercício 6 da lista 6. Ou, equivalentemente,

\begin{exercise}
%\label{exercise-}
Se $M^{2n+1}$ tem $K_M >0$ e $\gamma:S^1 \to M$ inverta orientação, então é
possível encurtar $\gamma$ na sua classe de homotopia.
\end{exercise}

\section{Lista 3}
\label{section-lista-3}

\begin{exercise}[Curvas minimizantes]
\label{exercise-curvas-minimizantes}
\begin{enumerate}
\item Seja $\gamma$ uma curva suave por partes parametrizada por comprimento de
 arco, conectando $p$ a $q$, pontos de uma variedade Riemanniana $(M,g)$.
 Mostre que se $d(p,q)=\ell(\gamma)$, então $\gamma$ é uma geodésica.
 Dizemos que $\gamma$ realiza a distância entre $p$ e $q$.
\item Suponha que $\gamma,\sigma:[0,2]\to M$ são geodésicas distintas
 e satisfazem: $\gamma(0)=\sigma(0):=p$, $\gamma(1)=\sigma(1):=q$, 
$\gamma$ e $\sigma$ realizam a distância entre $p$ e $q$.
 Mostre que $\gamma$ não realiza a distância entre $p$ e $\gamma(1+s)$ 
para nenhum $s>0$.
\end{enumerate}
\end{exercise}

\begin{proof}
\begin{enumerate}
\item Primeiro suponha que $\gamma$ é suave. Podemos ver que trata-se de uma 
geodésica usando seu campo de velocidades como campo variacional na primeira
fórmula da variação:
$$
E'(0)=\int_0^a |\gamma'|=0 \implies \gamma'\equiv 0
$$
Uma geodésica quebrada não pode ser minimizante porque em uma vizinhança 
totalmente normal (cf. \ref{proposition-totally-normal-neighbourhoods}) 
do ponto singular podemos achar uma geodésica radial (suave)
 que acurta a distância entre qualquer ponto antes do ponto singular
 e qualquer ponto depois do ponto singular.
\item Se as duas geodésicas se intersectam não tangencialmente, podemos 
construir um caminho mais curto entre $p$ e $\gamma(1+s)$ ao longo de $\sigma$,
a menos de suavizar a quina perto de $q$. Se as curvas se intersectam
 tangencialmente devem ser a mesma.
\end{enumerate}
\end{proof}

\section{Lista 5}
\label{section-lista-5}

\begin{exercise}
\label{exercise-l5-5}
\cite{doc} Capítulo XII, Exercício 6. Uma geodésica $\gamma:[0,\infty)\to M$ em
uma variedade Riemannana $M$ é um {\it raio partindo de $\gamma(0)$} se ela é
minimizante entre $\gamma(0)$ e $\gamma(s)$ para todo  $s \in (0,\infty)$.
Admita que $M$ é completa, não-compact, e seja $p \in M$. Mostre que $M$ contém
um raio partindo de $p$.
\end{exercise}

\begin{proof}
Como $M$ é compacta e completa, ela não pode ser limitada. Então
existe uma sequência de pontos $p_i$  tal que $\lim_{i \to \infty}
d(p_i,p)=\infty$. Para cada ponto podemos pegar uma geodésica minimizante
$\gamma_n$ ligando $p$ e $p_n$. Podemos associar a cada geodésica um vetor
unitário $v_i \in S^n \subset T_pM$ tal que $\gamma_i(t)=\text{exp}_p(tv_i)$.

Como $S^n \subset T_pM$ é compacta existe um vetor $v$ limite da sequência
$v_i$. A geodésica $\gamma(t):=\text{exp}_p(tv)$ é um raio, pois por
continuidade de $\text{exp}_p$ e da distância Riemanniana,
$$
d(\gamma(0),\gamma(t))=d\left(p,\text{exp}_p(tv)\right)
=d\left(p,\text{exp}_p\left(t\lim_{i \to \infty} v_i\right)\right)
=\lim_{i \to \infty} d(p,\gamma_i (t))
$$
Pegando $i$ suficientemente grande, teremos que $\gamma_i$ é minimizante entre
$p$ e $\gamma_i(t)$. Isso significa que $d(p,\gamma_i(t))$ está dada como o
comprimento de $\gamma_i$ até esse ponto, e pegando o limite concluímos que o
comprimento de $\gamma$ é a distância entre $\gamma(0)$ e $\gamma(t)$.
\end{proof}

\begin{definition}
\label{definition-line-riemannian-manifold}
Let $(M,g)$ be a Riemannian manifold. A geodesic $\gamma:\mathbb{R} \to M$ is
called a {\it line} if $d(\gamma(t),\gamma(s))=|t-s|$ for all $t,s \in
\mathbb{R}$.
\end{definition}

\begin{exercise}
\label{exercise-l5-6}
Mostre que toda métrica completa em $S^n \times \mathbb{R}$ admite uma linha.
\end{exercise}

\begin{proof}
Considere o conjunto $S^n\times \mathbb{R}\setminus(S^n \times\{0\})$.
Trata-se de um aberto disconexo, e portanto não fechado, nem compacto nem
limitado.  Dentro desse conjunto podemos pegar duas sequências de pontos $p_i$ e
$q_i$, onde a segunda coordenada de $p_i$ tem signo positivo, e a segunda
coordenada de $q_i$ tem  signo negativo para toda $i$. Considere a geodésica
$\gamma_i$ que liga $p_i$ com $q_i$, que deve passar pela esfera $S^n\times
\{0\}$. Então obtemos uma sequência $v_i$ de vetores no fibrado tangente
unitário de $S^n\times\{0\}$ dadas como as velocidades das geodésicas
$\gamma_i$. Como tal fibrado é compacto, achamos um limite $v$ dessa sequência
que por um argumento análogo ao exercício anterior, converge a um vetor cuja
geodésica associada é uma linha. ($\gamma$ é minimizante, e como está
parametrizada por comprimento de arco obtemos a condição dada na Definição
\ref{definition-line-riemannian-manifold}.
\end{proof}

\section{Lista 6}
\label{section-lista-6}

\begin{exercise}
\label{exercise-l6-6}
Seja $M^{2n}$ uma variedade Riemanniana de dimensão para, completa, orientável e
com curvatura seccional $K>0$. Seja $\gamma$ uma geodésica fechada em $M$ de
comprimento $\ell(\gamma)$. Mostre que existem curvas livremente homotópicas a
$\gamma$ em $M$, arbitrariamente próximas de $\gamma$, que possuem comprimento
menor que $\ell(\gamma)$.
\end{exercise}


\section{Lista 7}
\label{section-lista-7}

\begin{exercise}
\label{exercise-minimizing-implies-no-conjugate-points}
Seja $\gamma:[0,a]\to M$ uma geodésica em uma variedade Riemanniana $M$.

 Prove que se $\gamma$ é minimizante, então $\gamma$ não possui pontos
 conjugados em  $(0,a)$. Encontre um exemplo de geodésica $\gamma:[0,a] \to M$ 
sem pontos conjugados que não é minimizante.
\end{exercise}

\begin{proof}
Por contrapositiva, suponha que existem dois pontos conjugados
 ao longo de $\gamma$ e vamos provar que ela não pode ser minimizante. 
Por que não posso simplesmente pegar a variação associada ao campo de 
Jacobi? Essa variação me da duas geodésicas que convergem num ponto,
 e portanto $\gamma$ não pode ser minimizante depois desse ponto. 
Resposta: porque nada me assegura que essas geodésicas 
\end{proof}

\section{Lista 8}
\label{section-lista-8}

\begin{exercise}
\label{exercise-l8-1}
Prop. 2.12 do capítulo XIII, \cite{doc}. Seja $p \in M$. 
Suponha exista um ponto $q \in C_m(p)$ que realiza a distância de $p$ a 
$C_m(p)$. Então:
\begin{enumerate}
\item ou existe uma geodésica 
minimizante $\gamma$ de $p$ a $q$ ao longo da qual $q$ é 
conjugado a $p$,
\item ou existem exatamente duas geodésicas 
minimizantes $\gamma$ e $\sigma$ de $p$ a $q$; além disto,
$\gamma'(\ell)=-\sigma'(\ell)$, 
$\ell=d(p,q)$. 
\end{enumerate}
\end{exercise}

\begin{proof}
No final da secção 1 do Capítulo XIII está especificado que as variedades que se
consideram no capítulo são completas. Portanto podemos supor que existe
uma geodésica minimizante $\gamma$ ligando $p$ e $q$.

Então podemos aplicar a Proposição \ref{proposition-cut-point-characterization}:
 um ponto $q$ é o cut point de $p$ ao longo de uma 
geodésica minimizante $\gamma$ se e somente se
alguma das seguintes condições é verdadeira: 
(a) $q$ é o primeiro ponto conjugado a $p$ ao longo de $\gamma$, ou 
(b) existem duas geodésicas minimizantes ligando $p$ e $q$.

Se (a) é verdadeira, terminamos. Se (b) é verdadeira, temos que existe outra
geodésica minimizante $\sigma$ ligando $p$ e $q$.

{\bf Intento pessoal (não funcionou):}  considere a variação por geodésicas 
$$
f(s,t):=\operatorname{exp}_{\gamma(t)}
(s\operatorname{exp}_{\gamma(t)}^{-1}\sigma(t))
$$
cujo campo de Jacobi 
$$
J(t)=d_{t\text{exp}_{\gamma(t)}^{-1}\sigma(t)}
\text{exp}_p(\text{exp}_{\gamma(t)}^{-1}\sigma(t))
$$
se anula em $t=0,1$. Pensei que esse campo de Jacobi estava bem definido pelo 
Corolário 2.8, Cap. XIII (Lema 
\ref{lemma-outside-cut-locus-exists-minimizing-geodesic}), que assegura
 que $\text{exp}_p$ é injetiva fora do cut locus de $p$. Porém, precisaria
 que a exponencial ao longo de $\gamma$,
 $\text{exp}_{\gamma(t)}$, for injetiva fora do cut locus de $p$. 
Essa condição seria garantida se $d(p,q)=i(M)$, ou seja, se a exponencial
 $\text{exp}_{p'}$ for injetiva na bola de raio $d(p,q)$ para qualquer 
$p' \in M$. Em conclusão: minha variação não está bem definida. 
(Se estivesse, com
isso terminaria o exercício, pois obtemos que o único caso em que $p$ não é
conjugado a $q$ é se $\gamma'(\ell)=-\sigma'(\ell)$, i.e. se o campo de Jacobi 
associado à variação é nulo.) 

\medskip

A variação certa é produzida no \cite{doc} supondo que 
$\gamma'(\ell)\neq -\sigma'(\ell)$; isso implica que existe um vetor
 $V \in T_q M$ tal que 
$\left<\gamma'(\ell),V\right>< 0$ e $\left<\sigma'(\ell),V\right>< 0$.

Uma maneria simples de conferir a existência desse vetor $V$ é olhando para o 
plano gerado por $\gamma'(\ell)$ e $\sigma'(\ell)$ (usando que são linearmente
independentes). O conjunto de vetores $V$ nesse plano satisfazendo 
$\left<\gamma'(\ell),V\right><0$ é um semiespaço, e o mesmo acontece com 
 os vetores satisfazendo $\left<\sigma'(\ell),V\right><0$. Esses dois semiespaços
devem ter interseção não vazia precisamente porque 
$\gamma'(\ell)\neq -\sigma'(\ell)$.

Agora usamos o fato de que $p$ e $q$ não são conjugados para 
achar uma vizinhança do vetor $\ell \gamma'(0)$ onde $\text{exp}_p$ é um
difeomorfismo e assim levantar alguma
curva $r$ que realize o vetor $V$ ao espaço tangente,
 digamos $v:(-\varepsilon,\varepsilon) \to T_q M$.

A variação então está dada por
$$
f(s,t):=\text{exp}_p\left(\frac{t}{\ell}v(s)\right)
$$
Podemos aplicar a primeira fórmula da variação, na qual: o termo com a
integral é zero porque $\gamma$ é uma geodésica; o termo da sumatoria é zero
porque trata-se de uma variação suave; e o primeiro termo dos extremos é zero
porque a variação fixa o ponto de partida. Obtemos que 
$$
E'(0)=\left<V,\gamma'(0)\right><0
$$
Note que Manfredo escreve a equação anterior usando o funcional de distância. 
Isso também é válido: de fato a primeira fórmula da variação pode ser 
deduzida de maneira análoga (feito em sala para variações próprias) 
 para o funcional de distância no caso de geodésicas parametrizadas por 
comprimento de arco (cf. Teo. 6.3 \cite{ler}).

O fato da derivada do funcional de comprimento ser negativa nos diz que para
valores perto de $s=0$ podemos achar curvas com menor comprimento. Mais
precisamente, existe um $s>0$ tal que $\ell(f(s,\cdot))<\ell(\gamma)$.

É claro que o mesmo procedimento genera uma variação $\tilde{f}$ de $\sigma$, e 
podemos supor que a mesma $s>0$ faz $\ell(\tilde{f}(s,\cdot))<\ell(\sigma)$.

Concluímos seguindo o argumento do Professor Manfredo: se 
$\ell(\gamma_s)=\ell(\sigma_s)$, então temos duas geodésicas com o mesmo
comprimento ligando $p$ e $\gamma_s(\ell)=\sigma_s(\ell)$; isso significa 
que existe um ponto $\tilde{t} \in (0,\ell]$ tal que $\gamma_s(\tilde{t})$ é
 o cut point de $p$. Porém, isso contradiz o fato de que $q$ realiza a 
distância entre $p$ e o seu cut locus.

Finalmente, se $\ell(\gamma_s)<\ell(\sigma_s)$, segue que o cut point de $p$
 ao longo de $\sigma_s$ está a distância menor do que $q$, que de novo contradiz
a nossa hipótese.
\end{proof}

\begin{exercise}
\label{exercise-l8-2}
Proposição 2.13, Cap. XIII \cite{doc}. Se a curvatura seccional $K$ de uma
variedade Riemanniana completa $M$ satisfaz
$$
0< K_{\operatorname{min}}\leq K\leq K_{\operatorname{max}},
$$
Então
\begin{enumerate}
\item $i(M) \geq \pi/\sqrt{K_{\operatorname{max}}}$, ou
\item existe uma geodésica fechada $\gamma$ em $M$, cujo comprimento é menor do 
que o de qualquer outra geodésica fechada em $M$, tal que
$$
i(M)=\frac{1}{2}\ell(\gamma)
$$
\end{enumerate}
\end{exercise}

\begin{proof}
Suponha que (1) não é verdadeiro. 

Note que pelo Teorema de Bonnet-Myers 
\ref{theorem-Bonnet-Myers}, $M$ é compacta e portanto $C_m(p)$ é compacto para
todo $p$. Isso nos permite achar dois pontos $p$ e $q$ cuja distância é o raio
de injetividade  $i(M)$.

Então podemos usar o exercício anterior para $p$ e $q$.  Primeiro vamos ver o
que acontece no caso da segunda possibilidade daquele exercício, i.e. que
existam exatamente duas geodésicas ligando $p$ e $q$, digamos $\gamma$ é
$\sigma$, tais que $\gamma'(\ell)=-\sigma'(\ell)$ onde $\ell=d(p,q)$.  Considere
$\gamma * \overline{\sigma}$, onde $*$ é a concatenação de curvas e 
$\overline{\sigma}$ é a curva $\sigma$ percorrida em sentido contrário. É claro
que $\gamma * \overline{\sigma}$ é uma geodésica fechada de comprimento $2
d(p,q)$. Note que por definição  $d(p,q)=\ell=i(M)$, como queríamos.

Essa geodésica fechada tem a distância mínima entre as geodésicas fechadas, já
que se tivéssemos alguma outra com distância menor, podemos pegar um ponto
qualquer nela e o seu cut point ficaria a distância exatamente a metade do
comprimento do laço (pois existem duas geodésicas que chegam nele: cada
metade do laço partindo em direções opostas do ponto inicial escolhido),
contradizendo o fato de que $i(M)=d(p,q)$.

Portanto, basta descartar o primeiro caso do exercício anterior. Para chegar a
 uma contradição, suponha que $p $ é conjugado a $q$, i.e. que existe um campo
 de Jacobi $J$ que se anula em $p$ e em $q$. Podemos usar o teorema de Rauch
\ref{theorem-Rauch} para comparar esse campo com $\tilde{J}$, 
um campo em $S^n_{K_{\text{max}}}$, a esfera de curvatura constante
 $K_{\text{max}}$. Como $K \leq  K_{\text{max}}$,
concluímos que $\tilde{J}$ deve se anular em $\ell=d(p,q)=i(M)$. Absurdo, pois
estamos supondo que (1) não é verdadeiro, i.e. que
 $i(M) < \pi/\sqrt{K_{\text{max}}}$.
\end{proof}

\begin{exercise}
\label{exercise-l8-3}
\cite{doc}, Capítulo XIII, Proposição 3.4. Se a curvatura seccional $K$ de uma
variedade Riemanniana $M^n$, compacta, orientável e de dimensão par, satisfaz 
$0<K \leq 1$, então $i(M) \geq \pi$.
\end{exercise}

\begin{proof}
Para obter uma contradição, suponha que existe um ponto $p \in M$ tal que
 $d(p, C_m(p))<\pi$. Como $M$ é compacta, sabemos que $C_m(p)$ é compacto e
portanto existe uma geodésica $\gamma$ ligando $p$ com $q \in C_m(p)$ e
realizando a distância $d(p,C_m(p))$.

Pelo teorema de Rauch, sabemos que qualquer geodésica não pode ter pontos
conjugados antes de alcançar comprimento $\pi$. Explicitamente, se $J$ é um
campo de Jacobi ao longo de alguma geodésica $\gamma$ tal que $J(0)=0$ e 
$J(\ell(\gamma))=0$, comparando com um campo $\tilde{J}$ em $S^n$ tal que
 $\tilde{J}(0)=0$, $|\tilde{J}'(0)|=|J'(0)|$ e 
$\left<J,\gamma'\right>=\left<\tilde{J},\tilde{\gamma}\right>$, concluímos que
 $|\tilde{J}|\leq |J|$ ao longo de $\gamma$. Como as geodésicas de $S^n$ não
 tem pontos conjugados antes de atingir comprimento $\pi$, concluímos que 
$J$ não pode se anular antes desse ponto.

Isso mostra que, como estamos supondo que $d(p,C_m(p))<\pi$, devemos estar no
caso (2) do Exercício \ref{exercise-l8-1}, i.e. existem exatamente duas
geodésicas $\gamma$ e $\sigma$ ligando $p$ e $q$, e
$\gamma'(\ell)=-\sigma'(\ell)$. Repetindo o procedimento do Exercício anterior
 \ref{exercise-l8-3}, podemos usar essas duas geodésicas para achar uma 
geodésica fechada de comprimento $2i(M)$, cujo comprimento é menor do que o
comprimento de qualquer outra geodésica fechada.

\begin{remark}
\label{remark-curvatura-positiva-implica-maior-que-constante-positiva}
Na prova do teorema de Synge \ref{theorem-Synge}, o Professor
Manfredo afirma que ``Como $M$ é compacta e tem curvatura positiva, $K\geq
\delta>0$"; que não me parece imediato, pois a curvatura seccional não é uma
função definida em $M$. Porém, não preciso mostrar isso para garantir a
existência do loop geodésico (via o exercício anterior); apenas é necessário 
que $M$ seja compacta para garantir a existência do ponto $q$.
\end{remark}

\medskip

Para concluir seguimos a prova de \cite{doc}. Vamos a usar o exercício 6 da
lista 6, (cf. \ref{exercise-l6-6}), onde mostrei que, nestas condições, i.e.
$M$ de dimensão par, completa, orientável e com curvatura seccional positiva,
existem curvas livremente homotópicas a qualquer geodésica fechada $\gamma$, que
possuem comprimento menor que $\ell(\gamma)$. Lembre que essas curvas foram
construídas mediante a função exponencial aplicada a um campo paralelo ao longo
de $\gamma$, de forma que são curvas suaves; vou usar esse fato depois.

A ideia é chegar numa contradição mostrando que existe uma terceira geodésica
ligando $p$ e $q$. Essa geodésica se obtém como o limite de uma sequência de
geodésicas associada à variação dada pelo Exercício 6 da Lista 6.

Seja $c_s$ uma das curvas da variação com comprimento estritamente
menor do que $\ell(\gamma)$. Pegue o ponto inicial $c_s(0):=\tilde{p}_s$ e o 
ponto $\tilde{q}_s$ em $c_s$ tal que $d(\tilde{p}_s,\tilde{q}_s)$ é máxima ao 
longo de $c_s$. Existe uma geodésica $\tilde{\gamma}_s$ que liga 
$\tilde{p}_s$ e $\tilde{q}_s$.

A seguir mostrarei que tomando limite quando $s \to 0$, obteremos uma terceira
geodésica minimizante $\tilde{\gamma}_0$ que liga $p$ e $q$, que não é possível
de novo pelo Exercício \ref{exercise-l8-1}.

De fato, podemos dar explicitamente $\tilde{\gamma}_0$ como sendo $exp_q(tw)$
onde $w$ é o vetor limite das velocidades iniciais de cada $\tilde{\gamma}_s$,
supondo que elas são de tamanho 1 para assegurar a existência do limite por
compacidade do fibrado tangente unitário. Note que a geodésica obtida minimiza a
distância entre $p$ e $q$: a geodésica liga $p$ com  $q$ porque, por um lado, o
limite dos pontos iniciais é $p$ por definição, e, por outro lado, porque os
pontos finais são os pontos que maximizam a distância dentro de cada geodésica.
A geodésica limite é minimizante por continuidade da função distância.

Como cada curva $c_s$ é diferenciável, temos um vetor tangente a ela em cada
ponto $\tilde{q}_s$. Como $\tilde{\gamma}$ é minimizante, quando aplicamos a
fórmula da primeira variação,vemos que $\tilde{\gamma}'_s$ é ortogonal a
$c_s'(\tilde{q}_s)$. Como o a métrica é contínua, concluímos que
$\tilde{\gamma}_0'(q) \perp \gamma'(q)$.
\end{proof}

\begin{exercise}
\label{exercise-l8-11}
Seja $M$ uma variedade Riemanniana completa, de curvatura seccional não
negativa. Sejam $\gamma,\sigma:[0,\infty) \to M$ geodésicas tais que
$\gamma(0)=\sigma(0)$. Se $\gamma$ é um raio e
$\angle(\gamma'(0),\sigma'(0))<\frac{1}{2}\pi$, então $\lim_{t \to \infty}
\rho(\sigma(0),\sigma(t))=\infty$ (i.e. $\sigma$ vai para o infinito).
\end{exercise}

\begin{proof}
Seja $t \in \mathbb{R}$. Considere uma geodésica minimizante $\tau_t$ ligando
$\gamma(t)$ com $\sigma(t)$. Defina o angulo 
$\beta_t:=\angle(-\gamma'(t),\tau'_t(0))$ e o número $s_t$ como sendo o tempo em que
$\tau_t$ chega em $\sigma$, i.e. $\tau_t(s_t)=\sigma(t)$. Ver Figura
\ref{figure-toponogov-exercise}.

Como $0 \leq  K$, podemos comparar a ``hinge'',
i.e. a informação do angulo e vetores incidentes no ponto $\gamma(t)$, 
$(-\gamma'(t),\tau_t'(0),\beta)$ com uma hinge no espaço
euclidiano. Obtemos que 
$$
\rho_{\mathbb{R}^n}(\tilde{\gamma}(0),\tilde{\tau}_t(s_t))\leq
\rho(\gamma(0),\tau_t(s_t))=\rho(\sigma(0),\sigma(t))
$$
\begin{figure}[H]
\includegraphics[width=1\textwidth]{figures/toponogov-exercise}
\caption{}
\label{figure-toponogov-exercise}
\end{figure}
Podemos calcular a distância
$\rho_{\mathbb{R}^n}(\tilde{\gamma}(0),\tilde{\tau}_t(s_t))$ usando a lei de
cosenos euclidiana desde que conheçamos $\beta_t$ e $\ell(\tau_t)$:
$$
\rho_{\mathbb{R}^n}(\tilde{\gamma}(0),\tilde{\tau}_t(s_t))^2=
t^2+\ell(\tau_t)^2-2t\ell(\tau_t)\cos \beta_t
$$
Para calcular $\ell(\tau_t)$ podemos usar o Teorema de Toponogov de novo, dessa
vez comparando a hinge $(\gamma'(0), \sigma'(0),\alpha)$ com uma correspondente
no espaço euclidiano:
\begin{figure}[H]
\centering
\includegraphics[width=1\textwidth]{figures/toponogov-exercise2}
\label{figure-toponogov-exercise2}
\end{figure}
Obtemos que
$$
\rho_{\mathbb{R}^n}(\tilde{\gamma}(t),\tilde{\sigma}(t))\leq \ell(\tau_t)
$$
Como $\alpha$ é constante conforme $t$ muda, podemos fixar $\tilde{\gamma}$ e
$\tilde{\sigma}$ para fazer a nossa comparação. Conforme  $t$ avança, nos
movemos a velocidade constante ao longo de $\tilde{\gamma}$ en direção ao
infinito, e embora não sabemos se a distância $\rho(\sigma(0),\sigma(t))$ cresse
 ou diminue, a distância do nosso interesse 
$\rho_{\mathbb{R}^n}(\tilde{\gamma}(t),\tilde{\sigma}(t))$
vai sempre crescendo, e concluimos que diverge ao infinito.

Portanto o problema acaba se conseguimos achar uma cota inferior positiva para
$\beta_t$. Parece que isso é equivalente a provar que a distância entre
$\gamma$ e $\sigma$ está acotada inferiormente por uma constante positiva…

A lei de cosenos do lado Euclideano nos diz que
$$
\rho_{\mathbb{R}^n}(\tilde{\gamma}(t),\tilde{\sigma}(t))^2=
t^2+|\tilde{\sigma}(t)|^2-2t |\tilde{\sigma}(t)|\cos \alpha
\leq t^2+|\tilde{\sigma}(t)|^2-2t |\tilde{\sigma}(t)|
$$
Supondo que $\tilde{\gamma}(0)=\tilde{\sigma}(0)=0 \in \mathbb{R}^n$. 
Parece que para medir essa distância deveríamos conhecer
$|\tilde{\sigma}(t)|=\rho(\sigma(0),\sigma(t))$...
 o que parece levar o problema de volta ao
problema inicial.
\end{proof}



\clearpage
\bibliography{my}
\bibliographystyle{amsalpha}

\end{document}


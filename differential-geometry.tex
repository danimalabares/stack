\input{preamble}

\begin{document}

\title{Differential Geometry}
\maketitle

\phantomsection
\label{section-phantom}

\tableofcontents

\section{Variações da energia}
\label{section-variações-energia}

\begin{theorem}[Bonnet Myers]
\label{theorem-bonnet-myers}
If $M$ has  $\operatorname{Ric} \leq \frac{1}{r^2}$, then
$\operatorname{diam}M\leq \pi r$. In particular, $M$ is compact.
\end{theorem}

\section{Morse index theorem}
\label{section-morse-index}

\subsection{Cut locus}
\label{subsection-cut-locus}

\begin{definition}
\label{definition-injectivity-radius}
The {\it injectivity radius} of a manifold $M$ is
$$
i(M)=\operatorname{inf}\{d(p,C_m(M)):p \in M\}
$$
\end{definition}

\section{Bishop-Gromov Theorem}
\label{section-bishop-gromov}

\begin{theorem}[Bishop-Gromov]
\label{theorem-bishop-gromov}

\end{theorem}

\begin{proof}
Tome a função exponencial como uma carta parametrizando a esfera geodésica de
raio $r$ com centro em $p$. Então
$$
\operatorname{Vol}_{S_r}=\int
\operatorname{Vol}_{S_r}=
\int_{S_r(0)}\operatorname{exp}_p^*\operatorname{Vol}_{S_r}
$$
Portanto, estamos interessados em calcular 
$$
|\det d_{rv} \operatorname{exp}_p|
$$
Avaliando em uma base de vetores tangentes a $S_r(0)$, notamos que cada um deles
nos dá um campo de Jacobi $J$ ao longo de $\gamma$.

Como $\gamma'(r)\perp J$, derivando obtemos $A J=J'$ onde $A$ é o operador de
forma de $S_r$ respeito ao normal interior $\gamma'(r)$ (cf. conta feita antes).

Mmm… no sé: creo que lo que quiero calcular realmente es
$$
\det J
$$
entonces no sé dondé entran ni $A$ ni $J'$… O sea si derivas con respecto a $r$
pues va… pero si no…

\end{proof}

\section{Toponogov's theorem}
\label{section-toponogov}

First recall

\begin{theorem}[Toponogov, local version]
\label{theorem-toponogov,local}
Let $o,p_1,p_2 \in B$ where $B$ is a totally convex ball of $M$. Let $\gamma_1$
and $\gamma_2$ be the normalized geodesics joining $o$ to $p_1$ and to $p_2$.

Suppose $\tilde{o}, \tilde{p}_1,\tilde{p}_2 \in \tilde{B}$ is a triple with the
exact same properties on another manifold $\tilde{M}$.

Then for any $t \in [0,\ell(\gamma_1)]$ and $s \in [0,\ell(\sigma)]$,
$$
\tilde{d}(\tilde{\gamma}_1(t),\tilde{\gamma}_s)\leq d(\gamma_1(t),\gamma_2(s))
$$

\end{theorem}

\begin{exercise}[Lista 7]
\label{exercise-toponogov-from-rauch}
Prove the local version of Toponogov's theorem as a corollary of Rauch's
Comparison Theorem.
\end{exercise}

\begin{proof}

\end{proof}

\begin{theorem}[Toponogov, hinge version]
\label{theorem-toponogov-hinge}
Let $o,p_1,p_2 \in M$ and 
\end{theorem}






\section{Other exercises}
\label{section-other-exercises}

\begin{exercise}
Calcule o diâmetro de $S^2$, $\mathbb{T}^2$, $\mathbb{R}P^{2}$.
\end{exercise}

\begin{proof}[Solution]
O diâmetro de $S^2$ pode ser calculado via o teorema de Bonnet Myers
\ref{theorem-bonnet-myers}: nenhuma geodésica é minimizante depois de atingir
comprimento $\pi r$, e temos uma geodésica que atinge esse comprimento: qualquer
uma!

O diâmetro de $\mathbb{T}^2$ é $1$. Isso é por simples geometria
euclidiana: é o diâmetro do cubo! Por definição, a métrica de  $\mathbb{T}^2$ é
a induzida pela projeção quociente.

O diâmetro de $\mathbb{R}P^{2}$ é $\pi/2$. Qualquer geodésica que liga dois
pontos a distância $\pi/2$ é minimizante, pois estamos na métrica esférica e
podemos pegar cartas esféricas onde dois pontos a essa distância estão contidos. 
Por outro lado, se tivéssemos dois pontos a distância maior, a geodésica 
esférica que percorre o ponto antípoda ao inicial, chega no ponto final mais 
rapidamente que inicial; portanto geodésicas de comprimento maior que $\pi/2$ 
não são minimizantes.
\end{proof}

\section{Lista 8}
\label{section-l8}

\begin{exercise}
Prop. 2.12 do capítulo XIII, \cite{doc}. Seja $p \in M$. 
Suponha exista um ponto $q \in C_m(p)$ que realiza a distância de $p$ a 
$C_m(p)$. Então:
\begin{enumerate}
\item ou existe uma geodésica 
minimizante $\gamma$ de $p$ a $q$ ao longo da qual $q$ é 
conjungado a p,
\item ou existem exatamente duas geodésicas 
minimizantes $\gamma$ e $\sigma$ de $p$ a $q$; além disto, $\gamma'(\ell)$, 
$\ell=d(p,q)$. 
\end{enumerate}
\end{exercise}

\begin{proof}[Prova sem consultar outras referencias]
Foi provado em sala que um ponto $q$ está no cut locus $C_m(p)$ se e somente se
alguma das seguintes condições é verdadeira: (a) $q$ é o primeiro ponto
conjugado a $p$, ou (b) existem duas geodésicas minimizantes ligando $p$ e $q$.

Sponha que (a) não é verdadeira. Considere a variação por geodésicas 
$$
f(s,t):=\operatorname{exp}_{\gamma(t)}
(s\operatorname{exp}_{\gamma(t)}^{-1}\sigma(t))
$$
Note que se $\gamma'(\ell)=-\sigma'(\ell)$ o campo de Jacobi é nulo. Em outro
caso obtemos que $p$ é conjugado a $q$, absurdo.
\end{proof}

\begin{exercise}
Proposição 2.13, Cap. XIII \cite{doc}. Se a curvatura seccional $K$ de uma
variedade Riemanniana completa $M$ satisfaz
$$
0\leq K_{\operatorname{min}}\leq K\leq K_{\operatorname{max}},
$$
Então
\begin{enumerate}
\item $i(M) \geq \pi/\sqrt{K_{\operatorname{max}}}$, ou
\item existe uma geodésica fechada $\gamma$ em $M$, cujo comprimento é menor do 
que o de qualquer outra geodésica fechada em $M$, tal que
$$
i(M)=\frac{1}{2}\ell(\gamma)
$$

\end{enumerate}
\end{exercise}

\bibliography{my}
\bibliographystyle{amsalpha}

\end{document}


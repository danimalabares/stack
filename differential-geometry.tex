\IfFileExists{stacks-project.cls}{%
\documentclass{stacks-project}
}{%
\documentclass{amsart}
}

% For dealing with references we use the comment environment
\usepackage{verbatim}
\newenvironment{reference}{\comment}{\endcomment}
%\newenvironment{reference}{}{}
\newenvironment{slogan}{\comment}{\endcomment}
\newenvironment{history}{\comment}{\endcomment}

% For commutative diagrams we use Xy-pic
\usepackage[all]{xy}

% We use 2cell for 2-commutative diagrams.
\xyoption{2cell}
\UseAllTwocells

% We use multicol for the list of chapters between chapters
\usepackage{multicol}

% This is generally recommended for better output
\usepackage{lmodern}
\usepackage[T1]{fontenc}

% For cross-file-references
\usepackage{xr-hyper}

% Package for hypertext links:
\usepackage{hyperref}

% For any local file, say "hello.tex" you want to link to please
% use \externaldocument[hello-]{hello}
\externaldocument[introduction-]{introduction}
\externaldocument[conventions-]{conventions}
\externaldocument[sets-]{sets}
\externaldocument[categories-]{categories}
\externaldocument[topology-]{topology}
\externaldocument[sheaves-]{sheaves}
\externaldocument[sites-]{sites}
\externaldocument[stacks-]{stacks}
\externaldocument[fields-]{fields}
\externaldocument[algebra-]{algebra}
\externaldocument[brauer-]{brauer}
\externaldocument[homology-]{homology}
\externaldocument[derived-]{derived}
\externaldocument[simplicial-]{simplicial}
\externaldocument[more-algebra-]{more-algebra}
\externaldocument[smoothing-]{smoothing}
\externaldocument[modules-]{modules}
\externaldocument[sites-modules-]{sites-modules}
\externaldocument[injectives-]{injectives}
\externaldocument[cohomology-]{cohomology}
\externaldocument[sites-cohomology-]{sites-cohomology}
\externaldocument[dga-]{dga}
\externaldocument[dpa-]{dpa}
\externaldocument[sdga-]{sdga}
\externaldocument[hypercovering-]{hypercovering}
\externaldocument[schemes-]{schemes}
\externaldocument[constructions-]{constructions}
\externaldocument[properties-]{properties}
\externaldocument[morphisms-]{morphisms}
\externaldocument[coherent-]{coherent}
\externaldocument[divisors-]{divisors}
\externaldocument[limits-]{limits}
\externaldocument[varieties-]{varieties}
\externaldocument[topologies-]{topologies}
\externaldocument[descent-]{descent}
\externaldocument[perfect-]{perfect}
\externaldocument[more-morphisms-]{more-morphisms}
\externaldocument[flat-]{flat}
\externaldocument[groupoids-]{groupoids}
\externaldocument[more-groupoids-]{more-groupoids}
\externaldocument[etale-]{etale}
\externaldocument[chow-]{chow}
\externaldocument[intersection-]{intersection}
\externaldocument[pic-]{pic}
\externaldocument[weil-]{weil}
\externaldocument[adequate-]{adequate}
\externaldocument[dualizing-]{dualizing}
\externaldocument[duality-]{duality}
\externaldocument[discriminant-]{discriminant}
\externaldocument[derham-]{derham}
\externaldocument[local-cohomology-]{local-cohomology}
\externaldocument[algebraization-]{algebraization}
\externaldocument[curves-]{curves}
\externaldocument[resolve-]{resolve}
\externaldocument[models-]{models}
\externaldocument[functors-]{functors}
\externaldocument[equiv-]{equiv}
\externaldocument[pione-]{pione}
\externaldocument[etale-cohomology-]{etale-cohomology}
\externaldocument[proetale-]{proetale}
\externaldocument[relative-cycles-]{relative-cycles}
\externaldocument[more-etale-]{more-etale}
\externaldocument[trace-]{trace}
\externaldocument[crystalline-]{crystalline}
\externaldocument[spaces-]{spaces}
\externaldocument[spaces-properties-]{spaces-properties}
\externaldocument[spaces-morphisms-]{spaces-morphisms}
\externaldocument[decent-spaces-]{decent-spaces}
\externaldocument[spaces-cohomology-]{spaces-cohomology}
\externaldocument[spaces-limits-]{spaces-limits}
\externaldocument[spaces-divisors-]{spaces-divisors}
\externaldocument[spaces-over-fields-]{spaces-over-fields}
\externaldocument[spaces-topologies-]{spaces-topologies}
\externaldocument[spaces-descent-]{spaces-descent}
\externaldocument[spaces-perfect-]{spaces-perfect}
\externaldocument[spaces-more-morphisms-]{spaces-more-morphisms}
\externaldocument[spaces-flat-]{spaces-flat}
\externaldocument[spaces-groupoids-]{spaces-groupoids}
\externaldocument[spaces-more-groupoids-]{spaces-more-groupoids}
\externaldocument[bootstrap-]{bootstrap}
\externaldocument[spaces-pushouts-]{spaces-pushouts}
\externaldocument[spaces-chow-]{spaces-chow}
\externaldocument[groupoids-quotients-]{groupoids-quotients}
\externaldocument[spaces-more-cohomology-]{spaces-more-cohomology}
\externaldocument[spaces-simplicial-]{spaces-simplicial}
\externaldocument[spaces-duality-]{spaces-duality}
\externaldocument[formal-spaces-]{formal-spaces}
\externaldocument[restricted-]{restricted}
\externaldocument[spaces-resolve-]{spaces-resolve}
\externaldocument[formal-defos-]{formal-defos}
\externaldocument[defos-]{defos}
\externaldocument[cotangent-]{cotangent}
\externaldocument[examples-defos-]{examples-defos}
\externaldocument[algebraic-]{algebraic}
\externaldocument[examples-stacks-]{examples-stacks}
\externaldocument[stacks-sheaves-]{stacks-sheaves}
\externaldocument[criteria-]{criteria}
\externaldocument[artin-]{artin}
\externaldocument[quot-]{quot}
\externaldocument[stacks-properties-]{stacks-properties}
\externaldocument[stacks-morphisms-]{stacks-morphisms}
\externaldocument[stacks-limits-]{stacks-limits}
\externaldocument[stacks-cohomology-]{stacks-cohomology}
\externaldocument[stacks-perfect-]{stacks-perfect}
\externaldocument[stacks-introduction-]{stacks-introduction}
\externaldocument[stacks-more-morphisms-]{stacks-more-morphisms}
\externaldocument[stacks-geometry-]{stacks-geometry}
\externaldocument[moduli-]{moduli}
\externaldocument[moduli-curves-]{moduli-curves}
\externaldocument[examples-]{examples}
\externaldocument[exercises-]{exercises}
\externaldocument[guide-]{guide}
\externaldocument[desirables-]{desirables}
\externaldocument[coding-]{coding}
\externaldocument[obsolete-]{obsolete}
\externaldocument[fdl-]{fdl}
\externaldocument[index-]{index}

% Theorem environments.
%
\theoremstyle{plain}
\newtheorem{theorem}[subsection]{Theorem}
\newtheorem{proposition}[subsection]{Proposition}
\newtheorem{lemma}[subsection]{Lemma}

\theoremstyle{definition}
\newtheorem{definition}[subsection]{Definition}
\newtheorem{example}[subsection]{Example}
\newtheorem{exercise}[subsection]{Exercise}
\newtheorem{situation}[subsection]{Situation}

\theoremstyle{remark}
\newtheorem{remark}[subsection]{Remark}
\newtheorem{remarks}[subsection]{Remarks}

\numberwithin{equation}{subsection}

% Macros
%
\def\lim{\mathop{\mathrm{lim}}\nolimits}
\def\colim{\mathop{\mathrm{colim}}\nolimits}
\def\Spec{\mathop{\mathrm{Spec}}}
\def\Hom{\mathop{\mathrm{Hom}}\nolimits}
\def\Ext{\mathop{\mathrm{Ext}}\nolimits}
\def\SheafHom{\mathop{\mathcal{H}\!\mathit{om}}\nolimits}
\def\SheafExt{\mathop{\mathcal{E}\!\mathit{xt}}\nolimits}
\def\Sch{\mathit{Sch}}
\def\Mor{\mathop{\mathrm{Mor}}\nolimits}
\def\Ob{\mathop{\mathrm{Ob}}\nolimits}
\def\Sh{\mathop{\mathit{Sh}}\nolimits}
\def\NL{\mathop{N\!L}\nolimits}
\def\CH{\mathop{\mathrm{CH}}\nolimits}
\def\proetale{{pro\text{-}\acute{e}tale}}
\def\etale{{\acute{e}tale}}
\def\QCoh{\mathit{QCoh}}
\def\Ker{\mathop{\mathrm{Ker}}}
\def\Im{\mathop{\mathrm{Im}}}
\def\Coker{\mathop{\mathrm{Coker}}}
\def\Coim{\mathop{\mathrm{Coim}}}

% Boxtimes
%
\DeclareMathSymbol{\boxtimes}{\mathbin}{AMSa}{"02}

%
% Macros for moduli stacks/spaces
%
\def\QCohstack{\mathcal{QC}\!\mathit{oh}}
\def\Cohstack{\mathcal{C}\!\mathit{oh}}
\def\Spacesstack{\mathcal{S}\!\mathit{paces}}
\def\Quotfunctor{\mathrm{Quot}}
\def\Hilbfunctor{\mathrm{Hilb}}
\def\Curvesstack{\mathcal{C}\!\mathit{urves}}
\def\Polarizedstack{\mathcal{P}\!\mathit{olarized}}
\def\Complexesstack{\mathcal{C}\!\mathit{omplexes}}
% \Pic is the operator that assigns to X its picard group, usage \Pic(X)
% \Picardstack_{X/B} denotes the Picard stack of X over B
% \Picardfunctor_{X/B} denotes the Picard functor of X over B
\def\Pic{\mathop{\mathrm{Pic}}\nolimits}
\def\Picardstack{\mathcal{P}\!\mathit{ic}}
\def\Picardfunctor{\mathrm{Pic}}
\def\Deformationcategory{\mathcal{D}\!\mathit{ef}}

%Dani's additions
\usepackage{graphicx} %figures


\begin{document}

\title{Differential Geometry}
\maketitle

\phantomsection
\label{section-phantom}

\tableofcontents

\section{Variações da energia}
\label{section-variações-energia}

\begin{theorem}[Bonnet Myers]
\label{theorem-bonnet-myers}
If $M$ has  $\operatorname{Ric} \leq \frac{1}{r^2}$, then
$\operatorname{diam}M\leq \pi r$. In particular, $M$ is compact.
\end{theorem}

\section{Morse index theorem}
\label{section-morse-index}

\subsection{Cut locus}
\label{subsection-cut-locus}

\begin{definition}
\label{definition-injectivity-radius}
The {\it injectivity radius} of a manifold $M$ is
$$
i(M)=\operatorname{inf}\{d(p,C_m(M)):p \in M\}
$$
\end{definition}

\section{Bishop-Gromov Theorem}
\label{section-bishop-gromov}

\begin{theorem}[Bishop-Gromov]
\label{theorem-bishop-gromov}

\end{theorem}

\begin{proof}
Tome a função exponencial como uma carta parametrizando a esfera geodésica de
raio $r$ com centro em $p$. Então
$$
\operatorname{Vol}_{S_r}=\int
\operatorname{Vol}_{S_r}=
\int_{S_r(0)}\operatorname{exp}_p^*\operatorname{Vol}_{S_r}
$$
Portanto, estamos interessados em calcular 
$$
|\det d_{rv} \operatorname{exp}_p|
$$
Avaliando em uma base de vetores tangentes a $S_r(0)$, notamos que cada um deles
nos dá um campo de Jacobi $J$ ao longo de $\gamma$.

Como $\gamma'(r)\perp J$, derivando obtemos $A J=J'$ onde $A$ é o operador de
forma de $S_r$ respeito ao normal interior $\gamma'(r)$ (cf. conta feita antes).

Mmm… no sé: creo que lo que quiero calcular realmente es
$$
\det J
$$
entonces no sé dondé entran ni $A$ ni $J'$… O sea si derivas con respecto a $r$
pues va… pero si no…

\end{proof}

\section{Toponogov's theorem}
\label{section-toponogov}

First recall

\begin{theorem}[Toponogov, local version]
\label{theorem-toponogov,local}
Let $o,p_1,p_2 \in B$ where $B$ is a totally convex ball of $M$. Let $\gamma_1$
and $\gamma_2$ be the normalized geodesics joining $o$ to $p_1$ and to $p_2$.

Suppose $\tilde{o}, \tilde{p}_1,\tilde{p}_2 \in \tilde{B}$ is a triple with the
exact same properties on another manifold $\tilde{M}$.

Then for any $t \in [0,\ell(\gamma_1)]$ and $s \in [0,\ell(\sigma)]$,
$$
\tilde{d}(\tilde{\gamma}_1(t),\tilde{\gamma}_s)\leq d(\gamma_1(t),\gamma_2(s))
$$

\end{theorem}

\begin{exercise}[Lista 7]
\label{exercise-toponogov-from-rauch}
Prove the local version of Toponogov's theorem as a corollary of Rauch's
Comparison Theorem.
\end{exercise}

\begin{proof}

\end{proof}

\begin{theorem}[Toponogov, hinge version]
\label{theorem-toponogov-hinge}
Let $o,p_1,p_2 \in M$ and 
\end{theorem}






\section{Other exercises}
\label{section-other-exercises}

\begin{exercise}
Calcule o diâmetro de $S^2$, $\mathbb{T}^2$, $\mathbb{R}P^{2}$.
\end{exercise}

\begin{proof}[Solution]
O diâmetro de $S^2$ pode ser calculado via o teorema de Bonnet Myers
\ref{theorem-bonnet-myers}: nenhuma geodésica é minimizante depois de atingir
comprimento $\pi r$, e temos uma geodésica que atinge esse comprimento: qualquer
uma!

O diâmetro de $\mathbb{T}^2$ é $1$. Isso é por simples geometria
euclidiana: é o diâmetro do cubo! Por definição, a métrica de  $\mathbb{T}^2$ é
a induzida pela projeção quociente.

O diâmetro de $\mathbb{R}P^{2}$ é $\pi/2$. Qualquer geodésica que liga dois
pontos a distância $\pi/2$ é minimizante, pois estamos na métrica esférica e
podemos pegar cartas esféricas onde dois pontos a essa distância estão contidos. 
Por outro lado, se tivéssemos dois pontos a distância maior, a geodésica 
esférica que percorre o ponto antípoda ao inicial, chega no ponto final mais 
rapidamente que inicial; portanto geodésicas de comprimento maior que $\pi/2$ 
não são minimizantes.
\end{proof}

\section{Lista 8}
\label{section-l8}

\begin{exercise}
Prop. 2.12 do capítulo XIII, \cite{doc}. Seja $p \in M$. 
Suponha exista um ponto $q \in C_m(p)$ que realiza a distância de $p$ a 
$C_m(p)$. Então:
\begin{enumerate}
\item ou existe uma geodésica 
minimizante $\gamma$ de $p$ a $q$ ao longo da qual $q$ é 
conjungado a p,
\item ou existem exatamente duas geodésicas 
minimizantes $\gamma$ e $\sigma$ de $p$ a $q$; além disto, $\gamma'(\ell)$, 
$\ell=d(p,q)$. 
\end{enumerate}
\end{exercise}

\begin{proof}[Prova sem consultar outras referencias]
Foi provado em sala que um ponto $q$ está no cut locus $C_m(p)$ se e somente se
alguma das seguintes condições é verdadeira: (a) $q$ é o primeiro ponto
conjugado a $p$, ou (b) existem duas geodésicas minimizantes ligando $p$ e $q$.

Sponha que (a) não é verdadeira. Considere a variação por geodésicas 
$$
f(s,t):=\operatorname{exp}_{\gamma(t)}
(s\operatorname{exp}_{\gamma(t)}^{-1}\sigma(t))
$$
Note que se $\gamma'(\ell)=-\sigma'(\ell)$ o campo de Jacobi é nulo. Em outro
caso obtemos que $p$ é conjugado a $q$, absurdo.
\end{proof}

\begin{exercise}
Proposição 2.13, Cap. XIII \cite{doc}. Se a curvatura seccional $K$ de uma
variedade Riemanniana completa $M$ satisfaz
$$
0\leq K_{\operatorname{min}}\leq K\leq K_{\operatorname{max}},
$$
Então
\begin{enumerate}
\item $i(M) \geq \pi/\sqrt{K_{\operatorname{max}}}$, ou
\item existe uma geodésica fechada $\gamma$ em $M$, cujo comprimento é menor do 
que o de qualquer outra geodésica fechada em $M$, tal que
$$
i(M)=\frac{1}{2}\ell(\gamma)
$$

\end{enumerate}
\end{exercise}

\bibliography{my}
\bibliographystyle{amsalpha}

\end{document}


\input{preamble}

\begin{document}

\title{Complex Geometry}
\maketitle

\phantomsection
\label{section-phantom}
\hfill
\href{http://github.com/danimalabares/stack}{github.com/danimalabares/stack}

\tableofcontents

\section{Complex analysis in several variables}
\label{section-complex-analysis-in-several-variables}

\begin{lemma}
\label{lemma-holomorphic-function-characterization}
\cite{lec}, Theorem 1.21. Let $U\subseteq\mathbb{C}^n$ be open and $f:U\to
\mathbb{C}$. The following are equivalent:
\begin{enumerate}
\item $f$ is holomorphic (i.e. it is continuous and has a complex partial
derivative with respect to each variable at each point of $U$)
\item $f$ is smooth and satisfies the following Cauchy-Riemann equations:
\begin{equation}
\label{equation-Cauchy-Riemann-several-variables}
\frac{\partial u}{\partial x^j}=\frac{\partial v}{\partial y^j},\qquad 
\frac{\partial u}{\partial y^j}=-\frac{\partial v}{\partial x^j}
\end{equation}
where $z^j=x^j+\sqrt{-1}y^j$ and $f(s)=u(z)+\sqrt{-1}v(x)$.
\item For each $p=(p^1,\ldots,p^n)\in U$ there exists a neighbourhood of $p$ in
$U$ on which $f$ is equal to the sum of an absolutely convergent power series of
the form
\begin{equation}
\label{equation-Taylor-series-several-variables}
f(z)=\sum_{i_1,\ldots,i_n}a_{i_1,\ldots,i_n}(z^1-p^1)\ldots(z^n-p^n)
\end{equation}
\end{enumerate}
\end{lemma}

\begin{proof}
I will prove that if $f$ is holomorphic then it has a Taylor series for $n=2$. 
First apply Cauchy integral formula on each variable to obtain
$$
f(z^1,z^2)=\frac{1}{(2\pi\sqrt{-1})^2}
\int_{\substack{|z^1-w^1|=r \\ |z^2-w^2|=r}}
\frac{f(w^1,w^2)}{(w^1-z^1)(w^2-z^2)}dw^1dw^2
$$
Now observe:
\begin{equation}
\label{equation-multivariable-Cauchy}
\frac{1}{w^1-z^1}=\frac{1}{w^1-p^1+p^1-z^1}
=\frac{1}{w^1-p^1}\frac{1}{1-\frac{p^1-z^1}{w^1-p^1}}
\end{equation}
And on the right-hand-side we have a geometric series so that we may write
$$
\frac{1}{w^1-z^1}
=\frac{1}{w^1-p^1}\sum_{k=0}^\infty\left(\frac{p^1-z^1}{w^1-p^1}\right)^k
$$
finally substituting this into (\ref{equation-multivariable-Cauchy}) we may take
the products $(p^1-z^1)^{k_1}(p^2-z^2)^{k_2}$ out of the integral and define the
remaining term as $a_{k_1k_2}$.
\end{proof}

\section{Weierstrass preparation theorem}
\label{section-Weierstrass-preparation-theorem}

\begin{definition}
\label{definition-germ}
A {\it germ} of a function near a point is a function defined on some open 
neighbourhood of the point.
\end{definition}

\begin{definition}
\label{definition-Weierstrass-polynomial}
A {\it Weierstrass polynomial} is a polynomial whose coefficients are
holomorphic functions.
\end{definition}

\begin{theorem}
\label{theorem-Weierstrass-preparation}
If $f:U\subset\mathbb{C}^n\to\mathbb{C}$ is holomorphic and $f$ is not
identically zero in the coordinate axis $z_n:=w$, there
is a unique germ of a monic Weierstrass polynomial $g$ whose coefficients are
holomorphic functions on the first $n-1$ variables 
and a germ of a holomorphic 
function $h$ with $h(0)\neq 0$ such that $f=gh$.
\end{theorem}

\begin{proof}
Consider the function of one complex variable $f(0,\ldots,0,w)$. Since it has a
zero at $0$ and is holomorphic, by Lemma 
\ref{lemma-zeroes-are-isolated-and-have-finite-order} there is a smallest 
integer $m$ such that $f^{(m)}(0)$ is not zero called the order of the zero $0$.
By the Residue theorem $f(z,w)$ as a function of $w$ must have $m$ zeroes in a
disk $\Delta(0,r_n)$. (?)
\end{proof}

\begin{lemma}
\label{lemma-Gauss-UFD}
If $R$ is a UDF, then $R[x]$ is a UFD.
\end{lemma}

\begin{lemma}
\label{lemma-stalk-is-UFD}
The stalk $\mathcal{O}_n:=\mathcal{O}_{\mathbb{C}^n,0}$ is a UFD.
\end{lemma}

\begin{proof}
By induction on $n$. For $n=0$ it is trivial. Suppose $\mathcal{O}_{n-1}$ is a
UFD. Then by Gauss' Lemma \ref{lemma-Gauss}, $\mathcal{O}_{n-1}[w]$ is a UFD
too. Thus we may express any Weierstrass polynomial $g$ as a product of
irreducible elements (uniquely up to multiplication by units).

Let $f\in \mathcal{O}_n$. We want to express $f$ as a product of (unique up to
multiplication by units) of irreducible elements. By Weierstrass Preparation
Theorem \ref{theorem-Weierstrass-preparation} there is a Weierstrass polynomial
$g\in\mathcal{O}_n[w]$ and a holomorphic function not vanishing on $0$ (i.e. a
unit of $\mathcal{O}_n$) such that $f=gh$. By the previous remark $g$ is
factored uniquely up to multiplication by units as $g=g_1\ldots g_m$. This shows
existence of the factorization.

To prove uniqueness suppose that $f=f_1\ldots f_k$ for some irreducible
$f_1,\ldots,f_k\in\mathcal{O}_n$. Since $f$ does not vanish in the $w$ axis,
neither can each $f_i$, so that we may decompose each of them as  $f_i=g_i'h_i$
by Weierstrass Preparation Theorem. Since $f_i$ is irreducible, it follows that
$g_i'$ is irreducible. Then we have that $$ f=gh=\prod g_i'\prod h_i $$ so by
uniqueness in Weierstrass Preparation Theorem we conclude that $g=\prod g_i'$,
and by uniqueness from the fact that  $\mathcal{O}_n[w]$ is a UFD we conclude
that $g$ coincides with $\prod g_i'$ up to multiplication by units.
\end{proof}

\section{Complex manifolds}
\label{section-complex-manifolds}

\begin{definition}
\label{definition-complex-manifold}
A {\it complex manifold} $M$ is a smooth manifold admitting an open cover
$\{U_\alpha\}$ and coordinate maps $\varphi_\alpha:U_\alpha\to\mathbb{C}^n$ such
that $\varphi_\alpha\circ\varphi_\beta^{-1}$ is holomorphic on
$\varphi_\beta(U_\alpha\cap U_\beta)\subset\mathbb{C}^n$ for all $\alpha,\beta$.
\end{definition}

\begin{definition}
\label{definition-holomorphic-function-on-complex-manifold}
A function on an open set $U\subset M$ is {\it holomorphic} if for all $\alpha$,
$f\varphi_\alpha^{-1}$ is holomorphic on 
$\varphi_\alpha(U\cap U_\alpha)\subset\mathbb{C}^n$.
\end{definition}

\begin{lemma}
\label{lemma-holomorphic-function-sheaf}
The sheaf $\mathcal{O}_M$ of holomorphic functions is a sheaf.
\end{lemma}

\section{Analytic varieties}
\label{section-analytic-varieties}

\begin{definition}
\label{definition-analytic-variety}
An {\it analytic variety} is a subset $V$ of an open set $U\subset \mathbb{C}^n$
such that for any $p\in V$ there is a neighbourhood $U'\ni p$ such that $V\cap
U'$ is given as the zero locus of a finite set of holomorphic functions
$f_1,\ldots,f_k$ defined on $U'$.
\end{definition}

\begin{definition}
\label{definition-analytic-hypersurface}
An analytic variety is a {\it hypersurface} if it is given as the vanishing
locus of a single holomorphic function.
\end{definition}

\begin{definition}
\label{definition-irreducible-variety}
An analytic variety $V\subset U\subset \mathbb{C}^n$ is {\it irreducible} if
$V$ cannot be written as the union of two distinct analytic varieties 
$V_1,V_2\subset U$, both distinct to $V$.
\end{definition}

\begin{lemma}
\label{lemma-irreducible-analytic-hypersurface-irreducible-polynomial}
If $V$ is an analytic hypersurface given locally as $V=\{f=0\}$, then $f$ is
irreducible in $\mathcal{O}_p$.
\end{lemma}

\begin{proof}
If $f=gh$ and neither of $g$ and $h$ are units and they are distinct, we could 
express $V$ as the union of two distinct varieties: $V(g)$ and $V(h)$. (If one 
of them, was a unit then the vanishing set would be all of $U$.)
\end{proof}

\section{Analytic subvarieties}
\label{section-analytic-subvarieties}

\begin{definition}
\label{definition-analytic-subvariety}
\cite{gri}. An {\it analytic subvariety} $V$ of a complex manifold $M$ is a
subset given locally as the zeros of a finite collection of holomorphic
functions.
\end{definition}

For a smooth submanifold $M$ of a complex manifold $N$ we have the {\it normal 
short exact sequence}
$$
\xymatrix{
0\ar[r]&TM\ar[r]&TN\ar[r]&T^\perp M\ar[r]&0
}
$$
If $M$ is a singular subscheme we have the {\it ideal short exact sequence}
$$
\xymatrix{
0\ar[r]&\mathcal{I}_M\ar[r]&\mathcal{O}_N\ar[r]&\mathcal{O}_M\ar[r]&0
}
$$
which follows from definition of a closed embedded subscheme.

If $M$ is a singular analytic subvariety we also have an ideal short exact
sequence, this time using the holomorphic function sheaves.

\begin{lemma}
\label{lemma-ideal-sheaf-is-line-bundle-schemes}
The ideal sheaf of an irreducible codimension-1 closed subscheme of a smooth
scheme is a line bundle.
\end{lemma}

\begin{proof}
Since $M$ is codimension-1 irreducible, the ideal sheaf at every affine chart is
a codimension-1 prime ideal. This means that there aren't any nontrivial prime
ideals of $\mathcal{I}_X(\text{Spec}A):=I$. Let $f\in I$.
Since $X$ is smooth, $O_X\text{Spec}A=A$ is a UFD. Then there exists an
irreducible element $g\in I$ such that $gh=f$. The ideal generated by $g$ is
prime and contained in $I$, so that $I$ is principal. This shows that
$\mathcal{I}_M$ is locally principal, i.e. it is a line bundle.
\end{proof}

In the case of complex manifolds and analytic subvarieties, we have

\begin{lemma}
\label{lemma-ideal-sheaf-is-line-bundle-analytic-varieties}
The ideal sheaf of an irreducible codimension-1 analytic subvariety of a smooth
complex manifold is a line bundle.
\end{lemma}

\begin{proof}
By definition, our subvariety $M$ is locally defined by the zeroes of a single 
holomorphic function defined on the ambient manifold $X$. That is,
at every $p\in M$ there is an open set $U\subset X$ such that $U\cap
M=f^{-1}(0)$ for some $f\in\mathcal{O}_X(U)$. Since $U$ is an open set of a
complex manifold, the ring $\mathcal{O}_X(U)$ is isomorphic to the ring
$\mathcal{O}_n$ of holomorphic functions on $\mathbb{C}^n$ in a neighbourhood of
the origin; that's by Definition \ref{definition-holomorphic-function} since a
holomorphic function on a manifold is defined as such if its composition with
the coordinate chart is a holomorphic function on $\mathbb{C}^n$.

By Lemma
\ref{lemma-irreducible-analytic-hypersurface-irreducible-polynomial}, since $M$
is irreducible it follows that $f$ is irreducible and so is the ideal of
functions vanishing on $M$.

By Lemma \ref{lemma-stalk-is-UFD}, we prove identically as in Lemma 
\ref{lemma-ideal-sheaf-is-line-bundle-schemes} that the ideal of functions
vanishing on $M$ is principal.
\end{proof}

\section{Riemann-Roch for curves on surfaces}
\label{section-Riemann-Roch-for-curves-on-surfaces}





\section{Adjunction formula}
\label{section-adjunction-formula}

\begin{lemma}
\label{lemma-adjunction-formula-for-curves-on-surfaces}
Let $S$ be any curve on a surface $X$. Then
\begin{equation}
\label{equation-adjunction-formula-for-curves-on-surfaces}
2p_a(S)-2=(S.S-K_X)
\end{equation}
\end{lemma}

\begin{proof}
If $S$ is nonsingular we may use the normal bundle short exact sequence. If
$S$ is singular we need to use the ideal sheaf short exact sequence, which means
we think of $S$ either as a closed embedded subscheme of $X$ or as an analytic 
subvariety. Taking Euler class we obtain
$$
\chi(\mathcal{O}(-S))+\chi(\mathcal{O}_S)=\chi(\mathcal{O}_X)
$$
Since $p_a(S):=1-\chi(\mathcal{O}_S)$, we obtain that
$p_a(S)=1-\chi(\mathcal{O}_X)-\chi(\mathcal{O}(-S))$. To compute
$\chi(\mathcal{O}(-S))$ we use Riemann-Roch formula
\ref{equation-Riemann-Roch-for-curves-on-surfaces} for curves on surfaces, which
gives
$$
\chi(\mathcal{O}(-S))=\chi(\mathcal{O}_X)
+\frac{(\mathcal{O}(-S).\mathcal{O}(-S)+\omega_X^*)}{2}
$$
Taking duals on both entries of the intersection form and substituting in our
previous expression for $p_a(S)$ we obtain the result.
\end{proof}

\begin{exercise}
\label{exercise-singular-curve-on-K3-has-positive-self-intersection}
Let $S$ be a singular, irreducible complex curve on a K3 surface. Prove that
$(S.S)\geq 0$.
\end{exercise}

\begin{proof}
This is just an application of Eq.
\ref{equation-adjunction-formula-for-curves-on-surfaces}. Since
$K_X=\mathcal{O}_X$ it suffices to show that $p_a(S)$ is not zero. This follows
fact that $S$ is singular, since its arithmetic genus is defined as the genus of
its normalization, which must be strictly positive since there exists at least
one singularity.
\end{proof}

\section{Chow's theorem}
\label{section-chow-theorem}

\begin{proposition}
\label{proposition-slice-charts}
Every codimension-$p$ submanifold of a complex $n$-manifold has slice charts,
i.e. for any $p \in M$ there's a coordinate chart $U \ni p$ such that points of
$M$ have coordinates $(z_1,\ldots,z_n,0,\ldots 0)$.
\end{proposition}

\begin{proof}
This may be taken as the definition of smooth (real or complex) submanifold, or
as a proposition using inverse function theorem (real or complex).
\end{proof}

\begin{proposition}
\label{proposition-stalk-is-PID}
Stalk of holomorphic functions is a PID.
\end{proposition}

\begin{proof}
This can be proved via Weierstrass Preparation theorem
\ref{theorem-Weierstrass-preparation} which implies
 Lemma \ref{lemma-stalks-are-UFDs}, and this in turn implies Lemma 
\ref{lemma-codimension-1-ideals-in-UFD-are-principal}.

Also, it can be shown directly for smooth hypersurfaces
 as in \cite{lec} Lemma 3.38.
\end{proof}

\begin{theorem}[Chow for hypersurfaces]
\label{theorem-Chow-for-hypersurfaces}
Every complex codimension-1 submanifold of $\mathbb{C}P^{n}$ is algebraic.
\end{theorem}

\section{Ampleness}
\label{section-ampleness}

\begin{definition}
\label{definition-base-point}
\begin{reference}
\cite[Definition 2.3.25]{huc}
\end{reference}
Let $L$ be a holomorphic bundle on a complex manifold $X$. A point $x \in X$ is
a {\it base point} of $L$ if $s(x)=0$ for all $s\in H^{0}(X,L)$. 
\end{definition}

\begin{definition}
\label{definition-base-locus}
\begin{reference}
\cite[Definition 2.3.25]{huc}
\end{reference}
Let $L$ be a holomorphic bundle on a complex manifold $X$. 
The {\it base locus} $\text{Bs}(L)$ is the sate of all base points of $L$.
\end{definition}

\begin{proposition}
\label{proposition-canonical-map}
\begin{reference}
\cite[Proposition 2.3.26]{huc}
\end{reference}
Let $L$ be a holomorphic line bundle on a complex manifold $X$ and suppose that
$s_0,\ldots,s_N\in H^{0}(X,L)$ is a basis. Then
\begin{align*}
\varphi_L: X\setminus\text{Bs}(L) &\longrightarrow \mathbb{P}^N \\
x &\longmapsto (s_0(x):\ldots:s_N(x))
\end{align*}
defines a holomorphic map such that $\varphi^*
_L\mathcal{O}_{\mathbb{P}^N}(1)\cong L|_{X\setminus\text{Bs}(L)}$.
\end{proposition}

\begin{definition}
\label{definition-linear-system}
\begin{reference}
\cite[p. 86]{huc}
\end{reference}
The map $\varphi_L$ in Proposition \ref{proposition-canonical-map} is said to be
associated to the {\it complete linear system} $H^{0}(X,L)$ (i.e. the sections
of $L$ are what we call a complete linear system) whereas a subspace of 
$H^{0}(X,L)$ is called a {\it linear system} of $L$.
\end{definition}

\begin{definition}
\label{definition-globally-generated}
\begin{reference}
\cite[p. 86]{huc}
\end{reference}
$L$ is {\it globally generated} by the sections $s_0,\ldots,s_N$ if
$\text{Bs}(L,s_0,\ldots,s_N)=\emptyset$.
\end{definition}

\begin{definition}
\label{definition-ample}
\begin{reference}
\cite[Definition 2.3.28]{huc}
\end{reference}
A line bundle $L$ on a complex manifold is called {\it ample} if for some $k>0$
and some linear system in $H^{0}(X,L^k)$ the associated map $\varphi$ is an
embedding.
\end{definition}

\begin{slogan}
By definition, a compact complex manifold is projective if and only if it admits
an ample line bundle.
\end{slogan}

\begin{exercise}
\label{exercise-L-ample-implies-Lotimes2-globally-generated}
Let $L$ be an ample bundle on a K3 surface $M$. Prove that $L^{\otimes 2}$ is
globally generated (that is, for each $x \in M$ there exists a section $h \in
H^{0}(L^{\otimes 2})$ which does not vanish in $x$).
\end{exercise}

\begin{proof}
Let $D$ be a divisor in the line system $|L|:=\mathbb{P}H^{0}(L)$, which is a
smooth curve by Bertini's theorem \ref{theorem-Bertini}. Restrict the


First we must find a curve on the surface. We expect to know the genus of this
curve. Restrict the line bundle to the curve and find a contradiction if there
was a base point.
\end{proof}

\section{Bertini's theorem}
\label{section-Bertini-theorem}

\begin{slogan}
In \(\mathbb{C}^n\) many complex hypersurfaces can be written as regular level
sets of globally defined holomorphic functions. But in a compact complex
manifold, this is never possible, because all global holomorphic functions are
constants. Instead, we can use sections of line bundles.
\end{slogan}

\section{Serre duality}
\label{section-serre-duality}

\begin{theorem}[Serre duality]
\label{theorem-serre-duality}
\cite{voi} II.5.32. The pairing
$$
H^q(X,\mathcal{E})\otimes
H^{n-q}(X,\mathcal{E}^*\otimes K_X)\to H^{n}(X,K_X)\cong\mathbb{C}
$$
is perfect.
\end{theorem}

So when you put the dual \(\vee\) on one of these you get isomorphism.

Our course version says:
$$
H^{k}(X,\mathcal{L})^{\vee}=H^{n-k}(X,\omega_X \otimes \mathcal{L}^*)
$$
\section{Kodaira Vanishing theorem}
\label{section-Kodaira-vanishing-theorem}

\begin{theorem}[Kodaira]
\label{theorem-Kodaira-vanishing}
\cite{sea}, 21.5.8. Suppose $k$ is a field of characteristic 0, and $X$ is a
smooth projective $k$-variety. Then for any ample invertible sheaf $L$,
$H^{i}(X,K_X \otimes L)=0$ for $i>0$.
\end{theorem}

\begin{proof}
No proof in \cite{sea}.
\end{proof}

\bibliography{my}
\bibliographystyle{amsalpha}

\end{document}


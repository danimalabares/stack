\input{preamble}

\begin{document}

\title{Complex Geometry}
\maketitle

\phantomsection
\label{section-phantom}

\tableofcontents

\section{Chow's theorem}
\label{section-chow-theorem}

\begin{proposition}
\label{proposition-slice-charts}
Every codimension-$p$ submanifold of a complex $n$-manifold has slice charts,
i.e. for any $p \in M$ there's a coordinate chart $U \ni p$ such that points of
$M$ have coordinates $(z_1,\ldots,z_n,0,\ldots 0)$.
\end{proposition}

\begin{proof}
This may be taken as the definition of smooth (real or complex) submanifold, or
as a proposition using inverse function theorem (real or complex).
\end{proof}

\begin{proposition}
\label{proposition-stalk-principal-smooth}
Stalk of holomorphic functions is a PID at smooth points.
\end{proposition}

\begin{proof}
This can be proved via Weierstrass Preparation theorem
\ref{theorem-Weierstrass-preparation} which implies
 Lemma \ref{lemma-stalks-are-UFDs}, and this in turn implies Lemma 
\ref{lemma-codimension-1-ideals-in-UFD-are-principal}.

Also, it can be shown directly for smooth hypersurfaces
 as in \cite{lec} Lemma 3.38.
\end{proof}

\begin{theorem}[Chow for hypersurfaces]
\label{theorem-chow}
Every complex codimension-1 submanifold of $\mathbb{C}P^{n}$ is algebraic.
\end{theorem}

\begin{proof}

\end{proof}

\section{Serre duality}
\label{section-serre-duality}

\begin{theorem}[Serre duality] \label{theorem-serre-duality} \cite{voi}
II.5.32. The pairing
$$
H^q(X,\mathcal{E})\otimes
H^{n-q}(X,\mathcal{E}^*\otimes K_X)\to H^{n}(X,K_X)\cong\mathbb{C}
$$

\end{theorem}

So when you put the dual \(\vee\) on one of these you get isomorphism.

Our course version says:
$$
H^{k}(X,\mathcal{L})^{\vee}=H^{n-k}(X,\omega_X \otimes \mathcal{L}^*)
$$
\bibliography{my}
\bibliographystyle{amsalpha}

\end{document}


\input{preamble}

\begin{document}

\title{Complex Geometry}
\maketitle

\phantomsection
\label{section-phantom}

\tableofcontents

\section{Chow's theorem}
\label{section-chow-theorem}

\begin{proposition}
\label{proposition-slice-charts}
Every codimension-$p$ submanifold of a complex $n$-manifold has slice charts,
i.e. for any $p \in M$ there's a coordinate chart $U \ni p$ such that points of
$M$ have coordinates $(z_1,\ldots,z_n,0,\ldots 0)$.
\end{proposition}

\begin{proof}
This may be taken as the definition of smooth (real or complex) submanifold, or
as a proposition using inverse function theorem (real or complex).
\end{proof}

\begin{proposition}
\label{proposition-stalk-principal-smooth}
Stalks of holomorphic functions principal at smooth points.
\end{proposition}



\begin{theorem}[Chow for hypersurfaces]
\label{theorem-chow}
Every complex codimension-1 submanifold of $\mathbb{C}P^{n}$ is algebraic.
\end{theorem}

\begin{proof}

\end{proof}

\section{Morse index theorem}
\label{section-morse-index}

\subsection{Cut locus}
\label{subsection-cut-locus}

\begin{definition}
\label{definition-injectivity-radius}
The {\it injectivity radius} of a manifold $M$ is
$$
i(M)=\operatorname{inf}\{d(p,C_m(M)):p \in M\}
$$
\end{definition}
\section{Other exercises}
\label{section-other-exercises}

\begin{exercise}
Calcule o diâmetro de $S^2$, $\mathbb{T}^2$, $\mathbb{R}P^{2}$.
\end{exercise}

\begin{proof}[Solution]
O diâmetro de $S^2$ pode ser calculado via o teorema de Bonnet Myers
\ref{theorem-bonnet-myers}: nenhuma geodésica é minimizante depois de atingir
comprimento $\pi r$, e temos uma geodésica que atinge esse comprimento: qualquer
uma!

O diâmetro de $\mathbb{T}^2$ é $1$. Isso é por simples geometria
euclidiana: é o diâmetro do cubo! Por definição, a métrica de  $\mathbb{T}^2$ é
a induzida pela projeção quociente.

O diâmetro de $\mathbb{R}P^{2}$ é $\pi/2$. Qualquer geodésica que liga dois
pontos a distância $\pi/2$ é minimizante, pois estamos na métrica esférica e
podemos pegar cartas esféricas onde dois pontos a essa distância estão contidos. 
Por outro lado, se tivéssemos dois pontos a distância maior, a geodésica 
esférica que percorre o ponto antípoda ao inicial, chega no ponto final mais 
rapidamente que inicial; portanto geodésicas de comprimento maior que $\pi/2$ 
não são minimizantes.
\end{proof}

\section{Lista 8}
\label{section-l8}

\begin{exercise}
Prop. 2.12 do capítulo XIII, \cite{doc}. Seja $p \in M$. Suponha exista um ponto $q \in C_m(p)$ que realiza a distância de $p$ a $C_m(p)$. Então:
\begin{enumerate}
\item ou existe uma geodésica minimizante $\gamma$ de $p$ a $q$ ao longo da qual $q$ é conjungado a p,
\item ou existem exatamente duas geodésicas minimizantes $\gamma$ e $\sigma$ de $p$ a $q$; além disto, $\gamma'(\ell)$, $\ell=d(p,q)$.
\end{enumerate}
\end{exercise}

\begin{proof}[Prova sem consultar outras referencias]
Foi provado em sala que um ponto $q$ está no cut locus $C_m(p)$ se e somente se
alguma das seguintes condições é verdadeira: (a) $q$ é o primeiro ponto
conjugado a $p$, ou (b) existem duas geodésicas minimizantes ligando $p$ e $q$.

Sponha que (a) não é verdadeira. Considere a variação por geodésicas 
$$
f(s,t):=\operatorname{exp}_{\gamma(t)}
(s\operatorname{exp}_{\gamma(t)}^{-1}\sigma(t))
$$
Note que se $\gamma'(\ell)=-\sigma'(\ell)$ o campo de Jacobi é nulo. Em outro
caso obtemos que $p$ é conjugado a $q$, absurdo.
\end{proof}

\begin{exercise}
Proposição 2.13, Cap. XIII \cite{doc}. Se a curvatura seccional $K$ de uma
variedade Riemanniana completa $M$ satisfaz
$$
0\leq K_{\operatorname{min}}\leq K\leq K_{\operatorname{max}},
$$
Então
\begin{enumerate}
\item $i(M) \geq \pi/\sqrt{K_{\operatorname{max}}}$, ou
\item existe uma geodésica fechada $\gamma$ em $M$, cujo comprimento é menor do 
que o de qualquer outra geodésica fechada em $M$, tal que
$$
i(M)=\frac{1}{2}\ell(\gamma)
$$

\end{enumerate}
\end{exercise}

\bibliography{my}
\bibliographystyle{amsalpha}

\end{document}


\input{preamble}

\begin{document}

\title{Representation Theory}
\maketitle

\phantomsection
\label{section-phantom}
\hfill
\href{http://github.com/danimalabares/stack}{github.com/danimalabares/stack}

\tableofcontents

\section{Borel-Weil-Bott theorem}
\label{section-Borel-Weil-Bott-thoerem}

Borel-Weil-Bott theorem is a device for computing cohomologies of sheaves.

Take the Grassmanian $\text{Gr}(n,k)$, which may be obtained as a quotient $G$
of a group $G$ by a parabolic subgroup $P$. We would like it if $P$ was a
semisimple group since those are classified, but unfortunately it is not. 
So instead we use so-called Levi quotient denoted by $L$ which is semisimple
and allows us to understand the cohomologies of the variety $G/P$.

Fortunately the representation theory of $\text{Gr}(k,n)$ is well known, in fact
we get $\text{GL}_k \times \text{GL}_{n-k}$.

There are functors associated to $L$, which are described by a sequence of
numbers $a_1,\ldots,a_n$. Along with other sequences of numbers
$k_1,\ldots,k_\ell$, and $\rho$ (the latter is a concept in representation
theory but we may ultimately think of it as another sequence of numbers) we may
construct an action of the symmetric group $S^n$ acting on these sequences and
obtain a result concerning the cohomology $H^{\ell(\sigma)}(L)$, and in
particular we find that if two entries in our list of numbers coincide, the
cohomology will vanish.


\bibliography{my}
\bibliographystyle{amsalpha}

\end{document}


\input{preamble}

\begin{document}

\title{Representation Theory}
\maketitle

\phantomsection
\label{section-phantom}
\hfill
\href{http://github.com/danimalabares/stack}{github.com/danimalabares/stack}

\tableofcontents

\section{Basic definitions}
\label{section-basic-definitions}

\begin{definition}
\label{definition-representation}
A {\it representation of Lie algebra} $\mathfrak{g}$ is a vector space $V$ 
and homomorphism
$$
\rho:\mathfrak{g} \to \text{End}(V)
$$
of Lie algebras, i.e.,
$$
\rho([x,y])=\rho(x)\rho(y)-\rho(y)\rho(x).
$$
\end{definition}

\begin{remark}
\label{remark-representations-are-modules}
Another name for a representation of a Lie algebra $\mathfrak{g}$ is a 
{\it $\mathfrak{g}$-module}. This is like the abstract definition of an algebra
using a morphism. It's as if the morphism lets us define a sort of product of
elements in the codomain by elements in the domain.
\end{remark}

It is not possible to classify all representations of a Lie algebra
$\mathfrak{g}$. But there is a theorem by Weyl that says that if $\mathfrak{g}$
is finite-dimensional and (semi)simple, then every finite-dimensional
representation of $\mathfrak{g}$ 
(i.e. a representation where $V$ is finite-dimensional) 
is isomorphic to a direct sum of irreducible representations 
(i.e. {\bf missing!}).

\medskip\noindent
Now we review the classification of irreducible finite-dimensional
representations of $\mathfrak{sl}_2$. ({\bf Missing!})

\begin{definition}
\label{definition-weight-space-decomposition}

\end{definition}

\begin{exercise}
\label{exercise-any-finite-dimensional-representation-of-g-has-weight-space-
decomposition}
Any finite dimensional representation of $\mathfrak{g}$ has a weight space
decomposition.
\end{exercise}

The {\it highest weight} is 

\begin{exercise}
\label{exercise-if-V-is-irreducible-then-highest-weight-is-unique}
If $V$ is irreducible then the highest weight is unique.
\end{exercise}

\begin{definition}
\label{definition-Casimir-element}
(It is an element of $\text{End}(V)$.
\end{definition}

Computation of the Casimir element of some representations of $\mathfrak{sl}_2$.

The {\it fundamental weights} are …

There should be some relation between the bases of simple roots and of
fundamental weights.

\begin{exercise}
\label{exercise-}
Roots, coroots, Cartan matrix.
\end{exercise}

\section{Borel-Weil-Bott theorem}
\label{section-Borel-Weil-Bott-thoerem}

Borel-Weil-Bott theorem is a device for computing cohomologies of sheaves.

Take the Grassmanian $\text{Gr}(n,k)$, which may be obtained as a quotient $G$
of a group $G$ by a parabolic subgroup $P$. We would like it if $P$ was a
semisimple group since those are classified, but unfortunately it is not. 
So instead we use so-called Levi quotient denoted by $L$ which is semisimple
and allows us to understand the cohomologies of the variety $G/P$.

Fortunately the representation theory of $\text{Gr}(k,n)$ is well known, in fact
we get $\text{GL}_k \times \text{GL}_{n-k}$.

There are functors associated to $L$, which are described by a sequence of
numbers $a_1,\ldots,a_n$. Along with other sequences of numbers
$k_1,\ldots,k_\ell$, and $\rho$ (the latter is a concept in representation
theory but we may ultimately think of it as another sequence of numbers) we may
construct an action of the symmetric group $S^n$ acting on these sequences and
obtain a result concerning the cohomology $H^{\ell(\sigma)}(L)$, and in
particular we find that if two entries in our list of numbers coincide, the
cohomology will vanish.


\bibliography{my}
\bibliographystyle{amsalpha}

\end{document}


\input{preamble}

\begin{document}

\title{Complex Analysis}
\maketitle

\phantomsection
\label{section-phantom}
\hfill
\href{http://github.com/danimalabares/stack}{github.com/danimalabares/stack}

\tableofcontents

\section{Cauchy-Riemann equations}
\label{section-Cauchy-Riemann-equations}

\begin{definition}
\label{definition-holomorphic-function}
A function $f:W\subset \mathbb{C} \to \mathbb{C}$ is {\it holomorphic} at $z_0
\in U$ if
$$
\lim_{z\to z_0} \frac{f(z_0-z)-f(z_0)}{z-z_0}
$$
exists. This is equivalent to
$$
\lim_{h\to 0} \frac{f(z_0+h)-f(z_0)}{h}
$$
where $h$ is a complex parameter.
\end{definition}

\begin{theorem}
\label{theorem-Cauchy-Riemann}
Let $f:W\subset\mathbb{C}\to \mathbb{C}$ be a function. Let $z=x+iy$ coordinates
in $\mathbb{C}$. Define $f:=u+iv$ and
$z_0=x_0+iy_0 \in U$. $f$ is holomorphic at $z_0$ if and only if
$$
\frac{\partial u}{\partial x}=\frac{\partial v}{\partial y},\qquad \frac{\partial v}{\partial x}=-\frac{\partial u}{\partial y}
$$
\end{theorem}

\section{Integration}
\label{section-integration}

\begin{theorem}[Cauchy]
\label{theorem-Cauchy}
If $f(z)$ is analytic in an open disk $\Delta$, then
\begin{equation}
\label{equation-Cauchy-theorem}
\int_\gamma f(z)dz=0
\end{equation}
for every closed curve $\gamma$ in $\Delta$.
\end{theorem}

\begin{proof}
I prove this using Stokes theorem. The fact that $f$ is holomorphic says that
$\overline{\partial}f=0$. In turn, $d=\partial+\overline{\partial}$, so that
$df=\partial f$, which is a $(1,0)$-form (just as $\overline{\partial}$ maps
functions to $(0,1)$-forms… (must justify…)). This makes $d(fdz)=df\wedge dz$ a
$(2,0)$-form, but there are no such forms on $\mathbb{C}$ (because there are no
holomorphic or antiholomorphic 2-forms on a complex dimension 1 space).
\end{proof}

\begin{theorem}
\label{theorem-Cauchy-removable}
Let $f$ be analytic in the region $\Delta'$ by omitting a finite number of
points $\dseta_j$ from an open disk $\Delta$. If $f$ satisfies that
\begin{equation}
\label{equation-removable-singularity}
\lim_{z\to\zeta_j} (z-\zeta_j)f(z)=0
\end{equation}
for all $j$, then Eq. \ref{equation-Cauchy-theorem} holds for any closed curve
$\gamma$ in $\Delta'$.
\end{theorem}

\begin{proof}[Sketch of proof]
It suffices to prove that Eq. \ref{equation-removable-singularity} implies that
we can extend $f$ to a holomorphic function on all of $\Delta$ and apply Stokes
as in Theorem \ref{theorem-Cauchy}. But that won't work: for that we need Cauchy
integral formula, and that's why I'm here: to prove Cauchy integral formula.

So I think it might be using logarithmic derivative. The limit above allows to
write
$$
|f(z)|\leq \frac{\varepsilon}{|z-\zeta_j|}
$$
Then integrate. On the left we can bound the integral of $f$ after applying
integral triangle inequality, and on the right we have logarithmic derivative of
$z-\zeta_j$. This will be a fixed number (the index, see Definition
\ref{definition-index}), so that we have effectively bounded the integral.
\end{proof}

\section{Cauchy integral formula}
\label{section-Cauchy-integral-formula}

\begin{slogan}
The index of a curve about a point tells us how many times a curve winds about 
the point.
\end{slogan}

\begin{lemma}
\label{lemma-index-is-multiple-of-2pii}
\begin{reference}
\cite[Section 2.2, Lemma 1]{ahl}
\end{reference}
If the piecewise differentiable closed curve $\gamma$ does not pass through the
point $a$, then the value of the integral
$$
\int_\gamma\frac{dz}{z-a}
$$
is a multple of $2\pi i$.
\end{lemma}

\begin{proof}
We are tempted to simply write the integrand as the logarithmic derivative of
the function $f(z)=z-a$. But this isn't quite right, we must be careful with the
domain of the logarithm. But it is instructive to see the computation:
$$
\int_\gamma\frac{dz}{z-a}=\int_\gamma d \text{log}(z-a)=
\int_\gamma d\text{log}|z-a|+i\int_\gamma d \text{arg}(z-a)
$$
If  $\gamma$ is closed then $\text{log}|z-a|$ would return to its initial value
and $\text{arg}(z-a)$ increases or decreases by a multiple of $2\pi$.
\end{proof}

\begin{definition}
\label{definition-index}
The {\it index} of the point $a$ with respect to the closed curve $\gamma$ is
\begin{equation}
\label{equation-index}
n(\gamma,a):=\frac{1}{2\pi i}\int_\gamma\frac{dz}{z-a}
\end{equation}
\end{definition}

\begin{theorem}
\label{theorem-Cauchy-integral-formula}
\begin{reference}
\cite[Section 2.2, Theorem 6]{ahl}
\end{reference}
Suppose that $f(z)$ is analytic in an open disk $\Delta$, and let $\gamma$ be a
closed curve in $\Delta$. For any point not on $\gamma$,
\begin{equation}
\label{equation-Cauchy-formula-with-index}
n(\gamma,a)f(a)=\frac{1}{2\pi i}\int_\gamma\frac{f(z)dz}{z-a}
\end{equation}
where $n(\gamma,a)$ is the index of $a$ with respect to $\gamma$.
\end{theorem}

\begin{proof}
This is a simple application of Theorem \ref{theorem-Cauchy-removable} for the
function
$$
F(z):=\frac{f(z)-f(a)}{z-a}
$$
Notice the limit condition holds, so that the integral vanishes!
\end{proof}

\section{Zeroes of a holomorphic function}
\label{section-zeroes}

\begin{lemma}
\label{lemma-finite-Taylor-expansion}
If $f$ is analytic in a connected open set $W\subset\mathbb{C}$ and $a \in W$,
then we can write
$$
f(z)=f(a)+f'(a)(z-a)+\frac{f''(a)}{2!}(z-a)^2+\ldots
+\frac{f^{(n-1)}(a)}{(n-1)!}(z-a)^{n-1}+f_n(z)(z-a)^n
$$
for some function $f_n$ analytic on $W$.
\end{lemma}

\begin{theorem}
\label{theorem-zeroes-are-isolated-and-have-finite-order}
If $f:W\subset\mathbb{C}\to \mathbb{C}$ is a holomorphic function defined on an
open set $W$ and $f(a)=0$ for some $a\in W$, and $f$ is not identically zero,
then there is a disck $D_r(a)\subseteq W$ such that $f(z)\neq 0$ for $z \in
D_r(a)\setminus\{a\}$ and a positive integer $m$ called the {\it order} or
{\it multiplicity} of the zero $a$ such that  $f(z)=(z-a)^mh(z)$ for some
holomorphic function $h$ that does not vanish at $a$. The order of a zero is
equal to the smallest integer $m$ such that $f^{(m)}(a)\neq 0$.
\end{theorem}

\begin{proof}
If $f$ is not identically zero, there must exist a first derivative $f^{
(h)}(a)$ that is not zero, since otherwise $f$ would vanish identically in $W$.
This follows from … 

Then by Lemma \ref{lemma-finite-Taylor-expansion} we obtain that
$f(z)=f_h(z-a)^n$.
\end{proof}

\section{Argument Principle}
\label{section-argument-principle}

\begin{theorem}[Argument Principle and Rouché theorem]
\label{theorem-argument-principle-and-Rouche-theorem}
\begin{reference}
\cite[Chapter 5, Theorem 18]{ahl}
\end{reference}
If $f$ is meromorphic in $\Omega$ with zeros $a_j$ and $b_k$, then
\begin{equation}
\label{equation-argument-principle}
\frac{1}{2\pi i}\int_\gamma\frac{f'(\zeta)}{f(\zeta)}d\zeta
=\sum_{i}n(\gamma,a_i)-\sum_{k}n(\gamma,b_k)
\end{equation}
\end{theorem}

\begin{lemma}[Rouché theorem]
\label{lemma-Rouche-theorem}
\begin{reference}
\cite[Chapter 5, Corollary, p. 153]{ahl}
\end{reference}
Let $\gamma$ be homologous to zero in $\Omega$ and such that $n(\gamma,z)$ is
either 0 or 1 for any point $z$ not on $\gamma$. Suppose that $f$ and $g$ are
analytic in $\Omega$ and satisfy that $|f-g|<|f|$ on $\gamma$. Then $f$ and $g$
have the same number of zeros enclosed by $\gamma$.
\end{lemma}

Compare with Misha's version

\begin{theorem}[Rouché theorem]
\label{theorem-Rouche-theorem-Mishas-version}
Let $f_t$ be a family of holomorphic functions on a disk  $\Delta$, continuously
depending on a parameter $t\in \mathbb{R}$ and non-zero everywhere on its
boundary $\partial\Delta$. Prove that the number of zeros of $f_t$ in $\Delta$
is constant.
\end{theorem}


\bibliography{my}
\bibliographystyle{amsalpha}

\end{document}


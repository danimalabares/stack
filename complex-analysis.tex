\IfFileExists{stacks-project.cls}{%
\documentclass{stacks-project}
}{%
\documentclass{amsart}
}

% For dealing with references we use the comment environment
\usepackage{verbatim}
\newenvironment{reference}{\comment}{\endcomment}
%\newenvironment{reference}{}{}
\newenvironment{slogan}{\comment}{\endcomment}
\newenvironment{history}{\comment}{\endcomment}

% For commutative diagrams we use Xy-pic
\usepackage[all]{xy}

% We use 2cell for 2-commutative diagrams.
\xyoption{2cell}
\UseAllTwocells

% We use multicol for the list of chapters between chapters
\usepackage{multicol}

% This is generally recommended for better output
\usepackage{lmodern}
\usepackage[T1]{fontenc}

% For cross-file-references
\usepackage{xr-hyper}

% Package for hypertext links:
\usepackage{hyperref}

% For any local file, say "hello.tex" you want to link to please
% use \externaldocument[hello-]{hello}
\externaldocument[introduction-]{introduction}
\externaldocument[conventions-]{conventions}
\externaldocument[sets-]{sets}
\externaldocument[categories-]{categories}
\externaldocument[topology-]{topology}
\externaldocument[sheaves-]{sheaves}
\externaldocument[sites-]{sites}
\externaldocument[stacks-]{stacks}
\externaldocument[fields-]{fields}
\externaldocument[algebra-]{algebra}
\externaldocument[brauer-]{brauer}
\externaldocument[homology-]{homology}
\externaldocument[derived-]{derived}
\externaldocument[simplicial-]{simplicial}
\externaldocument[more-algebra-]{more-algebra}
\externaldocument[smoothing-]{smoothing}
\externaldocument[modules-]{modules}
\externaldocument[sites-modules-]{sites-modules}
\externaldocument[injectives-]{injectives}
\externaldocument[cohomology-]{cohomology}
\externaldocument[sites-cohomology-]{sites-cohomology}
\externaldocument[dga-]{dga}
\externaldocument[dpa-]{dpa}
\externaldocument[sdga-]{sdga}
\externaldocument[hypercovering-]{hypercovering}
\externaldocument[schemes-]{schemes}
\externaldocument[constructions-]{constructions}
\externaldocument[properties-]{properties}
\externaldocument[morphisms-]{morphisms}
\externaldocument[coherent-]{coherent}
\externaldocument[divisors-]{divisors}
\externaldocument[limits-]{limits}
\externaldocument[varieties-]{varieties}
\externaldocument[topologies-]{topologies}
\externaldocument[descent-]{descent}
\externaldocument[perfect-]{perfect}
\externaldocument[more-morphisms-]{more-morphisms}
\externaldocument[flat-]{flat}
\externaldocument[groupoids-]{groupoids}
\externaldocument[more-groupoids-]{more-groupoids}
\externaldocument[etale-]{etale}
\externaldocument[chow-]{chow}
\externaldocument[intersection-]{intersection}
\externaldocument[pic-]{pic}
\externaldocument[weil-]{weil}
\externaldocument[adequate-]{adequate}
\externaldocument[dualizing-]{dualizing}
\externaldocument[duality-]{duality}
\externaldocument[discriminant-]{discriminant}
\externaldocument[derham-]{derham}
\externaldocument[local-cohomology-]{local-cohomology}
\externaldocument[algebraization-]{algebraization}
\externaldocument[curves-]{curves}
\externaldocument[resolve-]{resolve}
\externaldocument[models-]{models}
\externaldocument[functors-]{functors}
\externaldocument[equiv-]{equiv}
\externaldocument[pione-]{pione}
\externaldocument[etale-cohomology-]{etale-cohomology}
\externaldocument[proetale-]{proetale}
\externaldocument[relative-cycles-]{relative-cycles}
\externaldocument[more-etale-]{more-etale}
\externaldocument[trace-]{trace}
\externaldocument[crystalline-]{crystalline}
\externaldocument[spaces-]{spaces}
\externaldocument[spaces-properties-]{spaces-properties}
\externaldocument[spaces-morphisms-]{spaces-morphisms}
\externaldocument[decent-spaces-]{decent-spaces}
\externaldocument[spaces-cohomology-]{spaces-cohomology}
\externaldocument[spaces-limits-]{spaces-limits}
\externaldocument[spaces-divisors-]{spaces-divisors}
\externaldocument[spaces-over-fields-]{spaces-over-fields}
\externaldocument[spaces-topologies-]{spaces-topologies}
\externaldocument[spaces-descent-]{spaces-descent}
\externaldocument[spaces-perfect-]{spaces-perfect}
\externaldocument[spaces-more-morphisms-]{spaces-more-morphisms}
\externaldocument[spaces-flat-]{spaces-flat}
\externaldocument[spaces-groupoids-]{spaces-groupoids}
\externaldocument[spaces-more-groupoids-]{spaces-more-groupoids}
\externaldocument[bootstrap-]{bootstrap}
\externaldocument[spaces-pushouts-]{spaces-pushouts}
\externaldocument[spaces-chow-]{spaces-chow}
\externaldocument[groupoids-quotients-]{groupoids-quotients}
\externaldocument[spaces-more-cohomology-]{spaces-more-cohomology}
\externaldocument[spaces-simplicial-]{spaces-simplicial}
\externaldocument[spaces-duality-]{spaces-duality}
\externaldocument[formal-spaces-]{formal-spaces}
\externaldocument[restricted-]{restricted}
\externaldocument[spaces-resolve-]{spaces-resolve}
\externaldocument[formal-defos-]{formal-defos}
\externaldocument[defos-]{defos}
\externaldocument[cotangent-]{cotangent}
\externaldocument[examples-defos-]{examples-defos}
\externaldocument[algebraic-]{algebraic}
\externaldocument[examples-stacks-]{examples-stacks}
\externaldocument[stacks-sheaves-]{stacks-sheaves}
\externaldocument[criteria-]{criteria}
\externaldocument[artin-]{artin}
\externaldocument[quot-]{quot}
\externaldocument[stacks-properties-]{stacks-properties}
\externaldocument[stacks-morphisms-]{stacks-morphisms}
\externaldocument[stacks-limits-]{stacks-limits}
\externaldocument[stacks-cohomology-]{stacks-cohomology}
\externaldocument[stacks-perfect-]{stacks-perfect}
\externaldocument[stacks-introduction-]{stacks-introduction}
\externaldocument[stacks-more-morphisms-]{stacks-more-morphisms}
\externaldocument[stacks-geometry-]{stacks-geometry}
\externaldocument[moduli-]{moduli}
\externaldocument[moduli-curves-]{moduli-curves}
\externaldocument[examples-]{examples}
\externaldocument[exercises-]{exercises}
\externaldocument[guide-]{guide}
\externaldocument[desirables-]{desirables}
\externaldocument[coding-]{coding}
\externaldocument[obsolete-]{obsolete}
\externaldocument[fdl-]{fdl}
\externaldocument[index-]{index}

% Theorem environments.
%
\theoremstyle{plain}
\newtheorem{theorem}[subsection]{Theorem}
\newtheorem{proposition}[subsection]{Proposition}
\newtheorem{lemma}[subsection]{Lemma}

\theoremstyle{definition}
\newtheorem{definition}[subsection]{Definition}
\newtheorem{example}[subsection]{Example}
\newtheorem{exercise}[subsection]{Exercise}
\newtheorem{situation}[subsection]{Situation}

\theoremstyle{remark}
\newtheorem{remark}[subsection]{Remark}
\newtheorem{remarks}[subsection]{Remarks}

\numberwithin{equation}{subsection}

% Macros
%
\def\lim{\mathop{\mathrm{lim}}\nolimits}
\def\colim{\mathop{\mathrm{colim}}\nolimits}
\def\Spec{\mathop{\mathrm{Spec}}}
\def\Hom{\mathop{\mathrm{Hom}}\nolimits}
\def\Ext{\mathop{\mathrm{Ext}}\nolimits}
\def\SheafHom{\mathop{\mathcal{H}\!\mathit{om}}\nolimits}
\def\SheafExt{\mathop{\mathcal{E}\!\mathit{xt}}\nolimits}
\def\Sch{\mathit{Sch}}
\def\Mor{\mathop{\mathrm{Mor}}\nolimits}
\def\Ob{\mathop{\mathrm{Ob}}\nolimits}
\def\Sh{\mathop{\mathit{Sh}}\nolimits}
\def\NL{\mathop{N\!L}\nolimits}
\def\CH{\mathop{\mathrm{CH}}\nolimits}
\def\proetale{{pro\text{-}\acute{e}tale}}
\def\etale{{\acute{e}tale}}
\def\QCoh{\mathit{QCoh}}
\def\Ker{\mathop{\mathrm{Ker}}}
\def\Im{\mathop{\mathrm{Im}}}
\def\Coker{\mathop{\mathrm{Coker}}}
\def\Coim{\mathop{\mathrm{Coim}}}

% Boxtimes
%
\DeclareMathSymbol{\boxtimes}{\mathbin}{AMSa}{"02}

%
% Macros for moduli stacks/spaces
%
\def\QCohstack{\mathcal{QC}\!\mathit{oh}}
\def\Cohstack{\mathcal{C}\!\mathit{oh}}
\def\Spacesstack{\mathcal{S}\!\mathit{paces}}
\def\Quotfunctor{\mathrm{Quot}}
\def\Hilbfunctor{\mathrm{Hilb}}
\def\Curvesstack{\mathcal{C}\!\mathit{urves}}
\def\Polarizedstack{\mathcal{P}\!\mathit{olarized}}
\def\Complexesstack{\mathcal{C}\!\mathit{omplexes}}
% \Pic is the operator that assigns to X its picard group, usage \Pic(X)
% \Picardstack_{X/B} denotes the Picard stack of X over B
% \Picardfunctor_{X/B} denotes the Picard functor of X over B
\def\Pic{\mathop{\mathrm{Pic}}\nolimits}
\def\Picardstack{\mathcal{P}\!\mathit{ic}}
\def\Picardfunctor{\mathrm{Pic}}
\def\Deformationcategory{\mathcal{D}\!\mathit{ef}}

%Dani's additions
\usepackage{graphicx} %figures


\begin{document}

\title{Complex Analysis}
\maketitle

\phantomsection
\label{section-phantom}
\hfill
\href{http://github.com/danimalabares/stack}{github.com/danimalabares/stack}

\tableofcontents

\section{Cauchy-Riemann equations}
\label{section-Cauchy-Riemann-equations}

\begin{definition}
\label{definition-holomorphic-function}
A function $f:W\subset \mathbb{C} \to \mathbb{C}$ is {\it holomorphic} at $z_0
\in U$ if
$$
\lim_{z\to z_0} \frac{f(z_0-z)-f(z_0)}{z-z_0}
$$
exists. This is equivalent to
$$
\lim_{h\to 0} \frac{f(z_0+h)-f(z_0)}{h}
$$
where $h$ is a complex parameter.
\end{definition}

\begin{theorem}
\label{theorem-Cauchy-Riemann}
Let $f:W\subset\mathbb{C}\to \mathbb{C}$ be a function. Let $z=x+iy$ coordinates
in $\mathbb{C}$. Define $f:=u+iv$ and
$z_0=x_0+iy_0 \in U$. $f$ is holomorphic at $z_0$ if and only if
$$
\frac{\partial u}{\partial x}=\frac{\partial v}{\partial y},\qquad \frac{\partial v}{\partial x}=-\frac{\partial u}{\partial y}
$$
\end{theorem}

\section{Integration}
\label{section-integration}

\begin{theorem}[Cauchy]
\label{theorem-Cauchy}
If $f(z)$ is analytic in an open disk $\Delta$, then
\begin{equation}
\label{equation-Cauchy-theorem}
\int_\gamma f(z)dz=0
\end{equation}
for every closed curve $\gamma$ in $\Delta$.
\end{theorem}

\begin{proof}
I prove this using Stokes theorem. The fact that $f$ is holomorphic says that
$\overline{\partial}f=0$. In turn, $d=\partial+\overline{\partial}$, so that
$df=\partial f$, which is a $(1,0)$-form (just as $\overline{\partial}$ maps
functions to $(0,1)$-forms… (must justify…)). This makes $d(fdz)=df\wedge dz$ a
$(2,0)$-form, but there are no such forms on $\mathbb{C}$ (because there are no
holomorphic or antiholomorphic 2-forms on a complex dimension 1 space).
\end{proof}

\begin{theorem}
\label{theorem-Cauchy-removable}
Let $f$ be analytic in the region $\Delta'$ by omitting a finite number of
points $\dseta_j$ from an open disk $\Delta$. If $f$ satisfies that
\begin{equation}
\label{equation-removable-singularity}
\lim_{z\to\zeta_j} (z-\zeta_j)f(z)=0
\end{equation}
for all $j$, then Eq. \ref{equation-Cauchy-theorem} holds for any closed curve
$\gamma$ in $\Delta'$.
\end{theorem}

\begin{proof}[Sketch of proof]
It suffices to prove that Eq. \ref{equation-removable-singularity} implies that
we can extend $f$ to a holomorphic function on all of $\Delta$ and apply Stokes
as in Theorem \ref{theorem-Cauchy}. But that won't work: for that we need Cauchy
integral formula, and that's why I'm here: to prove Cauchy integral formula.

So I think it might be using logarithmic derivative. The limit above allows to
write
$$
|f(z)|\leq \frac{\varepsilon}{|z-\zeta_j|}
$$
Then integrate. On the left we can bound the integral of $f$ after applying
integral triangle inequality, and on the right we have logarithmic derivative of
$z-\zeta_j$. This will be a fixed number (the index, see Definition
\ref{definition-index}), so that we have effectively bounded the integral.
\end{proof}

\section{Cauchy integral formula}
\label{section-Cauchy-integral-formula}

\begin{slogan}
The index of a curve about a point tells us how many times a curve winds about 
the point.
\end{slogan}

\begin{lemma}
\label{lemma-index-is-multiple-of-2pii}
\begin{reference}
\cite[Section 2.2, Lemma 1]{ahl}
\end{reference}
If the piecewise differentiable closed curve $\gamma$ does not pass through the
point $a$, then the value of the integral
$$
\int_\gamma\frac{dz}{z-a}
$$
is a multple of $2\pi i$.
\end{lemma}

\begin{proof}
We are tempted to simply write the integrand as the logarithmic derivative of
the function $f(z)=z-a$. But this isn't quite right, we must be careful with the
domain of the logarithm. But it is instructive to see the computation:
$$
\int_\gamma\frac{dz}{z-a}=\int_\gamma d \text{log}(z-a)=
\int_\gamma d\text{log}|z-a|+i\int_\gamma d \text{arg}(z-a)
$$
If  $\gamma$ is closed then $\text{log}|z-a|$ would return to its initial value
and $\text{arg}(z-a)$ increases or decreases by a multiple of $2\pi$.
\end{proof}

\begin{definition}
\label{definition-index}
The {\it index} of the point $a$ with respect to the closed curve $\gamma$ is
\begin{equation}
\label{equation-index}
n(\gamma,a):=\frac{1}{2\pi i}\int_\gamma\frac{dz}{z-a}
\end{equation}
\end{definition}

\begin{theorem}
\label{theorem-Cauchy-integral-formula}
\begin{reference}
\cite[Section 2.2, Theorem 6]{ahl}
\end{reference}
Suppose that $f(z)$ is analytic in an open disk $\Delta$, and let $\gamma$ be a
closed curve in $\Delta$. For any point not on $\gamma$,
\begin{equation}
\label{equation-Cauchy-formula-with-index}
n(\gamma,a)f(a)=\frac{1}{2\pi i}\int_\gamma\frac{f(z)dz}{z-a}
\end{equation}
where $n(\gamma,a)$ is the index of $a$ with respect to $\gamma$.
\end{theorem}

\begin{proof}
This is a simple application of Theorem \ref{theorem-Cauchy-removable} for the
function
$$
F(z):=\frac{f(z)-f(a)}{z-a}
$$
Notice the limit condition holds, so that the integral vanishes!
\end{proof}

\section{Zeroes of a holomorphic function}
\label{section-zeroes}

\begin{lemma}
\label{lemma-finite-Taylor-expansion}
If $f$ is analytic in a connected open set $W\subset\mathbb{C}$ and $a \in W$,
then we can write
$$
f(z)=f(a)+f'(a)(z-a)+\frac{f''(a)}{2!}(z-a)^2+\ldots
+\frac{f^{(n-1)}(a)}{(n-1)!}(z-a)^{n-1}+f_n(z)(z-a)^n
$$
for some function $f_n$ analytic on $W$.
\end{lemma}

\begin{theorem}
\label{theorem-zeroes-are-isolated-and-have-finite-order}
If $f:W\subset\mathbb{C}\to \mathbb{C}$ is a holomorphic function defined on an
open set $W$ and $f(a)=0$ for some $a\in W$, and $f$ is not identically zero,
then there is a disck $D_r(a)\subseteq W$ such that $f(z)\neq 0$ for $z \in
D_r(a)\setminus\{a\}$ and a positive integer $m$ called the {\it order} or
{\it multiplicity} of the zero $a$ such that  $f(z)=(z-a)^mh(z)$ for some
holomorphic function $h$ that does not vanish at $a$. The order of a zero is
equal to the smallest integer $m$ such that $f^{(m)}(a)\neq 0$.
\end{theorem}

\begin{proof}
If $f$ is not identically zero, there must exist a first derivative $f^{
(h)}(a)$ that is not zero, since otherwise $f$ would vanish identically in $W$.
This follows from … 

Then by Lemma \ref{lemma-finite-Taylor-expansion} we obtain that
$f(z)=f_h(z-a)^n$.
\end{proof}

\section{Argument Principle}
\label{section-argument-principle}

\begin{theorem}[Argument Principle and Rouché theorem]
\label{theorem-argument-principle-and-Rouche-theorem}
\begin{reference}
\cite[Chapter 5, Theorem 18]{ahl}
\end{reference}
If $f$ is meromorphic in $\Omega$ with zeros $a_j$ and $b_k$, then
\begin{equation}
\label{equation-argument-principle}
\frac{1}{2\pi i}\int_\gamma\frac{f'(\zeta)}{f(\zeta)}d\zeta
=\sum_{i}n(\gamma,a_i)-\sum_{k}n(\gamma,b_k)
\end{equation}
\end{theorem}

\begin{lemma}[Rouché theorem]
\label{lemma-Rouche-theorem}
\begin{reference}
\cite[Chapter 5, Corollary, p. 153]{ahl}
\end{reference}
Let $\gamma$ be homologous to zero in $\Omega$ and such that $n(\gamma,z)$ is
either 0 or 1 for any point $z$ not on $\gamma$. Suppose that $f$ and $g$ are
analytic in $\Omega$ and satisfy that $|f-g|<|f|$ on $\gamma$. Then $f$ and $g$
have the same number of zeros enclosed by $\gamma$.
\end{lemma}

Compare with Misha's version

\begin{theorem}[Rouché theorem]
\label{theorem-Rouche-theorem-Mishas-version}
Let $f_t$ be a family of holomorphic functions on a disk  $\Delta$, continuously
depending on a parameter $t\in \mathbb{R}$ and non-zero everywhere on its
boundary $\partial\Delta$. Prove that the number of zeros of $f_t$ in $\Delta$
is constant.
\end{theorem}


\bibliography{my}
\bibliographystyle{amsalpha}

\end{document}


\IfFileExists{stacks-project.cls}{%
\documentclass{stacks-project}
}{%
\documentclass{amsart}
}

% For dealing with references we use the comment environment
\usepackage{verbatim}
\newenvironment{reference}{\comment}{\endcomment}
%\newenvironment{reference}{}{}
\newenvironment{slogan}{\comment}{\endcomment}
\newenvironment{history}{\comment}{\endcomment}

% For commutative diagrams we use Xy-pic
\usepackage[all]{xy}

% We use 2cell for 2-commutative diagrams.
\xyoption{2cell}
\UseAllTwocells

% We use multicol for the list of chapters between chapters
\usepackage{multicol}

% This is generally recommended for better output
\usepackage{lmodern}
\usepackage[T1]{fontenc}

% For cross-file-references
\usepackage{xr-hyper}

% Package for hypertext links:
\usepackage{hyperref}

% For any local file, say "hello.tex" you want to link to please
% use \externaldocument[hello-]{hello}
\externaldocument[introduction-]{introduction}
\externaldocument[conventions-]{conventions}
\externaldocument[sets-]{sets}
\externaldocument[categories-]{categories}
\externaldocument[topology-]{topology}
\externaldocument[sheaves-]{sheaves}
\externaldocument[sites-]{sites}
\externaldocument[stacks-]{stacks}
\externaldocument[fields-]{fields}
\externaldocument[algebra-]{algebra}
\externaldocument[brauer-]{brauer}
\externaldocument[homology-]{homology}
\externaldocument[derived-]{derived}
\externaldocument[simplicial-]{simplicial}
\externaldocument[more-algebra-]{more-algebra}
\externaldocument[smoothing-]{smoothing}
\externaldocument[modules-]{modules}
\externaldocument[sites-modules-]{sites-modules}
\externaldocument[injectives-]{injectives}
\externaldocument[cohomology-]{cohomology}
\externaldocument[sites-cohomology-]{sites-cohomology}
\externaldocument[dga-]{dga}
\externaldocument[dpa-]{dpa}
\externaldocument[sdga-]{sdga}
\externaldocument[hypercovering-]{hypercovering}
\externaldocument[schemes-]{schemes}
\externaldocument[constructions-]{constructions}
\externaldocument[properties-]{properties}
\externaldocument[morphisms-]{morphisms}
\externaldocument[coherent-]{coherent}
\externaldocument[divisors-]{divisors}
\externaldocument[limits-]{limits}
\externaldocument[varieties-]{varieties}
\externaldocument[topologies-]{topologies}
\externaldocument[descent-]{descent}
\externaldocument[perfect-]{perfect}
\externaldocument[more-morphisms-]{more-morphisms}
\externaldocument[flat-]{flat}
\externaldocument[groupoids-]{groupoids}
\externaldocument[more-groupoids-]{more-groupoids}
\externaldocument[etale-]{etale}
\externaldocument[chow-]{chow}
\externaldocument[intersection-]{intersection}
\externaldocument[pic-]{pic}
\externaldocument[weil-]{weil}
\externaldocument[adequate-]{adequate}
\externaldocument[dualizing-]{dualizing}
\externaldocument[duality-]{duality}
\externaldocument[discriminant-]{discriminant}
\externaldocument[derham-]{derham}
\externaldocument[local-cohomology-]{local-cohomology}
\externaldocument[algebraization-]{algebraization}
\externaldocument[curves-]{curves}
\externaldocument[resolve-]{resolve}
\externaldocument[models-]{models}
\externaldocument[functors-]{functors}
\externaldocument[equiv-]{equiv}
\externaldocument[pione-]{pione}
\externaldocument[etale-cohomology-]{etale-cohomology}
\externaldocument[proetale-]{proetale}
\externaldocument[relative-cycles-]{relative-cycles}
\externaldocument[more-etale-]{more-etale}
\externaldocument[trace-]{trace}
\externaldocument[crystalline-]{crystalline}
\externaldocument[spaces-]{spaces}
\externaldocument[spaces-properties-]{spaces-properties}
\externaldocument[spaces-morphisms-]{spaces-morphisms}
\externaldocument[decent-spaces-]{decent-spaces}
\externaldocument[spaces-cohomology-]{spaces-cohomology}
\externaldocument[spaces-limits-]{spaces-limits}
\externaldocument[spaces-divisors-]{spaces-divisors}
\externaldocument[spaces-over-fields-]{spaces-over-fields}
\externaldocument[spaces-topologies-]{spaces-topologies}
\externaldocument[spaces-descent-]{spaces-descent}
\externaldocument[spaces-perfect-]{spaces-perfect}
\externaldocument[spaces-more-morphisms-]{spaces-more-morphisms}
\externaldocument[spaces-flat-]{spaces-flat}
\externaldocument[spaces-groupoids-]{spaces-groupoids}
\externaldocument[spaces-more-groupoids-]{spaces-more-groupoids}
\externaldocument[bootstrap-]{bootstrap}
\externaldocument[spaces-pushouts-]{spaces-pushouts}
\externaldocument[spaces-chow-]{spaces-chow}
\externaldocument[groupoids-quotients-]{groupoids-quotients}
\externaldocument[spaces-more-cohomology-]{spaces-more-cohomology}
\externaldocument[spaces-simplicial-]{spaces-simplicial}
\externaldocument[spaces-duality-]{spaces-duality}
\externaldocument[formal-spaces-]{formal-spaces}
\externaldocument[restricted-]{restricted}
\externaldocument[spaces-resolve-]{spaces-resolve}
\externaldocument[formal-defos-]{formal-defos}
\externaldocument[defos-]{defos}
\externaldocument[cotangent-]{cotangent}
\externaldocument[examples-defos-]{examples-defos}
\externaldocument[algebraic-]{algebraic}
\externaldocument[examples-stacks-]{examples-stacks}
\externaldocument[stacks-sheaves-]{stacks-sheaves}
\externaldocument[criteria-]{criteria}
\externaldocument[artin-]{artin}
\externaldocument[quot-]{quot}
\externaldocument[stacks-properties-]{stacks-properties}
\externaldocument[stacks-morphisms-]{stacks-morphisms}
\externaldocument[stacks-limits-]{stacks-limits}
\externaldocument[stacks-cohomology-]{stacks-cohomology}
\externaldocument[stacks-perfect-]{stacks-perfect}
\externaldocument[stacks-introduction-]{stacks-introduction}
\externaldocument[stacks-more-morphisms-]{stacks-more-morphisms}
\externaldocument[stacks-geometry-]{stacks-geometry}
\externaldocument[moduli-]{moduli}
\externaldocument[moduli-curves-]{moduli-curves}
\externaldocument[examples-]{examples}
\externaldocument[exercises-]{exercises}
\externaldocument[guide-]{guide}
\externaldocument[desirables-]{desirables}
\externaldocument[coding-]{coding}
\externaldocument[obsolete-]{obsolete}
\externaldocument[fdl-]{fdl}
\externaldocument[index-]{index}

% Theorem environments.
%
\theoremstyle{plain}
\newtheorem{theorem}[subsection]{Theorem}
\newtheorem{proposition}[subsection]{Proposition}
\newtheorem{lemma}[subsection]{Lemma}

\theoremstyle{definition}
\newtheorem{definition}[subsection]{Definition}
\newtheorem{example}[subsection]{Example}
\newtheorem{exercise}[subsection]{Exercise}
\newtheorem{situation}[subsection]{Situation}

\theoremstyle{remark}
\newtheorem{remark}[subsection]{Remark}
\newtheorem{remarks}[subsection]{Remarks}

\numberwithin{equation}{subsection}

% Macros
%
\def\lim{\mathop{\mathrm{lim}}\nolimits}
\def\colim{\mathop{\mathrm{colim}}\nolimits}
\def\Spec{\mathop{\mathrm{Spec}}}
\def\Hom{\mathop{\mathrm{Hom}}\nolimits}
\def\Ext{\mathop{\mathrm{Ext}}\nolimits}
\def\SheafHom{\mathop{\mathcal{H}\!\mathit{om}}\nolimits}
\def\SheafExt{\mathop{\mathcal{E}\!\mathit{xt}}\nolimits}
\def\Sch{\mathit{Sch}}
\def\Mor{\mathop{\mathrm{Mor}}\nolimits}
\def\Ob{\mathop{\mathrm{Ob}}\nolimits}
\def\Sh{\mathop{\mathit{Sh}}\nolimits}
\def\NL{\mathop{N\!L}\nolimits}
\def\CH{\mathop{\mathrm{CH}}\nolimits}
\def\proetale{{pro\text{-}\acute{e}tale}}
\def\etale{{\acute{e}tale}}
\def\QCoh{\mathit{QCoh}}
\def\Ker{\mathop{\mathrm{Ker}}}
\def\Im{\mathop{\mathrm{Im}}}
\def\Coker{\mathop{\mathrm{Coker}}}
\def\Coim{\mathop{\mathrm{Coim}}}

% Boxtimes
%
\DeclareMathSymbol{\boxtimes}{\mathbin}{AMSa}{"02}

%
% Macros for moduli stacks/spaces
%
\def\QCohstack{\mathcal{QC}\!\mathit{oh}}
\def\Cohstack{\mathcal{C}\!\mathit{oh}}
\def\Spacesstack{\mathcal{S}\!\mathit{paces}}
\def\Quotfunctor{\mathrm{Quot}}
\def\Hilbfunctor{\mathrm{Hilb}}
\def\Curvesstack{\mathcal{C}\!\mathit{urves}}
\def\Polarizedstack{\mathcal{P}\!\mathit{olarized}}
\def\Complexesstack{\mathcal{C}\!\mathit{omplexes}}
% \Pic is the operator that assigns to X its picard group, usage \Pic(X)
% \Picardstack_{X/B} denotes the Picard stack of X over B
% \Picardfunctor_{X/B} denotes the Picard functor of X over B
\def\Pic{\mathop{\mathrm{Pic}}\nolimits}
\def\Picardstack{\mathcal{P}\!\mathit{ic}}
\def\Picardfunctor{\mathrm{Pic}}
\def\Deformationcategory{\mathcal{D}\!\mathit{ef}}

%Dani's additions
\usepackage{graphicx} %figures


\begin{document}

\title{Algebra}
\maketitle

\phantomsection
\label{section-phantom}
\hfill
\href{http://github.com/danimalabares/stack}{github.com/danimalabares/stack}

\tableofcontents

\section{Rings}
\label{section-rings}

\noindent
So elementary that this is not in Stacks Project.

\begin{theorem}
\label{theorem-ideals-quotient-correspondence}
Let $R$ be a ring and $I \subset R$ an ideal.
There is a correspondence between ideals of $R$ containing $I$ 
and ideals of $R/I$.
\end{theorem}

\begin{proof}
We show that there are two maps between the set of ideals of $R$ 
containing $I$ and the ideals of $R/I$ whose compositions
are the identities in the corresponding set. One map is the projection
to the quotient, and the other map is taking the elements
whose equivalence class lies in a given ideal.

Suppose that $\tilde{J}$ is an ideal of $R/I$.
Define $J=\{j \in R : j+I \in \tilde{J}\}$, 
the set of elements that project to $\tilde{J}$.
Then $I \subset J$, because $i + I = I$ is the zero of $R/I$ 
which is contained in any ideal of $R/I$.
Also notice that $J$ is an ideal of $R$ since for any $r \in R$,
$rj+I \in \tilde{J}$ by $\tilde{J}$ being an ideal.
Notice that projecting $J$ back to $R/I$ gives $J$
by definition.

Conversely, we project to the quotient.
Let $J$ be any ideal of $R$ containing $I$ 
Let $\tilde{J}$ be the projection to $R/I$.
Then $\tilde{J}$ is an ideal of $R/I$ by $J$ being an ideal of $R$.
Consider the set $J'$ of all elements whose projection lies in $\tilde{J}$.
Then $J'=J$ by definition.
\end{proof}

\noindent
This is the essential tool in proving that $R/I$ is a field
when $I$ is a maximal ideal

\begin{lemma}
\label{lemma-quotient-ring-maximal-ideal-is-field}
Let $R$ be a ring and $I$ a maximal ideal of $R$.
Then $R/I$ is a field.
\end{lemma}

\begin{proof}
It's easy to prove that a ring is a field if and only if its
only ideals are $\{0\}$ and $R$. By Theorem \ref{theorem-correspondence-ideals},
the ideals of $R/I$ are in correspondence with ideals of $R$ containing $I$,
but only $R$ is such.
\end{proof}

\section{Fields}
\label{section-fields}

\noindent
Just for the record, any simple field extension, i.e. the smallest
field in $F$ containing a single element, is isomorphic
to either que ``rational function field $k(t)/k$''
or to one of the field extensions $k[t]/(P)$ 
with $P \in k[t]$ irreducible.
(Stacks Project tag \href{https://stacks.math.columbia.edu/tag/09G1}{09G1}.)

\section{Modules}
\label{section-modules}

It looks like most books (at least
\cite{eis} and \cite{Samuel-Zariski-Vol1})
define a module to be an abelian group
along with a certain operation with
elements of the ring.

Which secretly just says that

\begin{definition}
\label{definition-module}
Let $R$ be a ring.
An {\it (left) $R$-module} $M$ is an abelian group
such that there exists a (left) ring morphism
$$
R \to \text{End}(M).
$$
\end{definition}

\noindent
Indeed, the three usual requirements
correspond to:

\begin{enumerate}
\item The endomorphism corresponding to $a \in R$ 
respect the group structure of the group $M$:
$$
a(x+y)=ax+ay,
$$

\item The representation map is a map of rings
$$
(a+b)x=ax+by,\qquad (ab)x=a(bx).
$$
\end{enumerate}

\noindent


\section{Algebras}
\label{section-algebras}

\noindent
There appears to be no definition of ``algebra''
in Stacks Project.
But there is one in \cite{Eisenbud}:

\begin{definition}
\label{definition-algebra}
If $R$ is a commutative ring,
then a {\it commutative algebra} over $R$ 
is a commutative ring $S$ together
with a ring morphism $R \to S$.
\end{definition}

\noindent
But actually I was expecting that 
$S$ would be defined as 
a ring that is also an $R$-module.
So let us note that a morphism of rings
$R \to S$ gives a representation
$R \to \text{End}(S)$ via left multiplication.
But the other way around,
given an endomorphism associated
to some element in $R$,
how do we assign an element of $S$
so as to produce a map $R \to S$?

\section{Finitely-generated and finitely presented algebras}
\label{section-finitely-presented-generated}

\noindent
See Stacks Project tag \href{https://stacks.math.columbia.edu/tag/00F2}{00F2}.

Upshot: a finitely-generated $R$-algebra $S$ is such that
$S \simeq R[x_1,\ldots,x_n]/I$ for some ideal $I \subset R$.
Finitely-presentedness is when $I = (s_1,\ldots,s_k)$.

Fun fact: The second item in the following definition
is the algebraic counterpart to an
affine algebraic set (variety);
i.e. the reason why we hear that
``affine varieties are in correspondence
with finitely presented $k$-algebras''
(wasn't there a notion of reduced algebra in that phrase…?)
And the difference with that and finite type,
I think,
is that the kernel is finitely generated.

\begin{definition}
\label{definition-finite-type}
Let $R \to S$ be a ring map.
\begin{enumerate}
\item We say $R \to S$ is of {\it finite type}, or that {\it $S$ is a finite
type $R$-algebra} if there exist an $n \in \mathbf{N}$ and an surjection
of $R$-algebras $R[x_1, \ldots, x_n] \to S$.
\item We say $R \to S$ is of {\it finite presentation} if there
exist integers $n, m \in \mathbf{N}$ and polynomials
$f_1, \ldots, f_m \in R[x_1, \ldots, x_n]$
and an isomorphism of $R$-algebras
$R[x_1, \ldots, x_n]/(f_1, \ldots, f_m) \cong S$.
\end{enumerate}
\end{definition}

Informally, $M$ is a finitely presented $R$-module if and only if
it is finitely generated and the module of relations among these
generators is finitely generated as well.
A choice of an exact sequence as in the definition is called a
{\it presentation} of $M$.

So, probably an ``algebra'' $A$ over a ring  $R$ is when $A$ contains 
(or, is isomorphic to?)  $R[X]$ for some possibly very arbitrary set
 $X\subset R$.


\section{Finite and integral ring extensions}
\label{section-finite-ring-extensions}

\noindent
The upshot is: let $\varphi:R \to S$ be a ring map.
An element $s \in S$ is integral over $R$ if there
is a monic polynomial with coefficients in $R$
that gives zero when you put $s$ instead of the variable.

But it might be convenient to think about
finite (or, for humans, finitely-generated) $R$-modules:
$x$ is integral over $R$ if and only if $R[x]$
is a finitely-generated  $R$-module (i.e.,
not only finitely generated as an algebra
but also as a module).
See \href{
https://youtu.be/5fCHj80BQHw?si=o7tMC4tRsbBnwmJi}{Zvi Rosen's video}.
The forward implication is: suppose $x$ is integral,
then $x^n+r_1x^{n-1}+\ldots+r_n=0$, so
$x^n=-r_1x^{n-1}+\ldots+r_n$,
which ``says'' $R[x] \cong \bigoplus_{k=0}^{n-1}Rx^k$.

So maybe the details of that are not entirely trivial (Zvi uses some theorem,
and Stacks Project does lots of things).
But the point is: I think that after applying Spec,
the fibers of an integral extension are {\bf finite}.

Ok, now I go to Stacks Project:

\medskip\noindent
Trivial lemmas concerning finite and integral ring maps.
We recall the definition.

\begin{definition}
\label{definition-integral-ring-map}
Let $\varphi : R \to S$ be a ring map.
\begin{enumerate}
\item An element $s \in S$
is {\it integral over $R$} if there exists a monic
polynomial $P(x) \in R[x]$ such that
$P^\varphi(s) = 0$, where $P^\varphi(x) \in S[x]$
is the image of $P$ under $\varphi : R[x] \to S[x]$.
\item  The ring map $\varphi$ is {\it integral}
if every $s \in S$ is integral over $R$.
\end{enumerate}
\end{definition}

\begin{lemma}
\label{lemma-characterize-integral-element}
Let $\varphi : R \to S$ be a ring map. Let $y \in S$. If there exists a
finite $R$-submodule $M$ of $S$ such that $1 \in M$ and $yM \subset M$,
then $y$ is integral over $R$.
\end{lemma}

\begin{proof}
Consider the map $\varphi : M \to M$, $x \mapsto y \cdot x$.
By Lemma \ref{lemma-charpoly-module} there exists a monic polynomial
$P \in R[T]$ with $P(\varphi) = 0$. In the ring $S$ we get
$P(y) = P(y) \cdot 1 = P(\varphi)(1) = 0$.
\end{proof}

\begin{lemma}
\label{lemma-finite-is-integral}
A finite ring map is integral.
\end{lemma}

\begin{proof}
Let $R \to S$ be finite. Let $y \in S$. Apply
Lemma \ref{lemma-characterize-integral-element}
to $M = S$ to see that $y$ is integral over $R$.
\end{proof}

\begin{lemma}
\label{lemma-characterize-integral}
Let $\varphi : R \to S$ be a ring map. Let $s_1, \ldots, s_n$
be a finite set of elements of $S$.
In this case $s_i$ is integral over $R$ for all $i = 1, \ldots, n$
if and only if
there exists an $R$-subalgebra $S' \subset S$ finite over $R$
containing all of the $s_i$.
\end{lemma}

\begin{proof}
If each $s_i$ is integral, then the subalgebra
generated by $\varphi(R)$ and the $s_i$ is finite
over $R$. Namely, if $s_i$ satisfies a monic equation
of degree $d_i$ over $R$, then this subalgebra is generated as an
$R$-module by the elements $s_1^{e_1} \ldots s_n^{e_n}$
with $0 \leq e_i \leq d_i - 1$.
Conversely, suppose given a finite $R$-subalgebra
$S'$ containing all the $s_i$. Then all of the
$s_i$ are integral by Lemma \ref{lemma-finite-is-integral}.
\end{proof}

\begin{lemma}
\label{lemma-characterize-finite-in-terms-of-integral}
Let $R \to S$ be a ring map. The following are equivalent
\begin{enumerate}
\item $R \to S$ is finite,
\item $R \to S$ is integral and of finite type, and
\item there exist $x_1, \ldots, x_n \in S$ which generate $S$ as an
algebra over $R$ such that each $x_i$ is integral over $R$.
\end{enumerate}
\end{lemma}

\begin{proof}
Clear from Lemma \ref{lemma-characterize-integral}.
\end{proof}

\begin{lemma}
\label{lemma-integral-transitive}
\begin{slogan}
A composition of integral ring maps is integral
\end{slogan}
Suppose that $R \to S$ and $S \to T$ are integral
ring maps. Then $R \to T$ is integral.
\end{lemma}

\begin{proof}
Let $t \in T$. Let $P(x) \in S[x]$ be a
monic polynomial such that $P(t) = 0$.
Apply Lemma \ref{lemma-characterize-integral}
to the finite set of coefficients of $P$.
Hence $t$ is integral over some subalgebra
$S' \subset S$ finite over $R$. Apply Lemma
\ref{lemma-characterize-integral} again to find
a subalgebra $T' \subset T$ finite over $S'$ and
containing $t$. Lemma \ref{lemma-finite-transitive}
applied to $R \to S' \to T'$ shows that $T'$ is finite
over $R$. The integrality of $t$ over $R$
now follows from Lemma \ref{lemma-finite-is-integral}.
\end{proof}

\begin{lemma}
\label{lemma-integral-closure-is-ring}
Let $R \to S$ be a ring homomorphism.
The set
$$
S' = \{s \in S \mid s\text{ is integral over }R\}
$$
is an $R$-subalgebra of $S$.
\end{lemma}

\begin{proof}
This is clear from Lemmas \ref{lemma-characterize-integral}
and \ref{lemma-finite-is-integral}.
\end{proof}

\begin{lemma}
\label{lemma-finite-product-integral}
Let $R_i\to S_i$ be ring maps $i = 1, \ldots, n$.
Let $R$ and $S$ denote the product of the $R_i$ and $S_i$ respectively.
Then an element $s = (s_1, \ldots, s_n) \in S$ is integral over $R$
if and only if each $s_i$ is integral over $R_i$.
\end{lemma}

\begin{proof}
Omitted.
\end{proof}

\begin{definition}
\label{definition-integral-closure}
Let $R \to S$ be a ring map.
The ring $S' \subset S$ of elements integral over
$R$, see Lemma \ref{lemma-integral-closure-is-ring},
is called the {\it integral closure} of $R$
in $S$. If $R \subset S$ we say that $R$ is
{\it integrally closed} in $S$ if $R = S'$.
\end{definition}

\noindent
In particular, we see that $R \to S$ is integral if and only
if the integral closure of $R$ in $S$ is all of $S$.

\begin{lemma}
\label{lemma-finite-product-integral-closure}
Let $R_i\to S_i$ be ring maps $i = 1, \ldots, n$.
Denote the integral closure of $R_i$ in $S_i$ by $S'_i$.
Further let $R$ and $S$ denote the product of the $R_i$ and $S_i$ respectively.
Then the integral closure of $R$ in $S$
is the product of the $S'_i$. In particular $R \to S$ is
integrally closed if and only if each $R_i \to S_i$ is integrally closed.
\end{lemma}

\begin{proof}
This follows immediately from Lemma \ref{lemma-finite-product-integral}.
\end{proof}

\begin{lemma}
\label{lemma-integral-closure-localize}
Integral closure commutes with localization: If $A \to B$ is a ring
map, and $S \subset A$ is a multiplicative subset, then the integral
closure of $S^{-1}A$ in $S^{-1}B$ is $S^{-1}B'$, where $B' \subset B$
is the integral closure of $A$ in $B$.
\end{lemma}

\begin{proof}
Since localization is exact we see that $S^{-1}B' \subset S^{-1}B$.
Suppose $x \in B'$ and $f \in S$. Then
$x^d + \sum_{i = 1, \ldots, d} a_i x^{d - i} = 0$
in $B$ for some $a_i \in A$. Hence also
$$
(x/f)^d + \sum\nolimits_{i = 1, \ldots, d} a_i/f^i (x/f)^{d - i} = 0
$$
in $S^{-1}B$. In this way we see that $S^{-1}B'$ is contained in
the integral closure of $S^{-1}A$ in $S^{-1}B$. Conversely, suppose
that $x/f \in S^{-1}B$ is integral over $S^{-1}A$. Then we have
$$
(x/f)^d + \sum\nolimits_{i = 1, \ldots, d} (a_i/f_i) (x/f)^{d - i} = 0
$$
in $S^{-1}B$ for some $a_i \in A$ and $f_i \in S$. This means that
$$
(f'f_1 \ldots f_d x)^d +
\sum\nolimits_{i = 1, \ldots, d}
f^i(f')^if_1^i \ldots f_i^{i - 1} \ldots f_d^i a_i
(f'f_1 \ldots f_dx)^{d - i} = 0
$$
for a suitable $f' \in S$. Hence $f'f_1\ldots f_dx \in B'$ and thus
$x/f \in S^{-1}B'$ as desired.
\end{proof}

\begin{lemma}
\label{lemma-integral-closure-stalks}
\begin{slogan}
An element of an algebra over a ring is integral over the ring
if and only if it is locally integral at every prime ideal of the ring.
\end{slogan}
Let $\varphi : R \to S$ be a ring map.
Let $x \in S$. The following are equivalent:
\begin{enumerate}
\item $x$ is integral over $R$, and
\item for every prime ideal $\mathfrak p \subset R$ the element
$x \in S_{\mathfrak p}$ is integral over $R_{\mathfrak p}$.
\end{enumerate}
\end{lemma}

\begin{proof}
It is clear that (1) implies (2). Assume (2). Consider the $R$-algebra
$S' \subset S$ generated by $\varphi(R)$ and $x$. Let $\mathfrak p$ be
a prime ideal of $R$. Then we know that
$x^d + \sum_{i = 1, \ldots, d} \varphi(a_i) x^{d - i} = 0$
in $S_{\mathfrak p}$ for some $a_i \in R_{\mathfrak p}$. Hence we see,
by looking at which denominators occur, that
for some $f \in R$, $f \not \in \mathfrak p$ we have
$a_i \in R_f$ and
$x^d + \sum_{i = 1, \ldots, d} \varphi(a_i) x^{d - i} = 0$
in $S_f$. This implies that $S'_f$ is finite over $R_f$.
Since $\mathfrak p$ was arbitrary and $\Spec(R)$ is quasi-compact
(Lemma \ref{lemma-quasi-compact}) we can find finitely many elements
$f_1, \ldots, f_n \in R$
which generate the unit ideal of $R$ such that $S'_{f_i}$ is finite
over $R_{f_i}$. Hence we conclude from Lemma \ref{lemma-cover} that
$S'$ is finite over $R$. Hence $x$ is integral over $R$ by
Lemma \ref{lemma-characterize-integral}.
\end{proof}

\begin{lemma}
\label{lemma-base-change-integral}
\begin{slogan}
Integrality and finiteness are preserved under base change.
\end{slogan}
Let $R \to S$ and $R \to R'$ be ring maps.
Set $S' = R' \otimes_R S$.
\begin{enumerate}
\item If $R \to S$ is integral so is $R' \to S'$.
\item If $R \to S$ is finite so is $R' \to S'$.
\end{enumerate}
\end{lemma}

\begin{proof}
We prove (1).
Let $s_i \in S$ be generators for $S$ over $R$.
Each of these satisfies a monic polynomial equation $P_i$
over $R$. Hence the elements $1 \otimes s_i \in S'$ generate
$S'$ over $R'$ and satisfy the corresponding polynomial
$P_i'$ over $R'$. Since these elements generate $S'$ over $R'$
we see that $S'$ is integral over $R'$.
Proof of (2) omitted.
\end{proof}

\begin{lemma}
\label{lemma-integral-local}
Let $R \to S$ be a ring map.
Let $f_1, \ldots, f_n \in R$ generate the unit ideal.
\begin{enumerate}
\item If each $R_{f_i} \to S_{f_i}$ is integral, so is $R \to S$.
\item If each $R_{f_i} \to S_{f_i}$ is finite, so is $R \to S$.
\end{enumerate}
\end{lemma}

\begin{proof}
Proof of (1).
Let $s \in S$. Consider the ideal $I \subset R[x]$ of
polynomials $P$ such that $P(s) = 0$. Let $J \subset R$
denote the ideal (!) of leading coefficients of elements of $I$.
By assumption and clearing denominators
we see that $f_i^{n_i} \in J$ for all $i$
and certain $n_i \geq 0$. Hence $J$ contains $1$ and we see
$s$ is integral over $R$. Proof of (2) omitted.
\end{proof}

\begin{lemma}
\label{lemma-integral-permanence}
Let $A \to B \to C$ be ring maps.
\begin{enumerate}
\item If $A \to C$ is integral so is $B \to C$.
\item If $A \to C$ is finite so is $B \to C$.
\end{enumerate}
\end{lemma}

\begin{proof}
Omitted.
\end{proof}

\begin{lemma}
\label{lemma-integral-closure-transitive}
Let $A \to B \to C$ be ring maps.
Let $B'$ be the integral closure of $A$ in $B$,
let $C'$ be the integral closure of $B'$ in $C$. Then
$C'$ is the integral closure of $A$ in $C$.
\end{lemma}

\begin{proof}
Omitted.
\end{proof}

\begin{lemma}
\label{lemma-integral-overring-surjective}
Suppose that $R \to S$ is an integral
ring extension with $R \subset S$.
Then $\varphi : \Spec(S) \to \Spec(R)$
is surjective.
\end{lemma}

\begin{proof}
Let $\mathfrak p \subset R$ be a prime ideal.
We have to show $\mathfrak pS_{\mathfrak p} \not = S_{\mathfrak p}$, see
Lemma \ref{lemma-in-image}.
The localization $R_{\mathfrak p} \to S_{\mathfrak p}$ is injective
(as localization is exact) and integral by
Lemma \ref{lemma-integral-closure-localize} or
\ref{lemma-base-change-integral}.
Hence we may replace $R$, $S$ by $R_{\mathfrak p}$, $S_{\mathfrak p}$ and
we may assume $R$ is local with maximal ideal $\mathfrak m$ and
it suffices to show that $\mathfrak mS \not = S$.
Suppose $1 = \sum f_i s_i$ with $f_i \in \mathfrak m$
and $s_i \in S$ in order to get a contradiction.
Let $R \subset S' \subset S$
be such that $R \to S'$ is finite and $s_i \in S'$, see
Lemma \ref{lemma-characterize-integral}.
The equation $1 = \sum f_i s_i$ implies that
the finite $R$-module $S'$ satisfies $S' = \mathfrak m S'$. Hence by
Nakayama's Lemma \ref{lemma-NAK}
we see $S' = 0$. Contradiction.
\end{proof}

\begin{lemma}
\label{lemma-integral-under-field}
Let $R$ be a ring. Let $K$ be a field.
If $R \subset K$ and $K$ is integral over $R$,
then $R$ is a field and $K$ is an algebraic extension.
If $R \subset K$ and $K$ is finite over $R$,
then $R$ is a field and $K$ is a finite algebraic extension.
\end{lemma}

\begin{proof}
Assume that $R \subset K$ is integral.
By Lemma \ref{lemma-integral-overring-surjective} we see that
$\Spec(R)$ has $1$ point. Since clearly $R$ is a domain we see
that $R = R_{(0)}$ is a field (Lemma \ref{lemma-minimal-prime-reduced-ring}).
The other assertions are immediate from this.
\end{proof}

\begin{lemma}
\label{lemma-integral-over-field}
Let $k$ be a field. Let $S$ be a $k$-algebra over $k$.
\begin{enumerate}
\item If $S$ is a domain and finite dimensional over $k$,
then $S$ is a field.
\item If $S$ is integral over $k$ and a domain,
then $S$ is a field.
\item If $S$ is integral over $k$ then every prime of
$S$ is a maximal ideal (see
Lemma \ref{lemma-ring-with-only-minimal-primes}
for more consequences).
\end{enumerate}
\end{lemma}

\begin{proof}
The statement on primes follows from the statement
``integral $+$ domain $\Rightarrow$ field''.
Let $S$ integral over $k$ and assume $S$ is a domain,
Take $s \in S$. By Lemma
\ref{lemma-characterize-integral} we may find a
finite dimensional $k$-subalgebra $k \subset S' \subset S$
containing $s$. Hence $S$ is a field if we can prove the
first statement. Assume $S$ finite dimensional
over $k$ and a domain. Pick $s\in S$.
Since $S$ is a domain the multiplication
map $s : S \to S$ is surjective by dimension
reasons. Hence there exists an element $s_1 \in S$
such that $ss_1 = 1$. So $S$ is a field.
\end{proof}

\begin{lemma}
\label{lemma-integral-no-inclusion}
Suppose $R \to S$ is integral.
Let $\mathfrak q, \mathfrak q' \in \Spec(S)$
be distinct primes
having the same image in $\Spec(R)$.
Then neither $\mathfrak q \subset \mathfrak q'$
nor $\mathfrak q' \subset \mathfrak q$.
\end{lemma}

\begin{proof}
Let $\mathfrak p \subset R$ be the image.
By Remark \ref{remark-fundamental-diagram}
the primes $\mathfrak q, \mathfrak q'$
correspond to ideals in
$S \otimes_R \kappa(\mathfrak p)$.
Thus the lemma follows from Lemma \ref{lemma-integral-over-field}.
\end{proof}

\begin{lemma}
\label{lemma-finite-finite-fibres}
Suppose $R \to S$ is finite.
Then the fibres of $\Spec(S) \to \Spec(R)$ are finite.
\end{lemma}

\begin{proof}
By the discussion in
Remark \ref{remark-fundamental-diagram}
the fibres are the spectra of the rings $S \otimes_R \kappa(\mathfrak p)$.
As $R \to S$ is finite, these fibre rings are finite over
$\kappa(\mathfrak p)$ hence Noetherian by
Lemma \ref{lemma-Noetherian-permanence}.
By
Lemma \ref{lemma-integral-no-inclusion}
every prime of $S \otimes_R \kappa(\mathfrak p)$ is a minimal
prime. Hence by
Lemma \ref{lemma-Noetherian-irreducible-components}
there are at most finitely many.
\end{proof}

\begin{lemma}
\label{lemma-integral-going-up}
Let $R \to S$ be a ring map such that
$S$ is integral over $R$.
Let $\mathfrak p \subset \mathfrak p' \subset R$
be primes. Let $\mathfrak q$ be a prime of $S$ mapping
to $\mathfrak p$. Then there exists a prime $\mathfrak q'$
with $\mathfrak q \subset \mathfrak q'$
mapping to $\mathfrak p'$.
\end{lemma}

\begin{proof}
We may replace $R$ by $R/\mathfrak p$ and $S$ by $S/\mathfrak q$.
This reduces us to the situation of having an integral
extension of domains $R \subset S$ and a prime $\mathfrak p' \subset R$.
By Lemma \ref{lemma-integral-overring-surjective} we win.
\end{proof}

\noindent
The property expressed in the lemma above is called
the ``going up property'' for the ring map $R \to S$,
see Definition \ref{definition-going-up-down}.

\begin{lemma}
\label{lemma-finite-finitely-presented-extension}
Let $R \to S$ be a finite and finitely presented ring map.
Let $M$ be an $S$-module.
Then $M$ is finitely presented as an $R$-module if and only if
$M$ is finitely presented as an $S$-module.
\end{lemma}

\begin{proof}
One of the implications follows from
Lemma \ref{lemma-finitely-presented-over-subring}.
To see the other assume that $M$ is finitely presented as an $S$-module.
Pick a presentation
$$
S^{\oplus m} \longrightarrow
S^{\oplus n} \longrightarrow
M \longrightarrow 0
$$
As $S$ is finite as an $R$-module, the kernel of
$S^{\oplus n} \to M$ is a finite $R$-module. Thus from
Lemma \ref{lemma-extension}
we see that it suffices to prove that $S$ is finitely presented as an
$R$-module.

\medskip\noindent
Pick $y_1, \ldots, y_n \in S$ such that $y_1, \ldots, y_n$ generate $S$
as an $R$-module. By Lemma \ref{lemma-characterize-integral-element}
each $y_i$ is integral over $R$. Choose monic polynomials
$P_i(x) \in R[x]$ with $P_i(y_i) = 0$. Consider the ring
$$
S' = R[x_1, \ldots, x_n]/(P_1(x_1), \ldots, P_n(x_n))
$$
Then we see that $S$ is of finite presentation as an $S'$-algebra
by Lemma \ref{lemma-compose-finite-type}. Since $S' \to S$ is surjective,
the kernel $J = \Ker(S' \to S)$ is finitely generated as an ideal by
Lemma \ref{lemma-finite-presentation-independent}. Hence $J$ is a finite
$S'$-module (immediate from the definitions).
Thus $S = \Coker(J \to S')$ is  of finite presentation as an $S'$-module
by Lemma \ref{lemma-extension}.
Hence, arguing as in the first paragraph, it suffices to show that $S'$ is
of finite presentation as an $R$-module. Actually, $S'$ is free as an
$R$-module with basis the monomials $x_1^{e_1} \ldots x_n^{e_n}$
for $0 \leq e_i < \deg(P_i)$. Namely, write $R \to S'$ as the composition
$$
R \to R[x_1]/(P_1(x_1)) \to R[x_1, x_2]/(P_1(x_1), P_2(x_2)) \to
\ldots \to S'
$$
This shows that the $i$th ring in this sequence is free as a module over the
$(i - 1)$st one with basis $1, x_i, \ldots, x_i^{\deg(P_i) - 1}$. The result
follows easily from this by induction. Some details omitted.
\end{proof}

\begin{lemma}
\label{lemma-silly-normal}
Let $R$ be a ring. Let $x, y \in R$ be nonzerodivisors.
Let $R[x/y] \subset R_{xy}$ be the $R$-subalgebra generated
by $x/y$, and similarly for the subalgebras $R[y/x]$ and $R[x/y, y/x]$.
If $R$ is integrally closed in $R_x$ or $R_y$, then the sequence
$$
0 \to R \xrightarrow{(-1, 1)} R[x/y] \oplus R[y/x] \xrightarrow{(1, 1)}
R[x/y, y/x] \to 0
$$
is a short exact sequence of $R$-modules.
\end{lemma}

\begin{proof}
Since $x/y \cdot y/x = 1$ it is clear that the map
$R[x/y] \oplus R[y/x] \to R[x/y, y/x]$ is surjective.
Let $\alpha \in R[x/y] \cap R[y/x]$. To show exactness in the middle
we have to prove that $\alpha \in R$. By assumption we may write
$$
\alpha = a_0 + a_1 x/y + \ldots + a_n (x/y)^n =
b_0 + b_1 y/x + \ldots + b_m(y/x)^m
$$
for some $n, m \geq 0$ and $a_i, b_j \in R$.
Pick some $N > \max(n, m)$.
Consider the finite $R$-submodule $M$ of $R_{xy}$ generated by the elements
$$
(x/y)^N, (x/y)^{N - 1}, \ldots, x/y, 1, y/x, \ldots, (y/x)^{N - 1}, (y/x)^N
$$
We claim that $\alpha M \subset M$. Namely, it is clear that
$(x/y)^i (b_0 + b_1 y/x + \ldots + b_m(y/x)^m) \in M$ for
$0 \leq i \leq N$ and that
$(y/x)^i (a_0 + a_1 x/y + \ldots + a_n(x/y)^n) \in M$ for
$0 \leq i \leq N$. Hence $\alpha$ is integral over $R$ by
Lemma \ref{lemma-characterize-integral-element}. Note that
$\alpha \in R_x$, so if $R$ is integrally closed in $R_x$
then $\alpha \in R$ as desired.
\end{proof}










\bibliography{my}
\bibliographystyle{amsalpha}

\end{document}

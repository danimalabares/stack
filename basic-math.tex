\input{preamble}

\begin{document}

\title{Basic math}
\maketitle

\phantomsection
\label{section-phantom}

\tableofcontents

\section{Signature}
\label{section-signature}

The signature of a form quadratic form can be equivalently computed as the
maximal dimension of an isotropic subspace, or as the maximal dimension of the
subspace where the form is negative.

\section{Arzelà-Ascoli Theorem}
\label{section-Arzela-Ascoli-theorem}
\begin{definition}[Equicontinuous family]
\label{definition-equicontinuous}
From Wikipedia. Let $X$ be a compact Hausdorff space and let  $C(X)$ be the
space of real-valued continuous functions on $X$. A subset $\mathcal{F} \subset
C(X)$ is said to be {\it equicontinuous} if for every $x in X$ and every
$\varepsilon>0$, $x$ has a neighbourhood $U_x$ such that
$$
\forall y \in U_x, \forall f \in \mathcal{F}: \qquad |f(y)-f(x)|<\varepsilon
$$
\end{definition}

\begin{definition}[Pointwise bounded]
\label{definition-pointwise-bounded}
A set $\mathcal{F}\subset C(X,\mathbb{R})$ is said to be {\it pointwise
bounded} if for every $x \in X$,
$$
\text{sup}\{|f(x)|:f \in \mathcal{F}\}<\infty
$$
\end{definition}

\begin{theorem}[Arzelà-Ascoli]
\label{theorem-Arzela-Ascoli}
Let $X$ be a compact Hausdorff space. Then a subset $\mathcal{F}$ of $C(X)$ is
relatively compact in the topology induced by the uniform norm if and only if it
is equicontinuous and pointwise bounded.
\end{theorem}

\bibliography{my}
\bibliographystyle{amsalpha}

\end{document}


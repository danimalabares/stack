\input{preamble}

\begin{document}

\title{Basic math}
\maketitle

\phantomsection
\label{section-phantom}

\tableofcontents

\section{Trace}
\label{section-trace}

\begin{proposition}
\label{proposition-trace-is-coordinate-independent}

\end{proposition}

\section{Signature}
\label{section-signature}

The signature of a form quadratic form can be equivalently computed as the
maximal dimension of an isotropic subspace, or as the maximal dimension of the
subspace where the form is negative.

\section{Adjoint}
\label{section-adjoint}

\begin{definition}
\label{definition-adjoint}
Let  $A:(V,\left<\cdot,\cdot\right>_V) \to (W,\left<\cdot,\cdot\right>_W)$ be a 
linear transformation of inner product vector spaces, its {\it adjoint} is the
map
$$
A^{\dag}:W\to V
$$
satisfying
\begin{equation}
\label{equation-adjoint}
\left<Av,w\right>_W=\left<v,A^\dag w\right>_V
\end{equation}
\end{definition}

Let $G$ be a matrix representation for $\left<\cdot,\cdot\right>_V$ and $H$ for
$\left<\cdot,\cdot\right>_W$. Then Eq. \ref{equation-adjoint} is
$$
v^{\mathbf{T}}A^{\mathbf{T}}Hw=(Av)^{\mathbf{T}}Hw=v^{\mathbf{T}}GA^\dag w
$$
So
\begin{equation}
\label{equation-adjoint-transpose}
A^{\mathbf{T}}H=GA^\dag \iff A^\dag:=G^{-1}A^{\mathbf{T}}H
\end{equation}
which gives the relationship between the usual transpose matrix and the adjoint.

\begin{definition}
\label{definition-Frobenius-norm}
In the space of matrices we define {\it Frobenius norm} as
\begin{equation}
\label{equation-Frobenius-norm}
\|A\|^2=\text{tr}A^\dag A=\text{tr}(G^{-1}A^{\mathbf{T}}HA
\end{equation}
\end{definition}

\section{Fréchet derivative}
\label{section-Fréchet-derivative}

\begin{definition}
\label{definition-Frechet-derivative}
\cite{zo1}, Section 8.2.1, Definition 1. A function $f:E \to \mathbb{R}^n$
defined on a set $E \subset \mathcal{R}^m$ is {\it differentiable} at the point
$x \in E$, which is a limit point of $E$, if
$$
f(x+h)-f(x)=L(x)h+\alpha(x;h)
$$
where $L(x):\mathbb{R}^m\to \mathbb{R}^n$ is a function that is linear in $h$
and $\alpha(x;h)=o(h)$ as $h\to 0$, $x+h\in E$.
\end{definition}

This means that (\href{Alguém sabe um jeito bom de mostrar A~(1\s) * Id
?}{Wikipedia})
$$
\lim_{\|h\|\to 0} \frac{\|f(x+h)-f(x)-Ah\|}{\|h\|}=0
$$

\section{Arzelà-Ascoli Theorem}
\label{section-Arzela-Ascoli-theorem}
\begin{definition}[Equicontinuous family]
\label{definition-equicontinuous}
From Wikipedia. Let $X$ be a compact Hausdorff space and let  $C(X)$ be the
space of real-valued continuous functions on $X$. A subset $\mathcal{F} \subset
C(X)$ is said to be {\it equicontinuous} if for every $x in X$ and every
$\varepsilon>0$, $x$ has a neighbourhood $U_x$ such that
$$
\forall y \in U_x, \forall f \in \mathcal{F}: \qquad |f(y)-f(x)|<\varepsilon
$$
\end{definition}

\begin{definition}[Pointwise bounded]
\label{definition-pointwise-bounded}
A set $\mathcal{F}\subset C(X,\mathbb{R})$ is said to be {\it pointwise
bounded} if for every $x \in X$,
$$
\text{sup}\{|f(x)|:f \in \mathcal{F}\}<\infty
$$
\end{definition}

\begin{theorem}[Arzelà-Ascoli]
\label{theorem-Arzela-Ascoli}
Let $X$ be a compact Hausdorff space. Then a subset $\mathcal{F}$ of $C(X)$ is
relatively compact in the topology induced by the uniform norm if and only if it
is equicontinuous and pointwise bounded.
\end{theorem}

\bibliography{my}
\bibliographystyle{amsalpha}

\end{document}


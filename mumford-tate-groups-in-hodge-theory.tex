\input{preamble}
\begin{document}

\title{Mumford-Tate groups in Hodge theory}
\maketitle

\noindent
Minicourse by (Andrey Soldatenkov, UNICAMP, Brazil), 
IMPA Summer School, 2026.

\medskip\noindent
Notes at 
\href{http://github.com/danimalabares/cimpa-floripa}
{github.com/danimalabares/cimpa-floripa}

\bigskip\noindent

\tableofcontents

\section{Plan}
\label{section-plan}

\begin{enumerate}
\item Motivation: cohomology of algebraic varieties.
\item Definition. Hodge structures, Mumford-Tate group.
\item Characterizations of the MT groups and relations with representation
theory.
\item Variations of Hodge structures and moduli spaces.
\item Dichotomy: abelian vs non-abelian HS.
\item The Kuga-Satake construction.
\end{enumerate}

\section{Introduction}
\label{section-intro}

$X\subset \mathbb{C}P^N$ smooth complex subvariety,
$\dim_\mathbb{C}=n$.
First recall we have singular cohomology,
$H^k(X,\mathbb{C})$,
which is isomorphic to the cohomology
of the constant sheaf $\underline{\mathbb{C}}_X$.
This cohomology is nonzero for $0\leq k \leq 2n$.

Recall. $U \subset X$ open,
$\Gamma(U,\mathbb{C})=\{f:U \to \mathbb{C}:
\text{$f$ is locally constant}\}
=\prod_{\pi_0(U)}\mathbb{C}$.

\begin{example}
\label{example-singular-cohomology}
\begin{enumerate}
\item 
$X=\mathbb{C}P^n$,
$$
H^k(\mathbb{C}P^n,\mathbb{C})=
\begin{cases}
  \mathbb{C}\qquad &k=2m, 0 \leq  m \leq  n \\
  0\qquad &\text{otherwise.} 
\end{cases}
$$
A way to prove this is using the CW decomposition of $\mathbb{C}P^n$.

\item $X \subset \mathbb{C}P^2$ hypersurface of degree $d$;
$X$ a Riemann surface of genus  $g=\frac{(d-1)(d-2)}{2}$.
Then
$H^0(X,\mathbb{C})=\mathbb{C}$,
$H^1(X,\mathbb{C})=\mathbb{C}^{2g}$,
$H^2(X,\mathbb{C})=\mathbb{C}$.
\end{enumerate}
\end{example}

\noindent
We also have the following additional data 
(a Hodge structure) on $H^k(X,\mathbb{C})$:
\begin{itemize}
\item A lattice $H^k(X,\mathbb{Z})/\text{torsion}\subset H^k(X,\mathbb{C})$,
\item A $(p,q)$-decomposition,
$H^k(X,\mathbb{C})=\bigoplus_{p+q=k}H^{p,q}(X)$.
\end{itemize}

Why is it useful?
\begin{itemize}
\item It gives restrictions on possible Betti numbers of algebraic varieties.
(So it may tell us that certain complex variety cannot be algebraic,
for example.)

\item It $f:X \to Y$ is a morphism of algebraic varieties,
then $f^*:H^k(Y,\mathbb{C}) \to H^k(X,\mathbb{C})$
preserves the Hodge structure.
\end{itemize}

\noindent
As an example of the latter statement,

\begin{example}
\label{example-preserving-hodge-structure}
Let $X \subset \mathbb{C}P^4$ be a ``general'' hypersurface
of degree 5. Then there exists no abelian variety
(i.e. a projective variety that is biholomorphic
to $\mathbb{C}P^N/\Lambda$ where $\Lambda$ is a lattice;
so, a complex torus that is also a projective variety)
$A$ that admits a dominant birational map onto $X$
$f:A\xymatrix{\ar@{-->}[r]&}X$.
\end{example}

\section{The p,q decomposition}
\label{section-p-q-decomposition}

\noindent
We use the de Rham complex.
Let $\Omega_X^k$ be the sheaf of holomorphic $k$-forms.
The de Rham complex is
$$
\Omega_{dR}^\bullet
=(0 \to \mathcal{O}_X \to \Omega_X^1\xrightarrow{d}\cdots
\to\Omega_X^n \to 0)
$$

$\Omega_{dR}^\bullet$ is a resolution of $\mathbb{C}_X$.
Therefore $H^k(X,\mathbb{C})\cong H^k(X,\Omega_{dR}^\bullet$.

Let's define a subcomplex:
$$
F^p\Omega_{dR}^\bullet=
(0 \to \cdots \to \Omega_X^p \to O_X^{p+1}
\to \cdots \to \Omega_X^n \to 0)
$$
It's a subcomplex $F^p\Omega_{dr}^\bullet \subset \Omega_{dR}^\bullet$ 
since the sheaves coincide when $F^p\Omega_{dR}$ are nonzero
and they inject otherwise (because it's the zero sheaf).

This gives
$H^k(X,F^p\Omega_{dR}^\bullet)\xrightarrow{(\ast)}
H^k(X,\Omega_{dR}^\bullet)=H^k(X,\mathbb{C})$.

\begin{definition}
\label{definition-hodge-filtration}
The {\it Hodge filtration} is
$F^pH^k(X)=\text{Im}(\ast)$.
\end{definition}

\noindent
Hodge theory tells us the map $(\ast)$ is in fact injective.

Let $\Lambda^{p,q}_X$ be the sheaf of $C^\infty$-forms
on $X$ of type $(p,q)$.
They are given locally by
$\sum_{\substack{|I|=p \\ |J|=q}}
\alpha_{I J}dz_I \wedge d\overline{z}_J$,
for $\alpha_{I J}\in C^\infty_X$.

Then we have an acyclic resolution of $\Omega_X^p$:
$$
0 \to \Omega_X^p \hookrightarrow
\Lambda^{p,0}_X
\xrightarrow{\bar{\partial}}
\Lambda^{p,1}_X
\xrightarrow{\overline{\partial}}
\cdots
\to
\Lambda^{p,n}_X\to 0.
$$

$\Lambda^{\bullet,\bullet}_X$ is a fine resolution of $\Omega_X^\bullet$.

Then $H^k(X,\mathbb{C})\cong H^k(X,\text{Tot}\Lambda^{\bullet,\bullet}_X$,
which can be computed by a spectral sequence.
The first page of such spectral sequence is given by
$E^{p,q}_1=H^q(X,\Omega_X^p)$.
This converges to $H^k(X,\mathbb{C})$.
This is the Hodge-to-de Rham spectral sequence.

\medskip\noindent
Since our manifolds are projective they admit a Kähler metric
$\omega$ induced by the inclusion $X \xrightarrow{i}\mathbb{C}P^N$,
that is, $\omega=i^* (\text{Fubini-Study metric on }\mathbb{C}P^N)$.

Then $\Lambda_X^k=\bigoplus_{p+q=k}\Lambda_X^{p,q}$,
$\Lambda^\bullet_X$ becomes an elliptic complex.

We have
$$
… \to \Lambda_X^{k-1}\xrightarrow{d} \Lambda_X^k\to …,
\qquad 
…\to\Lambda_X^k \xrightarrow{d^*}\Lambda_X^{k-1}\to…
$$
where $d^*$ is the adjoint of $d$ w.r.t. $\omega$.
Then $\Lambda=d d^*+d^*d$ is an elliptic operator.
$\mathcal{H}^k=\Ker(\Lambda|_{\Lambda^k_X})
=\Ker(d|_{\Lambda^k_X})\cap \Ker(d^*|_{\Lambda^k_X})$ 
are the harmonic forms.

Consider a natural map from the harmonic forms of type $(p,q)$ 
to the cohomology:
$$
\mathcal{H}^{p,q}=\mathcal{H}^k\cap\Lambda^{p,q}_X
\xrightarrow{(\ast \ast)} H^q(X,\Omega_X^p).
$$
Fact: since $d\omega=0$ ($X$ is Kähler),
$(\ast\ast)$ is an isomorphism.

This means that for Kähler manifolds
$$
\dim H^k(X,\mathbb{C})
\leq 
\sum_{p+q=k}H^q(X,\Omega_X^p)
\underbrace{\leq}_{(\ast \ast)}
\dim H^k(X,\mathbb{C})
$$
This implies that the Hodge-to-de Rham spectral sequence
degenerates at $E_1$. Therefore,
$$
H^k(X,\mathbb{C})=\mathcal{H}^k=
\bigoplus_{p+q=k}\mathcal{H}^{p,q}\cong H^q(X,\Omega_X^p).
$$

The Hodge filtration is
$$
F^pH^k(X,\mathbb{C})=\bigoplus_{\substack{p'+q'=k \\ p' \geq p}}
H^{p',q'}(X).
$$

\section{Symmetries of the p,q decomposition}
\label{section-symmetries-of-p-q-decomposition}

\begin{enumerate}
\item Since $\overline{\Lambda^{p,q}_X}=\Lambda^{q,p}_X$,
we have $\mathcal{H}^{q,p}=\overline{\mathcal{H}^{p,q}}$.
This means that if $k \equiv 1 (\text{mod }2)$,
$H^k(X,\mathbb{C})=\mathcal{H}^{k,0} \oplus … \oplus 
\mathcal{H}^{\frac{k+1}{2},\frac{k-1}{2}}
\oplus \mathcal{H}^{0,k}\oplus … \oplus 
\mathcal{H}^{\frac{k-1}{2},\frac{k+1}{2}}$.
This means that the $k$-th Betti number is even,
$b_k(X)\equiv 0 \text{mod }2$.

\item (Poincare duality.) We have a perfect pairing
\begin{align*}
H^k(X,\mathbb{C}) \otimes H^{2n-k}(X,\mathbb{C})
&\longrightarrow \mathbb{C} \\
[\alpha]\otimes[\beta] &\longmapsto 
\int_X\alpha\wedge\beta
\end{align*}

\noindent
Notice that if
$\alpha \in \mathcal{H}^{p,q}$ 
then $\beta \in \mathcal{H}^{n-p,n-q}$.
This induces a perfect pairing
$$
\mathcal{H}^{p,q}\otimes\mathcal{H}^{n-p,n-q}
\to \mathbb{C}.
$$

\item (Polarization and the Lefschetz operator.)
The polarization is the Kähler class of the Kähler form.
By $X$ being projective we have that the Kähler class is integral.
Moreover, it is the Poincar'e dual of the hyperplane section class.
That is, let $h \in H^2(\mathbb{C}P^n,\mathbb{Z})$
be the class of a hyperplane,
then $i^*h=[\omega] \in H^2(X,\mathbb{Z})$.
We have $\omega \in \mathcal{H}^{1,1}$ 
and $[\omega]\in H^2(X,\mathbb{Z}) \cap H^{1,1}(X)$.
The Lefschetz operator is
$$
L_\omega:H^{p,q}(X) \to H^{p+1,q+1}(X).
$$
\begin{align*}
L_\omega: H^{p,q}(X) &\longrightarrow H^{p+1,q+1}(X) \\
[\alpha] &\longmapsto [\alpha\wedge\omega]=[\alpha]\cup [\omega].
\end{align*}

\noindent
Lefschetz theorem says
\begin{enumerate}
\item $L_\omega^k:H^{n-k}(X,\mathbb{Q})\xrightarrow{\sim}
H^{n+k}(X,\mathbb{Q})$
is an isomorphism for $0\leq k\leq n$

\item The dual of $L_\omega$ is
\begin{align*}
\Lambda_\omega: H^{p,q}(X) &\longrightarrow H^{p-1,q-1}(X) \\
[\alpha] &\longmapsto [i_\alpha \omega].
\end{align*}

$$
[L_\omega,\Lambda_\omega]=\Theta \in \text{End}(H^\bullet(X,\mathbb{Q}))
$$
$$
\Theta|_{H^k(X,\mathbb{Q})}
=(k-n)\text{Id}
$$
Then $L_\omega, \Lambda_\omega$ and $\Theta$
span a subalgebra of  $\text{End}(H^\bullet(X,\mathbb{Q}))$ 
isomorphic to $\mathfrak{sl}_2$.
This allows us to use what we know about the
representation theory of $\mathfrak{sl}_2$.
Let $H^k_{\text{prim}(X,\mathbb{Q})}(\Lambda|_{H^k(X,\mathbb{Q})}$.
Then $H^m(X,\mathbb{Q})=\bigoplus_{i \geq 0}L_\omega^i
H^{m-2i}_{\text{prim}}(X,\mathbb{Q})$
for $0 < m \leq n$.
(I think this corresponds to the usual
weight space decomposition.)

[Picture of Hodge diamond.
Reflection by vertical axis is complex conjugation,
180-degree rotation is Poincar'e duality,
$p+q=$constant is a horizontal line,
reflection along horizontal axis is Lefschetz theorem.
Warning! This depends on conventions of how we draw the diamond.]
\end{enumerate}
\end{enumerate}

\section{The Hodge-Riemann relations}
\label{section-hodge-riemann}

\noindent
For all $[\alpha]\in H^k_{\text{prim}}(X,\mathbb{C})\cap H^{p,q}(X)$ 
we have
$$
i^{p-q}(-1)^{\frac{k(k-1)}{2}}
\int_X \alpha \wedge \overline{\alpha}\wedge\omega^{n-k}>0.
$$
(Here $i$ is the imaginary unit.)

Define a pairing on $H^k_{\text{prim}}(X,\mathbb{C})$:
$$
\psi([\alpha],[\beta])
=(2\pi i)^{-k}
(-1)^{\frac{k(k-1)}{2}}
\int_X \alpha\wedge\beta\wedge\omega^{n-k}
$$
which is $(-1)^k$-symmetric.

\begin{definition}
\label{definition-weil-operator}
The {\it Weil operator} $C \in \text{End}(H^k(X,\mathbb{C}))$ is given by
$C|_{H^{p,q}}(X)=(i)^{p-q}\text{Id}$.
\end{definition}

\noindent
Let $Q([\alpha],[\beta])=(2\pi i)^k
\psi(C[\alpha],[\beta])$.

Then $Q$ is symmetric (exercise) and positive on $H^k(X,\mathbb{R})$.
Positive is just the Hodge-Riemann relation.





\end{document}

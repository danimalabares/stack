\input{preamble}
\begin{document}

\title{Mumford-Tate groups in Hodge theory}
\maketitle

\noindent
Minicourse by Andrey Soldatenkov, UNICAMP, Brazil, 
IMPA Summer School, 2026.

\tableofcontents

\section{Plan}
\label{section-plan}

\begin{enumerate}
\item Motivation: cohomology of algebraic varieties.
\item Definition. Hodge structures, Mumford-Tate group.
\item Characterizations of the MT groups and relations with representation
theory.
\item Variations of Hodge structures and moduli spaces.
\item Dichotomy: abelian vs non-abelian HS.
\item The Kuga-Satake construction.
\end{enumerate}

\section{Introduction}
\label{section-intro}

$X\subset \mathbb{C}P^N$ smooth complex subvariety,
$\dim_\mathbb{C}=n$.
First recall we have singular cohomology,
$H^k(X,\mathbb{C})$,
which is isomorphic to the cohomology
of the constant sheaf $\underline{\mathbb{C}}_X$.
This cohomology is nonzero for $0\leq k \leq 2n$.

Recall. $U \subset X$ open,
$\Gamma(U,\mathbb{C})=\{f:U \to \mathbb{C}:
\text{$f$ is locally constant}\}
=\prod_{\pi_0(U)}\mathbb{C}$.

\begin{example}
\label{example-singular-cohomology}
\begin{enumerate}
\item 
$X=\mathbb{C}P^n$,
$$
H^k(\mathbb{C}P^n,\mathbb{C})=
\begin{cases}
  \mathbb{C}\qquad &k=2m, 0 \leq  m \leq  n \\
  0\qquad &\text{otherwise.} 
\end{cases}
$$
A way to prove this is using the CW decomposition of $\mathbb{C}P^n$.

\item $X \subset \mathbb{C}P^2$ hypersurface of degree $d$;
$X$ a Riemann surface of genus  $g=\frac{(d-1)(d-2)}{2}$.
Then
$H^0(X,\mathbb{C})=\mathbb{C}$,
$H^1(X,\mathbb{C})=\mathbb{C}^{2g}$,
$H^2(X,\mathbb{C})=\mathbb{C}$.
\end{enumerate}
\end{example}

\noindent
We also have the following additional data 
(a Hodge structure) on $H^k(X,\mathbb{C})$:
\begin{itemize}
\item A lattice $H^k(X,\mathbb{Z})/\text{torsion}\subset H^k(X,\mathbb{C})$,
\item A $(p,q)$-decomposition,
$H^k(X,\mathbb{C})=\bigoplus_{p+q=k}H^{p,q}(X)$.
\end{itemize}

Why is it useful?
\begin{itemize}
\item It gives restrictions on possible Betti numbers of algebraic varieties.
(So it may tell us that certain complex variety cannot be algebraic,
for example.)

\item It $f:X \to Y$ is a morphism of algebraic varieties,
then $f^*:H^k(Y,\mathbb{C}) \to H^k(X,\mathbb{C})$
preserves the Hodge structure.
\end{itemize}

\noindent
As an example of the latter statement,

\begin{example}
\label{example-preserving-hodge-structure}
Let $X \subset \mathbb{C}P^4$ be a ``general'' hypersurface
of degree 5. Then there exists no abelian variety
(i.e. a projective variety that is biholomorphic
to $\mathbb{C}P^N/\Lambda$ where $\Lambda$ is a lattice;
so, a complex torus that is also a projective variety)
$A$ that admits a dominant birational map onto $X$
$f:A\xymatrix{\ar@{-->}[r]&}X$.
\end{example}

\section{The p,q decomposition}
\label{section-p-q-decomposition}

\noindent
We use the de Rham complex.
Let $\Omega_X^k$ be the sheaf of holomorphic $k$-forms.
The de Rham complex is
$$
\Omega_{dR}^\bullet
=(0 \to \mathcal{O}_X \to \Omega_X^1\xrightarrow{d}\cdots
\to\Omega_X^n \to 0)
$$

$\Omega_{dR}^\bullet$ is a resolution of $\mathbb{C}_X$.
Therefore $H^k(X,\mathbb{C})\cong H^k(X,\Omega_{dR}^\bullet$.

Let's define a subcomplex:
$$
F^p\Omega_{dR}^\bullet=
(0 \to \cdots \to \Omega_X^p \to O_X^{p+1}
\to \cdots \to \Omega_X^n \to 0)
$$
It's a subcomplex $F^p\Omega_{dr}^\bullet \subset \Omega_{dR}^\bullet$ 
since the sheaves coincide when $F^p\Omega_{dR}$ are nonzero
and they inject otherwise (because it's the zero sheaf).

This gives
$H^k(X,F^p\Omega_{dR}^\bullet)\xrightarrow{(\ast)}
H^k(X,\Omega_{dR}^\bullet)=H^k(X,\mathbb{C})$.

\begin{definition}
\label{definition-hodge-filtration}
The {\it Hodge filtration} is
$F^pH^k(X)=\text{Im}(\ast)$.
\end{definition}

\noindent
Hodge theory tells us the map $(\ast)$ is in fact injective.

Let $\Lambda^{p,q}_X$ be the sheaf of $C^\infty$-forms
on $X$ of type $(p,q)$.
They are given locally by
$\sum_{\substack{|I|=p \\ |J|=q}}
\alpha_{I J}dz_I \wedge d\overline{z}_J$,
for $\alpha_{I J}\in C^\infty_X$.

Then we have an acyclic resolution of $\Omega_X^p$:
$$
0 \to \Omega_X^p \hookrightarrow
\Lambda^{p,0}_X
\xrightarrow{\bar{\partial}}
\Lambda^{p,1}_X
\xrightarrow{\overline{\partial}}
\cdots
\to
\Lambda^{p,n}_X\to 0.
$$

$\Lambda^{\bullet,\bullet}_X$ is a fine resolution of $\Omega_X^\bullet$.

Then $H^k(X,\mathbb{C})\cong H^k(X,\text{Tot}\Lambda^{\bullet,\bullet}_X$,
which can be computed by a spectral sequence.
The first page of such spectral sequence is given by
$E^{p,q}_1=H^q(X,\Omega_X^p)$.
This converges to $H^k(X,\mathbb{C})$.
This is the Hodge-to-de Rham spectral sequence.

\medskip\noindent
Since our manifolds are projective they admit a Kähler metric
$\omega$ induced by the inclusion $X \xrightarrow{i}\mathbb{C}P^N$,
that is, $\omega=i^* (\text{Fubini-Study metric on }\mathbb{C}P^N)$.

Then $\Lambda_X^k=\bigoplus_{p+q=k}\Lambda_X^{p,q}$,
$\Lambda^\bullet_X$ becomes an elliptic complex.

We have
$$
… \to \Lambda_X^{k-1}\xrightarrow{d} \Lambda_X^k\to …,
\qquad 
…\to\Lambda_X^k \xrightarrow{d^*}\Lambda_X^{k-1}\to…
$$
where $d^*$ is the adjoint of $d$ w.r.t. $\omega$.
Then $\Lambda=d d^*+d^*d$ is an elliptic operator.
$\mathcal{H}^k=\Ker(\Lambda|_{\Lambda^k_X})
=\Ker(d|_{\Lambda^k_X})\cap \Ker(d^*|_{\Lambda^k_X})$ 
are the harmonic forms.

Consider a natural map from the harmonic forms of type $(p,q)$ 
to the cohomology:
$$
\mathcal{H}^{p,q}=\mathcal{H}^k\cap\Lambda^{p,q}_X
\xrightarrow{(\ast \ast)} H^q(X,\Omega_X^p).
$$
Fact: since $d\omega=0$ ($X$ is Kähler),
$(\ast\ast)$ is an isomorphism.

This means that for Kähler manifolds
$$
\dim H^k(X,\mathbb{C})
\leq 
\sum_{p+q=k}H^q(X,\Omega_X^p)
\underbrace{\leq}_{(\ast \ast)}
\dim H^k(X,\mathbb{C})
$$
This implies that the Hodge-to-de Rham spectral sequence
degenerates at $E_1$. Therefore,
$$
H^k(X,\mathbb{C})=\mathcal{H}^k=
\bigoplus_{p+q=k}\mathcal{H}^{p,q}\cong H^q(X,\Omega_X^p).
$$

The Hodge filtration is
$$
F^pH^k(X,\mathbb{C})=\bigoplus_{\substack{p'+q'=k \\ p' \geq p}}
H^{p',q'}(X).
$$

\section{Symmetries of the p,q decomposition}
\label{section-symmetries-of-p-q-decomposition}

\begin{enumerate}
\item Since $\overline{\Lambda^{p,q}_X}=\Lambda^{q,p}_X$,
we have $\mathcal{H}^{q,p}=\overline{\mathcal{H}^{p,q}}$.
This means that if $k \equiv 1 (\text{mod }2)$,
$H^k(X,\mathbb{C})=\mathcal{H}^{k,0} \oplus … \oplus 
\mathcal{H}^{\frac{k+1}{2},\frac{k-1}{2}}
\oplus \mathcal{H}^{0,k}\oplus … \oplus 
\mathcal{H}^{\frac{k-1}{2},\frac{k+1}{2}}$.
This means that the $k$-th Betti number is even,
$b_k(X)\equiv 0 \text{mod }2$.

\item (Poincare duality.) We have a perfect pairing
\begin{align*}
H^k(X,\mathbb{C}) \otimes H^{2n-k}(X,\mathbb{C})
&\longrightarrow \mathbb{C} \\
[\alpha]\otimes[\beta] &\longmapsto 
\int_X\alpha\wedge\beta
\end{align*}

\noindent
Notice that if
$\alpha \in \mathcal{H}^{p,q}$ 
then $\beta \in \mathcal{H}^{n-p,n-q}$.
This induces a perfect pairing
$$
\mathcal{H}^{p,q}\otimes\mathcal{H}^{n-p,n-q}
\to \mathbb{C}.
$$

\item (Polarization and the Lefschetz operator.)
The polarization is the Kähler class of the Kähler form.
By $X$ being projective we have that the Kähler class is integral.
Moreover, it is the Poincar'e dual of the hyperplane section class.
That is, let $h \in H^2(\mathbb{C}P^n,\mathbb{Z})$
be the class of a hyperplane,
then $i^*h=[\omega] \in H^2(X,\mathbb{Z})$.
We have $\omega \in \mathcal{H}^{1,1}$ 
and $[\omega]\in H^2(X,\mathbb{Z}) \cap H^{1,1}(X)$.
The Lefschetz operator is
\begin{align*}
L_\omega: H^{p,q}(X) &\longrightarrow H^{p+1,q+1}(X) \\
[\alpha] &\longmapsto [\alpha\wedge\omega]=[\alpha]\cup [\omega].
\end{align*}

\noindent
Lefschetz theorem says
\begin{enumerate}
\item $L_\omega^k:H^{n-k}(X,\mathbb{Q})\xrightarrow{\sim}
H^{n+k}(X,\mathbb{Q})$
is an isomorphism for $0\leq k\leq n$

\item The dual of $L_\omega$ is
\begin{align*}
\Lambda_\omega: H^{p,q}(X) &\longrightarrow H^{p-1,q-1}(X) \\
[\alpha] &\longmapsto [i_\alpha \omega].
\end{align*}

$$
[L_\omega,\Lambda_\omega]=\Theta \in \text{End}(H^\bullet(X,\mathbb{Q}))
$$
$$
\Theta|_{H^k(X,\mathbb{Q})}
=(k-n)\text{Id}
$$
Then $L_\omega, \Lambda_\omega$ and $\Theta$
span a subalgebra of  $\text{End}(H^\bullet(X,\mathbb{Q}))$ 
isomorphic to $\mathfrak{sl}_2$.
This allows us to use what we know about the
representation theory of $\mathfrak{sl}_2$.
Let $H^k_{\text{prim}}(X,\mathbb{Q})=(\Lambda|_{H^k(X,\mathbb{Q})})$.
Then $H^m(X,\mathbb{Q})=\bigoplus_{i \geq 0}L_\omega^i
H^{m-2i}_{\text{prim}}(X,\mathbb{Q})$
for $0 < m \leq n$.
(I think this corresponds to the usual
weight space decomposition.)

[Picture of Hodge diamond.
Reflection by vertical axis is complex conjugation,
180-degree rotation is Poincar'e duality,
$p+q=$constant is a horizontal line,
reflection along horizontal axis is Lefschetz theorem.
Warning! This depends on conventions of how we draw the diamond.]
\end{enumerate}
\end{enumerate}

\section{The Hodge-Riemann relations}
\label{section-hodge-riemann}

\noindent
For all $[\alpha]\in H^k_{\text{prim}}(X,\mathbb{C})\cap H^{p,q}(X)$ 
we have
$$
i^{p-q}(-1)^{\frac{k(k-1)}{2}}
\int_X \alpha \wedge \overline{\alpha}\wedge\omega^{n-k}>0.
$$
(Here $i$ is the imaginary unit.)

Define a pairing on $H^k_{\text{prim}}(X,\mathbb{C})$:
$$
\psi([\alpha],[\beta])
=(2\pi i)^{-k}
(-1)^{\frac{k(k-1)}{2}}
\int_X \alpha\wedge\beta\wedge\omega^{n-k}
$$
which is $(-1)^k$-symmetric.

\begin{definition}
\label{definition-weil-operator}
The {\it Weil operator} $C \in \text{End}(H^k(X,\mathbb{C}))$ is given by
$C|_{H^{p,q}}(X)=(i)^{p-q}\text{Id}$.
\end{definition}

\noindent
Let $Q([\alpha],[\beta])=(2\pi i)^k
\psi(C[\alpha],[\beta])$.

Then $Q$ is symmetric (exercise) and positive on $H^k(X,\mathbb{R})$.
Positive is just the Hodge-Riemann relation.

\section{Hodge structures and Mumford-Tate groups}
\label{section-hodge-structures-mumford-tate-groups}

\begin{definition}
\label{definition-rational-hodge-structure}
A {\it rational Hodge structure} ($\mathbb{Q}$-HS)
of weight $k\in\mathbb{Z}$ is a finite-dimensional
$\mathbb{Q}$-vector space $V$ 
and a decomposition $V \otimes_{\mathbb{Q}} \mathbb{C}
=\bigoplus_{p+q=k}V^{p,q}$ 
such that
$V^{q,p}=\overline{V^{p,q}}\forall p,q$.

A $\mathbb{Q}$-HS (without fixed weight)
is a $\mathbb{Q}$-vector space $V$ 
with a decomposition
$V=V_{k_1}\oplus … \oplus V_{k_n}$
where $V_{k_i}$ is a $\mathbb{Q}$-HS 
of weight $k_i$.

Analogously, define $\mathbb{Z}$-HS,
$\mathbb{R}$-HS, etc
(i.e. take $V$ to be a finitely generated
$\mathbb{Z}$-module, $\mathbb{R}$-vector space, etc.)
\end{definition}

\begin{example}
\label{example-hodge-structures}
\begin{enumerate}
\item $X \subset \mathbb{C}P^N$ smooth subvariety,
then $H^k(X,\mathbb{Z})$ is a $\mathbb{Z}$-HS
of weight $k$.

\item The {\it Tate HS} is $\mathbb{Z}(1):=2\pi i\mathbb{Z}
=\Ker(\mathbb{C}\xrightarrow{\text{exp}}\mathbb{C}^*$,
since $\mathbb{C}=
\mathbb{Z}(1)\otimes_{\mathbb{Z}}\mathbb{C}=\mathbb{Z}(1)^{-1,1}$,
so this is a $\mathbb{Q}$-HS of weight $-2$.

Note that
$H^2(\mathbb{C}P^1,\mathbb{Z})=\mathbb{Z}$
sits inside
$H^2(\mathbb{C}P^1,\mathbb{C}=\mathbb{Z}\otimes\mathbb{C} \cong \mathbb{C}
=H^{1,1}(\mathbb{C}P^1)$.

Analogously,
$\mathbb{Q}(1)=2\pi i\mathbb{Q} \subset \mathbb{C}$
and $\mathbb{Q}(1)\otimes \mathbb{C}=\mathbb{Q}(1)^{-1,-1}$ 
is of weight $-2$.
\end{enumerate}
\end{example}

\section{The Deligne torus}
\label{section-deligne-torus}

\begin{definition}
\label{definition-deligne-torus}
$\mathbb{S}$ is the algebraic group such that
$\mathbb{S}(\mathbb{R})$ is the $\mathbb{C}^*=\mathbb{C}\setminus\{0\}$.
\end{definition}

\noindent
The group of real points $\mathbb{S}(\mathbb{R})$ is a real Lie group.

Note that
\begin{align*}
\mathbb{C}^*&=
\{(x,y) \in \mathbb{R}^2:x^2+y^2\neq 0\}\\
&=\{(x,y,y)\in \mathbb{R}^2:(x^2+y^2)t=1\},
\end{align*}

\noindent
which allows to see $\mathbb{C}^\ast$
as the vanishing locus of the polynomial
$(x^2+y^2)t-1$, so to see it as an algebraic variety.
Now
\begin{align*}
\mathbb{S}(\mathbb{C})&=
\{(x,y,t)\in \mathbb{C}^3:(x^2+y^2)t=1\}\\
&=\{(x,y,t)\in \mathbb{C}^3:\underbrace{(x+iy)}_{z}
\underbrace{(x-iy)}_{w}t=1\}\\
&=\{(z,w,t)\in\mathbb{C}^3:zwt=1\}\\
&=
\{(z,w)\in\mathbb{C}^2:z\neq 0, w\neq 0\} \cong \mathbb{C}^*\times\mathbb{C}^*.
\end{align*}

\medskip\noindent
Let $V$ be a $\mathbb{Q}$-HS of weight $k$.
Define a representation over $\mathbb{R}$ 
$\rho:\mathbb{S}(\mathbb{R}) \to \text{GL}(V\otimes\mathbb{R})$ 
as follows:
$\forall z \in \mathbb{C}^*\forall v \in V^{p,q}$ 
$\rho(z)\cdot v=z^p \overline{z}^qv$ 
if $v \in V \otimes \mathbb{R}$,
then $v = \sum_{ p+q=k}v^{p,q},$ 
$v^{q,p}=\overline{v^{p,q}}$,
and then
$\overline{\rho\cdot v}
=\sum_{p+q=k}z^p \overline{z}^q
=\sum_{p+q=k}\overline{z}^pz^qv^{q,p}
=\rho(z)\cdot v$,
so it is in fact a representation.
But why? Why is this equality what we need
to make sure it is a representation?

Observe that $z \in \mathbb{R}^*\subset\mathbb{S}(\mathbb{R})=\mathbb{C}^*$,
the eigenvalue is
$\rho(z)\cdot v= z^{p+q}\cdot v=z^k\cdot v$.
This motivates the following:

Conversely,
given a $\mathbb{Q}$-vector space $V$
and a representation
$\rho:\mathbb{S}(\mathbb{R}) \to \text{GL}(V\otimes_{\mathbb{Q}}\mathbb{R})$ 
of $\mathbb{R}$-groups, such that
$\forall r \in \mathbb{R}^*$
$\rho (r)=r^k\cdot \text{Id}$,
we have
$V \otimes \mathbb{C}=\bigoplus_{p,q}V^{p,q}$ 
where $v \in V^{p,q}$,
$\rho(z,w)=z^pw^q\cdot v$.
Since $\rho$ is a representation of $\mathbb{R}$-groups
we have $V^{q,p}=\overline{V^{p,q}}$,
so by the computation above we have
$V^{p,q}=0$ when $p+q \neq k$,
then $V$ becomes a $\mathbb{Q}$ =HS
of weight $k$.

In conclusion, a $\mathbb{Q}$-HS is the same thing as a
$\mathbb{Q}$-vector space $V$ and a representation
of $\mathbb{R}$-groups $\rho:\mathbb{S}(\mathbb{R}) \to
\text{GL}(V\otimes_{\mathbb{Q}}\mathbb{R})$ 
such that $\rho |_{\mathbb{R}^*}$ 
is defined over $\mathbb{Q}$.

(We may also replace $\mathbb{Q}$ with $\mathbb{Z}$, etc. in this construction.)

\medskip\noindent
Back to the Tate group
$\mathbb{Q}(1)=2\pi i \mathbb{Q} \subset \mathbb{C}$,
define for all $z \in \mathbb{C}^*$ and
$v \in \mathbb{Q}(1) \otimes \mathbb{C}$ 
by $\rho(z)\cdot v=|z|^{-2}\cdot v$.

We have operations on HS:
$V_1 \oplus V_2, V_1 \otimes V_2, \Hom(V_1,V_2)$.
The $\mathbb{Q}$-HS form an abelian category.
Let $\mathbb{Q}(m)=\mathbb{Q}(1)^{\otimes m}$ 
when $m \geq 0$,
$\mathbb{Q}(-1)=\mathbb{Q}(1)^*$
and $\mathbb{Q}(-m)=\mathbb{Q}(-1)^{\otimes m}$ 
for $m \geq 0$.

If $V$ is a $\mathbb{Q}$-HS, then
$V(m)=V \otimes \mathbb{Q}(m)$
is the {\it Tate twist}.

\begin{example}
\label{example-chern-class}
\begin{enumerate}
\item $X \subset \mathbb{C}P^n$ subvariety.
Notice that while $\mathbb{C}$ is the
algebraic closure of $\mathbb{R}$,
such a closure can be obtained by 
choosing the imaginary unit $i$ or $-i$,
so this is not canonical.

However, the first Chern class is canonical:
$$
\xymatrix{
0\ar[r]
&\mathbb{Z}(1)\ar[r]
&\mathbb{C}\ar[r]^{\text{exp}}
&\mathbb{C}^*\ar[r]
&0
}
$$
gives the connecting homomorphism
$H^1(X,\mathbb{C}^* )\xrightarrow{c_1}H^2(X,\mathbb{Z}(1))$
which is in fact a HS of weight 0
(because $H^2$ has weight 2 and $\mathbb{Z}(1)$ has
weight $-2$).

Analogously, the $p$-th chern class is
$c_p(A\text{ coh. sheaf}) \in H^{2p}(X,\mathbb{Q}(p))$.

\item $cl:CH_\mathbb{Q}^p(X) \to H^{2p}(X,\mathbb{Q})$ 
the cycle class map, where
$$
CH^p_\mathbb{Q}(X)=\left\{\sum_{\substack{Z_i \subset X \\ 
\text{subvar.}}}\alpha_i[Z_i]:\alpha_i \in \mathbb{Q}\right\}
\Big/\text{rational equiv.}
$$
\end{enumerate}
\end{example}

Consider
$$
\xymatrix{
Z\ar@{^(->}[r]^i& X\\
\tilde{Z} \ar[ur]_j\ar[u]
}
$$
where 
$Z$ is a subvariety of codimension $p$ and 
$\tilde{Z}$ is a resolution of singularities.
Then we have the pushforward of the fundamental class,
$j_*[\tilde{Z}] \in H_{2n-2p}(X,\mathbb{Q})$
where $[\tilde{Z}]\in H_{2n-2p}(\tilde{Z},\mathbb{Q})
\cong H^{2n-2p}(\tilde{Z}, \mathbb{Q})^*$.
Then the Poincare dual of
$j_*[\tilde{Z}]:=d[Z]
\in H^{2p}(X,\mathbb{Q}) \cap H^{p,p}(X)$.
So $H_{2n-2p}(\tilde{Z},\mathbb{Q})$ 
is a HS of weight $2(p-n)$
and  $[\tilde{Z}] \in H_{p-n,p-n}$.

\begin{definition}
\label{definition-space-of-hodge-classes}
The {\it space of Hodge classes}
is $H^{2p}_{\text{Hdg}}(X)
=H^{2p}(X,\mathbb{Q})
\cap H^{p,p}(X)$
\end{definition}



\noindent
Then we have
$cl:CH^p_{\mathbb{Q}}(X)
\to H^{2p}_{\text{Hdg}}(X)
\cong \Hom_{\mathbb{Q}-HS}(\mathbb{Q}(-p),H^{2p}(X,\mathbb{Q})$.
The Hodge conjecture is that cl
is surjective onto the space of Hodge classes.

\section{Polarizations}
\label{section-polarizations}

\noindent
Let $V$ be a $\mathbb{Q}$-HS of weight $k$
and $\rho:\mathbb{S}(\mathbb{R})\to \text{GL}(V\otimes_{\mathbb{Q}}\mathbb{R})$ 
the corresponding representation of the Deligne torus.
The Weil operator in terms of $\rho$ is
$C=\rho(i)$.

\begin{definition}
\label{definition-polarization}
A {\it polarization} on $V$ is a morphism
of $\mathbb{Q}$-HS is a $(-1)^k$-symmetric morphism
$\psi:V \otimes V \to \mathbb{Q}(-k)$
such that the bilinear form
$Q:V_{\mathbb{R}}\otimes V_{\mathbb{R}}\to \mathbb{R}$
given by $Q(x,y)=(2\pi i)^k\psi(Cx,y)$
is symmetric and positive definite.
\end{definition}

\begin{example}
\label{example-polarization}
$V=H^k_{\text{prim}}(X,\mathbb{Q})$
with $\psi$ the Hodge-Riemann pairing.
\end{example}

\noindent
Observe: if $V$ is polarizable
(i.e. it admits a polarization;
not every HS admits a polarization e.g. HS
on nonprojetive varieties)
then $V$ is semisimple
$V=V_1 \oplus  … \oplus  V_m$,
$V_i$ is simple.

Assume $W \subset V$ is a sub-HS,
then $W^{\perp_\psi} \subset V$
is a sub-HS.
Then 
$0 \underbrace{=}_{\substack{Q \\ \text{positive}\\\text{def.}}} 
W \cap W^{\perp_\psi} \subset V$,
Then $V=W \oplus  W^{\perp_\psi}$.

\section{The Mumford-Tate group}
\label{section-mumford-tate-group}

\noindent
Let $V$ be a $\mathbb{Q}$-HS
and $\rho:\mathbb{S}(\mathbb{R}) \to \text{GL}(V \otimes \mathbb{R})$ 
a representation of $\mathbb{R}$-groups.

Consider $G \subset \text{GL}(V)$ such that
$\text{Im}(\rho) \subset G(\mathbb{R})$.

(The idea is that the Mumford-Tate group
will recover the rational structure of $V$,
because the representation $\rho$ knows nothing
about this rational structure, which must exist
since $V$ is a rational vector space.)

\begin{definition}
\label{definition-mumford-tate-group}
The {\it Mumford-Tate group} is
$$
MT(V)=\bigcap_{\substack{G \subset \text{GL}(V) \\ 
\text{subgroup s.t.}\\
\text{Im}(\rho) \subset G(\mathbb{R})}}G
=
\text{smallest $Q$-subgroup of $\text{GL}(V)$ 
containing $\text{Im}\rho$}.
$$
\end{definition}

\begin{remark}
\label{remark-Hodge-group}
We can also consider $U(1) \subset \mathbb{S}(\mathbb{R})$ 
and $\rho':=\rho|_{U(1)}: U(1) \to \text{GL}(V \otimes \mathbb{R})$.
Then the {\it Hodge group} is $\text{Hdg}(V)=$
smallest $Q$-subgroup of $\text{GL}(V)$ such that
$\text{Im}(\rho')\subset \text{Hdg}(V)(\mathbb{R})$.

$\text{Hdg}(V)$
is always smaller than $MT(V)$.
\end{remark}

\begin{example}
\label{example-mumford-tate-group}
\begin{enumerate}
\item $MT(\mathbb{Q}(1))=\mathbb{Q}^*$
$\mathbb{Q}(1)=2\pi i \mathbb{Q} \subset \mathbb{C}$,
$\mathbb{Q}(1) \otimes \mathbb{R}=2\pi i\mathbb{R}\subset \mathbb{C}$,
$\text{GL}(\mathbb{Q}(1)\otimes \mathbb{R})=\mathbb{R}^*$,
since $z \in \mathbb{S}(\mathbb{R})$ acts as $|z|^{-2}\cdot \text{Id}$.

\item $\mathbb{Q}(0)=\mathbb{Q} \subset \mathbb{C}$,
$MT(\mathbb{Q}(0)) = \{1\}$.
In general, if $V$ is of weight $k$,
then $\mathbb{Q}^* =$ center of $\text{GL}(V)\subset MT(V)$ 
and $MT(V)$ is generated by  $\mathbb{Q}^*$
and $Hdg(V)$.
\end{enumerate}
\end{example}


\section{Tensor construction}
\label{section-tensor-construction}

\noindent
Let $V$ be a $\mathbb{Q}$-HS with $MT(V) \subset \text{GL}(V)$
and $\rho:\mathbb{S}(\mathbb{R}) \to MT(V)(\mathbb{R})$.
Then
$$
T^\bullet(V)=\bigoplus_{e,f \geq 0}
V^{\otimes e}\otimes (V^* )^{\otimes f}
$$
is a $MT(V)$-representation.

\begin{proposition}
\label{proposition-sub-HS-iff}
A finite-dimensional subspace $W \subset T^\bullet(V)$
is a sub-HS if and only if $W$ is
a $MT(V)$-subspace.
\end{proposition}

\begin{proof}
$(\impliedby)$. If $W$ is a $MT(V)$-subrepresentation,
then from $\rho$ we can compose with the representation that $MT(V)$ is
to obtain
$\rho':\mathbb{S}(\mathbb{R}) \to \text{GL}(W \otimes \mathbb{R})$.
Note that $\rho|_{\mathbb{R}^*}$ is defined over $Q$,
We get that $W$ is a sub-HS.

($\implies$). Assume that $W \subset T^\bullet(V)$ 
is a sub-HS, $G - \text{Stab}(W) \subset \text{GL}(V)$ 
is a $\mathbb{Q}$-subgroup.
Since $W$ is a sub-HS,
the action of $\mathbb{S}(\mathbb{R})$ on $T^\bullet(V)$ 
preserves $W$,
then $\text{Im}(\rho) \subset G(\mathbb{R})$
and thus $MT(V)\subset G$,
which implies that $W$ is a $MT(V)$-subrepresentation.
\end{proof}


As a corollary,

\begin{lemma}
\label{lemma-mumford-tate-invariant-iff}
$x \in T^\bullet(V)$ is $MT(V)$-invariant
if and only if $X$ is a $(0,0)$ Hodge element.
\end{lemma}

\begin{proposition}
\label{proposition-}
Assume that $V$ is a $\mathbb{Q}$-HS of weight $k$ 
and $\psi:V \otimes V \to \mathbb{Q}(-k)$ is
a polarization.
Then 
$$
Hdg(V) \subset
\begin{cases}
\text{SO}(V,\psi)\qquad &\text{ if }k\equiv 0(\text{mod }2) \\
Sp(V,\psi)\qquad &\text{ if }k \equiv 1(\text{mod }2)
\end{cases}
$$
\end{proposition}

\begin{proof}
$\psi: V \otimes V \to \mathbb{Q}(-k)$,
where the $\mathbb{Q}(-k)$ is a trivial $Hdg(V)$-module,
so that $\psi$ is a morphism of $Hdg(V)$-modules.
The action of $Hdg(V)$ preserves the symmetric form
$\psi$ if $k \equiv 0 (\text{mod }2)$ 
and the antisymmetric form $\psi$ 
if $k \equiv 1 (\text{mod }2)$.
\end{proof}

\begin{example}
\label{example-elliptic-curve}
Let $V = H^1(E,\mathbb{Q})$ where $E$ is an elliptic curve
and $\psi$ the Hodge-Riemann pairing.
Then $V\otimes \mathbb{C}=V^{1,0}\oplus V^{0,1}$ 
and $\dim V^{1,0}=1$.
$Hdg(V) \subset Sp(V,\psi) \cong \text{SL}_2(\mathbb{Q})$.

Observe thati, in general, if $V$ is polarizable,
then $V=V_1 \oplus  … \oplus  V_m$, $V_i$ are
irreducible $MT(V)$-representations,
so $MT(V)$ is reductive.

Back to our elliptic curve example,
$Hdg(V) \subset \text{SL}_2(\mathbb{Q})$ is reductive.

There are two possibilities:
\begin{enumerate}
\item $Hdg(V)=\text{SL}^2(\mathbb{Q})$,
which implies that $MT(V) = \text{GL}_2(V)$.
This happens when $E$ is generic in the moduli space.

\item $Hdg(V)$ is properly contained in  $\text{SL}_2(\mathbb{Q})$.
Then $Hdg(V)$ is a 1-dimensional torus.
Then $\text{End}_{\mathbb{Q}-HS}(V) \neq  \mathbb{Q}$,
which implies that $E$ has complex multiplication.
\end{enumerate}
\end{example}

\begin{definition}
\label{definition-cm-type}
A HS $V$ is of  {\it CM-type} if $MT(V)$ is abelian
(such HS defines a special point in the moduli space of HS.)
\end{definition}






























\end{document}

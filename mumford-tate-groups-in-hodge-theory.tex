\input{preamble}
\begin{document}

\title{Mumford-Tate groups in Hodge theory}
\maketitle

\noindent
Minicourse by Andrey Soldatenkov, UNICAMP, Brazil, 
IMPA Summer School, 2026.

\medskip\noindent\hfill
\href{http://github.com/danimalabares/stack}{github.com/danimalabares/stack}


\tableofcontents

\noindent
Upshot. The Mumford-Tate group is
the stabilizer of the Hodge classes,
and is constant on the complement of the Hodge locus
(see Definition \ref{definition-generic-mt-group}).

\section{Plan}
\label{section-plan}

\begin{enumerate}
\item Motivation: cohomology of algebraic varieties.
\item Definition. Hodge structures, Mumford-Tate group.
\item Characterizations of the MT groups and relations with representation
theory.
\item Variations of Hodge structures and moduli spaces.
\item Dichotomy: abelian vs non-abelian HS.
\item The Kuga-Satake construction.
\end{enumerate}

\section{Introduction}
\label{section-intro}

$X\subset \mathbb{C}P^N$ smooth complex subvariety,
$\dim_\mathbb{C}=n$.
First recall we have singular cohomology,
$H^k(X,\mathbb{C})$,
which is isomorphic to the cohomology
of the constant sheaf $\underline{\mathbb{C}}_X$.
This cohomology is nonzero for $0\leq k \leq 2n$.

Recall. $U \subset X$ open,
$\Gamma(U,\mathbb{C})=\{f:U \to \mathbb{C}:
\text{$f$ is locally constant}\}
=\prod_{\pi_0(U)}\mathbb{C}$.

\begin{example}
\label{example-singular-cohomology}
\begin{enumerate}
\item 
$X=\mathbb{C}P^n$,
$$
H^k(\mathbb{C}P^n,\mathbb{C})=
\begin{cases}
  \mathbb{C}\qquad &k=2m, 0 \leq  m \leq  n \\
  0\qquad &\text{otherwise.} 
\end{cases}
$$
A way to prove this is using the CW decomposition of $\mathbb{C}P^n$.

\item $X \subset \mathbb{C}P^2$ hypersurface of degree $d$;
$X$ a Riemann surface of genus  $g=\frac{(d-1)(d-2)}{2}$.
Then
$H^0(X,\mathbb{C})=\mathbb{C}$,
$H^1(X,\mathbb{C})=\mathbb{C}^{2g}$,
$H^2(X,\mathbb{C})=\mathbb{C}$.
\end{enumerate}
\end{example}

\noindent
We also have the following additional data 
(a Hodge structure) on $H^k(X,\mathbb{C})$:
\begin{itemize}
\item A lattice $H^k(X,\mathbb{Z})/\text{torsion}\subset H^k(X,\mathbb{C})$,
\item A $(p,q)$-decomposition,
$H^k(X,\mathbb{C})=\bigoplus_{p+q=k}H^{p,q}(X)$.
\end{itemize}

Why is it useful?
\begin{itemize}
\item It gives restrictions on possible Betti numbers of algebraic varieties.
(So it may tell us that certain complex variety cannot be algebraic,
for example.)

\item It $f:X \to Y$ is a morphism of algebraic varieties,
then $f^*:H^k(Y,\mathbb{C}) \to H^k(X,\mathbb{C})$
preserves the Hodge structure.
\end{itemize}

\noindent
As an example of the latter statement,

\begin{example}
\label{example-preserving-hodge-structure}
Let $X \subset \mathbb{C}P^4$ be a ``general'' hypersurface
of degree 5. Then there exists no abelian variety
(i.e. a projective variety that is biholomorphic
to $\mathbb{C}P^N/\Lambda$ where $\Lambda$ is a lattice;
so, a complex torus that is also a projective variety)
$A$ that admits a dominant birational map onto $X$
$f:A\xymatrix{\ar@{-->}[r]&}X$.
\end{example}

\section{The p,q decomposition}
\label{section-p-q-decomposition}

\noindent
We use the de Rham complex.
Let $\Omega_X^k$ be the sheaf of holomorphic $k$-forms.
The de Rham complex is
$$
\Omega_{dR}^\bullet
=(0 \to \mathcal{O}_X \to \Omega_X^1\xrightarrow{d}\cdots
\to\Omega_X^n \to 0)
$$

$\Omega_{dR}^\bullet$ is a resolution of $\mathbb{C}_X$.
Therefore $H^k(X,\mathbb{C})\cong H^k(X,\Omega_{dR}^\bullet$.

Let's define a subcomplex:
$$
F^p\Omega_{dR}^\bullet=
(0 \to \cdots \to \Omega_X^p \to O_X^{p+1}
\to \cdots \to \Omega_X^n \to 0)
$$
It's a subcomplex $F^p\Omega_{dr}^\bullet \subset \Omega_{dR}^\bullet$ 
since the sheaves coincide when $F^p\Omega_{dR}$ are nonzero
and they inject otherwise (because it's the zero sheaf).

This gives
$H^k(X,F^p\Omega_{dR}^\bullet)\xrightarrow{(\ast)}
H^k(X,\Omega_{dR}^\bullet)=H^k(X,\mathbb{C})$.

\begin{definition}
\label{definition-hodge-filtration}
The {\it Hodge filtration} is
$F^pH^k(X)=\text{Im}(\ast)$.
\end{definition}

\noindent
Hodge theory tells us the map $(\ast)$ is in fact injective.

Let $\Lambda^{p,q}_X$ be the sheaf of $C^\infty$-forms
on $X$ of type $(p,q)$.
They are given locally by
$\sum_{\substack{|I|=p \\ |J|=q}}
\alpha_{I J}dz_I \wedge d\overline{z}_J$,
for $\alpha_{I J}\in C^\infty_X$.

Then we have an acyclic resolution of $\Omega_X^p$:
$$
0 \to \Omega_X^p \hookrightarrow
\Lambda^{p,0}_X
\xrightarrow{\bar{\partial}}
\Lambda^{p,1}_X
\xrightarrow{\overline{\partial}}
\cdots
\to
\Lambda^{p,n}_X\to 0.
$$

$\Lambda^{\bullet,\bullet}_X$ is a fine resolution of $\Omega_X^\bullet$.

Then $H^k(X,\mathbb{C})\cong H^k(X,\text{Tot}\Lambda^{\bullet,\bullet}_X$,
which can be computed by a spectral sequence.
The first page of such spectral sequence is given by
$E^{p,q}_1=H^q(X,\Omega_X^p)$.
This converges to $H^k(X,\mathbb{C})$.
This is the Hodge-to-de Rham spectral sequence.

\medskip\noindent
Since our manifolds are projective they admit a Kähler metric
$\omega$ induced by the inclusion $X \xrightarrow{i}\mathbb{C}P^N$,
that is, $\omega=i^* (\text{Fubini-Study metric on }\mathbb{C}P^N)$.

Then $\Lambda_X^k=\bigoplus_{p+q=k}\Lambda_X^{p,q}$,
$\Lambda^\bullet_X$ becomes an elliptic complex.

We have
$$
… \to \Lambda_X^{k-1}\xrightarrow{d} \Lambda_X^k\to …,
\qquad 
…\to\Lambda_X^k \xrightarrow{d^*}\Lambda_X^{k-1}\to…
$$
where $d^*$ is the adjoint of $d$ w.r.t. $\omega$.
Then $\Lambda=d d^*+d^*d$ is an elliptic operator.
$\mathcal{H}^k=\Ker(\Lambda|_{\Lambda^k_X})
=\Ker(d|_{\Lambda^k_X})\cap \Ker(d^*|_{\Lambda^k_X})$ 
are the harmonic forms.

Consider a natural map from the harmonic forms of type $(p,q)$ 
to the cohomology:
$$
\mathcal{H}^{p,q}=\mathcal{H}^k\cap\Lambda^{p,q}_X
\xrightarrow{(\ast \ast)} H^q(X,\Omega_X^p).
$$
Fact: since $d\omega=0$ ($X$ is Kähler),
$(\ast\ast)$ is an isomorphism.

This means that for Kähler manifolds
$$
\dim H^k(X,\mathbb{C})
\leq 
\sum_{p+q=k}H^q(X,\Omega_X^p)
\underbrace{\leq}_{(\ast \ast)}
\dim H^k(X,\mathbb{C})
$$
This implies that the Hodge-to-de Rham spectral sequence
degenerates at $E_1$. Therefore,
$$
H^k(X,\mathbb{C})=\mathcal{H}^k=
\bigoplus_{p+q=k}\mathcal{H}^{p,q}\cong H^q(X,\Omega_X^p).
$$

The Hodge filtration is
$$
F^pH^k(X,\mathbb{C})=\bigoplus_{\substack{p'+q'=k \\ p' \geq p}}
H^{p',q'}(X).
$$

\section{Symmetries of the p,q decomposition}
\label{section-symmetries-of-p-q-decomposition}

\begin{enumerate}
\item Since $\overline{\Lambda^{p,q}_X}=\Lambda^{q,p}_X$,
we have $\mathcal{H}^{q,p}=\overline{\mathcal{H}^{p,q}}$.
This means that if $k \equiv 1 (\text{mod }2)$,
$H^k(X,\mathbb{C})=\mathcal{H}^{k,0} \oplus … \oplus 
\mathcal{H}^{\frac{k+1}{2},\frac{k-1}{2}}
\oplus \mathcal{H}^{0,k}\oplus … \oplus 
\mathcal{H}^{\frac{k-1}{2},\frac{k+1}{2}}$.
This means that the $k$-th Betti number is even,
$b_k(X)\equiv 0 \text{mod }2$.

\item (Poincare duality.) We have a perfect pairing
\begin{align*}
H^k(X,\mathbb{C}) \otimes H^{2n-k}(X,\mathbb{C})
&\longrightarrow \mathbb{C} \\
[\alpha]\otimes[\beta] &\longmapsto 
\int_X\alpha\wedge\beta
\end{align*}

\noindent
Notice that if
$\alpha \in \mathcal{H}^{p,q}$ 
then $\beta \in \mathcal{H}^{n-p,n-q}$.
This induces a perfect pairing
$$
\mathcal{H}^{p,q}\otimes\mathcal{H}^{n-p,n-q}
\to \mathbb{C}.
$$

\item (Polarization and the Lefschetz operator.)
The polarization is the Kähler class of the Kähler form.
By $X$ being projective we have that the Kähler class is integral.
Moreover, it is the Poincar'e dual of the hyperplane section class.
That is, let $h \in H^2(\mathbb{C}P^n,\mathbb{Z})$
be the class of a hyperplane,
then $i^*h=[\omega] \in H^2(X,\mathbb{Z})$.
We have $\omega \in \mathcal{H}^{1,1}$ 
and $[\omega]\in H^2(X,\mathbb{Z}) \cap H^{1,1}(X)$.
The Lefschetz operator is
\begin{align*}
L_\omega: H^{p,q}(X) &\longrightarrow H^{p+1,q+1}(X) \\
[\alpha] &\longmapsto [\alpha\wedge\omega]=[\alpha]\cup [\omega].
\end{align*}

\noindent
Lefschetz theorem says
\begin{enumerate}
\item $L_\omega^k:H^{n-k}(X,\mathbb{Q})\xrightarrow{\sim}
H^{n+k}(X,\mathbb{Q})$
is an isomorphism for $0\leq k\leq n$

\item The dual of $L_\omega$ is
\begin{align*}
\Lambda_\omega: H^{p,q}(X) &\longrightarrow H^{p-1,q-1}(X) \\
[\alpha] &\longmapsto [i_\alpha \omega].
\end{align*}

$$
[L_\omega,\Lambda_\omega]=\Theta \in \text{End}(H^\bullet(X,\mathbb{Q}))
$$
$$
\Theta|_{H^k(X,\mathbb{Q})}
=(k-n)\text{Id}
$$
Then $L_\omega, \Lambda_\omega$ and $\Theta$
span a subalgebra of  $\text{End}(H^\bullet(X,\mathbb{Q}))$ 
isomorphic to $\mathfrak{sl}_2$.
This allows us to use what we know about the
representation theory of $\mathfrak{sl}_2$.
Let $H^k_{\text{prim}}(X,\mathbb{Q})=(\Lambda|_{H^k(X,\mathbb{Q})})$.
Then $H^m(X,\mathbb{Q})=\bigoplus_{i \geq 0}L_\omega^i
H^{m-2i}_{\text{prim}}(X,\mathbb{Q})$
for $0 < m \leq n$.
(I think this corresponds to the usual
weight space decomposition.)

[Picture of Hodge diamond.
Reflection by vertical axis is complex conjugation,
180-degree rotation is Poincar'e duality,
$p+q=$constant is a horizontal line,
reflection along horizontal axis is Lefschetz theorem.
Warning! This depends on conventions of how we draw the diamond.]
\end{enumerate}
\end{enumerate}

\section{The Hodge-Riemann relations}
\label{section-hodge-riemann}

\noindent
For all $[\alpha]\in H^k_{\text{prim}}(X,\mathbb{C})\cap H^{p,q}(X)$ 
we have
$$
i^{p-q}(-1)^{\frac{k(k-1)}{2}}
\int_X \alpha \wedge \overline{\alpha}\wedge\omega^{n-k}>0.
$$
(Here $i$ is the imaginary unit.)

Define a pairing on $H^k_{\text{prim}}(X,\mathbb{C})$:
$$
\psi([\alpha],[\beta])
=(2\pi i)^{-k}
(-1)^{\frac{k(k-1)}{2}}
\int_X \alpha\wedge\beta\wedge\omega^{n-k}
$$
which is $(-1)^k$-symmetric.

\begin{definition}
\label{definition-weil-operator}
The {\it Weil operator} $C \in \text{End}(H^k(X,\mathbb{C}))$ is given by
$C|_{H^{p,q}}(X)=(i)^{p-q}\text{Id}$.
\end{definition}

\noindent
Let $Q([\alpha],[\beta])=(2\pi i)^k
\psi(C[\alpha],[\beta])$.

Then $Q$ is symmetric (exercise) and positive on $H^k(X,\mathbb{R})$.
Positive is just the Hodge-Riemann relation.

\section{Hodge structures and Mumford-Tate groups}
\label{section-hodge-structures-mumford-tate-groups}

\begin{definition}
\label{definition-rational-hodge-structure}
A {\it rational Hodge structure} ($\mathbb{Q}$-HS)
of weight $k\in\mathbb{Z}$ is a finite-dimensional
$\mathbb{Q}$-vector space $V$ 
and a decomposition $V \otimes_{\mathbb{Q}} \mathbb{C}
=\bigoplus_{p+q=k}V^{p,q}$ 
such that
$V^{q,p}=\overline{V^{p,q}}\forall p,q$.

A $\mathbb{Q}$-HS (without fixed weight)
is a $\mathbb{Q}$-vector space $V$ 
with a decomposition
$V=V_{k_1}\oplus … \oplus V_{k_n}$
where $V_{k_i}$ is a $\mathbb{Q}$-HS 
of weight $k_i$.

Analogously, define $\mathbb{Z}$-HS,
$\mathbb{R}$-HS, etc
(i.e. take $V$ to be a finitely generated
$\mathbb{Z}$-module, $\mathbb{R}$-vector space, etc.)
\end{definition}

\begin{example}
\label{example-hodge-structures}
\begin{enumerate}
\item $X \subset \mathbb{C}P^N$ smooth subvariety,
then $H^k(X,\mathbb{Z})$ is a $\mathbb{Z}$-HS
of weight $k$.

\item The {\it Tate HS} is $\mathbb{Z}(1):=2\pi i\mathbb{Z}
=\Ker(\mathbb{C}\xrightarrow{\text{exp}}\mathbb{C}^*$,
since $\mathbb{C}=
\mathbb{Z}(1)\otimes_{\mathbb{Z}}\mathbb{C}=\mathbb{Z}(1)^{-1,1}$,
so this is a $\mathbb{Q}$-HS of weight $-2$.

Note that
$H^2(\mathbb{C}P^1,\mathbb{Z})=\mathbb{Z}$
sits inside
$H^2(\mathbb{C}P^1,\mathbb{C}=\mathbb{Z}\otimes\mathbb{C} \cong \mathbb{C}
=H^{1,1}(\mathbb{C}P^1)$.

Analogously,
$\mathbb{Q}(1)=2\pi i\mathbb{Q} \subset \mathbb{C}$
and $\mathbb{Q}(1)\otimes \mathbb{C}=\mathbb{Q}(1)^{-1,-1}$ 
is of weight $-2$.
\end{enumerate}
\end{example}

\section{The Deligne torus}
\label{section-deligne-torus}

\begin{definition}
\label{definition-deligne-torus}
$\mathbb{S}$ is the algebraic group such that
$\mathbb{S}(\mathbb{R})$ is the $\mathbb{C}^*=\mathbb{C}\setminus\{0\}$.
\end{definition}

\noindent
The group of real points $\mathbb{S}(\mathbb{R})$ is a real Lie group.

Note that
\begin{align*}
\mathbb{C}^*&=
\{(x,y) \in \mathbb{R}^2:x^2+y^2\neq 0\}\\
&=\{(x,y,y)\in \mathbb{R}^2:(x^2+y^2)t=1\},
\end{align*}

\noindent
which allows to see $\mathbb{C}^\ast$
as the vanishing locus of the polynomial
$(x^2+y^2)t-1$, so to see it as an algebraic variety.
Now
\begin{align*}
\mathbb{S}(\mathbb{C})&=
\{(x,y,t)\in \mathbb{C}^3:(x^2+y^2)t=1\}\\
&=\{(x,y,t)\in \mathbb{C}^3:\underbrace{(x+iy)}_{z}
\underbrace{(x-iy)}_{w}t=1\}\\
&=\{(z,w,t)\in\mathbb{C}^3:zwt=1\}\\
&=
\{(z,w)\in\mathbb{C}^2:z\neq 0, w\neq 0\} \cong \mathbb{C}^*\times\mathbb{C}^*.
\end{align*}

\medskip\noindent
Let $V$ be a $\mathbb{Q}$-HS of weight $k$.
Define a representation over $\mathbb{R}$ 
$\rho:\mathbb{S}(\mathbb{R}) \to \text{GL}(V\otimes\mathbb{R})$ 
as follows:
$\forall z \in \mathbb{C}^*\forall v \in V^{p,q}$ 
$\rho(z)\cdot v=z^p \overline{z}^qv$ 
if $v \in V \otimes \mathbb{R}$,
then $v = \sum_{ p+q=k}v^{p,q},$ 
$v^{q,p}=\overline{v^{p,q}}$,
and then
$\overline{\rho\cdot v}
=\sum_{p+q=k}z^p \overline{z}^q
=\sum_{p+q=k}\overline{z}^pz^qv^{q,p}
=\rho(z)\cdot v$,
so it is in fact a representation.
But why? Why is this equality what we need
to make sure it is a representation?

Observe that $z \in \mathbb{R}^*\subset\mathbb{S}(\mathbb{R})=\mathbb{C}^*$,
the eigenvalue is
$\rho(z)\cdot v= z^{p+q}\cdot v=z^k\cdot v$.
This motivates the following:

Conversely,
given a $\mathbb{Q}$-vector space $V$
and a representation
$\rho:\mathbb{S}(\mathbb{R}) \to \text{GL}(V\otimes_{\mathbb{Q}}\mathbb{R})$ 
of $\mathbb{R}$-groups, such that
$\forall r \in \mathbb{R}^*$
$\rho (r)=r^k\cdot \text{Id}$,
we have
$V \otimes \mathbb{C}=\bigoplus_{p,q}V^{p,q}$ 
where $v \in V^{p,q}$,
$\rho(z,w)=z^pw^q\cdot v$.
Since $\rho$ is a representation of $\mathbb{R}$-groups
we have $V^{q,p}=\overline{V^{p,q}}$,
so by the computation above we have
$V^{p,q}=0$ when $p+q \neq k$,
then $V$ becomes a $\mathbb{Q}$ =HS
of weight $k$.

In conclusion, a $\mathbb{Q}$-HS is the same thing as a
$\mathbb{Q}$-vector space $V$ and a representation
of $\mathbb{R}$-groups $\rho:\mathbb{S}(\mathbb{R}) \to
\text{GL}(V\otimes_{\mathbb{Q}}\mathbb{R})$ 
such that $\rho |_{\mathbb{R}^*}$ 
is defined over $\mathbb{Q}$.

(We may also replace $\mathbb{Q}$ with $\mathbb{Z}$, etc. in this construction.)

\medskip\noindent
Back to the Tate group
$\mathbb{Q}(1)=2\pi i \mathbb{Q} \subset \mathbb{C}$,
define for all $z \in \mathbb{C}^*$ and
$v \in \mathbb{Q}(1) \otimes \mathbb{C}$ 
by $\rho(z)\cdot v=|z|^{-2}\cdot v$.

We have operations on HS:
$V_1 \oplus V_2, V_1 \otimes V_2, \Hom(V_1,V_2)$.
The $\mathbb{Q}$-HS form an abelian category.
Let $\mathbb{Q}(m)=\mathbb{Q}(1)^{\otimes m}$ 
when $m \geq 0$,
$\mathbb{Q}(-1)=\mathbb{Q}(1)^*$
and $\mathbb{Q}(-m)=\mathbb{Q}(-1)^{\otimes m}$ 
for $m \geq 0$.

If $V$ is a $\mathbb{Q}$-HS, then
$V(m)=V \otimes \mathbb{Q}(m)$
is the {\it Tate twist}.

\begin{example}
\label{example-chern-class}
\begin{enumerate}
\item $X \subset \mathbb{C}P^n$ subvariety.
Notice that while $\mathbb{C}$ is the
algebraic closure of $\mathbb{R}$,
such a closure can be obtained by 
choosing the imaginary unit $i$ or $-i$,
so this is not canonical.

However, the first Chern class is canonical:
$$
\xymatrix{
0\ar[r]
&\mathbb{Z}(1)\ar[r]
&\mathbb{C}\ar[r]^{\text{exp}}
&\mathbb{C}^*\ar[r]
&0
}
$$
gives the connecting homomorphism
$H^1(X,\mathbb{C}^* )\xrightarrow{c_1}H^2(X,\mathbb{Z}(1))$
which is in fact a HS of weight 0
(because $H^2$ has weight 2 and $\mathbb{Z}(1)$ has
weight $-2$).

Analogously, the $p$-th chern class is
$c_p(A\text{ coh. sheaf}) \in H^{2p}(X,\mathbb{Q}(p))$.

\item $cl:CH_\mathbb{Q}^p(X) \to H^{2p}(X,\mathbb{Q})$ 
the cycle class map, where
$$
CH^p_\mathbb{Q}(X)=\left\{\sum_{\substack{Z_i \subset X \\ 
\text{subvar.}}}\alpha_i[Z_i]:\alpha_i \in \mathbb{Q}\right\}
\Big/\text{rational equiv.}
$$
\end{enumerate}
\end{example}

Consider
$$
\xymatrix{
Z\ar@{^(->}[r]^i& X\\
\tilde{Z} \ar[ur]_j\ar[u]
}
$$
where 
$Z$ is a subvariety of codimension $p$ and 
$\tilde{Z}$ is a resolution of singularities.
Then we have the pushforward of the fundamental class,
$j_*[\tilde{Z}] \in H_{2n-2p}(X,\mathbb{Q})$
where $[\tilde{Z}]\in H_{2n-2p}(\tilde{Z},\mathbb{Q})
\cong H^{2n-2p}(\tilde{Z}, \mathbb{Q})^*$.
Then the Poincare dual of
$j_*[\tilde{Z}]:=d[Z]
\in H^{2p}(X,\mathbb{Q}) \cap H^{p,p}(X)$.
So $H_{2n-2p}(\tilde{Z},\mathbb{Q})$ 
is a HS of weight $2(p-n)$
and  $[\tilde{Z}] \in H_{p-n,p-n}$.

\begin{definition}
\label{definition-space-of-hodge-classes}
The {\it space of Hodge classes}
is $H^{2p}_{\text{Hdg}}(X)
=H^{2p}(X,\mathbb{Q})
\cap H^{p,p}(X)$
\end{definition}



\noindent
Then we have
$cl:CH^p_{\mathbb{Q}}(X)
\to H^{2p}_{\text{Hdg}}(X)
\cong \Hom_{\mathbb{Q}-HS}(\mathbb{Q}(-p),H^{2p}(X,\mathbb{Q})$.
The Hodge conjecture is that cl
is surjective onto the space of Hodge classes.

\section{Polarizations}
\label{section-polarizations}

\noindent
Let $V$ be a $\mathbb{Q}$-HS of weight $k$
and $\rho:\mathbb{S}(\mathbb{R})\to \text{GL}(V\otimes_{\mathbb{Q}}\mathbb{R})$ 
the corresponding representation of the Deligne torus.
The Weil operator in terms of $\rho$ is
$C=\rho(i)$.

\begin{definition}
\label{definition-polarization}
A {\it polarization} on $V$ is a morphism
of $\mathbb{Q}$-HS is a $(-1)^k$-symmetric morphism
$\psi:V \otimes V \to \mathbb{Q}(-k)$
such that the bilinear form
$Q:V_{\mathbb{R}}\otimes V_{\mathbb{R}}\to \mathbb{R}$
given by $Q(x,y)=(2\pi i)^k\psi(Cx,y)$
is symmetric and positive definite.
\end{definition}

\begin{example}
\label{example-polarization}
$V=H^k_{\text{prim}}(X,\mathbb{Q})$
with $\psi$ the Hodge-Riemann pairing.
\end{example}

\noindent
Observe: if $V$ is polarizable
(i.e. it admits a polarization;
not every HS admits a polarization e.g. HS
on nonprojetive varieties)
then $V$ is semisimple
$V=V_1 \oplus  … \oplus  V_m$,
$V_i$ is simple.

Assume $W \subset V$ is a sub-HS,
then $W^{\perp_\psi} \subset V$
is a sub-HS.
Then 
$0 \underbrace{=}_{\substack{Q \\ \text{positive}\\\text{def.}}} 
W \cap W^{\perp_\psi} \subset V$,
Then $V=W \oplus  W^{\perp_\psi}$.

\section{The Mumford-Tate group}
\label{section-mumford-tate-group}

\noindent
Let $V$ be a $\mathbb{Q}$-HS
and $\rho:\mathbb{S}(\mathbb{R}) \to \text{GL}(V \otimes \mathbb{R})$ 
a representation of $\mathbb{R}$-groups.

Consider $G \subset \text{GL}(V)$ such that
$\text{Im}(\rho) \subset G(\mathbb{R})$.

(The idea is that the Mumford-Tate group
will recover the rational structure of $V$,
because the representation $\rho$ knows nothing
about this rational structure, which must exist
since $V$ is a rational vector space.)

\begin{definition}
\label{definition-mumford-tate-group}
The {\it Mumford-Tate group} is
$$
MT(V)=\bigcap_{\substack{G \subset \text{GL}(V) \\ 
\text{subgroup s.t.}\\
\text{Im}(\rho) \subset G(\mathbb{R})}}G
=
\text{smallest $Q$-subgroup of $\text{GL}(V)$ 
containing $\text{Im}\rho$}.
$$
\end{definition}

\begin{remark}
\label{remark-Hodge-group}
We can also consider $U(1) \subset \mathbb{S}(\mathbb{R})$ 
and $\rho':=\rho|_{U(1)}: U(1) \to \text{GL}(V \otimes \mathbb{R})$.
Then the {\it Hodge group} is $\text{Hdg}(V)=$
smallest $Q$-subgroup of $\text{GL}(V)$ such that
$\text{Im}(\rho')\subset \text{Hdg}(V)(\mathbb{R})$.

$\text{Hdg}(V)$
is always smaller than $MT(V)$.
\end{remark}

\begin{example}
\label{example-mumford-tate-group}
\begin{enumerate}
\item $MT(\mathbb{Q}(1))=\mathbb{Q}^*$
$\mathbb{Q}(1)=2\pi i \mathbb{Q} \subset \mathbb{C}$,
$\mathbb{Q}(1) \otimes \mathbb{R}=2\pi i\mathbb{R}\subset \mathbb{C}$,
$\text{GL}(\mathbb{Q}(1)\otimes \mathbb{R})=\mathbb{R}^*$,
since $z \in \mathbb{S}(\mathbb{R})$ acts as $|z|^{-2}\cdot \text{Id}$.

\item $\mathbb{Q}(0)=\mathbb{Q} \subset \mathbb{C}$,
$MT(\mathbb{Q}(0)) = \{1\}$.
In general, if $V$ is of weight $k$,
then $\mathbb{Q}^* =$ center of $\text{GL}(V)\subset MT(V)$ 
and $MT(V)$ is generated by  $\mathbb{Q}^*$
and $Hdg(V)$.
\end{enumerate}
\end{example}


\section{Tensor construction}
\label{section-tensor-construction}

\noindent
Let $V$ be a $\mathbb{Q}$-HS with $MT(V) \subset \text{GL}(V)$
and $\rho:\mathbb{S}(\mathbb{R}) \to MT(V)(\mathbb{R})$.
Then
$$
T^\bullet(V)=\bigoplus_{e,f \geq 0}
V^{\otimes e}\otimes (V^* )^{\otimes f}
$$
is a $MT(V)$-representation.

\begin{proposition}
\label{proposition-sub-HS-iff}
A finite-dimensional subspace $W \subset T^\bullet(V)$
is a sub-HS if and only if $W$ is
a $MT(V)$-subspace.
\end{proposition}

\begin{proof}
$(\impliedby)$. If $W$ is a $MT(V)$-subrepresentation,
then from $\rho$ we can compose with the representation that $MT(V)$ is
to obtain
$\rho':\mathbb{S}(\mathbb{R}) \to \text{GL}(W \otimes \mathbb{R})$.
Note that $\rho|_{\mathbb{R}^*}$ is defined over $Q$,
We get that $W$ is a sub-HS.

($\implies$). Assume that $W \subset T^\bullet(V)$ 
is a sub-HS, $G - \text{Stab}(W) \subset \text{GL}(V)$ 
is a $\mathbb{Q}$-subgroup.
Since $W$ is a sub-HS,
the action of $\mathbb{S}(\mathbb{R})$ on $T^\bullet(V)$ 
preserves $W$,
then $\text{Im}(\rho) \subset G(\mathbb{R})$
and thus $MT(V)\subset G$,
which implies that $W$ is a $MT(V)$-subrepresentation.
\end{proof}


As a corollary,

\begin{lemma}
\label{lemma-mumford-tate-invariant-iff}
$x \in T^\bullet(V)$ is $MT(V)$-invariant
if and only if $X$ is a $(0,0)$ Hodge element.
\end{lemma}

\begin{proposition}
\label{proposition-}
Assume that $V$ is a $\mathbb{Q}$-HS of weight $k$ 
and $\psi:V \otimes V \to \mathbb{Q}(-k)$ is
a polarization.
Then 
$$
Hdg(V) \subset
\begin{cases}
\text{SO}(V,\psi)\qquad &\text{ if }k\equiv 0(\text{mod }2) \\
Sp(V,\psi)\qquad &\text{ if }k \equiv 1(\text{mod }2)
\end{cases}
$$
\end{proposition}

\begin{proof}
$\psi: V \otimes V \to \mathbb{Q}(-k)$,
where the $\mathbb{Q}(-k)$ is a trivial $Hdg(V)$-module,
so that $\psi$ is a morphism of $Hdg(V)$-modules.
The action of $Hdg(V)$ preserves the symmetric form
$\psi$ if $k \equiv 0 (\text{mod }2)$ 
and the antisymmetric form $\psi$ 
if $k \equiv 1 (\text{mod }2)$.
\end{proof}

\begin{example}
\label{example-elliptic-curve}
Let $V = H^1(E,\mathbb{Q})$ where $E$ is an elliptic curve
and $\psi$ the Hodge-Riemann pairing.
Then $V\otimes \mathbb{C}=V^{1,0}\oplus V^{0,1}$ 
and $\dim V^{1,0}=1$.
$Hdg(V) \subset Sp(V,\psi) \cong \text{SL}_2(\mathbb{Q})$.

Observe thati, in general, if $V$ is polarizable,
then $V=V_1 \oplus  … \oplus  V_m$, $V_i$ are
irreducible $MT(V)$-representations,
so $MT(V)$ is reductive.

Back to our elliptic curve example,
$Hdg(V) \subset \text{SL}_2(\mathbb{Q})$ is reductive.

There are two possibilities:
\begin{enumerate}
\item $Hdg(V)=\text{SL}^2(\mathbb{Q})$,
which implies that $MT(V) = \text{GL}_2(V)$.
This happens when $E$ is generic in the moduli space.

\item $Hdg(V)$ is properly contained in  $\text{SL}_2(\mathbb{Q})$.
Then $Hdg(V)$ is a 1-dimensional torus.
Then $\text{End}_{\mathbb{Q}-HS}(V) \neq  \mathbb{Q}$,
which implies that $E$ has complex multiplication.
\end{enumerate}
\end{example}

\begin{definition}
\label{definition-cm-type}
A HS $V$ is of  {\it CM-type} if $MT(V)$ is abelian
(such HS defines a special point in the moduli space of HS.)
\end{definition}


\section{Summary so far}
\label{section-summary}

\noindent
Recall:
\begin{itemize}
\item A $\mathbb{Q}$-HS is a finite dimensional $\mathbb{Q}$ 
vector space $V$ with a representation of $\mathbb{R}$-group
$\rho:\underbrace{\mathbb{S}(\mathbb{R})}_{\cong\mathbb{C}^*} 
\to \text{GL}(V \otimes \mathbb{R})$
such that $\rho|_{\mathbb{R}^*}$ is defined over $\mathbb{Q}$.

This gives a Hodge structure
$V\otimes \mathbb{C}=\bigoplus_{p,q}V^{p,q}$,
$V^{p,q}=\overline{V^{q,p}}$.

$V$ is of weight $k$ if
$\forall  r \in \mathbb{R}^k$, $\rho(r)=r^k\cdot \text{Id}$.
$\forall z \in \mathbb{S}(\mathbb{R})$,
$\rho(z)|_{V^{p,q}}=z^p\overline{v}^q\text{Id}$

\item $MT(V)=\bigcap_{\substack{G \in \text{GL}(V) \\ 
\text{s.t.}\text{Im}(\rho) \subset G(\mathbb{R})}}G$
is the Mumford-Tate group.

\item $Hdg(V)=\bigcap_{\substack{
G \in \text{GL}(V)\\ 
\text{s.t.}\text{Im}(\rho|_{\text{U}(1)})\subset G(\mathbb{R})}}G$ 
is the Hodge group.

\item $T^\bullet V=\bigoplus_{e,f \geq 0}
V^{\otimes e}\otimes(V^* )^{\otimes f}$.
A sub-HS of $T^\bullet V$ is the same thing as a
sub-$MT(V)$-representation.
$T^\bullet V^{MT(V)}=$ $MT(V)$-invariants.
$T^\bullet V \cap(T^*V)^{0,0}=$ the space of Hodge $(0,0)$-classes.

\item (Polarization.)
Assume that $V$ has weight $k$ 
$\psi:V \otimes V \to \mathbb{Q}(-k)$ a $\mathbb{Q}$-HS morphism
such that the following form is positive definite:
\begin{align*}
Q: V_{\mathbb{R}}\otimes V_{\mathbb{R}} &\longrightarrow \mathbb{R} \\
Q(x,y) &=(2\pi i)^k \psi(Cx,y)
\end{align*}

\noindent
where $C=\rho(i) \in Hdg(V)(\mathbb{R})$,
then $MT(V),Hdg(V)$ are reductive.

\end{itemize}

\noindent
``We are interested in polarizable Hodge structures
because that is the case of cohomologies of algebraic varieties.''

\section{Some facts about reductive groups}
\label{section-reductive-groups}

\noindent
Let $V$ be a finite dimensional vector space over $k$, $\text{char}(k)=0$
and $G \subset \text{GL}(V)$ an irreducible algebraic $k$-group.

\begin{remark}
\label{remark-connected}
$MT(V)$ and $Hdg(V)$ are connected.
\end{remark}

\noindent
Since by definition $G$ is a subgroup of $\text{GL}(V)$,
we can view $V$ as a faithful representation
(because the inclusion has no kernel).
We say $G$ is {\it reductive} if $V$ is semisimple,
i.e.  $V$ is a direct sum of irreducible representations
$V=V_1 \oplus  … \oplus  V_n$.
Equivalently, Every $G$-representation is semisimple.
Equivalently, the center  $Z^0(G)$ is an algebraic torus
and $G/Z(G)$ is semisimple.
Equivalently, $G(\mathbb{C})$ admits a compact real form.

$\sigma\in \text{Aut}(G(\mathbb{C}))$
is a {\it Cartan involution} 
if $\sigma^2=\text{id}$ and $G^\sigma=\{g\in G(\mathbb{C}):\sigma(\bar{g})=g \}$
is compact.

\begin{example}
\label{example-polarizable-hs}
Let $V$ be a polarizable $\mathbb{Q}$-HS of weight $k$.
$C^2=\rho(i^2)=\rho(-1)\in Z(\underbrace{Hdg(V)}_{G})$,
$G=Hdg(V)$.

Note that  $\sigma=Ad_C:G \to G, g \mapsto CgC^{-1}$,
$\text{id}=Ad_C^2, g \mapsto  C^2g C^{-2}=g$.

We claim that $\sigma$ is a Cartan involution on $Hdg(V)=G$.
Indeed,  $Q$ defines a Hermitian product on $V \otimes \mathbb{C}$.
$g \in G^\sigma$,
\begin{align*}
Q(gx,\overline{gy})&=
Q(gx,\sigma(g)\bar{y})\\
&=(2\pi i)^k
\psi(\underbrace{Cg}_{\in Hdg}x,\underbrace{Cg}_{\in Hdg}C^{-1}\bar{y})\\
&=(2\pi i)^k\psi(x,C^{-1}\bar{y})\\
&=(2\pi i)^k\psi(Cx,\bar{y})\\
&=Q(x,\bar{y}).
\end{align*}

\noindent
Thus, $G^\sigma \subset \underbrace{U(V \otimes \mathbb{C}}_{\text{compact}}$ 
as claimed.
\end{example}

\begin{remark}
\label{remark-polarizable}
A $\mathbb{Q}$-HS is polarizable if and only if
$G=Ad_C$ is a Cartan involution.
\end{remark}

\section{Reductive groups and their representations}
\label{section-reductive-groups-and-their-representations}

\noindent
Let $V$ be a finite-dimensional vector space over $k$, $\text{char}k=0$.
$T^\bullet V=\bigoplus_{e,f \geq 0}V^{\otimes e}\otimes (V^* )^{\otimes f}$.

\begin{theorem}
\label{theorem-reductive-groups-representations}
Let $G\subset \text{GL}(V)$ be a reductive group.
Then
\begin{enumerate}
\item Any finite-dimensional representation of $G$ 
is a subgroup of $(T^\bullet V)^{\oplus  N}$ 
for some  $N>0$.

\item Assume that $H \subset G$ 
is an algebraic subgroup.
There exists a $G$-representation $W$ 
and a 1-dimensional subspace $L \subset W$ 
such that $H=Stab_W(L)$.

\item If $H$ is reductive,
then there exists a $G$-representation $W$ 
and $x \in W$ such that $H=Stab_W(x)$.
\end{enumerate}
\end{theorem}

\noindent
We need some preparations for the proof:

\begin{enumerate}
\item $G\subset \text{GL}(V)$
Let $k[G]$ be the algebra of regular functions on  $G$.

$$
\xymatrix{
\text{GL}(V)\ar[rr]\ar[ddrr]&&\text{End}(V) \oplus  \text{End}(V^* )\\
g \ar@{|->}[rd]\\
&(g,(g^t)^{-1})&
\{(A,B) : A \cdot B^t=\text{Id}\}\ar@{^{(}->}[uu].
}
$$
So we have
$$
\text{Sym}(V \otimes V^* )^{\otimes 2}
\xymatrix{\ar@{->>}[r]&}K[\text{GL}(V)]
\xymatrix{\ar@{->>}[r]&}K[G].
$$

\item $K[G]$ is a Hopf algebra. Indeed, we have
\begin{align*}
\begin{aligned}
G\times G &  \xrightarrow{m}G\\
\{1\}&\xymatrix{\ar@{^{(}->}[r]^e&} G\\
G &  \xrightarrow{inv}G
\end{aligned}\qquad 
\begin{aligned}
K[G] &\xrightarrow{\Delta}K[G]\otimes K[G]\\
K[G]&  \xrightarrow{\varepsilon}K\\
K[G]&\xrightarrow{coinv}K[G].
\end{aligned}
\end{align*}

\noindent
and
$$
\xymatrix{
G \times G \times G\ar[r]^{m \times \text{id}}\ar[d]^{\text{id} \times m}
&G \times G\ar[d]^{m}\\
G \times G\ar[r]^{m}
&G
}
\qquad 
\xymatrix{
K[C]\ar[r]^{\Delta}\ar[d]^{\Delta}
&K[G]^{\otimes 2}\ar[d]^{\text{id} \otimes \Delta}\\
K[G]^{\otimes 2}\ar[r]^{\Delta \otimes \text{id}}
&K[G]
}
$$

Then $V$ is a $G$-representation, i.e. a {\it comodule}.
We have
\begin{align*}
G \times V^*&\to V^*\\
\text{Sym}(V) & \to \text{Sym}(V) \otimes K[G]\\
V & \to V \otimes K[G]\\
\mu:V & \to V \otimes K[G]
\end{align*}

and the following diagram commutes
\begin{equation}
\label{equation-diagram}
\xymatrix{
V\ar[r]^{\mu}\ar[d]_{\mu}
&V \otimes K[G]\ar[d]^{\mu \otimes \text{id}}\\
V \otimes K[G]\ar[r]_{\text{id} \otimes \Delta}
&V \otimes K[G] \otimes K[G].
}
\end{equation}

\noindent
Equation \ref{equation-diagram} defines a $K[G]$-comodule
structure on $K[G]$.

\item Any $G$-representation is a union
of finite-dimensional subrepresentations.
(In particular this applies to $K[G]$.)

Indeed, let $\{a_i\}$ be a basis of $K[G]$.
Fix $x \in V$ and denote
$$
\mu(x)=\sum_{i, \text{finite}} v_i \otimes e_i 
\qquad \text{ and } \qquad 
\Delta e_i=\sum_{j,k}a_{ijk}e_j \otimes e_k,\qquad 
a_{ijk}\in k.
$$
Then
\begin{align*}
\sum_i\mu(v_i)\otimes e_i&=
\sum_{i,j,k}a_{i j k}v_i \otimes e_j \otimes e_k\\
&=\sum_i\left(\sum_{j,k}
a_{i j k}v_k \otimes v_k \otimes e_j \right) \otimes e_i.
\end{align*}

\noindent
$\implies \mu(v_i)=\sum_{j,k}a_{k j i}v_k \otimes e_j$.
$W=\text{span}(x,v_i)\implies \mu:W \to W \otimes K[G]$,
$\dim W < \infty$.
\end{enumerate}

Now we can prove Theorem \ref{theorem-reductive-groups-representations}.

\begin{proof}
\begin{enumerate}
\item Let $W$ be a finite-dimensional $G$-representation.
It is enough to show that $W$ 
is a subrepresentation of $K[G]^{\oplus N}$
for some $N>0$.
\begin{equation}
\label{equation-diagram2}
\xymatrix{
W\ar[r]^{\mu}\ar[d]_{\mu}
&W \otimes K[G]\ar[d]^{\mu \otimes \text{id}}\\
W \otimes K[G]\ar[r]_{\text{id} \otimes \Delta}
& W \otimes K[G] \otimes K[G]
}
\end{equation}

\noindent
Equation \ref{equation-diagram2}
means that $\mu$ defines a morphism of $K[G]$-comodules
$$
W \xymatrix{\ar@{^{(}->}[r]&}\underbrace{W \otimes K[G]}_{
K[G]^{\oplus \dim W}}.
$$
\item Let $H \subset G$ be an algebraic $k$-subgroup.
[We want to show] There exists a $G$-representation $W$
and $L \subset W$ with $\dim L=1$ such that
$H=Stab_W(L)$.

$k[H]=k[G]/I$ for an ideal  $I$.
For $f \in K[G],h,g \in G$ we have
$gf(h)=f(g^{-1}h)$.

Suppose that  $H=V(I)$ for an ideal  $I$.
Note that $g \in H$ if and only if $g\cdot I=I$
since $g\cdot H=V(g I)$
since $0=f(x)=f(g^{-1}gx)=gf(gx)=0$.
That is, $H=Stab_{K[G]}(I)$.

Let $f_1,…,f_m$ 
be generators of $I$.
Then there exists $\tilde{W} \subset k[G]$,
a finite-dimensional $G$-subgroup
with $f_1,…,f_m \in W$,
$H=Stab_{\tilde{W}}(\tilde{W} \cap I)$.
Consider $W = \Lambda^d\tilde{W}$
and $L=\Lambda^d(W \cap I)$, which satisfy the conditions.


\item If $H$ is reductive,  $H=Stab_W(L)$ 
$W=L \oplus  W'$,
$W \otimes W^* =(\underbrace{L \otimes L^*}_{\text{trivial}}
\oplus (…)$.
Let $0\neq x \in L \otimes L^*$.
Then $H=Stab_{W \otimes W^*}Stab(x)$.

\end{enumerate}
\end{proof}

\noindent
As a corollary,

\begin{lemma}
\label{lemma-mt-is-reductive-for-polarizable-hs}
If $V$ is a polarizable $\mathbb{Q}$-HS,
then [the Mumford-Tate group is reductive]
$MT(V)=$ Stabilizer of all $(0,0)$-Hodge classes
in $T^\bullet V$.
\end{lemma}

\begin{proof}
Let $G=MT(V)$ and
$$
G'=\bigcap_{\substack{x \in (0,0)\text{-Hodge}\\\text{classes}}}Stab(x)
\subset \text{GL}(V).
$$
$G \subset G'$.

By part 3 of Theorem \ref{theorem-},
$G=Stab_W(x)$ where $W$ is a representation of $\text{GL}(V)$.
By part 1 of Theorem \ref{theorem-},
$W \subset (T^\bullet V)^{\oplus N}$.
$x$ is fixed by $G=MT(V)$,
then  $x$ is a $(0,0)$-Hodge class.
Thus $G' \subset G$.
\end{proof}

\noindent

[Approximate comment]
We have shown that the representations of $MT(V)$ 
live in that tensor algebra.
We would like to find a universal object
such that any Hodge structure is realised as a
representation of such group.
This is analogous to Galois theory,
where we take an inverse limit over Glois groups
of field extensions,
which provides a universal object and
an equivalence of categories.

\section{Short summary}
\label{section-summary2}

$V$ a $\mathbb{Q}$-Hs, polarizable.
$\rho:\mathbb{S}(\mathbb{R})\to \text{GL}(V \otimes \mathbb{R})$,
$MT(V)\subset\text{GL}(V)$, the smallest
subgroup over $\mathbb{Q}$ such that
$MT(V)(\mathbb{R})$ contains $\text{Im}(\rho)$.
Polarizable $\implies$ $MT(V)$ is reductive.
$MT(V)=$ stabilizer of all $(0,0)$ Hodge classes
in $T^\bullet V=\bigoplus_{e,f\geq 0}V^{\otimes e}\otimes(V^* )^{\otimes f}$.

\section{Absolute Mumford-Tate group}
\label{section-absolute}

Let $V' \subset V$ be a sub-HS.
$MT(V) \subset Stab_V(V')=\left\{\begin{pmatrix}\ast&\ast\\ 
0&\ast\end{pmatrix} \right\} $.

We also have a map
$MT(V) \xymatrix{\ar@{->>}[r]&}MT(V')$.

\medskip\noindent
The absolute MT group should be something
like the inverse limit $\lim_{\xymatrix{&\ar[l]}}MT(V) $.

\medskip\noindent
Let $(\mathbb{Q}-HS)$ be the category
of $\mathbb{Q}$-HS.
\begin{enumerate}
\item $\mathbb{Q}$-linear abelian category.

\item Has $\otimes$, which is commutative with unit $1=\mathbb{Q}(0)$,
which a Hodge structure such that $\text{End}(\mathbb{Q}(0))=\mathbb{Q}$.
We also have duality, $\forall V\exists V^*$
and $\mathbb{Q}(0)=q \to V \otimes V^*$
which is the ``diagonal embedding'',
and $V \otimes V^* \to 1=\mathbb{Q}(0)$,
which is the trace.

\item There exists a functor
$\omega:(\mathbb{Q}-HS)\to (\mathbb{Q}-v.s.)$,
forgetful functor.
$\omega$ is called {\it fibre functor}.
A $\otimes$-functor that is exact and faithful
(injective on Hom's).
\end{enumerate}

\noindent
The pair $(\mathbb{Q}$-HS, $\omega)$ is called
a {\it neutral Tannakian category}.

\begin{theorem}
\label{theorem-tannakian}
A neutral Tannakian category $(\mathcal{C},\omega)$ 
(where $\mathcal{C}$ is $K$-linear, $\text{char}K=0$)
is equivalent to the category of finite
dimensional representations of an affine group scheme over $K$
$Spc K[G]$, $K[G]$ Hopf algebra.
$G$ is a pro-alg. group called the {\it Tannakian fundamental group}.
\end{theorem}

\noindent
Apply this to $(\mathbb{Q}-HS),\omega$,
$\implies $ $G=MT_{\mathbb{Q}-HS}$ is the
{\it absolute MT group}.

Let  $(\mathbb{Q}-HS)^{pol}$ be the full $\otimes$-subcategory
of polarizable HS.
$MT_{(\mathbb{Q}-HS)^{\text{pol}}}$ is pro-reductive.

\section{How to construct the Tannakian fundamental group}
\label{section-tannakian-fundamental-group}

\noindent
We have $\omega:\mathcal{C}\to (K-v.s.)$.
Let $\text{Aut}(\omega)^{\otimes}$ be the automorphisms
ot $\omega$ as a $\otimes$-functor.
That is, if $g \in \text{Aut}(\omega)^{\otimes}$,
we have $\forall  X \in \mathcal{C}$, $g_x\in \text{GL}(\omega(X))$.

We also have:

\begin{itemize}
\item 
Compatibility with morphisms,
i.e. for every $f:X \to Y$ we have
$$
\xymatrix{
  \omega(X)\ar[r]^{\omega(f)}\ar[d]_{g_X}
&\omega(Y)\ar[d]^{g_Y}\\
\omega(X)\ar[r]_{\omega(f)}
&\omega(Y)
}
$$
commutes, i.e. $g_Y \circ \omega(f)=\omega(f) \circ g_X$.

\item Compatibility with $\otimes$:
\begin{align*}
g_{X \otimes Y}&=g_X \otimes g_Y\\
g_1&=1 \in K^* = \text{GL}(K),
  g_{X^*}=(g^*_X)^{-1}.
\end{align*}
\end{itemize}

\noindent
$\text{Aut}(\omega)^\otimes$ is the group of $K$-points of
the Tannakian fundamental group.

\medskip\noindent
[Our category $(\mathbb{Q}-HS)^{\text{pol}}$ is equivalent to the
category of representations of this group.
But the problem is that this group is huge.
So we shall restrict to a smaller subcategory.]

Let $V \in (\mathbb{Q}-HS)^{\text{pol}}$ 
and $\left<V\right>$ the full $\otimes$-subcategory generated by $V$.
Objects are sub-HS of $T^\bullet V$ and their direct sums.

What is the Tannakian fundamental group of $\left<V\right>$?

Consider $\omega:\left<V\right>\to (\mathbb{Q}-v.s.)$.
For $g \in \text{Aut}(\omega)^{\otimes}$
we have that $g_V \in \text{GL}(V)$ 
is uniquely determined by $g$.

Let $x \in T^\bullet V \cap(T^\bullet V)^{0,0}$.
We have an embedding
\begin{align*}
1=\mathbb{Q}(0)&\xymatrix{\ar@{^{(}->}[r]&}T^\bullet V\\
1 & \mapsto x.
\end{align*}

\noindent
We have $g_{\mathbb{Q}(0)}=1 \implies g_V \in \text{Stab}_{T^\bullet V}(x)$.
Thus  $g_V \in MT(V)$.
We also have the converse, so we see that
the Tannakian fundamental group of $\left<V\right>$ 
is $MT(V)$.
This could be taken as an equivalent definition of $MT(V)$


\section{Hodge structures of abelian varieties}
\label{section-hodge-structures-of-abelian-varieties}

\noindent
Let $(\mathbb{Q}-HS)^{\text{ab}}$ 
be the full $\otimes$-subcategory of $(\mathbb{Q}-HS)^{\text{pol}}$ 
genertated by $H^1(A,\mathbb{Q})$
where $A$ is an abelian varieties.

Recall that if $A$ is a projective variety it
is biholomorphic to $\mathbb{C}^n/\Lambda$
and $H^1(A,\mathbb{C})=H^{1,0}(A) \oplus  H^{0,1}(A)$.

\medskip\noindent
[Suppose that you have some Hodge structure,
say, on the cohomology of your favourite variety.
Does it belong to this category or not?
It is not clear --- there might be some complicated Hodge structures in there]

Let $V=H^1(A,\mathbb{Q})$, $MT(V) \subset \text{GL}(V)$.
Let $\mathfrak{mt}(V)=\text{Lie}(MT(V))\subset \mathfrak{gl}(V)=V \otimes V^*$,
a sub-HS (of $\mathfrak{gl}(V)$, which is always a Hodge structure).
That is, the Lie algebra of $MT(V)$ always carries a Hodge structure.
This HS may only have types
$(-1,1),(0,0),(1,-1)$
because $((1,0)+(0,1)) \otimes ((-1,0),(0,-1))$.

\begin{proposition}
\label{proposition-lie-algebra-of-mtv}
Let $V \in (\mathbb{Q}-HS)^{\text{ab}}$.
Then the HS on $\mathfrak{mt}(V)$ may have only Hodge types
$(-1,1),(0,0),(1,-1)$.
\end{proposition}

\begin{proof}
We may assume that $V \subset T^\bullet W$
where $W=H^1(A,\mathbb{Q})$ by the tensor construction.
Then we must have $MT(W) \xymatrix{\ar@{->>}[r]&} MT(V)$,
which on the level of Lie algebras becomes
$\mathfrak{mt}(W) \xymatrix{\ar@{->>}[r]&}\mathfrak{mt}(V)$,
which is a surjection of HS.
\end{proof}

\noindent
The next example shows how we may use this result.

\begin{example}
\label{example-hodge-type-prop}
\begin{enumerate}
\item Let $X=$ a K3 surface, $X \subset \mathbb{P}^3$, a smooth quartic.
Then $K_X=\mathcal{O}_X$, $h^{2,0}=h^{0,2}=1$,
$H^2(X,\mathbb{C})=H^{2,0}\oplus H^{1,1}\oplus H^{0,2}$.
The HS is $V=H^2(X,\mathbb{Q})$
and the intersection product is a map
$V \otimes V \to \mathbb{Q}(-2)$.
Recall that the Hodge group is the image of $U(n)$ under the representation
from the Deligne torus.
In this case we have
$MT(V) \supset Hdg(V)\subset \text{SO}(V,\psi)$.
Also
$$
\mathfrak{mt}(V)=\underbrace{\mathfrak{hdg}(V)}_{\text{Lie}(Hdg(V))}
\oplus \underbrace{\mathbb{Q}(0)}_{\text{center of }\mathfrak{gl}(V)}.
$$
And also $(1,-1),(0,0),(-1,1)$.
Now since $H^{2,0}$ is 1-dimensional we can compute
$$
\Lambda^2(V \otimes \mathbb{C})
=\underbrace{(H^{2,0}\otimes H^{1,1})}_{3,1}
\oplus \underbrace{(\Lambda^2H^{1,1}\oplus H^{2,0}\otimes H^{0,2})}_{2,2} 
\oplus\underbrace{(H^{1,1}\otimes H^{0,2})}_{1,3} 
$$
\item $X \subset \mathbb{C}P^3$ hypersurface of degree $\geq 5$.
Then $h^{2,0}>1$, $V=H^2(X,\mathbb{Q})$,
$\mathfrak{hdg}(V) \subset \mathfrak{so}(V,\psi)$.
Hodge types: $(2,2),(1,-1),(0,0),(-1,1),(-2,-2)$.
\end{enumerate}
\end{example}

\section{Variations of Hodge structures}
\label{section-variations-of-hs}

\noindent
A {\it family} is $\pi:\mathcal{X} \to B$ with $\mathcal{X},B$ complex
manifolds, $\pi$ is a submersion, $B$ is connected
and $\mathcal{X} \xymatrix{\ar@{^{(}->}[r]&} \mathbb{P}^N \times B$.

Then $R^k\pi_*\mathbb{Z}/\text{tors.}$ is a local system over $B$.
In fact, $(R^k\pi_*\mathbb{Z})_t=H^k(\mathcal{X}_t,\mathbb{Z})$
where $\mathcal{X}_t=\pi^{-1}(t)$.
That is, $R^k\pi_*\mathbb{Z}$ is a sheaf with the information
of the cohomologies of the fibres.

\medskip\noindent
Consider
$$
\xymatrix{
0\ar[r]
&\pi^*\Omega^1_B\ar[r]
&\Omega^1_\mathcal{X}\ar[r]
&\Omega_{\mathcal{X}/B}^1\ar[r]
&0
}
$$
Assume $\dim B=1$
(This is used to prove the property of the
Gauss-Manin connection, though it can be
also be done by restricting to curves.)
Then
$$
\xymatrix{
0\ar[r]
&\pi^*\Omega^1_B \otimes \Omega_{\mathcal{X}}^{k-1}\ar[r]
&\Omega_{\mathcal{X}}^k\ar[r]
&\Omega_{\mathcal{X}/B}^k\ar[r]
&0
}
$$
\begin{equation}
\label{equation-diagram3}
\xymatrix{
0\ar[r]
&\pi^*\Omega_B^1 \otimes \Omega_{\mathcal{X}/B}^\bullet[-1]\ar[r]
&\Omega_{\mathcal{X}}^\bullet\ar[r]
&\Omega^\bullet_{\mathcal{X}/B}\simeq \pi^{-1}\mathcal{O}_B\ar[r]
&0
}
\end{equation}

We can construct the vector bundle
$$
\mathcal{V}_k=(R^k\pi_*\mathbb{C})\otimes\mathcal{O}_B
=R^k\pi_*(\pi^{-1}\mathcal{O}_\Big)
\simeq R^k\pi_*(\Omega^\bullet_{\mathcal{X}/B}).
$$

On $\mathcal{V}_k$ we have a flat connection
from Equation \ref{equation-diagram3}.
We get
$$
\xymatrix{
  R^k\pi_*(\Omega^\bullet_{\mathcal{X}/B}\ar[r]
&R^{k+1}\pi_*(\pi^*\Omega^1_B \otimes \Omega^\bullet_{\mathcal{X}/B}[-1]\ar[r]
&\mathcal{V}_k\ar[r]^{\nabla}
&\Omega_B^1 \otimes R^k\pi_*\Omega^\bullet_{\mathcal{X}/B}
=\Omega^1_B \otimes \mathcal{V}^k.
}
$$
The map $\nabla$ is called the {\it Gauss-Manin} connection.

\medskip\noindent
Define $F^pV_k=R^k\pi_*(\Omega_{\mathcal{X}/B}^{\geq p})$.
Then $F^p\mathcal{V}_k \subset \mathcal{V}_k$
is a holomorphic subbundle.

The {\it Hodge filtration} on $\mathcal{V}_k$.
\begin{enumerate}
\item $\nabla:\mathcal{V}_k \to \Omega_B^1 \otimes \mathcal{V}_k$ 
a flat connection.

\item The Hodge filtration
$$
0 \subset … \subset F^p\mathcal{V}_k
\subset F^{p-1}\mathcal{V}_k
\subset…\subset V_k.
$$
\item (Griffiths transversality.)
When we restrict this construction to $F^p$
the relation we obtain is {\it Griffiths transversality}.
$\nabla(F^k\mathcal{V}_k) \subset
\Omega^1_B \otimes F^{p-1}\mathcal{V}_k$.
\end{enumerate}

\begin{definition}
\label{definition-polarization-on-vk}
A {\it polarization} on $\mathcal{V}_k$
is defined by a section of $R^2\pi_*\mathbb{Z}(1)$.
\end{definition}

\noindent
By the exponential sequence
$$
\xymatrix{
0\ar[r]
&\mathbb{Z}(1)\ar[r]
&\mathbb{C}\ar[r]^{\text{exp}}
&\mathbb{C}^*\ar[r]
&1
}
$$
we obtain maps
$$
R^1\pi_*\mathbb{C}^*\to R^2\pi_*\mathbb{Z}(1),
$$
and
$$
\mathcal{O}_{\mathbb{P}^N}(1) \in
H^1(\mathcal{X},\mathbb{C}^*)
\to H^0(B,R^1\pi_*\mathbb{C}^*)
\to H^0(B,R^2\pi_*\mathbb{Z}(1)).
$$

From there we get
$$
\psi:V_B \otimes V_B \to \mathbb{Z}_B(-k),
$$
where $V_B=R^k\pi_*\mathbb{Z}/\text{tors.}$,
a morphism of local systems that defines
polarizations on the fibres of $\pi$.

\begin{definition}
\label{definition-z-variation}
A {\it $\mathbb{Z}$-variation of Hodge-Structure over $B$} 
is a $\mathbb{Z}$-local system $V_B$ with a 
fibre  $\mathbb{Z}^r$ for some $r\geq 0$ 
and a finite filtration 
of holomorphic subbundles on $\mathcal{V}=V_B \otimes \mathcal{O}_B$ 
$$
0\subset…\subset F^p\mathcal{V}
\subset F^{p-1}\mathcal{V}
\subset…\subset \mathcal{V}
$$
[which satisfies Griffiths transversality]
such that
\begin{enumerate}
\item for all $t \in B$ $(V_{B,t},F^\bullet \mathcal{V}_t)$ 
is a $\mathbb{Z}$-HS.
\item $\nabla(F^p\mathcal{V})\subset \Omega^1_B \otimes F^{p-1}\mathcal{V}$
where $\nabla$ is the flat connection on $\mathcal{V}$.
\end{enumerate}

\end{definition}

\section{The Hodge locus}
\label{section-hodge-locus}

\noindent
[Next we shall try to study the MT group
on families. Since MT essentially is the satbilizer of the Hodge 
classes, we look for those. There are fibres that may
have more Hodge classes than others. This leads to the following concept.]

Let $B$ be a complex manifold and $V_B$ a $\mathbb{Z}$-local system
with a VHS. $F^p\mathcal{V}$ of weight $2k$.
We have
$$
\xymatrix{
\text{Tot}(\mathcal{V})\ar[r]^p
& B\\
\text{Tot}(V_B)\ar@{^{(}->}[u]\ar[ur]_{\text{covering}}.
}
$$
Note that $\text{Tot}(V_B)=\amalg W_i$
where $W_i$ are the connected components of $\text{Tot}(V_B)$.

In this setting, a Hodge class is a point $x$ in one of $W_i$ 
such that $x \in F^k\mathcal{V} \cap \overline{F^k\mathcal{V}}$ 
$\iff$
$x \in F^k\mathcal{V} \cap W_i$.

Let $Z_i=\text{Tot}(F^k\mathcal{V})\cap W_i$.
Note that $Z_i$ is a closed subvariety (maybe singular)
of $\text{Tot}(\mathcal{V})$.

If the projections of these $Z_i$, $p(Z_i)$, is not equal to $B$,
then $p(Z_i)$ is in the Hodge locus.
That is, the {\it Hodge locus} is
\begin{align*}
B^{\text{Hdg}}(V_B)&=\bigcup_{i:p(Z_i)\neq B}p(Z_i)\\
B^{\text{Hdg}}=\bigcup_{\substack{e,f \\ \text{countable}\\\text{union}}}
B^{\text{Hdg}}(\underbrace{T^{e,f}V_B}_{
V_B^{\otimes e}\otimes(V^*_B)^{\otimes f}}).
\end{align*}

\noindent
[Our goal is to consider the complement of
the Hodge locus and study there the MT group.]

\section{Third summary}
\label{section-summary}

\noindent
Recall: $\mathbb{Z}$-VHS of weight $k$
\begin{itemize}
\item $B$ complex manifold,

\item $V_B$ a local system over $B$ with fibre $\mathbb{Z}^2$,
$\mathcal{V}=V_B \otimes \mathcal{O}_B$,

\item The Hodge filtration $F^p\mathcal{V} \subset \mathcal{V}$
(filtration by holomorphic subbundles)
$$
0=F^{k+1}\mathcal{V} \subset F^k\mathcal{V}
\subset…\subset F^0\mathcal{V}=\mathcal{V}.
$$

\item $\nabla:\mathcal{V} \to \Omega^1_B \otimes \mathcal{V}$
the flat connection induced by $V_B$.
Griffiths transversality condition
$\nabla(F^p\mathcal{V})\subset\Omega_B^1\otimes F^{p-1}\mathcal{V}$.

\item Polarization given by a morphism of local systems
$\psi:V_B \otimes V_B \to \mathbb{Z}_B$
such that for every $t \in B$, $\psi_t$ defines a polarization
on $(V_{B,t},F^p\mathcal{V}_t)$.
$\psi$ is $(-1)^k$-symmetric.
We have a representation 
$\rho:\mathbb{S}(\mathbb{R}) \to \text{GL}(V_{B,t}\otimes \mathbb{R})$,
$C=\rho(1)$,
$Q(x,y)=(2\pi i)^k \psi(Cx,y)$,
$Q>0$ on  $V_{B,t} \otimes \mathbb{R}$.

\item Hodge locus.
Assume weight = $2k$.
$x \in V_{B,t}$ is a Hodge class if and only if
$x \in F^k\mathcal{V} \cap V_{B,t}$.

 $$
\xymatrix{
\text{Tot}(V_B)\ar@{^{(}->}[r]\ar@{=}[d]
&\text{Tot}(\mathcal{V})\ar[d]^{p}\\
\amalg W_i\ar[r]
&B
}
$$
where $W_i$ are the connected components and
$p|_{W_i}\to B$ is a covering.

Let $Z_i=W_i\cap \text{Tot}(F^k\mathcal{V})$,
which is a closed complex subvariety of $\text{Tot}(\mathcal{V})$.
There are two possibilities:
\begin{enumerate}
\item Either $Z_i=W_i$ and $p(Z_i)=B$,
\item or $\dim Z_i=\dim W_i= \dim B \implies p(Z_i)$
is nowhere dense. [This one corresponds to the Hodge locus.]
\end{enumerate}

\noindent
$$
B^{Hdg}(V_B)
=\bigcup_{p(Z_i)\neq B}p(Z_i)
$$
Then the Hodge locus is
$$
B^{Hdg}=\bigcup_{e,f \geq 0}B^{Hdg}(T^{e,f}V_B
$$
where $T^{e,f}V_B=V_B^{\otimes e}\otimes (V_B^* )^{\otimes f}$.

Note that $B \neq  B^{Hdg}$ 
because $B^{Hdg}$ is a countable union of nowhere dense subsets.

\end{itemize}

\section{Hodge locus (continued)}
\label{section-hodge-locus-cont}

\begin{proposition}
\label{proposition-proper}
$\forall  i$, $p|_{Z_i}:Z_i \to B$
is a proper map.
Thus $p(Z_i)\subset B$ is a closed subvariety.
\end{proposition}

\begin{proof}
Let $x \in Z_i$.
Define
$$
A=(2\pi i) \psi(x,x)
\underbrace{=}_{\substack{\text{$x$ is of} \\ \text{type $(k,k)$}}}
Q(x,x)\geq 0
$$
$A$ does not depend on the choice of $x \in Z_i$
because $\psi$ is flat, i.e. $\nabla \psi=0$.

Let $S_A=\{z \in \text{Tot}(\mathcal{V}^{k,k})
:Q(z,\bar{z})=A\}$.
Then $p|_{S_A}:S_A \to B$
is a sphere bundle $\implies $ proper map.
Since $Z_i$ is a closed subset of $S_A$,
we conclude that $p|_{Z_i}$ is also a proper map.
\end{proof}

\begin{remark}
\label{remark-quasi-projective}
When $B$ is quasi-projective
$p(Z_i)$ is algebraic (Cattani-Deligne-Kaplan).
\end{remark}

\noindent
If $W_i=Z_i \implies W_i$ spans
a sub-local system on $V_B$,
that is in  $\mathcal{V}^{k,k}$.

\begin{definition}
\label{definition-hodge-tate-local-system}
$V_B^{HT}$ is the biggest sub-local system in $V_B$
that is Hodge-Tate, i.e. in $\mathcal{V}^{k,k}$.
\end{definition}

\noindent
Let $t_0 \in B\setminus B^{Hdg}$.
Denote $MT_{t_0}$ the MT group of the corresponding Hodge structure,
i.e.  $MT_{t_0}=MT(V_{B,t},F^\bullet\mathcal{V}_{t_0})$.
We do parallel transport along the loops in $B$ based in $t_0$,
and obtain an action of the fundamental group:
the monodromy representation is
$\mu:\pi_1(B,t_0) \to \text{GL}(V_{B,t_0})$.

\begin{proposition}
\label{proposition-monodromy}
$MT_{t_0}$ contains a finite index subgroup of $\text{Im}(\mu)$.
\end{proposition}

\begin{proof}
Recall that $MT_{t_0}$ is the stabilizer of all
$(0,0)$ Hodge classes in
all tensor powers, i.e. in $(T^\bullet V_B)_{t_0}$.

There exists Hodge classes
$x_j \in (T^{e_j,f_j}V_B)_{t_0}\cap T^{e_j,f_j}\mathcal{V})^{0,0}_{t_0}$
where $j=1,…,m$ 
such that $MT_{t_0}=\bigcap_{j=1}^m Stab(x_j)$.

Note that $t_0 \not\in B^{Hdg}\implies 
x_j\in(T^{e_j,f_j}V_B)^{HT}_{t_0}$,
which is a sublocal system.
Now consider the monodromy action:
$\pi_1(B,t_0)$ acts on $(T^{e_j,f_j}V_B)^{HT}_{t_0}$.
By Proposition \ref{proposition-proper},
all $x_j$'s have finite $\pi_1(B,t_0)$ orbits.
Thus there exists a finite index subgroup
$\pi_1(B,t_0)$ that stabilizes all $x_j$'s.
Thus, the image of this subgroup under the
monodromy representation lies in $MT_{t_0}$ 
(because it stabilizes all $x_j$'s).
\end{proof}

\noindent
\begin{definition}
\label{definition-generic-mt-group}
For any $t \in B$, define
$$
G_t=\bigcap_{x \in (T^\bullet V_B)_t^{HT}}Stab(x).
$$
Then $G_t$ define a local system of algebraic groups
with fibre $G$ called the {\it generic MT group}.
\end{definition}

For any $t \in B\setminus B^{Hdg}$,
$G_t=MT_t$ 
and for any $t \in B^{Hdg}$, $MT_t \subset G_t$

\begin{example}
\label{example-hypersurface}
Let $B$ be the space of smooth hypersurfaces $X \subset \mathbb{C}P^n$
of degree $d$.
Then $B$ is a Zariski open subset of $\mathbb{C}P^{\binom{n+d}{n}-1}$,
i.e. it is a quasi-projective variety.

Let $\pi:\mathcal{X} \to B$ be the universal family.

Assume that  $n=2r$.
$V_B=R^{2r-1}\pi_\ast\mathbb{Z}$
is a $\mathbb{Z}$-VHS$/B$.

$\psi:V_B \otimes V_B \to \mathbb{Z}_B$
is skew-symmetric,
$\mu:\pi_1(B,t_0) \to Sp(V_{B,t_0},\psi)$.

\begin{theorem}[Kazhdan-Margulis]
\label{theorem-kazhdan-margulis}
$\text{Im}(\mu)$ is Zariski dense in 
$Sp(V_{B,t_0}\otimes \mathbb{Q},\psi)$.
\end{theorem}

And a corollary is that
\begin{lemma}
\label{lemma-generic-hodge-is-sp}
The generic Hodge group of the family $\mathcal{X}$
is $Sp(V_{B,t_0} \otimes \mathbb{Q},\psi)$.
\end{lemma}

\medskip\noindent
Consider $n=4$, $d=5$, i.e.
$X \subset \mathbb{C}P^5$ a quintic 3-fold.
This is a Calabi-Yau, i.e. $K_X = \mathcal{O}_X$.
Thus $h^{3,0}=1,h^{2,1}=101,h^{1,2}=101$, $h^{0,3}=1$.

Assume  $X=\mathcal{X}_t$ for $t \in B \setminus B^{Hdg}$,
$Hdg(X)=Sp(H^3,\psi)$,
$\text{Lie}(Hdg(X))\cong S^2H^3(3)$
where $S^2$ is the symmetric square, and it is twisted by 3.

In the Hodge structure $S^2H^3(3)$ 
we have Hodge classes of types
$(3,-3),(2,-2),(1,-1),(0,0),…,(-3,3)$.
Thus $H^3(X,\mathbb{Q}) \notin (\mathbb{Q}-HS)^{ab}$.
\end{example}

\section{Period domains and period maps}
\label{section-periods}

\noindent
[A variation of Hodge structures is just a bundle.
The classifying space of these bundles is called period.]

\medskip\noindent
Let $V_B$ be a local system over $B$.
Pass to the universal covering $u:\tilde{B}\to B$.
Consider $u^* V_B=V \otimes \mathbb{Z}_{\tilde{B}}$.
Let $V=\mathbb{Z}^2$,
$\mathcal{V}=V \otimes \mathcal{O}_{\tilde{B}}$.
$F^p\mathcal{V} \subset \mathcal{V}$,
$t_0 \in \tilde{B}\setminus B^{Hdg}$,
$Hdg(V_{B,u(t_0)})=G$, a reductive group.
We also have $\rho_0:U(1) \to G(\mathbb{R})$.

\begin{definition}
\label{definition-connected-component}
Let $\mathcal{D}$ be the connected component of
$\Hom(U(1),G(\mathbb{R}))$ containing $\rho_0$.
\end{definition}

\begin{lemma}
\label{lemma-d-orbit}
$\mathcal{D} \cong G(\mathbb{R})^0/Stab(\rho_0)$,
where $G(\mathbb{R})$ acts on
$\Hom(U(1),G(\mathbb{R}))$ by conjugation on the image,
i.e. for $g \in G(\mathbb{R})$ we put
$(g\cdot \rho_0)(z)=g\rho_0(z)g^{-1}$.
\end{lemma}

\begin{proof}
$\text{Im}(\rho_0)$ is a compact abelian subgroup
of $G(\mathbb{R})$, thus $\text{Im}(\rho_0)$ is contained in a
maximal compact torus in $G(\mathbb{R})$.

Fact: $G$-reductive implies that all maximal compact
tori are conjugate by the action of $G(\mathbb{R})^0$.

$\rho:U(1) \to G(\mathbb{R})$ such that $[\rho] \in \mathcal{D}$,
then there exists $g \in G(\mathbb{R})$ such that
$g\cdot\rho \in T_0$,
$[g,\rho]\in \mathcal{D}$,
$\Hom(U(1),T_0)$ is a discrete topological space,
thus $g\rho=\rho_0 \iff \rho=g^{-1}\rho_0$.
\end{proof}

\medskip\noindent
Let $K=Stab(\rho_0)\subset G(\mathbb{R})^0$.

$C=\rho_0(i)$, $\sigma=Ad_C$ 
is a Cartan involution on $G(\mathbb{R})$.
$G^\sigma = \{g \in G(\mathbb{C}):\bar{g}=(g)\}$ 
is compact.
We also have that $\sigma|_K=\text{id}
\implies K\subset G^\sigma
\implies K$ is compact.

Consider the Lie algebra of $G$, i.e. $\mathfrak{g}=\text{Lie}(G)$.
Consider the Hodge decomposition induced by $\rho_0$, namely
$\mathfrak{g}_\mathbb{C}=\bigoplus_{p \in \mathbb{Z}}\mathfrak{g}^{p,-p}$.
We see that the action is trivial on the component $(0,0)$:
\begin{align*}
z \in U(1)&\implies \bar{z}=z^{-1},\\
\rho_0(z)|_{\mathfrak{g}^{p,-p}}&=z^{2p}\text{Id}\\
&\implies \mathfrak{g}^{0,0}\cap \mathfrak{g}_\mathbb{R}=\text{Lie} K.
\end{align*}

\noindent
Note that $F^p\mathfrak{g}_\mathbb{C}
=\bigoplus_{r \geq p}\mathfrak{g}^{r,-2}$,
$F^0\mathfrak{g}_\mathbb{C}=$ Lie subalgebra that preserves the Hodge
filtration on $V_{B,t_0}$.
$P \subset G(\mathbb{C})$ be such that 
$\text{Lie}(P)=F^0\mathfrak{g}_\mathbb{C}$.

\begin{definition}
\label{definition-dhat}
$\hat{\mathcal{D}}=G(\mathbb{C})/P$ where $P$ is the stabilizer
of the Hodge filtration on $V_{B,t_0}\otimes \mathbb{C}$.
\end{definition}

\noindent
$\hat{\mathcal{D}}$ is a flag variety.

[We want to see how to embed $\mathcal{D}$ in $\hat{\mathcal{D}}$.]
Consider $\text{Lie}(P) \cap \text{Lie}(G(\mathbb{R}))
=F^0\mathfrak{g}_\mathbb{C} \cap \mathfrak{g}_\mathbb{R}
=\text{Lie}(K)\implies $
Let $x_0=[1] \in \hat{\mathcal{D}}$ 
be the $G(\mathbb{R})^\bullet$-orbit of $x_0$ is
$\mathcal{D}=G(\mathbb{R})^\bullet/K
\implies \mathcal{D} \subset \hat{\mathcal{D}}$.

Note that
\begin{align*}
\dim_\mathbb{R}(\mathcal{D})
&=\dim_\mathbb{R} \mathfrak{g}_\mathbb{R}
-\dim_\mathbb{R}(\mathfrak{g}^{0,0}\cap \mathfrak{g}_\mathbb{R})\\
&=\dim_\mathbb{C}\mathfrak{g}_\mathbb{C}-\dim\mathbb{C}\mathfrak{g}^{0,0}\\
&=2\cdot \dim_\mathbb{C}\left(\bigoplus_{r<0}\mathfrak{g}^{r,-r}\right)\\
&=2\cdot \dim_\mathbb{C}\hat{\mathcal{D}}\\
&=\dim_\mathbb{R}\hat{\mathcal{D}}.
\end{align*}

\noindent
Thus we have an open embedding $\mathcal{D} \subset \hat{\mathcal{D}}$.

\medskip\noindent
Over $\tilde{B}$ we have a flat bundle
$V\otimes \mathcal{O}_{\tilde{B}}$ 
and a filtration $F^p \mathcal{V} \subset \mathcal{V}$ 
by holomorphic subbundles.
Thus we get a period map
$p:\tilde{B}\to \mathcal{D} \subset \hat{\mathcal{D}}$.

Griffiths transversality is [means] the following condition
$$
dp:T\tilde{B}\to p^*\mathcal{H}
$$
where $\mathcal{H}$ is given as follows.
$F^{-1}\mathfrak{g}_\mathbb{C}/F^0\mathfrak{g}_\mathbb{C}$ defines a homogeneous
subbundle of $T\hat{\mathcal{D}}$ which we call
the {\it horizontal subbundle} $\mathcal{H}$.

\medskip\noindent
If 
\begin{equation}
\label{equation-hermitian}
\mathfrak{g}_\mathbb{C}=\mathfrak{g}^{-1,1}\oplus \mathfrak{g}^{0,0}
\oplus \mathfrak{g}^{1,-1}
\end{equation}

\noindent
then $F^{-1}\mathfrak{g}_\mathbb{C}=\mathfrak{g}_\mathbb{C}$
which implies that $\mathcal{H}=T\hat{\mathcal{D}}$.
Thus Griffiths transversality holds automatically
for any holomorphic map $\tilde{B} \to \mathcal{D}$.

\begin{proposition}
\label{proposition-hermitian}
If Equation \ref{equation-hermitian} holds,
then $\mathcal{D}$ is a Hermitian symmetric space.
\end{proposition}

\begin{proof}
The $Ad_C$ [Weil operator] acts on 
$\mathfrak{g}^{1,-1}\oplus \mathfrak{g}^{-1,1}$ as $-\text{Id}$.
Thus $Ad_C$ gives a holomorphic isometry fixing $x_0\in \mathcal{D}$
and acting as $-\text{Id}$ on $T_{x_0}\mathcal{D}$.
\end{proof}


\section{The Kuga-Satake construction}
\label{section-kuga-satake}

\noindent
Let $X$ be a compact hyperkähler manifold
(aka IHS manifold).
Recall that this means that $X$ is a
compact smooth manifold, simply connected
equipped with a Riemannian metric $g$ such that
$\text{Hol}(\nabla^g)=\text{Sp}(n)$ 
where $\nabla^g$ is the Levi-Civita connection
and $\text{Sp}(n)$ is the compact form of
$\text{Sp}(2n,\mathbb{C})$,
i.e. $\text{Aut}(\mathbb{H}^n,\left<\cdot,\cdot\right>)$,
the quaternions equipped with the canonical
quaternionic Hermitian product
$\left<v,w\right>:=\sum_{i=1}^nv_i\overline{w_i}$.

The holonomy principle implies
that there exist complex structures $I,J$ and $K$ 
such that $I J=J I=K$.

Let $X_I$ be the complex manifold $X$ equipped with $I$.
Again by the holonomy princple, there is a holomorphically symplectic form
$\sigma_I \in H^0(X,\Omega_{X_I}^2)$.
In fact, this is unique up to scaling and
$H^0(X,\Omega_{X_I}^2)\cong \mathbb{C}\sigma_I$.
Also, $\sigma_I^n$ is a holomorphic volume form,
so that $K_{X_I}\cong\mathcal{O}_{X_I}$,
so that these are Calabi-Yau manifolds.
$\dim_{\mathbb{C}}X_I=2n$.

\begin{example}[IHS manifolds]
\label{example-hk}
\begin{enumerate}
\item $S$ complex K3 surfaces. E.g. quartics in $\mathbb{C}P^3$
[by CY theorem there is a hyperkähler metric;
it's difficult to construct such a metric explicitly].

\item Let $S$ be a complex K3 surface.
Then $\text{Hilb}^n(S)$, the Hilbert scheme
of length-$n$ 0-dimensional subschemes of $S$.

\item Let $A =\mathbb{C}^2/\Lambda$.
Consider the Albanese map $\text{Hilb}^{n+1}(A)\to A$.
Then the fiber is called a {\it Generalized Kummer variety} 
$K^nA:=Alb^{-1}(0)$.

\item O'Grady: exceptional examples of dimension 6, 10.
\end{enumerate}
\end{example}

\section{Cohomology of an IHS manifold}
\label{section-cohomology-ihs}

\noindent
$V:=H^2(X,\mathbb{Q})$ is a K3-type Hodge structure.
Since $H^0(X,\Omega_{X_T}^2)=\mathbb{C}\sigma$
where $\sigma$ is the holomorphic symplectic form,
we must have $h^{2,0}=h^{0,2}=1$.

\begin{theorem}[Beauville-Bogomolov-Fujiki]
\label{theorem-bbf}
There exists a constant $c_X \in \mathbb{Q}$ and a quadratic form
$q \in S^2V^*$ of signature $(3,b_2(X)-3)$ 
such that $\forall  a \in V$ 
$$
\int_X a^{2n}=c_X q(a)^n \qquad \text{(Fujiki relation)}
$$
where $\int_X$ is the pairing with the fundamental class
defined by $I$.
\end{theorem}

[We have the integral of a polynomial on $V$,
and it happens to be the $n$-th power of a quadratic form $q$.
In fact, the integral does not depend on the choice of complex
structure.]


We call $q$ the  {\it BBF form}.
The HS on $H^2$ is given by
$$
\rho:U(1) \to \text{SO}(V\otimes \mathbb{R},q).
$$

If $\omega$ is a Kähler form on $X$,
then $q([\omega])>0$.
If the class of  $\omega$ is rational,
i.e. $[\omega] \in H^2(X,\mathbb{Q})$,
then $H^2(X,\mathbb{Q})_{\text{prim}}=[\omega]^\perp$ 
is polarized by $q$.

Since $\omega$ is Kähler, we have the Lefschetz triple
\begin{align*}
L_\omega&=[\omega] \cup  -,\qquad  \Lambda_\omega=\text{dual of }L_\omega,
\qquad [L_\omega,\Lambda_\omega]=\Theta,
\end{align*}

\noindent
[which give us a representation of $\mathfrak{sl}_2$].

Consider $L_\omega, \Lambda_\omega$ for all Kähler forms
on all deformations of  $(X,I)$.
The smallest Lie subalgebra of $\text{End}(H^\bullet(X,\mathbb{R}))$
that contains all $L_\omega,\Lambda_\omega$?

\section{Mukai extension}
\label{section-mukai-extension}

\noindent
Let $(V,q)$, $\tilde{V}=V \oplus U$,
where $U$ is the hyperbolic plane
$q_n=\begin{pmatrix}0&1\\ 1&0\end{pmatrix}$.
Let $\tilde{q}=q \oplus q_n$.
We have a grading 
$$
\tilde{V}=\underbrace{\mathbb{Q}e_0}_{0}
\oplus \underbrace{V}_{2}\oplus\underbrace{\mathbb{Q}e_4}_{4}.
$$

Then
\begin{align*}
\mathfrak{so}(\tilde{V},\tilde{q})
&\simeq \Lambda^2\tilde{V}\\
&=\underbrace{e_0 \wedge V}_{\substack{-2 \\ \Lambda_\omega \in }}
\oplus\underbrace{(\underbrace{\Lambda^2V}_{\mathfrak{so}(V,q)} 
\oplus e_0 \wedge e_n)}_{0}
\oplus \underbrace{e_n\wedge V}_{\substack{2 \\ L_\omega \in}}
\end{align*}

\noindent
is in fact the minimal Lie subalgebra containing
the Lefschetz operators:

\begin{theorem}[Verbitsky, Looijenga-Lunts]
\label{theorem-vll}
The smallest subalegbra of $\text{End}(H^\bullet(X,\mathbb{R}))$
containing $L_\omega, \Lambda_\omega$ 
is isomorphic to $\mathfrak{so}(\tilde{V},\tilde{q})$.
\end{theorem}

\noindent
This implies that we have an action
of $\mathfrak{so}(V,q)$ on $H^k(X,\mathbb{Q})$ for all $k$.
Thus, $H^k(X,\mathbb{Q})$ is a representation of
$\text{Spin}(X,q)$ for $k \equiv 0 (\text{mod})$ of $\text{SO}(V,q)$.

The HS of $H^k(X,\mathbb{Q})$ are induced by
$\rho:U(1) \to \text{Spin}(V\otimes \mathbb{R},q)$.

[The idea is to understand the cohomology
as a representation of $\mathfrak{so}(\tilde{V},\tilde{q})$.]

Let $(\mathbb{Q}-HS)^{\text{tori}}$ be the full
$\otimes$-subcategory of $(\mathbb{Q}-HS)$ generated
by $H^1(T,\mathbb{Q})$ for tori $T=\mathbb{C}/\Lambda$.
We have
$$
(\mathbb{Q}-HS)^{\text{tori}}\supset
(\mathbb{Q}-HS)^{ab}
\subset (\mathbb{Q}-HS)^{pol}.
$$

\begin{theorem}[Kurnosov-Soldatenkov-Verbitsky]
\label{theorem-ksv}
Let $X$ be an ISH manifold.
Then for all $k$,
$H^k(X,\mathbb{Q}) \in (Q-HS)^{\text{tori}}$.
If $X$ is projective then
$$
H^k(X,\mathbb{Q}) \in (\mathbb{Q}-HS)^{ab}.
$$
\end{theorem}

\section{Idea of the proof}
\label{section-proof}

\noindent
[We need to understand cohomology groups
as representations of $Spin(V,q)$.]

\begin{enumerate}
\item (Clifford algebras.)
Let $(V,q)$ be a quadratic vector space over $\mathbb{Q}$.
The Clifford algebra $Cl(V,q)=\bigoplus_{i \geq 0}V^{\otimes i}/J$
where $J$ is the ideal generated by
$v \otimes v-q(v)\cdot 1$ for all $v \in V$.

In $Cl(V,q)$, $v^2=q(v)\cdot 1$,
$uv-vu=2q(u,v)\cdot 1$
Also,
$$
\alpha(v_1 \otimes … \otimes v_k)
=(-1)^k v_1 \otimes … \otimes v_k,\qquad 
\beta(v_1 \otimes … \otimes v_k)
=v_k \otimes v_{k-1} \otimes … \otimes v_1
$$
descend to $Cl(V,q)$.

Foro all  $a \in cl(V,q)$ write $\bar{a}=\alpha\beta(a)$,
$Cl(V,q)=Cl^0(V,q) \oplus Cl^1(V,q)$ 
is a $\mathbb{Z}/2$-grading.

Denote $Cl(V,q)=\mathcal{C}$ and $\mathcal{C}^\times$
the invertible elements of $\mathcal{C}$.
For all $x \in V$ such that $q(x) \neq 0$,
$x \in \mathcal{C}^\times$ since
$x^2=q(x)\cdot 1 \implies  x^{-1}=\frac{x}{q(x)}$.

We have a natural embedding $V \xymatrix{\ar@{^{(}->}[r]&} \mathcal{C}$.
Let $\{ x \in \mathcal{C}^\times:\alpha(x)Vx^{-1}\subset V\}:=G$.

If $x \in V$, $q(x) \neq 0$, $y \in V$,
\begin{align*}
\alpha(x) y x^{-1}&=-xy\frac{x}{q(x)}\\
&=(yx-2q(x,y)\frac{x}{q(x)}\\
&=y-2\frac{q(x,y)}{q(x)}x,
\end{align*}

\noindent
i.e. a reflection.

$x \in G$, $y \in V$,
\begin{align*}
q(\alpha(x)yx^{-1})&=\underbrace{\alpha(x)yx^{-1}}_{V}\alpha(x)yx^{-1}\\
&=-\alpha(\alpha(x)yx^{-1})\alpha(x)yx^{-1}\\
&=xy\alpha(x)^{-1}\alpha(x)yx^{-1}\\
&=q(y).
\end{align*}

\noindent
We get a representation $\mu:G \to O(V,q)$.
$\mu$ is surjective (because $\text{Im}(\mu)$ contains reflections
[and there's a theorem that $O(V,q)$ is generated by reflections]).

$\Ker(\mu)=\mathbb{Q}^*\cdot 1$.
[The kernel is quite big so let's make it a bit smaller.]
Define a norm $N:G \to \mathbb{Q}^*$ by $N(a)=a\cdot \bar{a}$.
Define $\Ker(N):=Pin(V,q)$.
Then $Pin(V,q) \cap \mathcal{C}^0=\text{Spin}(V,q)$.
Then $\mu:\text{Spin}(V,q)\to \text{SO}(V,q)$.
Thus, $\Ker \mu=\{\pm 1\}$,
[the double cover we were looking for].

\item (Key observation.)
The morphism $\rho:U(1) \to \text{SO}(V\otimes \mathbb{R},q)$ 
defining the HS on $H^2(X,\mathbb{Q})$
lifts to $\text{Spin}(V\otimes \mathbb{R},q)$.

To see why, write
$$
V \otimes \mathbb{R}= \underbrace{(V^{2,0}\oplus V^{0,2})}_{V^+_{\mathbb{R}}}
\oplus V^{1,1}_\mathbb{R}.
$$
Then $\dim V^+_\mathbb{R}=2$.

We have a double cover
$$
\xymatrix{
\rho:U(1)\ar[rd]_{2:1}\ar[r]&  \text{SO}(V_{\mathbb{R}}^+)\ar@{=}[d]\\
& U(1).
}
$$

We have
$$
Cl(V \otimes \mathbb{R},q)
= \underbrace{ Cl(V_\mathbb{R}^+)}_{4-\dim}
\otimes Cl(V^{1,1}_\mathbb{R}).
$$
Then $V_\mathbb{R}^+=\left<e,f\right>$ --- an orthonormal basis,
$Cl^0(V_\mathbb{R}^+)=\left<1,ef\right>\simeq \mathbb{R}^2$.

$$
\xymatrix{
\text{Spin}(V_\mathbb{R}^+)\cong U(1)
\ar[d]^{2:1}\\
\text{SO}(V_\mathbb{R}^+).
}
$$

\item (Definition of the Hodge structure.)
[What's the use of lifting?]
We have a lift $\tilde{\rho}:U(1) \to \text{Spin}(V\otimes \mathbb{R},q)$.
Note that $Cl(V_\mathbb{R}^+)$ is two copies of the standard
representation of $U(1)$.

$$
Cl(V \otimes \mathbb{R},q)
\simeq 
\bigoplus \text{standard reps of }U(1) \simeq \text{Spin}(V_\mathbb{R}^+).
$$
Denote $H:=Cl(V,q)$,
$\implies H \otimes \mathbb{C}=H^{1,0} \oplus H^{0,1}$.
Then $H \in (\mathbb{Q}-HS)^{\text{tori}}$.

\item (Constructing the embedding.)
We want to construct $V \xymatrix{\ar@{^{(}->}[r]&} H^*$.
[We should use some bilinear form on the Clifford algebra.]
In fact, there exists a non-degenerate pairing
$\tau:\mathcal{C} \times \mathcal{C} \to \mathbb{Q}$,
$\tau(x,y)=\text{Tr}(x\bar{y})$,
where we consider $x\bar{y}$ as an endomorphism of $\mathcal{C}$ 
acting by left multiplication and 
Tr is the trace of this endomorphism.
In fact, $\tau(x,y)=\tau(y,x)$,
$\tau(ax,y)=\tau(x,\bar{a}y)$ and
$\tau$ is Pin-invariant, i.e. 
$\forall  \in Pin(V,q)$, $\tau(ax,ay)=\tau(x,y)$.

Analogously, on $H=\mathcal{C}^{\oplus m}$
we have $\tau:H \otimes H \to \mathbb{Q}$.

To construct the embedding, for any $x \in V$ 
let $\omega_x(u,v)=\tau(xu,v)$.
Then $\omega_x(u,v)=-\omega_x(v,u)$,
$\omega_x \in \Lambda^2H^*$.
This defines $\varphi:V \to \Lambda^2H^*$,
$x \mapsto \omega_x$, the embedding we wanted.

$\varphi$ is a morphism of $\text{Spin}(V,q)$-modules
[representations].
Thus, $\varphi$ is a morphism of Hodge structures.

[This finishes the work for K3 surfaces.
For general HK manifolds we need to work a bit more,
how to embed all cohomology groups?
In fact, this is where we will use the $m$ 
in $H=\mathcal{C}^{\oplus m}$.]

\item (Extra work needed for general HK manifolds.)
Let $\dim H=4N$ and $\text{Im}(\varphi)=W$.
For $p \in S^{2N}H^*$, 
$p(\omega)=\omega^{2N}\in \Lambda^{2N}(H^*)\simeq\mathbb{Q}$.

[$p$ is a polynomial, when we restrict it:]
$p|_W \in (S^{2N}W^* )^{\text{Spin}(V,q)}$.
The classical invariant theory implies [that this polynomial
is a power of the quadratic form, i.e.]
$p = A \cdot q^N, A \in \mathbb{Q}$.
So $p(\omega)=\omega^{2N}=A q(\omega)^N$,
a ``Fujiki relation''.

\begin{align*}
\mathfrak{so}(\tilde{V},\tilde{q})&=
\underbrace{V}_{\substack{2 \\ \Lambda_x\in}}
\oplus \underbrace{\mathfrak{so}(\tilde{V},\tilde{q})}_{0}
\oplus \underbrace{V}_{\substack{2 \\ L_x\in}}
\end{align*}

\noindent
Let $\eta \in \Lambda^\bullet H^*$, $x \in V$,
\begin{equation}
\label{equation-rep}
L_x \eta=\omega_x \wedge \eta,\qquad 
\Lambda_x\eta=i_\eta \omega_x
\end{equation}
[Where $i$ denotes interior multiplication].
\begin{align*}
\varphi: V &\xymatrix{\ar@{^{(}->}[r]&} \Lambda^2H^*\\
x &\longmapsto \omega_x.
\end{align*}

\begin{proposition}
\label{proposition-rep}
Equation \ref{equation-rep} define a $
\mathfrak{so}(\tilde{V},\tilde{q})$-representation
on $\Lambda^\bullet H^*$.
The corresponding $\text{Spin}(\tilde{V},\tilde{q})$-representation
is faithful.
\end{proposition}

In conclusion,
for all $\mathfrak{so}(\tilde{V},\tilde{q})$ representation $U$ 
exists $m>0$ such that for $H=\mathcal{C}^{\oplus m}$,
$U \subset \Lambda^\bullet H^*$ 
as a $\mathfrak{so}(\tilde{V},\tilde{q})$ subrepresentation.
$U$ is contained in a direct sum of 
 $(\Lambda^\bullet \mathcal{C}^* )^{\otimes i}
\simeq \Lambda^\bullet({\mathcal{C}^* }^{\oplus i})$.

Take $U = H^2(X,\mathbb{Q})$,
qed an embedding of $\text{Spin}(\tilde{V},\tilde{q})$ 
representations,
$H^\bullet(X,\mathbb{Q}) \xymatrix{\ar@{^{(}->}[r]&} \Lambda^\bullet H^*$
is an embedding of HS.
\end{enumerate}

[Thus the HS on the cohomology of $X$ are abelian.]








\section{References}
\label{section-references}

\begin{enumerate}
\item Deligne, Milne, Ogus, Shih,
``Hodge cycles, motives and Shimura varieties''.
We have seen mostly Chapter 1 and a bit of Chapter 2.

\item Deligne: 
\begin{itemize}
\item 
``La conj. de Weil pour les surfaces K3''

\item ``Théorie de Hodge II''

\item ``Variétes de Shimura: …''
\end{itemize}

\item Kingler: ``Hodge theory, between
algebraicity and transcendence 
+ reference therein''.
\end{enumerate}

\end{document}
